\documentclass[]{article}
\usepackage[T1]{fontenc}
\usepackage{lmodern}
\usepackage{amssymb,amsmath}
\usepackage{ifxetex,ifluatex}
\usepackage{fixltx2e} % provides \textsubscript
% use upquote if available, for straight quotes in verbatim environments
\IfFileExists{upquote.sty}{\usepackage{upquote}}{}
\ifnum 0\ifxetex 1\fi\ifluatex 1\fi=0 % if pdftex
  \usepackage[utf8]{inputenc}
\else % if luatex or xelatex
  \ifxetex
    \usepackage{mathspec}
    \usepackage{xltxtra,xunicode}
  \else
    \usepackage{fontspec}
  \fi
  \defaultfontfeatures{Mapping=tex-text,Scale=MatchLowercase}
  \newcommand{\euro}{€}
\fi
% use microtype if available
\IfFileExists{microtype.sty}{\usepackage{microtype}}{}
\ifxetex
  \usepackage[setpagesize=false, % page size defined by xetex
              unicode=false, % unicode breaks when used with xetex
              xetex]{hyperref}
\else
  \usepackage[unicode=true]{hyperref}
\fi
\hypersetup{breaklinks=true,
            bookmarks=true,
            pdfauthor={機械設計工程系二甲},
            pdftitle={2014 協同產品設計實習報告},
            colorlinks=true,
            citecolor=blue,
            urlcolor=blue,
            linkcolor=magenta,
            pdfborder={0 0 0}}
\urlstyle{same}  % don't use monospace font for urls
\setlength{\parindent}{0pt}
\setlength{\parskip}{6pt plus 2pt minus 1pt}
\setlength{\emergencystretch}{3em}  % prevent overfull lines
\setcounter{secnumdepth}{0}

\title{2014 協同產品設計實習報告}
\author{機械設計工程系二甲}
\date{April 23, 2014}

 
\usepackage{xeCJK}    % 中英文字行分開設置 
\usepackage[T1]{fontspec}    %設定字體用 
\usepackage{graphicx} 
\usepackage{fancyvrb} % for frame on Verbatim 
\setCJKmainfont{新細明體}
\begin{document}
\maketitle

{
\hypersetup{linkcolor=black}
\setcounter{tocdepth}{3}
\tableofcontents
}
\section{前言}\label{ux524dux8a00}

協同產品設計實習課程目標\footnote{這是註解的用法.}

Here is an inline note.\footnote{Inlines notes are easier to write,
  since you don't have to pick an identifier and move down to type the
  note.}

學習協同產品設計流程與環境的基本原理與架構.

學習如何在網際專案管理系統的協助下, 進行協同產品設計.

學習如何在協同設計流程中, 進行有效率的工程設計表達與產品資料管理.

延續程式語言與電腦輔助設計實習課程,
學習如何建構協同產品設計環境所需的工具.

\section{Pandoc 手冊}\label{pandoc-ux624bux518a}

http://johnmacfarlane.net/pandoc/README.html

footnotes

tables

flexible ordered lists

definition lists

fenced code blocks

superscript

subscript

strikeout

title blocks

automatic tables of contents

embedded LaTeX math

citations

markdown inside HTML block elements

\section{網際正齒輪減速機設計(2ag1)}\label{ux7db2ux969bux6b63ux9f52ux8f2aux6e1bux901fux6a5fux8a2dux8a082ag1}

正齒輪相關設計公式

新增 commit 9 之後的資料

\section{設計程式架構}\label{ux8a2dux8a08ux7a0bux5f0fux67b6ux69cb}

傳遞功率

新增 commit 9 之後的資料

\section{結果與討論}\label{ux7d50ux679cux8207ux8a0eux8ad6}

這裡是結果與討論

新增 commit 9 之後的資料

\section{網際鼓式煞車設計(2ag2)}\label{ux7db2ux969bux9f13ux5f0fux715eux8ecaux8a2dux8a082ag2}

有關鼓式煞車

\section{程式設計架構}\label{ux7a0bux5f0fux8a2dux8a08ux67b6ux69cb}

鼓式煞車

\section{結果與討論}\label{ux7d50ux679cux8207ux8a0eux8ad6-1}

這裡是結果與討論

\section{W12任務(2ag3)}\label{w12ux4efbux52d92ag3}

請各組將第九週考試的摘要報告放入 Github 協同專案中的分組報告區

\section{摘要報告}\label{ux6458ux8981ux5831ux544a}

\begin{verbatim}
第一題:
\end{verbatim}

請寫一個執行時可以列出以十為底對數表的網際 Python 程式,然後 Push 到個人
bitbucket 空間,而且同步指到 OpenShift 個人帳號上執行。

\begin{verbatim}
解題心得:
\end{verbatim}

可利用程式算出對數的值,就不需要一個一個按計算機。

\begin{verbatim}
第二題:
\end{verbatim}

請在個人的 OpenShift
平台上建立一個能夠列印出與九九乘法表結果完全相同的網際程式,接著在乘法表上端加上兩個輸入表單,讓使用者輸入兩個整數,按下送出鍵後,程式會列出以此兩個整數為基底的乘法表,例如:若兩個欄位都輸入:
9,則列出九九乘法表,若輸入 9,20,則列出 9×20 的乘法表。

解題心得:

利用網站輸入值並算出,並需要有兩個輸入表單。

\begin{verbatim}
第三題:
\end{verbatim}

請在各組的雲端 dokuwiki 中,新增帳號與密碼都是由 abc001
\textasciitilde{} abc399 字串所組成的 399
名用戶登入對應資料,並將製作過程與驗證流程拍成 flv 後上傳到個人的 Vimeo
資料區,並將連結放在個人第九週頁面。

解題心得:

可以利用程式創造多人帳密,一起管理網站,減少一個一個創建帳密,又會有被盜用的風險。

import math count = 0 text=`\%04d' for i in range(100,200+10):
print(text\%round(math.log(i/100,10)*10000,0),end=`-') count += 1
if(count == 10): print() count = 0

class Example(object): \emph{cp}config = \{ \# if there is no utf-8
encoding, no Chinese input available `tools.encode.encoding': `utf-8',
`tools.sessions.on' : True, `tools.sessions.storage\_type' : `file',
`tools.sessions.locking' : `explicit', `tools.sessions.storage\_path' :
data\_dir+`/tmp', \# session timeout is 60 minutes
`tools.sessions.timeout' : 60 \}

@cherrypy.expose def index(self): output = ''

\begin{verbatim}
form = '''
<form action='action'>
num1:<INPUT type='text' name='num1'>
num2:<INPUT type='text'  name='num2'>
<input type=submit>
<input type=reset>
</form>
'''
output += form
return output
\end{verbatim}

@cherrypy.expose def action(self, num1=9, num2=9): num1 = int(num1) num2
= int(num2) output = `' for i in range(num1): for j in range(num2):
output += str(i) +'\emph{`+ str(j) +'=' + str(i}j) + `' return output

import hashlib \#convert user\_password into sha1 encoded string def
gen\_password(user\_password): return
hashlib.sha1(user\_password.encode(``utf-8'')).hexdigest()
text=`abc\%03d' for i in range(1,399+1):
print(text\%(i)+`:'+gen\_password(text\%(i))+`:'+text\%(i)+`@gmail.com:'+text\%(i)+`@gmail.com:'+`user')

\section{網際鼓式煞車設計(2ag4)}\label{ux7db2ux969bux9f13ux5f0fux715eux8ecaux8a2dux8a082ag4}

有關鼓式煞車

\section{程式設計架構}\label{ux7a0bux5f0fux8a2dux8a08ux67b6ux69cb-1}

鼓式煞車

\section{結果與討論}\label{ux7d50ux679cux8207ux8a0eux8ad6-2}

這裡是結果與討論

\section{網際四連桿機構設計(2ag5)}\label{ux7db2ux969bux56dbux9023ux687fux6a5fux69cbux8a2dux8a082ag5}

有關連桿設計

\section{程式設計架構}\label{ux7a0bux5f0fux8a2dux8a08ux67b6ux69cb-2}

連桿計算

\section{結果與討論}\label{ux7d50ux679cux8207ux8a0eux8ad6-3}

這裡是結果與討論

by 2014cdag5

\section{網際鼓式煞車設計(2ag6)}\label{ux7db2ux969bux9f13ux5f0fux715eux8ecaux8a2dux8a082ag6}

有關鼓式煞車

\section{程式設計架構}\label{ux7a0bux5f0fux8a2dux8a08ux67b6ux69cb-3}

鼓式煞車

\section{結果與討論}\label{ux7d50ux679cux8207ux8a0eux8ad6-4}

這裡是結果與討論

\section{網際鼓式煞車設計(2ag7)}\label{ux7db2ux969bux9f13ux5f0fux715eux8ecaux8a2dux8a082ag7}

有關鼓式煞車

\section{程式設計架構}\label{ux7a0bux5f0fux8a2dux8a08ux67b6ux69cb-4}

鼓式煞車

\section{結果與討論}\label{ux7d50ux679cux8207ux8a0eux8ad6-5}

這裡是結果與討論

\section{網際鼓式煞車設計(2ag7)}\label{ux7db2ux969bux9f13ux5f0fux715eux8ecaux8a2dux8a082ag7-1}

有關鼓式煞車

\section{程式設計架構}\label{ux7a0bux5f0fux8a2dux8a08ux67b6ux69cb-5}

鼓式煞車

\section{結果與討論}\label{ux7d50ux679cux8207ux8a0eux8ad6-6}

這裡是結果與討論

\section{網際鼓式煞車設計(2ag10)}\label{ux7db2ux969bux9f13ux5f0fux715eux8ecaux8a2dux8a082ag10}

有關鼓式煞車

\section{程式設計架構}\label{ux7a0bux5f0fux8a2dux8a08ux67b6ux69cb-6}

鼓式煞車

\section{結果與討論}\label{ux7d50ux679cux8207ux8a0eux8ad6-7}

這裡是結果與討論

\section{網際鼓式煞車設計(2ag11)}\label{ux7db2ux969bux9f13ux5f0fux715eux8ecaux8a2dux8a082ag11}

有關鼓式煞車

\section{程式設計架構}\label{ux7a0bux5f0fux8a2dux8a08ux67b6ux69cb-7}

鼓式煞車

\section{結果與討論}\label{ux7d50ux679cux8207ux8a0eux8ad6-8}

這裡是結果與討論

\section{網際鼓式煞車設計(2ag12)}\label{ux7db2ux969bux9f13ux5f0fux715eux8ecaux8a2dux8a082ag12}

有關鼓式煞車

\section{程式設計架構}\label{ux7a0bux5f0fux8a2dux8a08ux67b6ux69cb-8}

鼓式煞車

\section{結果與討論}\label{ux7d50ux679cux8207ux8a0eux8ad6-9}

這裡是結果與討論

\section{網際鼓式煞車設計(2ag13)}\label{ux7db2ux969bux9f13ux5f0fux715eux8ecaux8a2dux8a082ag13}

有關鼓式煞車

\section{程式設計架構}\label{ux7a0bux5f0fux8a2dux8a08ux67b6ux69cb-9}

鼓式煞車

\section{結果與討論}\label{ux7d50ux679cux8207ux8a0eux8ad6-10}

這裡是結果與討論

\section{網際鼓式煞車設計(2ag14)}\label{ux7db2ux969bux9f13ux5f0fux715eux8ecaux8a2dux8a082ag14}

有關鼓式煞車

\section{程式設計架構}\label{ux7a0bux5f0fux8a2dux8a08ux67b6ux69cb-10}

鼓式煞車

\section{結果與討論}\label{ux7d50ux679cux8207ux8a0eux8ad6-11}

這裡是結果與討論

\section{網際鼓式煞車設計(2ag16)}\label{ux7db2ux969bux9f13ux5f0fux715eux8ecaux8a2dux8a082ag16}

有關鼓式煞車

\section{程式設計架構}\label{ux7a0bux5f0fux8a2dux8a08ux67b6ux69cb-11}

鼓式煞車

\section{結果與討論}\label{ux7d50ux679cux8207ux8a0eux8ad6-12}

這裡是結果與討論

\section{網際鼓式煞車設計(2ag17)}\label{ux7db2ux969bux9f13ux5f0fux715eux8ecaux8a2dux8a082ag17}

有關鼓式煞車g17

\section{程式設計架構}\label{ux7a0bux5f0fux8a2dux8a08ux67b6ux69cb-12}

鼓式煞車

\section{結果與討論}\label{ux7d50ux679cux8207ux8a0eux8ad6-13}

這裡是結果與討論

\section{網際鼓式煞車設計(2ag21)}\label{ux7db2ux969bux9f13ux5f0fux715eux8ecaux8a2dux8a082ag21}

有關鼓式煞車

\section{程式設計架構}\label{ux7a0bux5f0fux8a2dux8a08ux67b6ux69cb-13}

鼓式煞車

\section{結果與討論}\label{ux7d50ux679cux8207ux8a0eux8ad6-14}

這裡是結果與討論00123

\section{網際 OpenJSCAD
程式設計(coursemdetw)}\label{ux7db2ux969b-openjscad-ux7a0bux5f0fux8a2dux8a08coursemdetw}

將 Spur 改為凸輪零件成型

\section{設計程式架構}\label{ux8a2dux8a08ux7a0bux5f0fux67b6ux69cb-1}

定義凸輪設計公式

\section{結果與討論}\label{ux7d50ux679cux8207ux8a0eux8ad6-15}

有關凸倫程式設計的結果與討論

\end{document}
