\documentclass[]{article}
\usepackage[T1]{fontenc}
\usepackage{lmodern}
\usepackage{amssymb,amsmath}
\usepackage{ifxetex,ifluatex}
\usepackage{fixltx2e} % provides \textsubscript
% use upquote if available, for straight quotes in verbatim environments
\IfFileExists{upquote.sty}{\usepackage{upquote}}{}
\ifnum 0\ifxetex 1\fi\ifluatex 1\fi=0 % if pdftex
  \usepackage[utf8]{inputenc}
\else % if luatex or xelatex
  \ifxetex
    \usepackage{mathspec}
    \usepackage{xltxtra,xunicode}
  \else
    \usepackage{fontspec}
  \fi
  \defaultfontfeatures{Mapping=tex-text,Scale=MatchLowercase}
  \newcommand{\euro}{€}
\fi
% use microtype if available
\IfFileExists{microtype.sty}{\usepackage{microtype}}{}
\usepackage{color}
\usepackage{fancyvrb}
\newcommand{\VerbBar}{|}
\newcommand{\VERB}{\Verb[commandchars=\\\{\}]}
\DefineVerbatimEnvironment{Highlighting}{Verbatim}{commandchars=\\\{\}}
% Add ',fontsize=\small' for more characters per line
\newenvironment{Shaded}{}{}
\newcommand{\KeywordTok}[1]{\textcolor[rgb]{0.00,0.44,0.13}{\textbf{{#1}}}}
\newcommand{\DataTypeTok}[1]{\textcolor[rgb]{0.56,0.13,0.00}{{#1}}}
\newcommand{\DecValTok}[1]{\textcolor[rgb]{0.25,0.63,0.44}{{#1}}}
\newcommand{\BaseNTok}[1]{\textcolor[rgb]{0.25,0.63,0.44}{{#1}}}
\newcommand{\FloatTok}[1]{\textcolor[rgb]{0.25,0.63,0.44}{{#1}}}
\newcommand{\CharTok}[1]{\textcolor[rgb]{0.25,0.44,0.63}{{#1}}}
\newcommand{\StringTok}[1]{\textcolor[rgb]{0.25,0.44,0.63}{{#1}}}
\newcommand{\CommentTok}[1]{\textcolor[rgb]{0.38,0.63,0.69}{\textit{{#1}}}}
\newcommand{\OtherTok}[1]{\textcolor[rgb]{0.00,0.44,0.13}{{#1}}}
\newcommand{\AlertTok}[1]{\textcolor[rgb]{1.00,0.00,0.00}{\textbf{{#1}}}}
\newcommand{\FunctionTok}[1]{\textcolor[rgb]{0.02,0.16,0.49}{{#1}}}
\newcommand{\RegionMarkerTok}[1]{{#1}}
\newcommand{\ErrorTok}[1]{\textcolor[rgb]{1.00,0.00,0.00}{\textbf{{#1}}}}
\newcommand{\NormalTok}[1]{{#1}}
\usepackage{longtable,booktabs}
\ifxetex
  \usepackage[setpagesize=false, % page size defined by xetex
              unicode=false, % unicode breaks when used with xetex
              xetex]{hyperref}
\else
  \usepackage[unicode=true]{hyperref}
\fi
\hypersetup{breaklinks=true,
            bookmarks=true,
            pdfauthor={機械設計工程系二甲},
            pdftitle={2014 協同產品設計實習報告},
            colorlinks=true,
            citecolor=blue,
            urlcolor=blue,
            linkcolor=magenta,
            pdfborder={0 0 0}}
\urlstyle{same}  % don't use monospace font for urls
\setlength{\parindent}{0pt}
\setlength{\parskip}{6pt plus 2pt minus 1pt}
\setlength{\emergencystretch}{3em}  % prevent overfull lines
\setcounter{secnumdepth}{0}

\title{2014 協同產品設計實習報告}
\author{機械設計工程系二甲}
\date{April 23, 2014}

 
\usepackage{xeCJK}    % 中英文字行分開設置 
\usepackage[T1]{fontspec}    %設定字體用 
\usepackage{graphicx} 
\usepackage{fancyvrb} % for frame on Verbatim 
\setCJKmainfont{新細明體}
\begin{document}
\maketitle

{
\hypersetup{linkcolor=black}
\setcounter{tocdepth}{3}
\tableofcontents
}
\section{前言}\label{ux524dux8a00}

協同產品設計實習課程目標\footnote{這是註解的用法.}

Here is an inline note.\footnote{Inlines notes are easier to write,
  since you don't have to pick an identifier and move down to type the
  note.}

學習協同產品設計流程與環境的基本原理與架構.

學習如何在網際專案管理系統的協助下, 進行協同產品設計.

學習如何在協同設計流程中, 進行有效率的工程設計表達與產品資料管理.

延續程式語言與電腦輔助設計實習課程,
學習如何建構協同產品設計環境所需的工具.

\section{Pandoc 手冊}\label{pandoc-ux624bux518a}

http://johnmacfarlane.net/pandoc/README.html

footnotes

tables

flexible ordered lists

definition lists

fenced code blocks

superscript

subscript

strikeout

title blocks

automatic tables of contents

embedded LaTeX math

citations

markdown inside HTML block elements

\section{協同產品設計實習專案(2ag1)}\label{ux5354ux540cux7522ux54c1ux8a2dux8a08ux5be6ux7fd2ux5c08ux68482ag1}

\subsection{組員:}\label{ux7d44ux54e1}

40123101

40123102

40123132

OpenShift 網站: http://2014cdag1-cadp13ag6.rhcloud.com/

\subsection{w8考試}\label{w8ux8003ux8a66}

\begin{enumerate}
\def\labelenumi{\arabic{enumi}.}
\itemsep1pt\parskip0pt\parsep0pt
\item
  請寫一個執行時可以列出 9×9 乘法表的網際 Python 程式, 然後 Push 到個人
  bitbucket 空間, 而且同步指到 OpenShift 個人帳號上執行.
\end{enumerate}

\begin{itemize}
\item
  程式碼:

\begin{Shaded}
\begin{Highlighting}[]
    \KeywordTok{for} \NormalTok{x in }\DataTypeTok{range}\NormalTok{(}\DecValTok{0}\NormalTok{,}\DecValTok{10}\NormalTok{):}
        \KeywordTok{for} \NormalTok{y in }\DataTypeTok{range}\NormalTok{(}\DecValTok{0}\NormalTok{,}\DecValTok{10}\NormalTok{):}
            \DataTypeTok{print}\NormalTok{(x,}\StringTok{'*'}\NormalTok{,y,}\StringTok{'='}\NormalTok{,x*y)}
\end{Highlighting}
\end{Shaded}
\item
  解題過程:

  打完程式後,之後我把它上傳到bitbucket,git add .→ git commit -m
  ``99''→ git push,之後為了要把99乘法表顯示在openshift
  CMSimply上,進去openshift CMSimply的資料夾→ wsgi資料夾→
  application檔案,在裡面新增一個class 的格式,之後連線FileZilla ,
  把application檔案覆蓋到app-root/runtime/repo/wsgi的application檔案下,之後進去openshift
  CMSimply的網站,就顯示99乘法表了。
\item
  解題心得:

  雖然要打一個99乘法表的程式很快,但要如何顯示在遠端的網頁上,思考了很久,在課堂後請教了TA,最後應用cherrypy的方式,在application檔案,新增一個class
  的格式,就完成了。
\end{itemize}

\begin{enumerate}
\def\labelenumi{\arabic{enumi}.}
\setcounter{enumi}{1}
\itemsep1pt\parskip0pt\parsep0pt
\item
  請將上述執行過程錄為 flv 後, 上傳到個人的 Vemeo 空間中,
  並將網址回報到各組網站 (dokuwiki 與 CMSimply)與報告中,
  並且將相關心得與報告連結登錄到 wiki.mde.tw 第八週的分組頁面中.
\end{enumerate}

\begin{itemize}
\item
  Bitbucket 連結: https://bitbucket.org/40123102/40123102cmsimply/src
\item
  Vemeo 空間: https://vimeo.com/user26935042/videos
\item
  copy 空間: https://copy.com/XohJLdUBRJdk
\item
  dokuwiki網站:
  https://40123102cdg1dokuwiki-cadp13ag6.rhcloud.com/doku.php?id=start
\item
  CMSimply網址: http://40123102cdg1-cadp13ag6.rhcloud.com/w8test/
\end{itemize}

\begin{enumerate}
\def\labelenumi{\arabic{enumi}.}
\setcounter{enumi}{2}
\itemsep1pt\parskip0pt\parsep0pt
\item
  請在各組的雲端 dowiki 中,根據下列 40 個帳號與密碼,
  新增對應的使用者帳號與密碼後, 將雲端網址登錄在 wiki.mde.tw
  各組第八週頁面中, 並說明操作過程與心得後, 將心得整理成 pdf 後繳交到
  course@mde.tw.
\end{enumerate}

\begin{itemize}
\item
  解題過程:

  程式做不出來,只好一個一個建立。
\item
  解題心得:

  想很久,還是無法想出程式,只好用最勤勞的方法,一個一個建立,唉!!!
\end{itemize}

\subsection{w9考試}\label{w9ux8003ux8a66}

\begin{enumerate}
\def\labelenumi{\arabic{enumi}.}
\itemsep1pt\parskip0pt\parsep0pt
\item
  請寫一個執行時可以列出以十為底對數表的網際 Python 程式, 然後 Push
  到個人 bitbucket 空間, 而且同步指到 OpenShift 個人帳號上執行.
\end{enumerate}

\begin{itemize}
\item
  程式碼:

\begin{Shaded}
\begin{Highlighting}[]
\CharTok{import} \NormalTok{math}
\NormalTok{count = }\DecValTok{0}
\NormalTok{text=}\StringTok{'}\OtherTok\DataTypeTok{round}\NormalTok{(math.log(i/}\DecValTok{100}\NormalTok{,}\DecValTok{10}\NormalTok{)*}\DecValTok{10000}\NormalTok{,}\DecValTok{0}\NormalTok{),end=}\StringTok{'-'}\NormalTok{)}
    \NormalTok{count += }\DecValTok{1}
    \KeywordTok{if}\NormalTok{(count == }\DecValTok{10}\NormalTok{):}
        \DataTypeTok{print}\NormalTok{()}
        \NormalTok{count = }\DecValTok{0}
\end{Highlighting}
\end{Shaded}
\item
  解題過程:

  打完程式後,有顯示出來,但無法顯示表單在CMSimply網址上,只能一個一個慢慢打。
\item
  解題心得:

  想了很久,還是無法,只會打程式,只好把顯示出來的數字一個一個加上,慢慢打。
\end{itemize}

\begin{enumerate}
\def\labelenumi{\arabic{enumi}.}
\setcounter{enumi}{1}
\itemsep1pt\parskip0pt\parsep0pt
\item
  請在個人的 OpenShift
  平台上建立一個能夠列印出與九九乘法表結果完全相同的網際程式,
  接著在乘法表上端加上兩個輸入表單, 讓使用者輸入兩個整數, 按下送出鍵後,
  程式會列出以此兩個整數為基底的乘法表, 例如: 若兩個欄位都輸入: 9,
  則列出九九乘法表, 若輸入 9, 20, 則列出 9×20 的乘法表.
\end{enumerate}

\begin{itemize}
\item
  解題過程:

  打完程式後,之後我把它上傳到bitbucket,git add .→ git commit -m
  ``w9\_2''→ git push,之後為了要把99乘法表顯示在openshift
  CMSimply上,進去openshift CMSimply的資料夾→ wsgi資料夾→
  application檔案,在裡面新增一個class 的格式,之後連線FileZilla ,
  把application檔案覆蓋到app-root/runtime/repo/wsgi的application檔案下,之後進去openshift
  CMSimply的網站,輸入 9, 20, 則顯示 9×20 的乘法表.
\item
  解題心得:

  這題跟上週小考差不多,只是還要再新增兩個輸入表單,所以還要在回想一下,但大致上ok,所以完成了。
\end{itemize}

\begin{enumerate}
\def\labelenumi{\arabic{enumi}.}
\setcounter{enumi}{2}
\itemsep1pt\parskip0pt\parsep0pt
\item
  請在各組的雲端 dokuwiki 中, 新增帳號與密碼都是由 abc001
  \textasciitilde{} abc399 字串所組成的 399 名用戶登入對應資料,
  並將製作過程與驗證流程拍成 flv 後上傳到個人的 Vimeo 資料區,
  並將連結放在個人第九週頁面.
\end{enumerate}

\begin{itemize}
\item
  程式碼:

\begin{Shaded}
\begin{Highlighting}[]
\CharTok{import} \NormalTok{hashlib}
\CommentTok{# convert user_password into sha1 encoded string}
\KeywordTok{def} \NormalTok{gen_password(user_password):}
    \KeywordTok{return} \NormalTok{hashlib.sha1(user_password.encode(}\StringTok{"utf-8"}\NormalTok{)).hexdigest()}
\NormalTok{text=}\StringTok{'abc}\OtherTok{%03d}\StringTok{'}
\KeywordTok{for} \NormalTok{i in }\DataTypeTok{range}\NormalTok{(}\DecValTok{1}\NormalTok{,}\DecValTok{399+1}\NormalTok{):}
    \DataTypeTok{print}\NormalTok{(text%(i)+}\StringTok{':'}\NormalTok{+gen_password(text%(i))+}\StringTok{':'}\NormalTok{+text%(i)+}\StringTok{'@gmail.com:'}\NormalTok{+text%(i)+}\StringTok{'@gmail.com:'}\NormalTok{+}\StringTok{'user'}\NormalTok{)}
\end{Highlighting}
\end{Shaded}
\item
  操作過程:

  先寫一個程式使他能產生帳號與密碼都是由 abc001 \textasciitilde{} abc399
  字串所組成的 399 名用戶登入對應資料,之後連線到openshift
  dokuwiki的FileZilla,把產生出來的程式碼貼到app-root/data/conf/users.auth.php下,之後開啟dokuwiki網站即可。
\item
  心得:

  一開始這個程式需要思考一下,如何使帳號與密碼都是由 abc001
  \textasciitilde{} abc399 字串所組成的 399
  名用戶,打出來後還要去讓所產生的程式碼符合一開始的格式,如同users.auth.php,思考很久最後終於成功了。
\end{itemize}

\subsection{w12}\label{w12}

\begin{itemize}
\itemsep1pt\parskip0pt\parsep0pt
\item
  第十二週任務:
\end{itemize}

\begin{enumerate}
\def\labelenumi{\arabic{enumi}.}
\itemsep1pt\parskip0pt\parsep0pt
\item
  請各組將第八週與第九週考試的摘要報告放入 Github
  協同專案中的分組報告區, 並將內容放入各組控管的同步 OpenShift 網站中.
  (佔期末成績 5分)
\item
  請各組設法利用 CherryPy 與 Pro/Web.Link 技術, 在 Github
  協同專案中建立一個能夠透過連結或表單控制 Cube 零件, a, b, 或 c
  零件尺寸的網際協同程式, 讓使用者可以藉以利用近端的 Creo
  嵌入式瀏覽器控制 Cube 的尺寸後列出該零件的體積大小. (佔期末成績 5分)
\end{enumerate}

\begin{itemize}
\itemsep1pt\parskip0pt\parsep0pt
\item
  心得:
\end{itemize}

\begin{enumerate}
\def\labelenumi{\arabic{enumi}.}
\itemsep1pt\parskip0pt\parsep0pt
\item
  在URL直接更改參數:在對應的程式內,把迴圈改成可在URL後面輸入變數,EX:http://127.0.0.1:8080/cdag1/cube1?w=20\&h=20\&l=20
\item
  在更改FOR迴圈時切記\{\}!
  在此DEF最下面還有一個括弧要刪除,否則會跑出CATCH
\end{enumerate}

\begin{itemize}
\item
  組員自評:

  40123101:85分

  40123102:95分

  40123132:95分
\end{itemize}

\section{網際鼓式煞車設計(2ag2)}\label{ux7db2ux969bux9f13ux5f0fux715eux8ecaux8a2dux8a082ag2}

有關鼓式煞車

\section{程式設計架構}\label{ux7a0bux5f0fux8a2dux8a08ux67b6ux69cb}

鼓式煞車

\section{結果與討論}\label{ux7d50ux679cux8207ux8a0eux8ad6}

這裡是結果與討論

\section{W12任務(2ag3)}\label{w12ux4efbux52d92ag3}

請各組將第九週考試的摘要報告放入 Github 協同專案中的分組報告區

\section{摘要報告}\label{ux6458ux8981ux5831ux544a}

\begin{verbatim}
第一題:
\end{verbatim}

請寫一個執行時可以列出以十為底對數表的網際 Python 程式,然後 Push 到個人
bitbucket 空間,而且同步指到 OpenShift 個人帳號上執行。

\begin{Shaded}
\begin{Highlighting}[]
\CharTok{import} \NormalTok{math}
\NormalTok{count = }\DecValTok{0}
\NormalTok{text=}\StringTok{'}\OtherTok\DataTypeTok{round}\NormalTok{(math.log(i/}\DecValTok{100}\NormalTok{,}\DecValTok{10}\NormalTok{)*}\DecValTok{10000}\NormalTok{,}\DecValTok{0}\NormalTok{),end=}\StringTok{'-'}\NormalTok{)}
    \NormalTok{count += }\DecValTok{1}
    \KeywordTok{if}\NormalTok{(count == }\DecValTok{10}\NormalTok{):}
        \DataTypeTok{print}\NormalTok{()}
        \NormalTok{count = }\DecValTok{0}
\end{Highlighting}
\end{Shaded}

\begin{verbatim}
解題心得:
\end{verbatim}

可利用程式算出對數的值,就不需要一個一個按計算機。

\begin{verbatim}
第二題:
\end{verbatim}

請在個人的 OpenShift
平台上建立一個能夠列印出與九九乘法表結果完全相同的網際程式,接著在乘法表上端加上兩個輸入表單,讓使用者輸入兩個整數,按下送出鍵後,程式會列出以此兩個整數為基底的乘法表,例如:若兩個欄位都輸入:
9,則列出九九乘法表,若輸入 9,20,則列出 9×20 的乘法表。

\begin{Shaded}
\begin{Highlighting}[]
\KeywordTok{class} \NormalTok{Example(}\DataTypeTok{object}\NormalTok{):}
\NormalTok{_cp_config = \{}
\CommentTok{# if there is no utf-8 encoding, no Chinese input available}
\StringTok{'tools.encode.encoding'}\NormalTok{: }\StringTok{'utf-8'}\NormalTok{,}
\CommentTok{'tools.sessions.on'} \NormalTok{: }\OtherTok{True}\NormalTok{,}
\CommentTok{'tools.sessions.st\textbackslash{}orage_type'} \NormalTok{: }\StringTok{'file'}\NormalTok{,}
\CommentTok{'tools.sessions.locking'} \NormalTok{: }\StringTok{'explicit'}\NormalTok{,}
\CommentTok{'tools.sessions.storage_path'} \NormalTok{: data_dir+}\StringTok{'/tmp'}\NormalTok{,}
\CommentTok{# session timeout is 60 minutes}
\CommentTok{'tools.sessions.timeout'} \NormalTok{: }\DecValTok{60}
\NormalTok{\}}

\OtherTok{@cherrypy.expose}
\KeywordTok{def} \NormalTok{index(}\OtherTok{self}\NormalTok{):}
    \NormalTok{output = }\StringTok{''}

    \NormalTok{form = }\StringTok{'''}
\StringTok{    <form action='action'>}
\StringTok{    num1:<INPUT type='text' name='num1'>}
\StringTok{    num2:<INPUT type='text'  name='num2'>}
\StringTok{    <input type=submit>}
\StringTok{    <input type=reset>}
\StringTok{    </form>}
\StringTok{    '''}
    \NormalTok{output += form}
    \KeywordTok{return} \NormalTok{output}
\OtherTok{@cherrypy.expose}
\KeywordTok{def} \NormalTok{action(}\OtherTok{self}\NormalTok{, num1=}\DecValTok{9}\NormalTok{, num2=}\DecValTok{9}\NormalTok{):}
    \NormalTok{num1 = }\DataTypeTok{int}\NormalTok{(num1)}
    \NormalTok{num2 = }\DataTypeTok{int}\NormalTok{(num2)}
    \NormalTok{output = }\StringTok{''}
    \KeywordTok{for} \NormalTok{i in }\DataTypeTok{range}\NormalTok{(num1):}
        \KeywordTok{for} \NormalTok{j in }\DataTypeTok{range}\NormalTok{(num2):}
            \NormalTok{output += }\DataTypeTok{str}\NormalTok{(i) + }\StringTok{'*'} \NormalTok{+ }\DataTypeTok{str}\NormalTok{(j) + }\StringTok{'='} \NormalTok{+ }\DataTypeTok{str}\NormalTok{(i*j) + }\StringTok{'<br />'}
    \KeywordTok{return} \NormalTok{output}
\end{Highlighting}
\end{Shaded}

解題心得:

利用網站輸入值並算出,並需要有兩個輸入表單。

\begin{verbatim}
第三題:
\end{verbatim}

請在各組的雲端 dokuwiki 中,新增帳號與密碼都是由 abc001
\textasciitilde{} abc399 字串所組成的 399
名用戶登入對應資料,並將製作過程與驗證流程拍成 flv 後上傳到個人的 Vimeo
資料區,並將連結放在個人第九週頁面。

\begin{Shaded}
\begin{Highlighting}[]
\CharTok{import} \NormalTok{hashlib}
\CommentTok{#convert user_password into sha1 encoded string}
\KeywordTok{def} \NormalTok{gen_password(user_password):}
    \KeywordTok{return} \NormalTok{hashlib.sha1(user_password.encode(}\StringTok{"utf-8"}\NormalTok{)).hexdigest()}
\NormalTok{text=}\StringTok{'abc}\OtherTok{%03d}\StringTok{'}
\KeywordTok{for} \NormalTok{i in }\DataTypeTok{range}\NormalTok{(}\DecValTok{1}\NormalTok{,}\DecValTok{399+1}\NormalTok{):}
    \DataTypeTok{print}\NormalTok{(text%(i)+}\StringTok{':'}\NormalTok{+gen_password(text%(i))+}\StringTok{':'}\NormalTok{+text%(i)+}\StringTok{'@gmail.com:'}\NormalTok{+text%(i)+}\StringTok{'@gmail.com:'}\NormalTok{+}\StringTok{'user'}\NormalTok{)}
\end{Highlighting}
\end{Shaded}

解題心得:

可以利用程式創造多人帳密,一起管理網站,減少一個一個創建帳密,又會有被盜用的風險。

import math count = 0 text=`\%04d' for i in range(100,200+10):
print(text\%round(math.log(i/100,10)*10000,0),end=`-') count += 1
if(count == 10): print() count = 0

class Example(object): \emph{cp}config = \{ \# if there is no utf-8
encoding, no Chinese input available `tools.encode.encoding': `utf-8',
`tools.sessions.on' : True, `tools.sessions.storage\_type' : `file',
`tools.sessions.locking' : `explicit', `tools.sessions.storage\_path' :
data\_dir+`/tmp', \# session timeout is 60 minutes
`tools.sessions.timeout' : 60 \}

@cherrypy.expose def index(self): output = ''

\begin{verbatim}
form = '''
<form action='action'>
num1:<INPUT type='text' name='num1'>
num2:<INPUT type='text'  name='num2'>
<input type=submit>
<input type=reset>
</form>
'''
output += form
return output
\end{verbatim}

@cherrypy.expose def action(self, num1=9, num2=9): num1 = int(num1) num2
= int(num2) output = `' for i in range(num1): for j in range(num2):
output += str(i) +'\emph{`+ str(j) +'=' + str(i}j) + `' return output

import hashlib \#convert user\_password into sha1 encoded string def
gen\_password(user\_password): return
hashlib.sha1(user\_password.encode(``utf-8'')).hexdigest()
text=`abc\%03d' for i in range(1,399+1):
print(text\%(i)+`:'+gen\_password(text\%(i))+`:'+text\%(i)+`@gmail.com:'+text\%(i)+`@gmail.com:'+`user')

\section{網際鼓式煞車設計(2ag4)}\label{ux7db2ux969bux9f13ux5f0fux715eux8ecaux8a2dux8a082ag4}

github連結 40123107: https://github.com/40123107 40123120:
https://github.com/40123120 cdag4:
https://github.com/2014cdag4/2014cdag4

cmsimply連結 40123107: https://github.com/40123107 40123120:
https://github.com/40123120 40123150: http://cdg4-40123150.rhcloud.com/
===

第九週 1.(第一題 30\%) 請寫一個執行時可以列出以十為底對數表的網際 Python
程式, 然後 Push 到個人 bitbucket 空間, 而且同步指到 OpenShift
個人帳號上執行.
http://cdg4-40123150.rhcloud.com/get\_page?heading=\%E7\%AC\%AC\%E4\%B8\%80\%E9\%A1\%8C
2.(第二題 40\%) 請在個人的 OpenShift
平台上建立一個能夠列印出與九九乘法表結果完全相同的網際程式,
接著在乘法表上端加上兩個輸入表單, 讓使用者輸入兩個整數, 按下送出鍵後,
程式會列出以此兩個整數為基底的乘法表, 例如: 若兩個欄位都輸入: 9,
則列出九九乘法表, 若輸入 9, 20, 則列出 9×20 的乘法表.
http://cdg4-40123150.rhcloud.com/example/ http://vimeo.com/92577008
3.(第三題 30\%) 請在各組的雲端 dokuwiki 中, 新增帳號與密碼都是由 abc001
\textasciitilde{} abc399 字串所組成的 399 名用戶登入對應資料,
並將製作過程與驗證流程拍成 flv 後上傳到個人的 Vimeo 資料區,
並將連結放在個人第九週頁面.
https://php-40123150.rhcloud.com/doku.php?id=start
http://vimeo.com/92577072 === w12 請連結
https://github.com/2014cdag4/2014cdag4 ===

這裡是結果與討論

\section{網際四連桿機構設計(2ag5)}\label{ux7db2ux969bux56dbux9023ux687fux6a5fux69cbux8a2dux8a082ag5}

有關連桿設計

\section{程式設計架構}\label{ux7a0bux5f0fux8a2dux8a08ux67b6ux69cb-1}

連桿計算

\section{結果與討論}\label{ux7d50ux679cux8207ux8a0eux8ad6-1}

這裡是結果與討論

by 2014cdag5

\section{網際鼓式煞車設計(2ag6)}\label{ux7db2ux969bux9f13ux5f0fux715eux8ecaux8a2dux8a082ag6}

有關鼓式煞車

\section{程式設計架構}\label{ux7a0bux5f0fux8a2dux8a08ux67b6ux69cb-2}

鼓式煞車

\section{結果與討論}\label{ux7d50ux679cux8207ux8a0eux8ad6-2}

這裡是結果與討論

\section{網際鼓式煞車設計(2ag7)}\label{ux7db2ux969bux9f13ux5f0fux715eux8ecaux8a2dux8a082ag7}

有關鼓式煞車

\section{程式設計架構}\label{ux7a0bux5f0fux8a2dux8a08ux67b6ux69cb-3}

鼓式煞車

\section{結果與討論}\label{ux7d50ux679cux8207ux8a0eux8ad6-3}

這裡是結果與討論

\section{網際四連桿機構設計(2ag8)}\label{ux7db2ux969bux56dbux9023ux687fux6a5fux69cbux8a2dux8a082ag8}

有關連桿設計

\section{程式設計架構}\label{ux7a0bux5f0fux8a2dux8a08ux67b6ux69cb-4}

連桿計算

\section{結果與討論}\label{ux7d50ux679cux8207ux8a0eux8ad6-4}

這裡是結果與討論

by 2014cdag5

\section{網際鼓式煞車設計(2ag9)}\label{ux7db2ux969bux9f13ux5f0fux715eux8ecaux8a2dux8a082ag9}

有關鼓式煞車

\section{程式設計架構}\label{ux7a0bux5f0fux8a2dux8a08ux67b6ux69cb-5}

鼓式煞車

\section{結果與討論}\label{ux7d50ux679cux8207ux8a0eux8ad6-5}

這裡是結果與討論

這是第九組

\section{分組工作日誌}\label{ux5206ux7d44ux5de5ux4f5cux65e5ux8a8c}

\section{分組評分}\label{ux5206ux7d44ux8a55ux5206}

\begin{Shaded}
\begin{Highlighting}[]
\CommentTok{#coding: utf-8}
\CommentTok{'''}
\CommentTok{http://www.beyondmech.com/pro-e/cad-topic-27.html}
\CommentTok{"本程式的目的在輔助設計者選擇齒輪的尺寸大小,";}
\CommentTok{"由於相囓合的兩齒輪其徑節 (Diametral Pitch) 相同";}
\CommentTok{",齒的大小也相同。因徑節為每單位直徑的齒數,因此徑節愈大,則其齒的尺寸愈小";}
\CommentTok{";反之,徑節愈小,則齒的尺寸則愈大。";}
\CommentTok{"一般在設計齒輪對時,為避免使用過大的齒及過寬的齒面厚度,因此必須要就齒輪大小與強度與負載加以設計。";}
\CommentTok{"一般而言是希望齒輪面的寬度 (Face Width) 能大於3倍周節 (Circular Pitch),以避免選用太大的齒尺寸。";}
\CommentTok{"並且希望齒輪面的寬度 (Face Width) 能小於5倍周節,以便齒面傳遞負載時能有較為均勻的分佈,因此";}
\CommentTok{"設 d 為齒輪的節圓直徑(Pitch Diameter),單位為英吋";}
\CommentTok{"N 為齒數";}
\CommentTok{"P 為徑節, 即單位英吋的齒數";}
\CommentTok{"因此 d=N/P";}
\CommentTok{"設 V 為節線速度(Pitch Line Velocity),單位為英呎/分鐘";}
\CommentTok{"因此 V=(PI) * d * n/12";}
\CommentTok{"其中 n 為齒輪轉速,單位為 rpm";}
\CommentTok{"設傳輸負載大小為 W,單位為 pounds";}
\CommentTok{"因此 W=33000H/V";}
\CommentTok{"其中 H 為傳輸功率,單位為 hourse power";}
\CommentTok{"若設 K 為速度因子(Velocity Factor)";}
\CommentTok{"因此 K=1200/(1200+V)";}
\CommentTok{"最後可求出齒輪的齒面寬度(Face Width) F ,單位為英吋";}
\CommentTok{"即 F=WP/KYS";}
\CommentTok{"其中 S 為齒面的材料彎曲應力強度";}
\CommentTok{"設計要求:控制所選齒的尺寸大小,在滿足強度與傳輸負載的要求下,讓齒面厚度介於3倍周節與5倍周節之間。";}
\CommentTok{"設計者可以選擇的參數:";}
\CommentTok{"安全係數(建議值為3以上)";}
\CommentTok{"齒輪減速比";}
\CommentTok{"馬達傳輸功率,單位為 horse power";}
\CommentTok{"馬達轉速,單位為 rpm";}
\CommentTok{"齒制(Gear System)";}
\CommentTok{"齒輪材料與強度";}
\CommentTok{'''}
\CommentTok{# 這個程式要計算正齒輪的齒面寬, 資料庫連結希望使用 pybean 與 SQLite}
\CommentTok{# 導入 pybean 模組與所要使用的 Store 及 SQLiteWriter 方法}
\CharTok{from} \NormalTok{pybean }\CharTok{import} \NormalTok{Store, SQLiteWriter}
\CharTok{import} \NormalTok{math}

\KeywordTok{def} \NormalTok{interpolation(小齒輪齒數, 齒形):}
    \KeywordTok{global} \NormalTok{SQLite連結}
    \CommentTok{# 使用內插法求值}
    \CommentTok{# 找出比目標齒數大的其中的最小的,就是最鄰近的大值}
    \NormalTok{lewis_factor = SQLite連結.find_one(}\StringTok{"lewis"}\NormalTok{,}\StringTok{"gearno > ?"}\NormalTok{,[小齒輪齒數])}
    \KeywordTok{if}\NormalTok{(齒形 == }\DecValTok{1}\NormalTok{):}
        \NormalTok{larger_formfactor = lewis_factor.type1}
    \KeywordTok{elif}\NormalTok{(齒形 == }\DecValTok{2}\NormalTok{):}
        \NormalTok{larger_formfactor = lewis_factor.type2}
    \KeywordTok{elif}\NormalTok{(齒形 == }\DecValTok{3}\NormalTok{):}
        \NormalTok{larger_formfactor = lewis_factor.type3}
    \KeywordTok{else}\NormalTok{:}
        \NormalTok{larger_formfactor = lewis_factor.type4}
    \NormalTok{larger_toothnumber = lewis_factor.gearno}

    \CommentTok{# 找出比目標齒數小的其中的最大的,就是最鄰近的小值}
    \NormalTok{lewis_factor = SQLite連結.find_one(}\StringTok{"lewis"}\NormalTok{,}\StringTok{"gearno < ? order by gearno DESC"}\NormalTok{,[小齒輪齒數])}
    \KeywordTok{if}\NormalTok{(齒形 == }\DecValTok{1}\NormalTok{):}
        \NormalTok{smaller_formfactor = lewis_factor.type1}
    \KeywordTok{elif}\NormalTok{(齒形 == }\DecValTok{2}\NormalTok{):}
        \NormalTok{smaller_formfactor = lewis_factor.type2}
    \KeywordTok{elif}\NormalTok{(齒形 == }\DecValTok{3}\NormalTok{):}
        \NormalTok{smaller_formfactor = lewis_factor.type3}
    \KeywordTok{else}\NormalTok{:}
        \NormalTok{smaller_formfactor = lewis_factor.type4}
    \NormalTok{smaller_toothnumber = lewis_factor.gearno}
    \NormalTok{calculated_factor = larger_formfactor + (小齒輪齒數 - larger_toothnumber) * (larger_formfactor - smaller_formfactor) / (larger_toothnumber - smaller_toothnumber)}
    \CommentTok{# 只傳回小數點後五位數}
    \KeywordTok{return} \DataTypeTok{round}\NormalTok{(calculated_factor, }\DecValTok{5}\NormalTok{)}

\CommentTok{# 取得設計參數}
\NormalTok{馬力 = }\DecValTok{100}
\NormalTok{轉速 = }\DecValTok{1120}
\NormalTok{減速比 = }\DecValTok{4}
\NormalTok{齒形 = }\DecValTok{4}
\NormalTok{安全係數 = }\DecValTok{3}
\CommentTok{#unsno_treatment}
\NormalTok{材料 = }\StringTok{"G10350_CD"}
\NormalTok{小齒輪齒數 = }\DecValTok{18}

\CommentTok{# 改寫為齒面寬的設計函式}
\KeywordTok{def} \NormalTok{gear_width(馬力, 轉速, 減速比, 齒形, 安全係數, 材料, 小齒輪齒數):}
    \CommentTok{# 根據所選用的齒形決定壓力角}
    \KeywordTok{if}\NormalTok{(齒形 == }\DecValTok{1} \NormalTok{or 齒形 == }\DecValTok{2}\NormalTok{):}
        \NormalTok{壓力角 = }\DecValTok{20}
    \KeywordTok{else}\NormalTok{:}
        \NormalTok{壓力角 = }\DecValTok{25}

    \CommentTok{# 根據壓力角決定最小齒數}
    \KeywordTok{if}\NormalTok{(壓力角== }\DecValTok{20}\NormalTok{):}
        \NormalTok{最小齒數 = }\DecValTok{18}
    \KeywordTok{else}\NormalTok{:}
        \NormalTok{最小齒數 = }\DecValTok{12}

    \CommentTok{# 直接設最小齒數}
    \KeywordTok{if} \NormalTok{小齒輪齒數 <= 最小齒數:}
        \NormalTok{小齒輪齒數 = 最小齒數}
    \CommentTok{# 大於400的齒數則視為齒條(Rack)}
    \KeywordTok{if} \NormalTok{小齒輪齒數 >= }\DecValTok{400}\NormalTok{:}
        \NormalTok{小齒輪齒數 = }\DecValTok{400}

    \CommentTok{# 根據所選用的材料查詢強度值}
    \CommentTok{# 由 material之序號查 steel 表以得材料之降伏強度S單位為 kpsi 因此查得的值要成乘上1000}
    \CommentTok{# 利用 Store  建立資料庫檔案對應物件, 並且設定 frozen=True 表示不要開放動態資料表的建立}
    \NormalTok{SQLite連結 = Store(SQLiteWriter(}\StringTok{"lewis.db"}\NormalTok{, frozen=}\OtherTok{True}\NormalTok{))}
    \CommentTok{# 指定 steel 資料表}
    \NormalTok{steel = SQLite連結.new(}\StringTok{"steel"}\NormalTok{)}
    \CommentTok{# 資料查詢}
    \CommentTok{# 將 unsno 與 treatment 從材料字串中隔開}
    \NormalTok{unsno, treatment = 材料.split(}\StringTok{"_"}\NormalTok{)}
    \CommentTok{#print(unsno, treatment)}

    \NormalTok{material = SQLite連結.find_one(}\StringTok{"steel"}\NormalTok{,}\StringTok{"unsno=? and treatment=?"}\NormalTok{,[unsno, treatment])}
    \CommentTok{# 列出 steel 資料表中的資料筆數}
    \CommentTok{#print(SQLite連結.count("steel"))}
    \DataTypeTok{print} \NormalTok{(material.yield_str)}
    \NormalTok{strengthstress = material.yield_str*}\DecValTok{1000}
    \CommentTok{# 由小齒輪的齒數與齒形類別,查詢lewis form factor}
    \CommentTok{# 先查驗是否有直接對應值}
    \NormalTok{on_table = SQLite連結.count(}\StringTok{"lewis"}\NormalTok{,}\StringTok{"gearno=?"}\NormalTok{,[小齒輪齒數])}
    \KeywordTok{if} \NormalTok{on_table == }\DecValTok{1}\NormalTok{:}
        \CommentTok{# 直接進入設計運算}
        \DataTypeTok{print}\NormalTok{(}\StringTok{"直接運算"}\NormalTok{)}
        \DataTypeTok{print}\NormalTok{(on_table)}
        \NormalTok{lewis_factor = SQLite連結.find_one(}\StringTok{"lewis"}\NormalTok{,}\StringTok{"gearno=?"}\NormalTok{,[小齒輪齒數])}
        \CommentTok{#print(lewis_factor.type1)}
        \CommentTok{# 根據齒形查出 formfactor 值}
        \KeywordTok{if}\NormalTok{(齒形 == }\DecValTok{1}\NormalTok{):}
            \NormalTok{formfactor = lewis_factor.type1}
        \KeywordTok{elif}\NormalTok{(齒形 == }\DecValTok{2}\NormalTok{):}
            \NormalTok{formfactor = lewis_factor.type2}
        \KeywordTok{elif}\NormalTok{(齒形 == }\DecValTok{3}\NormalTok{):}
            \NormalTok{formfactor = lewis_factor.type3}
        \KeywordTok{else}\NormalTok{:}
            \NormalTok{formfactor = lewis_factor.type4}
    \KeywordTok{else}\NormalTok{:}
        \CommentTok{# 沒有直接對應值, 必須進行查表內插運算後, 再執行設計運算}
        \DataTypeTok{print}\NormalTok{(}\StringTok{"必須內插"}\NormalTok{)}
        \CommentTok{#print(interpolation(小齒輪齒數, 齒形))}
        \NormalTok{formfactor = interpolation(小齒輪齒數, 齒形)}

    \CommentTok{# 開始進行設計運算}

    \NormalTok{ngear = 小齒輪齒數 * 減速比}

    \CommentTok{# 重要的最佳化設計---儘量用整數的diametralpitch}
    \CommentTok{# 先嘗試用整數算若 diametralpitch 找到100 仍無所獲則改用 0.25 作為增量再不行則宣告 fail}
    \NormalTok{counter = }\DecValTok{0}
    \NormalTok{i = }\FloatTok{0.1}
    \NormalTok{facewidth = }\DecValTok{0}
    \NormalTok{circularpitch = }\DecValTok{0}
    \KeywordTok{while} \NormalTok{(facewidth <= }\DecValTok{3} \NormalTok{* circularpitch or facewidth >= }\DecValTok{5} \NormalTok{* circularpitch):}
        \NormalTok{diametralpitch = i}
        \CommentTok{#circularpitch = 3.14159/diametralpitch}
        \NormalTok{circularpitch = math.pi/diametralpitch}
        \NormalTok{pitchdiameter = 小齒輪齒數/diametralpitch}
        \CommentTok{#pitchlinevelocity = 3.14159*pitchdiameter*轉速/12}
        \NormalTok{pitchlinevelocity = math.pi * pitchdiameter * 轉速/}\DecValTok{12}
        \NormalTok{transmittedload = }\DecValTok{33000} \NormalTok{* 馬力/pitchlinevelocity}
        \NormalTok{velocityfactor = }\DecValTok{1200}\NormalTok{/(}\DecValTok{1200} \NormalTok{+ pitchlinevelocity)}
        \CommentTok{# formfactor is Lewis form factor}
        \CommentTok{# formfactor need to get from table 13-3 and determined ty teeth number and type of tooth}
        \CommentTok{# formfactor = 0.293}
        \CommentTok{# 90 is the value get from table corresponding to material type}
        \NormalTok{facewidth = transmittedload * diametralpitch * 安全係數/velocityfactor/formfactor/strengthstress}
        \KeywordTok{if}\NormalTok{(counter>}\DecValTok{5000}\NormalTok{):}
            \DataTypeTok{print}\NormalTok{(}\StringTok{"超過5000次的設計運算,仍無法找到答案!"}\NormalTok{)}
            \DataTypeTok{print}\NormalTok{(}\StringTok{"可能所選用的傳遞功率過大,或無足夠強度的材料可以使用!"}\NormalTok{)}
            \CommentTok{# 離開while迴圈}
            \KeywordTok{break}
        \NormalTok{i += }\FloatTok{0.1}
        \NormalTok{counter += }\DecValTok{1}
    \NormalTok{facewidth = }\DataTypeTok{round}\NormalTok{(facewidth, }\DecValTok{4}\NormalTok{)}
    \KeywordTok{if}\NormalTok{(counter<}\DecValTok{5000}\NormalTok{):}
        \DataTypeTok{print}\NormalTok{(}\StringTok{"進行"}\NormalTok{+}\DataTypeTok{str}\NormalTok{(counter)+}\StringTok{"次重複運算後,得到合用的facewidth值為:"}\NormalTok{+}\DataTypeTok{str}\NormalTok{(facewidth))}

\NormalTok{gear_width(馬力, 轉速, 減速比, 齒形, 安全係數, 材料, 小齒輪齒數)}
\end{Highlighting}
\end{Shaded}

\begin{Shaded}
\begin{Highlighting}[]
\CommentTok{#coding: utf-8}
\CommentTok{'''}
\CommentTok{"本程式的目的在輔助設計者選擇齒輪的尺寸大小,";}
\CommentTok{"由於相囓合的兩齒輪其徑節 (Diametral Pitch) 相同";}
\CommentTok{",齒的大小也相同。因徑節為每單位直徑的齒數,因此徑節愈大,則其齒的尺寸愈小";}
\CommentTok{";反之,徑節愈小,則齒的尺寸則愈大。";}
\CommentTok{"一般在設計齒輪對時,為避免使用過大的齒及過寬的齒面厚度,因此必須要就齒輪大小與強度與負載加以設計。";}
\CommentTok{"一般而言是希望齒輪面的寬度 (Face Width) 能大於3倍周節 (Circular Pitch),以避免選用太大的齒尺寸。";}
\CommentTok{"並且希望齒輪面的寬度 (Face Width) 能小於5倍周節,以便齒面傳遞負載時能有較為均勻的分佈,因此";}
\CommentTok{"設 d 為齒輪的節圓直徑(Pitch Diameter),單位為英吋";}
\CommentTok{"N 為齒數";}
\CommentTok{"P 為徑節, 即單位英吋的齒數";}
\CommentTok{"因此 d=N/P";}
\CommentTok{"設 V 為節線速度(Pitch Line Velocity),單位為英呎/分鐘";}
\CommentTok{"因此 V=(PI) * d * n/12";}
\CommentTok{"其中 n 為齒輪轉速,單位為 rpm";}
\CommentTok{"設傳輸負載大小為 W,單位為 pounds";}
\CommentTok{"因此 W=33000H/V";}
\CommentTok{"其中 H 為傳輸功率,單位為 hourse power";}
\CommentTok{"若設 K 為速度因子(Velocity Factor)";}
\CommentTok{"因此 K=1200/(1200+V)";}
\CommentTok{"最後可求出齒輪的齒面寬度(Face Width) F ,單位為英吋";}
\CommentTok{"即 F=WP/KYS";}
\CommentTok{"其中 S 為齒面的材料彎曲應力強度";}
\CommentTok{"設計要求:控制所選齒的尺寸大小,在滿足強度與傳輸負載的要求下,讓齒面厚度介於3倍周節與5倍周節之間。";}
\CommentTok{"設計者可以選擇的參數:";}
\CommentTok{"安全係數(建議值為3以上)";}
\CommentTok{"齒輪減速比";}
\CommentTok{"馬達傳輸功率,單位為 horse power";}
\CommentTok{"馬達轉速,單位為 rpm";}
\CommentTok{"齒制(Gear System)";}
\CommentTok{"齒輪材料與強度";}
\CommentTok{'''}
\CommentTok{# 這個程式要計算正齒輪的齒面寬, 資料庫連結希望使用 pybean 與 SQLite}
\CommentTok{# 導入 pybean 模組與所要使用的 Store 及 SQLiteWriter 方法}
\CharTok{from} \NormalTok{pybean }\CharTok{import} \NormalTok{Store, SQLiteWriter}
\CharTok{import} \NormalTok{math}

\NormalTok{SQLite連結 = Store(SQLiteWriter(}\StringTok{"lewis.db"}\NormalTok{, frozen=}\OtherTok{True}\NormalTok{))}
\NormalTok{overall_counter = }\DecValTok{0}
\NormalTok{查表次數 = }\DecValTok{0}
\NormalTok{材料計數 = }\DecValTok{0}

\CommentTok{# 執行 formfactor 內插運算的函式}
\KeywordTok{def} \NormalTok{interpolation(小齒輪齒數, 齒形):}
    \KeywordTok{global} \NormalTok{SQLite連結}
    \CommentTok{# 使用內插法求值}
    \CommentTok{# 找出比目標齒數大的其中的最小的,就是最鄰近的大值}
    \NormalTok{lewis_factor = SQLite連結.find_one(}\StringTok{"lewis"}\NormalTok{,}\StringTok{"gearno > ?"}\NormalTok{,[小齒輪齒數])}
    \KeywordTok{if}\NormalTok{(齒形 == }\DecValTok{1}\NormalTok{):}
        \NormalTok{larger_formfactor = lewis_factor.type1}
    \KeywordTok{elif}\NormalTok{(齒形 == }\DecValTok{2}\NormalTok{):}
        \NormalTok{larger_formfactor = lewis_factor.type2}
    \KeywordTok{elif}\NormalTok{(齒形 == }\DecValTok{3}\NormalTok{):}
        \NormalTok{larger_formfactor = lewis_factor.type3}
    \KeywordTok{else}\NormalTok{:}
        \NormalTok{larger_formfactor = lewis_factor.type4}
    \NormalTok{larger_toothnumber = lewis_factor.gearno}

    \CommentTok{# 找出比目標齒數小的其中的最大的,就是最鄰近的小值}
    \NormalTok{lewis_factor = SQLite連結.find_one(}\StringTok{"lewis"}\NormalTok{,}\StringTok{"gearno < ? order by gearno DESC"}\NormalTok{,[小齒輪齒數])}
    \KeywordTok{if}\NormalTok{(齒形 == }\DecValTok{1}\NormalTok{):}
        \NormalTok{smaller_formfactor = lewis_factor.type1}
    \KeywordTok{elif}\NormalTok{(齒形 == }\DecValTok{2}\NormalTok{):}
        \NormalTok{smaller_formfactor = lewis_factor.type2}
    \KeywordTok{elif}\NormalTok{(齒形 == }\DecValTok{3}\NormalTok{):}
        \NormalTok{smaller_formfactor = lewis_factor.type3}
    \KeywordTok{else}\NormalTok{:}
        \NormalTok{smaller_formfactor = lewis_factor.type4}
    \NormalTok{smaller_toothnumber = lewis_factor.gearno}
    \NormalTok{calculated_factor = larger_formfactor + (小齒輪齒數 - larger_toothnumber) * (larger_formfactor - smaller_formfactor) / (larger_toothnumber - smaller_toothnumber)}
    \CommentTok{# 只傳回小數點後五位數}
    \DataTypeTok{print}\NormalTok{(}\StringTok{"小齒輪齒數:"}\NormalTok{+}\DataTypeTok{str}\NormalTok{(小齒輪齒數)+}\StringTok{", 齒形"}\NormalTok{+}\DataTypeTok{str}\NormalTok{(齒形)+}\StringTok{", formfactor:"}\NormalTok{+}\DataTypeTok{str}\NormalTok{(}\DataTypeTok{round}\NormalTok{(calculated_factor, }\DecValTok{5}\NormalTok{)))}
    \KeywordTok{return} \DataTypeTok{round}\NormalTok{(calculated_factor, }\DecValTok{5}\NormalTok{)}

\CommentTok{# 取得設計參數}
\CommentTok{'''}
\CommentTok{馬力 = 100}
\CommentTok{轉速 = 1120}
\CommentTok{減速比 = 4}
\CommentTok{齒形 = 4}
\CommentTok{安全係數 = 3}
\CommentTok{#unsno_treatment}
\CommentTok{材料 = "G10350_CD"}
\CommentTok{小齒輪齒數 = 18}
\CommentTok{'''}

\CommentTok{# 改寫為齒面寬的設計函式}
\KeywordTok{def} \NormalTok{gear_width(馬力, 轉速, 減速比, 齒形, 安全係數, 材料, 小齒輪齒數):}
    \KeywordTok{global} \NormalTok{SQLite連結}
    \KeywordTok{global} \NormalTok{overall_counter}
    \KeywordTok{global} \NormalTok{查表次數}
    \KeywordTok{global} \NormalTok{材料計數}
    \CommentTok{# 根據所選用的齒形決定壓力角}
    \KeywordTok{if}\NormalTok{(齒形 == }\DecValTok{1} \NormalTok{or 齒形 == }\DecValTok{2}\NormalTok{):}
        \NormalTok{壓力角 = }\DecValTok{20}
    \KeywordTok{else}\NormalTok{:}
        \NormalTok{壓力角 = }\DecValTok{25}

    \CommentTok{# 根據壓力角決定最小齒數}
    \KeywordTok{if}\NormalTok{(壓力角== }\DecValTok{20}\NormalTok{):}
        \NormalTok{最小齒數 = }\DecValTok{18}
    \KeywordTok{else}\NormalTok{:}
        \NormalTok{最小齒數 = }\DecValTok{12}

    \CommentTok{# 直接設最小齒數}
    \KeywordTok{if} \NormalTok{小齒輪齒數 <= 最小齒數:}
        \NormalTok{小齒輪齒數 = 最小齒數}
    \CommentTok{# 大於400的齒數則視為齒條(Rack)}
    \KeywordTok{if} \NormalTok{小齒輪齒數 >= }\DecValTok{400}\NormalTok{:}
        \NormalTok{小齒輪齒數 = }\DecValTok{400}

    \CommentTok{# 根據所選用的材料查詢強度值}
    \CommentTok{# 由 material之序號查 steel 表以得材料之降伏強度S單位為 kpsi 因此查得的值要成乘上1000}
    \CommentTok{# 利用 Store  建立資料庫檔案對應物件, 並且設定 frozen=True 表示不要開放動態資料表的建立}
    \CommentTok{#SQLite連結 = Store(SQLiteWriter("lewis.db", frozen=True))}
    \CommentTok{# 指定 steel 資料表}
    \NormalTok{steel = SQLite連結.new(}\StringTok{"steel"}\NormalTok{)}
    \CommentTok{# 資料查詢}
    \CommentTok{# 將 unsno 與 treatment 從材料字串中隔開}
    \NormalTok{unsno, treatment = 材料.split(}\StringTok{"_"}\NormalTok{, }\DecValTok{1}\NormalTok{)}
    \CommentTok{#print(unsno, treatment)}
    \NormalTok{treatment = treatment.replace(}\StringTok{"_"}\NormalTok{, }\StringTok{" "}\NormalTok{)}
    \CommentTok{#print(treatment)}
    \NormalTok{material = SQLite連結.find_one(}\StringTok{"steel"}\NormalTok{,}\StringTok{"unsno=? and treatment=?"}\NormalTok{,[unsno, treatment])}
    \CommentTok{# 列出 steel 資料表中的資料筆數}
    \CommentTok{#print(SQLite連結.count("steel"))}
    \CommentTok{#print (material.yield_str)}
    \KeywordTok{if} \NormalTok{material.yield_str > }\DecValTok{100}\NormalTok{:}
        \NormalTok{材料計數 += }\DecValTok{1}
    \NormalTok{strengthstress = material.yield_str*}\DecValTok{1000}
    \CommentTok{# 由小齒輪的齒數與齒形類別,查詢lewis form factor}
    \CommentTok{# 先查驗是否有直接對應值}
    \NormalTok{on_table = SQLite連結.count(}\StringTok{"lewis"}\NormalTok{,}\StringTok{"gearno=?"}\NormalTok{,[小齒輪齒數])}
    \KeywordTok{if} \NormalTok{on_table == }\DecValTok{1}\NormalTok{:}
        \CommentTok{# 直接進入設計運算}
        \CommentTok{#print("直接運算")}
        \CommentTok{#print(on_table)}
        \NormalTok{lewis_factor = SQLite連結.find_one(}\StringTok{"lewis"}\NormalTok{,}\StringTok{"gearno=?"}\NormalTok{,[小齒輪齒數])}
        \CommentTok{#print(lewis_factor.type1)}
        \CommentTok{# 根據齒形查出 formfactor 值}
        \KeywordTok{if}\NormalTok{(齒形 == }\DecValTok{1}\NormalTok{):}
            \NormalTok{formfactor = lewis_factor.type1}
        \KeywordTok{elif}\NormalTok{(齒形 == }\DecValTok{2}\NormalTok{):}
            \NormalTok{formfactor = lewis_factor.type2}
        \KeywordTok{elif}\NormalTok{(齒形 == }\DecValTok{3}\NormalTok{):}
            \NormalTok{formfactor = lewis_factor.type3}
        \KeywordTok{else}\NormalTok{:}
            \NormalTok{formfactor = lewis_factor.type4}
    \KeywordTok{else}\NormalTok{:}
        \CommentTok{# 沒有直接對應值, 必須進行查表內插運算後, 再執行設計運算}
        \CommentTok{#print("必須內插")}
        \CommentTok{#print(interpolation(小齒輪齒數, 齒形))}
        \NormalTok{查表次數 += }\DecValTok{1}
        \DataTypeTok{print}\NormalTok{(}\StringTok{"第"}\NormalTok{+}\DataTypeTok{str}\NormalTok{(查表次數)+}\StringTok{"次查表"}\NormalTok{)}
        \NormalTok{formfactor = interpolation(小齒輪齒數, 齒形)}

    \CommentTok{# 開始進行設計運算}

    \NormalTok{ngear = 小齒輪齒數 * 減速比}

    \CommentTok{# 重要的最佳化設計---儘量用整數的diametralpitch}
    \CommentTok{# 先嘗試用整數算若 diametralpitch 找到100 仍無所獲則改用 0.25 作為增量再不行則宣告 fail}
    \NormalTok{counter = }\DecValTok{0}
    \NormalTok{i = }\FloatTok{0.1}
    \NormalTok{facewidth = }\DecValTok{0}
    \NormalTok{circularpitch = }\DecValTok{0}
    \KeywordTok{while} \NormalTok{(facewidth <= }\DecValTok{3} \NormalTok{* circularpitch or facewidth >= }\DecValTok{5} \NormalTok{* circularpitch):}
        \NormalTok{diametralpitch = i}
        \CommentTok{#circularpitch = 3.14159/diametralpitch}
        \NormalTok{circularpitch = math.pi/diametralpitch}
        \NormalTok{pitchdiameter = 小齒輪齒數/diametralpitch}
        \CommentTok{#pitchlinevelocity = 3.14159*pitchdiameter*轉速/12}
        \NormalTok{pitchlinevelocity = math.pi * pitchdiameter * 轉速/}\DecValTok{12}
        \NormalTok{transmittedload = }\DecValTok{33000} \NormalTok{* 馬力/pitchlinevelocity}
        \NormalTok{velocityfactor = }\DecValTok{1200}\NormalTok{/(}\DecValTok{1200} \NormalTok{+ pitchlinevelocity)}
        \CommentTok{# formfactor is Lewis form factor}
        \CommentTok{# formfactor need to get from table 13-3 and determined ty teeth number and type of tooth}
        \CommentTok{# formfactor = 0.293}
        \CommentTok{# 90 is the value get from table corresponding to material type}
        \NormalTok{facewidth = transmittedload * diametralpitch * 安全係數/velocityfactor/formfactor/strengthstress}
        \KeywordTok{if}\NormalTok{(counter>}\DecValTok{5000}\NormalTok{):}
            \DataTypeTok{print}\NormalTok{(}\StringTok{"超過5000次的設計運算,仍無法找到答案!"}\NormalTok{)}
            \DataTypeTok{print}\NormalTok{(}\StringTok{"可能所選用的傳遞功率過大,或無足夠強度的材料可以使用!"}\NormalTok{)}
            \CommentTok{# 離開while迴圈}
            \KeywordTok{break}
        \NormalTok{i += }\FloatTok{0.1}
        \NormalTok{counter += }\DecValTok{1}
    \NormalTok{facewidth = }\DataTypeTok{round}\NormalTok{(facewidth, }\DecValTok{4}\NormalTok{)}
    \KeywordTok{if}\NormalTok{(counter<}\DecValTok{5000}\NormalTok{):}
        \DataTypeTok{print}\NormalTok{(}\StringTok{"進行"}\NormalTok{+}\DataTypeTok{str}\NormalTok{(counter)+}\StringTok{"次重複運算後,得到合用的facewidth值為:"}\NormalTok{+}\DataTypeTok{str}\NormalTok{(facewidth))}
    \NormalTok{overall_counter += counter}

\CommentTok{# 執行正齒輪齒面寬的設計運算}
\CommentTok{#gear_width(馬力, 轉速, 減速比, 齒形, 安全係數, 材料, 小齒輪齒數)}

\CommentTok{# 執行輸入檔案的解讀}
\NormalTok{輸入檔案 = }\DataTypeTok{open}\NormalTok{(}\StringTok{'design_input.txt'}\NormalTok{, encoding=}\StringTok{"UTF-8"}\NormalTok{) }\CommentTok{# 開檔案的內建模式為 read}
\CommentTok{# 先將數字檔案中各行資料打包成為 list}
\CommentTok{# 先讀第一行的標題資料行}
\NormalTok{各行資料 = 輸入檔案.readline()}
\NormalTok{design_count = }\DecValTok{0}
\NormalTok{輸入= []}
\KeywordTok{while} \OtherTok{True}\NormalTok{:}
    \CommentTok{# readline() 讀取單行}
    \CommentTok{# readlines() 讀取多行, 並放入串列資料格式中}
    \NormalTok{各行資料 = 輸入檔案.readline()}
    \NormalTok{design_count += }\DecValTok{1}
    \DataTypeTok{print}\NormalTok{(}\StringTok{"design#:"}\NormalTok{+}\DataTypeTok{str}\NormalTok{(design_count))}
    \CommentTok{#print(各行資料,end="")}
    \CommentTok{# 以下兩行判斷式在確定檔案讀到最後一行後就會跳出 while 迴圈, 不會無限執行}
    \KeywordTok{if} \DataTypeTok{len}\NormalTok{(各行資料) == }\DecValTok{0}\NormalTok{: }\CommentTok{# 若該行的字數為 0, 表示已經到底}
        \KeywordTok{break}
    \CommentTok{# 去掉各行最後面的跳行符號}
    \NormalTok{各行資料 = 各行資料.rstrip()}
    \CommentTok{#print(各行資料,end="")}
    \CommentTok{# 依照資料的區隔符號 "\textbackslash{}t" 將各行資料拆開, 並且存為 list, 到這裡各行資料為 list}
    \NormalTok{各行資料 = 各行資料.split(}\StringTok{"}\CharTok{\textbackslash{}t}\StringTok{"}\NormalTok{)}
    \CommentTok{'''}
\CommentTok{    # 取得設計參數}
\CommentTok{    馬力 = 100}
\CommentTok{    轉速 = 1120}
\CommentTok{    減速比 = 4}
\CommentTok{    齒形 = 4}
\CommentTok{    安全係數 = 3}
\CommentTok{    #unsno_treatment}
\CommentTok{    材料 = "G10350_CD"}
\CommentTok{    小齒輪齒數 = 18}
\CommentTok{    '''}
    \NormalTok{馬力 = }\DataTypeTok{int}\NormalTok{(各行資料[}\DecValTok{0}\NormalTok{])}
    \NormalTok{轉速 = }\DataTypeTok{int}\NormalTok{(各行資料[}\DecValTok{1}\NormalTok{])}
    \NormalTok{減速比 = }\DataTypeTok{float}\NormalTok{(各行資料[}\DecValTok{2}\NormalTok{])}
    \NormalTok{齒形 = }\DataTypeTok{int}\NormalTok{(各行資料[}\DecValTok{3}\NormalTok{])}
    \NormalTok{安全係數 = }\DataTypeTok{float}\NormalTok{(各行資料[}\DecValTok{4}\NormalTok{])}
    \NormalTok{材料 = 各行資料[}\DecValTok{5}\NormalTok{]}
    \NormalTok{小齒輪齒數 = }\DataTypeTok{int}\NormalTok{(各行資料[}\DecValTok{6}\NormalTok{])}
    \NormalTok{gear_width(馬力, 轉速, 減速比, 齒形, 安全係數, 材料, 小齒輪齒數)}
\DataTypeTok{print}\NormalTok{(}\StringTok{"總運算次數:"}\NormalTok{+}\DataTypeTok{str}\NormalTok{(overall_counter))}
    \CommentTok{# 可以將各行資料印出檢查}
    \CommentTok{#print(各行資料)}
    \CommentTok{# 將各行資料數列再疊成 數字 list}
    \CommentTok{#輸入.append(各行資料)}
\CommentTok{#print(輸入)}
\CommentTok{# 取得各行輸入值後, 再呼叫 gear_width 執行齒面寬的設計運算}
\NormalTok{輸入檔案.close()}
\DataTypeTok{print}\NormalTok{(}\StringTok{"查表次數:"}\NormalTok{+}\DataTypeTok{str}\NormalTok{(查表次數))}
\DataTypeTok{print}\NormalTok{(}\StringTok{"材料計數:"}\NormalTok{+}\DataTypeTok{str}\NormalTok{(材料計數))}
\end{Highlighting}
\end{Shaded}

Welcome to the 2014cdb wiki!

\section{注意事項}\label{ux6ce8ux610fux4e8bux9805}

\begin{itemize}
\itemsep1pt\parskip0pt\parsep0pt
\item
  Python 版次必須使用 3.3 版
\item
  Leo Editor 必須使用 4.11 版
\item
  Creo 必須使用 64 位元教育版本
\item
  IE 建議使用 11 版
\item
  建立 pdf 與 html 文件則需要 pandoc 與 portableLatex 可攜應用程式(位於
  apps 目錄下)
\item
  Git 則使用可攜版本(位於 apps 目錄下)
\end{itemize}

\section{各組 push
資料確認事項}\label{ux5404ux7d44-push-ux8cc7ux6599ux78baux8a8dux4e8bux9805}

\begin{itemize}
\itemsep1pt\parskip0pt\parsep0pt
\item
  git config --list 中的 user.name 是否為各組代號 (代號錯誤將無法取分)
\item
  application 網站是否可以正確執行
\item
  .leo 檔案是否可以被 Leo Editor 正確開啟
\item
  專案是否可以正確產生協同報告 pdf 與 html 檔案
\item
  與 .leo 專案對應的外部目錄與檔案是否正確設定
\end{itemize}

\section{同步雲端網站}\label{ux540cux6b65ux96f2ux7aefux7db2ux7ad9}

http://2014cda-mdenfu.rhcloud.com/

\hyperdef{}{scda-ux5206ux7d44ux65e5ux8a8c}{\section{2014scda
分組日誌}\label{scda-ux5206ux7d44ux65e5ux8a8c}}

{[}\hyperref[scda-ux5206ux7d44ux65e5ux8a8c]{2014scda 分組日誌}{]}

2014cdag1 分組網站: http://2014cdag1-cadp13ag6.rhcloud.com/

2014cdag1 分組 Github: https://github.com/2014cdag1

\begin{longtable}[c]{@{}l@{}}
\toprule\addlinespace
\begin{minipage}[t]{0.09\columnwidth}\raggedright
2014cdag2
分組網站:https://2014cdag2-40123103.rhcloud.com/get\_page?heading=
\end{minipage}
\\\addlinespace
\begin{minipage}[t]{0.09\columnwidth}\raggedright
2014cdag2 分組 Github:https://github.com/2014cdag2
\end{minipage}
\\\addlinespace
\bottomrule
\end{longtable}

2014cdag3 分組網站:http://2014cdag3-cadp13ag10.rhcloud.com/

2014cdag3 分組 Github:https://github.com/2014cdag3/cdg3

\begin{center}\rule{3in}{0.4pt}\end{center}

2014cdag4 分組網站:http://cdg4-40123150.rhcloud.com/

2014cdag4 分組 Github:https://github.com/2014cdag4

\begin{longtable}[c]{@{}l@{}}
\toprule\addlinespace
\begin{minipage}[t]{0.09\columnwidth}\raggedright
2014cdag5 分組網站:http://2014cdag5-yimin40123157.rhcloud.com/
\end{minipage}
\\\addlinespace
\begin{minipage}[t]{0.09\columnwidth}\raggedright
2014cdag5 分組 Github:https://github.com/2014cdag5
\end{minipage}
\\\addlinespace
\bottomrule
\end{longtable}

2014cdag6 分組網站:http://2014cdag6-40123109cd2014.rhcloud.com/

2014cdag6 分組 Github:https://github.com/2014cdag6/2014cdag6

\begin{longtable}[c]{@{}l@{}}
\toprule\addlinespace
\begin{minipage}[t]{0.09\columnwidth}\raggedright
2014cdag7 分組網站:http://python-49823207.rhcloud.com/
\end{minipage}
\\\addlinespace
\begin{minipage}[t]{0.09\columnwidth}\raggedright
2014cdag7 分組 Github:https://github.com/2014cdag7/2014cdag7
\end{minipage}
\\\addlinespace
\bottomrule
\end{longtable}

2014cdag8 分組網站:http://2014cdag8-40123141.rhcloud.com/

2014cdag8 分組 Github:https://github.com/2014cdag8

\begin{center}\rule{3in}{0.4pt}\end{center}

2014cdag9 分組網站:http://2014cadg9-40123114.rhcloud.com/

2014cdag9 分組 Github:https://github.com/2014cdag9

\begin{longtable}[c]{@{}l@{}}
\toprule\addlinespace
\begin{minipage}[t]{0.09\columnwidth}\raggedright
2014cdag10 分組網站:http://cdag10-40123153.rhcloud.com/
\end{minipage}
\\\addlinespace
\begin{minipage}[t]{0.09\columnwidth}\raggedright
2014cdag10 分組 Github:https://github.com/2014cdag10/2014cdag10
\end{minipage}
\\\addlinespace
\bottomrule
\end{longtable}

2014cdag12 分組網站:http://cadpa1-40123123.rhcloud.com/

2014cdag12 分組 Github:https://github.com/2014cdag12

\begin{longtable}[c]{@{}l@{}}
\toprule\addlinespace
\begin{minipage}[t]{0.09\columnwidth}\raggedright
2014cdag13 分組網站:http://python-40123124.rhcloud.com/
\end{minipage}
\\\addlinespace
\begin{minipage}[t]{0.09\columnwidth}\raggedright
2014cdag13 分組 Github:https://github.com/2014cdag13
\end{minipage}
\\\addlinespace
\bottomrule
\end{longtable}

2014cdag14 分組網站:https://github-cadpag5.rhcloud.com/

2014cdag14 分組 Github:https://github.com/2014cdag14

\begin{longtable}[c]{@{}l@{}}
\toprule\addlinespace
\begin{minipage}[t]{0.09\columnwidth}\raggedright
2014cdag15 分組網站:http://2014cdag15-40123146.rhcloud.com/
\end{minipage}
\\\addlinespace
\begin{minipage}[t]{0.09\columnwidth}\raggedright
2014cdag15 分組 Github:https://github.com/2014cdag15
\end{minipage}
\\\addlinespace
\bottomrule
\end{longtable}

2014cdag16 分組網站:http://github-cda2014g16.rhcloud.com/

2014cdag16 分組 Github:https://github.com/2014cdag16/2014cdag16

\begin{longtable}[c]{@{}l@{}}
\toprule\addlinespace
\begin{minipage}[t]{0.09\columnwidth}\raggedright
2014cdag17 分組網站:http://2014cd17-ccwei.rhcloud.com/
\end{minipage}
\\\addlinespace
\begin{minipage}[t]{0.09\columnwidth}\raggedright
2014cdag17 分組 Github:https://github.com/2014cdag17/CDAG17
\end{minipage}
\\\addlinespace
\bottomrule
\end{longtable}

2014cdag18 分組網站:http://2014cdag18-40123126.rhcloud.com/

2014cdag18 分組 Github:https://github.com/2014cdag18

\begin{longtable}[c]{@{}l@{}}
\toprule\addlinespace
\begin{minipage}[t]{0.09\columnwidth}\raggedright
2014cdag21 分組網站:https://cmsimply-40123156.rhcloud.com/
\end{minipage}
\\\addlinespace
\begin{minipage}[t]{0.09\columnwidth}\raggedright
2014cdag21 分組 Github:https://github.com/2014cdag21/c21
\end{minipage}
\\\addlinespace
\bottomrule
\end{longtable}

\section{各週課後註記}\label{ux5404ux9031ux8ab2ux5f8cux8a3bux8a18}

自 W11 使用 pandoc\_auto 產生協同報告文件, 不再使用手動增加分組報告,
只要在 doc/分組代號/ 中加入 doc.txt, 並且使用正確的 pandoc markdown
格式就可以從 pandoc\_auto 按鈕產生對應的 html 與 pdf 檔案.

自 W12 起各組必須在分組 doc/分組代號/doc.txt
中加入第八週與第九週考試內容與心得.

\section{Creo 討論區}\label{creo-ux8a0eux8ad6ux5340}

http://www.mcadcentral.com/creo-user-forum-topics/

var drwmodels = drawing.ListModels(); newWin.document.writeln
(``\textless{}br\textgreater{}'' + ``Drawing:'' + drawing.FileName + " (
" + drwmodels.Count + " drw models )``);

var modelitemtype = pfcCreate(``pfcModelItemType'');

/\emph{--Retrieve the symbol definition from the system--}/

var symDef = drawing.RetrieveSymbolDefinition (``sym\_name'', void null,
void null, true);

for (var i = 0; i \textless{} drwmodels.Count; i++) \{

drwmodel = drwmodels.Item(i); var showndims =
drawing.ListShownDimensions(drwmodel, modelitemtype.ITEM\_DIMENSION);
newWin.document.writeln (``\textless{}br\textgreater{}'' + " ~ Model " +
(i+1) + ``:'' + drwmodel.FileName + " ( " + showndims.Count + " shown
dims )``);

for (var j = 0; j \textless{} showndims.Count; j++) \{

var pnt3d = void null; var showndim = showndims.Item(j); var symbol =
showndim.Symbol; var value = showndim.DimValue; pnt3d =
showndim.Location;

newWin.document.writeln (``\textless{}br\textgreater{}'' + " ~ ~ Dim " +
(j+1) + ``:'' + symbol + " = " + value + " ( " + pnt3d.Item(0) + " , " +
pnt3d.Item(1) + " , " + pnt3d.Item(2) + " )``);

var position = void null; var instrs = pfcCreate
(``pfcDetailSymbolInstInstructions'').Create (symDef); position =
pfcCreate (``pfcFreeAttachment'').Create (pnt3d); var allAttachments =
pfcCreate (``pfcDetailLeaders'').Create ();

position.AttachmentPoint = pnt3d; allAttachments.ItemAttachment =
position; instrs.InstAttachment = allAttachments;

/\emph{--Set the VariantTexts (for var text)--}/

var vtpp = pfcCreate (``pfcDetailVariantText'').Create (``vartext'',
j+1); var vtpp2 = pfcCreate (``pfcDetailVariantTexts'');
vtpp2.append(vtpp); instrs.TextValues = vtpp2;

/\emph{--Create and display the symbol--}/

var symInst = drawing.CreateDetailItem (instrs); symInst.Show(); \} \}

pfcWindow.Window.ExportRasterImage to save graphics

ProWindowPanZoomMatrixSet

ComponentConstraintType---Using the TYPE options, you can specify the
placement constraint types. They are as follows:

ASM\_CONSTRAINT\_MATE---Use this option to make two surfaces touch one
another, that is coincident and facing each other.

ASM\_CONSTRAINT\_MATE\_OFF---Use this option to make two planar surfaces
parallel and facing each other.

ASM\_CONSTRAINT\_ALIGN---Use this option to make two planes coplanar,
two axes coaxial and two points coincident. You can also align revolved
surfaces or edges. ○ ASM\_CONSTRAINT\_ALIGN\_OFF---Use this option to
align two planar surfaces at an offset.

ASM\_CONSTRAINT\_INSERT---Use this option to insert a ``male'' revolved
surface into a ``female'' revolved surface, making their respective axes
coaxial.

ASM\_CONSTRAINT\_ORIENT---Use this option to make two planar surfaces to
be parallel in the same direction.

ASM\_CONSTRAINT\_CSYS---Use this option to place a component in an
assembly by aligning the coordinate system of the component with the
coordinate system of the assembly.

ASM\_CONSTRAINT\_TANGENT---Use this option to control the contact of two
surfaces at their tangents.

ASM\_CONSTRAINT\_PNT\_ON\_SRF---Use this option to control the contact
of a surface with a point.

ASM\_CONSTRAINT\_EDGE\_ON\_SRF---Use this option to control the contact
of a surface with a straight edge.

ASM\_CONSTRAINT\_DEF\_PLACEMENT---Use this option to align the default
coordinate system of the component tothe default coordinate system of
the assembly.

ASM\_CONSTRAINT\_SUBSTITUTE---Use this option in simplified
representations when a component has been substituted with some other
model

ASM\_CONSTRAINT\_PNT\_ON\_LINE---Use this option to control the contact
of a line with a point.

ASM\_CONSTRAINT\_FIX---Use this option to force the component to remain
in its current packaged position.

ASM\_CONSTRAINT\_AUTO---Use this option in the user interface to allow
an automatic choice of constraint type based upon the references.

getBodyDiagonalInfo: function() \{ try \{ var cModel =
this.getCurrentModel();\\ var mitems =
cModel.ListItems(pfcCreate(``pfcModelItemType'').ITEM\_COORD\_SYS); var
cor = mitems.Item(0); var trans3 = cor.CoordSys;

\begin{verbatim}
var modtyp = pfcCreate("pfcModelItemTypes");
var item1 = pfcCreate("pfcModelItemType").ITEM_AXIS;
var item2 = pfcCreate("pfcModelItemType").ITEM_POINT;
var item3 = pfcCreate("pfcModelItemType").ITEM_COORD_SYS;

modtyp.Insert(0,item1);
modtyp.Insert(1,item2);
modtyp.Insert(2,item3);
var outline = cModel.EvalOutline(trans3,modtyp);
var point2 = outline.Item(0);
var x = point2.Item(0);
var y = point2.Item(1);
var z = point2.Item(2);
var point3 = outline.Item(1);
var p = point3.Item(0);
var q = point3.Item(1);
var r = point3.Item(2);
var sum = Math.pow(p - x, 2) + Math.pow(q - y, 2) + Math.pow(r - z, 2);
var diagonal = Math.sqrt(sum);
return diagonal;
\end{verbatim}

\} catch (e) \{ throw e; \} \}

建立元件的約束物件

pfcComponentConstraint.Create()

ComponentConstraintType Using the TYPE options, you can specify the
placement constraint types. They are as follows:

兩平面進行面接約束 (ASM\_CONSTRAINT\_MATE)

○ ASM\_CONSTRAINT\_MATE Use this option to make two surfaces touch one
another, that is coincident and facing each other.

兩平面進行面接約束, 且相隔一段距離 (ASM\_CONSTRAINT\_MATE\_OFF)

○ ASM\_CONSTRAINT\_MATE\_OFF Use this option to make two planar surfaces
parallel and facing each other.

兩平面或兩軸對齊 (ASM\_CONSTRAINT\_ALIGN)

○ ASM\_CONSTRAINT\_ALIGN Use this option to make two planes coplanar,
two axes coaxial and two points coincident. You can also align revolved
surfaces or edges.

兩平面或兩軸對齊, 且維持一特定距離 (ASM\_CONSTRAINT\_ALIGN\_OFF)

○ ASM\_CONSTRAINT\_ALIGN\_OFF Use this option to align two planar
surfaces at an offset.

插入約束 (ASM\_CONSTRAINT\_INSERT)

○ ASM\_CONSTRAINT\_INSERT Use this option to insert a ``male'' revolved
surface into a ``female'' revolved surface, making their respective axes
coaxial.

同方位約束 (ASM\_CONSTRAINT\_ORIENT)

○ ASM\_CONSTRAINT\_ORIENT Use this option to make two planar surfaces to
be parallel in the same direction.

座標系統定位約束 (ASM\_CONSTRAINT\_CSYS)

○ ASM\_CONSTRAINT\_CSYS Use this option to place a component in an
assembly by aligning the coordinate system of the component with the
coordinate system of the assembly.

垂直約束 (ASM\_CONSTRAINT\_TANGENT)

○ ASM\_CONSTRAINT\_TANGENT Use this option to control the contact of two
surfaces at their tangents.

控制與某曲面上一點接觸約束 (ASM\_CONSTRAINT\_PNT\_ON\_SRF)

○ ASM\_CONSTRAINT\_PNT\_ON\_SRF Use this option to control the contact
of a surface with a point.

控制與某曲面上一線接觸約束 (ASM\_CONSTRAINT\_EDGE\_ON\_SRF)

○ ASM\_CONSTRAINT\_EDGE\_ON\_SRF Use this option to control the contact
of a surface with a straight edge.

與組立中的特定座標系統對齊 (ASM\_CONSTRAINT\_DEF\_PLACEMENT)

○ ASM\_CONSTRAINT\_DEF\_PLACEMENT Use this option to align the default
coordinate system of the component to the default coordinate system of
the assembly.

取代約束 (ASM\_CONSTRAINT\_SUBSTITUTE)

○ ASM\_CONSTRAINT\_SUBSTITUTE Use this option in simplified
representations when a component has been substituted with some other
model

線與點進行接觸約束 (ASM\_CONSTRAINT\_PNT\_ON\_LINE)

○ ASM\_CONSTRAINT\_PNT\_ON\_LINE Use this option to control the contact
of a line with a point.

固定約束 (ASM\_CONSTRAINT\_FIX)

○ ASM\_CONSTRAINT\_FIX Use this option to force the component to remain
in its current packaged position.

○ ASM\_CONSTRAINT\_AUTO Use this option in the user interface to allow
an automatic choice of constraint type based upon the
references.AssemblyDatumSide Orientation of the assembly. This can have
the following values:

進行正面方向約束

○ Yellow The primary side of the datum plane which is the default
direction of the arrow.

進行反面方向約束

○ Red The secondary side of the datum plane which is the direction
opposite to that of the arrow.

ComponentReference A reference on the placed component.

ComponentDatumSide Orientation of the assembly component. This can have
the following values:

正面約束

○ Yellow The primary side of the datum plane which is the default
direction of the arrow.

反面約束 ○ Red The secondary side of the datum plane which is the
direction opposite to that of the arrow.

Offset The mate or align offset value from the reference.

In Pro/ENGINEER, a component's constraints are defined to restrict its
movement with respect to other components. Under certain conditions, a
component's definition dialog box shows an option to ``Allow
assumptions'' on the component.

When you select this option, Pro/ENGINEER adds more constraints to
completely constrain the component's motion with respect to its
connected, neighboring components.

Because these extra constraints are hidden, the exporter cannot
accurately determine the DoF restrictions between a component and its
connected, neighboring components.

To Allow Constraint Orientation Assumptions

The Allow Assumptions check box, located in the Placement Status area of
the Component Placement dialog box, allows you to switch system
constraint orientation assumptions on and off. The Allow Assumptions
check box is available whenever assumptions are made or could be made;
when a component is fully constrained, the check box disappears. The
setting of Allow Assumptions is component specific, and the setting is
saved with the component.

Allow assumptions - will allow Pro/E to make assumptions as to what you
think something to be fully constrained, like a bushing in a cylinder
after you use the insert and align will be listed as fully constrained
even though a rotational degree of freedom still exist.

Typically, three placement constraints are required to fully constrain a
component. Fewer constraints may be applied if the system is able to
position the component by making assumptions regarding orientation.

In reality, there is still a rotational degree of freedom open for this
component, but that may not be critical in our design, so we can allow
the system to fix this rotational degree of freedom for us. If we were
to uncheck the ``Allow Assumptions'' box, we would have to define
additional placement constraints to fix this rotational degree of
freedom.

\begin{Shaded}
\begin{Highlighting}[]
\CommentTok{#coding: utf-8}
\CharTok{import} \NormalTok{sqlite3}
\CharTok{from} \NormalTok{pkg_resources }\CharTok{import} \NormalTok{parse_version}

\NormalTok{__version__ = }\StringTok{"0.2.1"}
\NormalTok{__author__ = }\StringTok{"Mickael Desfrenes"}
\NormalTok{__email__ = }\StringTok{"desfrenes@gmail.com"}

\CommentTok{# Yen 2013.04.08, 將 Python2 的 .next() 改為 next(), 以便在 Python 3 中使用}

\KeywordTok{class} \NormalTok{SQLiteWriter(}\DataTypeTok{object}\NormalTok{):}

    \CommentTok{"""}
\CommentTok{    In frozen mode (the default), the writer will not alter db schema.}
\CommentTok{    Just add frozen=False to enable column creation (or just add False}
\CommentTok{    as second parameter):}

\CommentTok{    query_writer = SQLiteWriter(":memory:", False)}
\CommentTok{    """}
    \KeywordTok{def} \OtherTok{__init__}\NormalTok{(}\OtherTok{self}\NormalTok{, db_path=}\StringTok{":memory:"}\NormalTok{, frozen=}\OtherTok{True}\NormalTok{):}
        \OtherTok{self}\NormalTok{.db = sqlite3.}\OtherTok{connect}\NormalTok{(db_path)}
        \OtherTok{self}\NormalTok{.db.isolation_level = }\OtherTok{None}
        \OtherTok{self}\NormalTok{.db.row_factory = sqlite3.Row}
        \OtherTok{self}\NormalTok{.frozen = frozen}
        \OtherTok{self}\NormalTok{.cursor = }\OtherTok{self}\NormalTok{.db.cursor()}
        \OtherTok{self}\NormalTok{.cursor.execute(}\StringTok{"PRAGMA foreign_keys=ON;"}\NormalTok{)}
        \OtherTok{self}\NormalTok{.cursor.execute(}\StringTok{'PRAGMA encoding = "UTF-8";'}\NormalTok{)}
        \OtherTok{self}\NormalTok{.cursor.execute(}\StringTok{'BEGIN;'}\NormalTok{)}
    \KeywordTok{def} \OtherTok{__del__}\NormalTok{(}\OtherTok{self}\NormalTok{):}
        \OtherTok{self}\NormalTok{.db.close()}

    \KeywordTok{def} \NormalTok{replace(}\OtherTok{self}\NormalTok{, bean):}
        \NormalTok{keys = []}
        \NormalTok{values = []}
        \NormalTok{write_operation = }\StringTok{"replace"}
        \KeywordTok{if} \StringTok{"id"} \NormalTok{not in bean.__dict__:}
            \NormalTok{write_operation = }\StringTok{"insert"}
            \NormalTok{keys.append(}\StringTok{"id"}\NormalTok{)}
            \NormalTok{values.append(}\OtherTok{None}\NormalTok{)}
        \OtherTok{self}\NormalTok{.__create_table(bean.__class__.}\OtherTok{__name__}\NormalTok{)}
        \NormalTok{columns = }\OtherTok{self}\NormalTok{.__get_columns(bean.__class__.}\OtherTok{__name__}\NormalTok{)}
        \KeywordTok{for} \NormalTok{key in bean.__dict__:}
            \NormalTok{keys.append(key)}
            \KeywordTok{if} \NormalTok{key not in columns:}
                \OtherTok{self}\NormalTok{.__create_column(bean.__class__.}\OtherTok{__name__}\NormalTok{, key,}
                        \DataTypeTok{type}\NormalTok{(bean.__dict__[key]))}
            \NormalTok{values.append(bean.__dict__[key])}
        \NormalTok{sql  = write_operation + }\StringTok{" into "} \NormalTok{+ bean.__class__.}\OtherTok{__name__} \NormalTok{+ }\StringTok{"("}
        \NormalTok{sql += }\StringTok{","}\NormalTok{.join(keys) + }\StringTok{") values ("} 
        \NormalTok{sql += }\StringTok{","}\NormalTok{.join([}\StringTok{"?"} \KeywordTok{for} \NormalTok{i in keys])  +  }\StringTok{")"}
        \OtherTok{self}\NormalTok{.cursor.execute(sql, values)}
        \KeywordTok{if} \NormalTok{write_operation == }\StringTok{"insert"}\NormalTok{:}
            \NormalTok{bean.}\DataTypeTok{id} \NormalTok{= }\OtherTok{self}\NormalTok{.cursor.lastrowid}
        \KeywordTok{return} \NormalTok{bean.}\DataTypeTok{id}

    \KeywordTok{def} \NormalTok{__create_column(}\OtherTok{self}\NormalTok{, table, column, sqltype):}
        \KeywordTok{if} \OtherTok{self}\NormalTok{.frozen:}
            \KeywordTok{return}
        \KeywordTok{if} \NormalTok{sqltype in [}\DataTypeTok{float}\NormalTok{, }\DataTypeTok{int}\NormalTok{, }\DataTypeTok{bool}\NormalTok{]:}
            \NormalTok{sqltype = }\StringTok{"NUMERIC"}
        \KeywordTok{else}\NormalTok{:}
            \NormalTok{sqltype = }\StringTok{"TEXT"}
        \NormalTok{sql = }\StringTok{"alter table "} \NormalTok{+ table + }\StringTok{" add "} \NormalTok{+ column + }\StringTok{" "} \NormalTok{+ sqltype    }
        \OtherTok{self}\NormalTok{.cursor.execute(sql)}

    \KeywordTok{def} \NormalTok{__get_columns(}\OtherTok{self}\NormalTok{, table):}
        \NormalTok{columns = []}
        \KeywordTok{if} \OtherTok{self}\NormalTok{.frozen:}
            \KeywordTok{return} \NormalTok{columns}
        \OtherTok{self}\NormalTok{.cursor.execute(}\StringTok{"PRAGMA table_info("} \NormalTok{+ table  + }\StringTok{")"}\NormalTok{)}
        \KeywordTok{for} \NormalTok{row in }\OtherTok{self}\NormalTok{.cursor:}
            \NormalTok{columns.append(row[}\StringTok{"name"}\NormalTok{])}
        \KeywordTok{return} \NormalTok{columns}

    \KeywordTok{def} \NormalTok{__create_table(}\OtherTok{self}\NormalTok{, table):}
        \KeywordTok{if} \OtherTok{self}\NormalTok{.frozen:}
            \KeywordTok{return}
        \NormalTok{sql = }\StringTok{"create table if not exists "} \NormalTok{+ table + }\StringTok{"(id INTEGER PRIMARY KEY AUTOINCREMENT)"}
        \OtherTok{self}\NormalTok{.cursor.execute(sql)}

    \KeywordTok{def} \NormalTok{get_rows(}\OtherTok{self}\NormalTok{, table_name, sql = }\StringTok{"1"}\NormalTok{, replace = }\OtherTok{None}\NormalTok{):}
        \KeywordTok{if} \NormalTok{replace is }\OtherTok{None} \NormalTok{: replace = []}
        \OtherTok{self}\NormalTok{.__create_table(table_name)}
        \NormalTok{sql = }\StringTok{"SELECT * FROM "} \NormalTok{+ table_name + }\StringTok{" WHERE "} \NormalTok{+ sql}
        \KeywordTok{try}\NormalTok{:}
            \OtherTok{self}\NormalTok{.cursor.execute(sql, replace)}
            \KeywordTok{for} \NormalTok{row in }\OtherTok{self}\NormalTok{.cursor:}
                \KeywordTok{yield} \NormalTok{row}
        \KeywordTok{except} \NormalTok{sqlite3.OperationalError:}
            \KeywordTok{return}
   
    \KeywordTok{def} \NormalTok{get_count(}\OtherTok{self}\NormalTok{, table_name, sql=}\StringTok{"1"}\NormalTok{, replace = }\OtherTok{None}\NormalTok{):}
        \KeywordTok{if} \NormalTok{replace is }\OtherTok{None} \NormalTok{: replace = []}
        \OtherTok{self}\NormalTok{.__create_table(table_name)}
        \NormalTok{sql = }\StringTok{"SELECT count(*) AS cnt FROM "} \NormalTok{+ table_name + }\StringTok{" WHERE "} \NormalTok{+ sql}
        \KeywordTok{try}\NormalTok{:}
            \OtherTok{self}\NormalTok{.cursor.execute(sql, replace)}
        \KeywordTok{except} \NormalTok{sqlite3.OperationalError:}
            \KeywordTok{return} \DecValTok{0}
        \KeywordTok{for} \NormalTok{row in }\OtherTok{self}\NormalTok{.cursor:}
            \KeywordTok{return} \NormalTok{row[}\StringTok{"cnt"}\NormalTok{]}

    \KeywordTok{def} \NormalTok{delete(}\OtherTok{self}\NormalTok{, bean):}
        \OtherTok{self}\NormalTok{.__create_table(bean.__class__.}\OtherTok{__name__}\NormalTok{)}
        \NormalTok{sql = }\StringTok{"delete from "} \NormalTok{+ bean.__class__.}\OtherTok{__name__} \NormalTok{+ }\StringTok{" where id=?"}
        \OtherTok{self}\NormalTok{.cursor.execute(sql,[bean.}\DataTypeTok{id}\NormalTok{])}
    
    \KeywordTok{def} \NormalTok{link(}\OtherTok{self}\NormalTok{, bean_a, bean_b):}
        \OtherTok{self}\NormalTok{.replace(bean_a)}
        \OtherTok{self}\NormalTok{.replace(bean_b)}
        \NormalTok{table_a = bean_a.__class__.}\OtherTok{__name__}
        \NormalTok{table_b = bean_b.__class__.}\OtherTok{__name__}
        \NormalTok{assoc_table = }\OtherTok{self}\NormalTok{.__create_assoc_table(table_a, table_b)}
        \NormalTok{sql = }\StringTok{"replace into "} \NormalTok{+ assoc_table + }\StringTok{"("} \NormalTok{+ table_a + }\StringTok{"_id,"} \NormalTok{+ table_b}
        \NormalTok{sql += }\StringTok{"_id) values(?,?)"}
        \OtherTok{self}\NormalTok{.cursor.execute(sql,}
                \NormalTok{[bean_a.}\DataTypeTok{id}\NormalTok{, bean_b.}\DataTypeTok{id}\NormalTok{])}
    
    \KeywordTok{def} \NormalTok{unlink(}\OtherTok{self}\NormalTok{, bean_a, bean_b):}
        \NormalTok{table_a = bean_a.__class__.}\OtherTok{__name__}
        \NormalTok{table_b = bean_b.__class__.}\OtherTok{__name__}
        \NormalTok{assoc_table = }\OtherTok{self}\NormalTok{.__create_assoc_table(table_a, table_b)}
        \NormalTok{sql = }\StringTok{"delete from "} \NormalTok{+ assoc_table + }\StringTok{" where "} \NormalTok{+ table_a}
        \NormalTok{sql += }\StringTok{"_id=? and "} \NormalTok{+ table_b + }\StringTok{"_id=?"}
        \OtherTok{self}\NormalTok{.cursor.execute(sql,}
                \NormalTok{[bean_a.}\DataTypeTok{id}\NormalTok{, bean_b.}\DataTypeTok{id}\NormalTok{])}
    
    \KeywordTok{def} \NormalTok{get_linked_rows(}\OtherTok{self}\NormalTok{, bean, table_name):}
        \NormalTok{bean_table = bean.__class__.}\OtherTok{__name__}
        \NormalTok{assoc_table = }\OtherTok{self}\NormalTok{.__create_assoc_table(bean_table, table_name)}
        \NormalTok{sql = }\StringTok{"select t.* from "} \NormalTok{+ table_name + }\StringTok{" t inner join "} \NormalTok{+ assoc_table }
        \NormalTok{sql += }\StringTok{" a on a."} \NormalTok{+ table_name + }\StringTok{"_id = t.id where a."}
        \NormalTok{sql += bean_table + }\StringTok{"_id=?"}
        \OtherTok{self}\NormalTok{.cursor.execute(sql,[bean.}\DataTypeTok{id}\NormalTok{])}
        \KeywordTok{for} \NormalTok{row in }\OtherTok{self}\NormalTok{.cursor:}
            \KeywordTok{yield} \NormalTok{row}

    \KeywordTok{def} \NormalTok{__create_assoc_table(}\OtherTok{self}\NormalTok{, table_a, table_b):}
        \NormalTok{assoc_table = }\StringTok{"_"}\NormalTok{.join(}\DataTypeTok{sorted}\NormalTok{([table_a, table_b]))}
        \KeywordTok{if} \NormalTok{not }\OtherTok{self}\NormalTok{.frozen:}
            \NormalTok{sql = }\StringTok{"create table if not exists "} \NormalTok{+ assoc_table + }\StringTok{"("}
            \NormalTok{sql+= table_a + }\StringTok{"_id NOT NULL REFERENCES "} \NormalTok{+ table_a + }\StringTok{"(id) ON DELETE cascade,"}
            \NormalTok{sql+= table_b + }\StringTok{"_id NOT NULL REFERENCES "} \NormalTok{+ table_b + }\StringTok{"(id) ON DELETE cascade,"}
            \NormalTok{sql+= }\StringTok{" PRIMARY KEY ("} \NormalTok{+ table_a + }\StringTok{"_id,"} \NormalTok{+ table_b + }\StringTok{"_id));"}
            \OtherTok{self}\NormalTok{.cursor.execute(sql)}
            \CommentTok{# no real support for foreign keys until sqlite3 v3.6.19}
            \CommentTok{# so here's the hack}
            \KeywordTok{if} \DataTypeTok{cmp}\NormalTok{(parse_version(sqlite3.sqlite_version),parse_version(}\StringTok{"3.6.19"}\NormalTok{)) < }\DecValTok{0}\NormalTok{:}
                \NormalTok{sql = }\StringTok{"create trigger if not exists fk_"} \NormalTok{+ table_a + }\StringTok{"_"} \NormalTok{+ assoc_table}
                \NormalTok{sql+= }\StringTok{" before delete on "} \NormalTok{+ table_a}
                \NormalTok{sql+= }\StringTok{" for each row begin delete from "} \NormalTok{+ assoc_table + }\StringTok{" where "} \NormalTok{+ table_a + }\StringTok{"_id = OLD.id;end;"}
                \OtherTok{self}\NormalTok{.cursor.execute(sql)}
                \NormalTok{sql = }\StringTok{"create trigger if not exists fk_"} \NormalTok{+ table_b + }\StringTok{"_"} \NormalTok{+ assoc_table}
                \NormalTok{sql+= }\StringTok{" before delete on "} \NormalTok{+ table_b}
                \NormalTok{sql+= }\StringTok{" for each row begin delete from "} \NormalTok{+ assoc_table + }\StringTok{" where "} \NormalTok{+ table_b + }\StringTok{"_id = OLD.id;end;"}
                \OtherTok{self}\NormalTok{.cursor.execute(sql)}
        \KeywordTok{return} \NormalTok{assoc_table}

    \KeywordTok{def} \NormalTok{delete_all(}\OtherTok{self}\NormalTok{, table_name, sql = }\StringTok{"1"}\NormalTok{, replace = }\OtherTok{None}\NormalTok{):}
        \KeywordTok{if} \NormalTok{replace is }\OtherTok{None} \NormalTok{: replace = []}
        \OtherTok{self}\NormalTok{.__create_table(table_name)}
        \NormalTok{sql = }\StringTok{"DELETE FROM "} \NormalTok{+ table_name + }\StringTok{" WHERE "} \NormalTok{+ sql}
        \KeywordTok{try}\NormalTok{:}
            \OtherTok{self}\NormalTok{.cursor.execute(sql, replace)}
            \KeywordTok{return} \OtherTok{True}
        \KeywordTok{except} \NormalTok{sqlite3.OperationalError:}
            \KeywordTok{return} \OtherTok{False}

    \KeywordTok{def} \NormalTok{commit(}\OtherTok{self}\NormalTok{):}
        \OtherTok{self}\NormalTok{.db.commit()}



\KeywordTok{class} \NormalTok{Store(}\DataTypeTok{object}\NormalTok{):}
    \CommentTok{"""}
\CommentTok{    A SQL writer should be passed to the constructor:}

\CommentTok{    beans_save = Store(SQLiteWriter(":memory"), frozen=False)}
\CommentTok{    """}
    \KeywordTok{def} \OtherTok{__init__}\NormalTok{(}\OtherTok{self}\NormalTok{, SQLWriter):}
        \OtherTok{self}\NormalTok{.writer = SQLWriter }
    
    \KeywordTok{def} \NormalTok{new(}\OtherTok{self}\NormalTok{, table_name):}
        \NormalTok{new_object = }\DataTypeTok{type}\NormalTok{(table_name,(}\DataTypeTok{object}\NormalTok{,),\{\})()}
        \KeywordTok{return} \NormalTok{new_object}

    \KeywordTok{def} \NormalTok{save(}\OtherTok{self}\NormalTok{, bean):}
        \OtherTok{self}\NormalTok{.writer.replace(bean)}
    
    \KeywordTok{def} \NormalTok{load(}\OtherTok{self}\NormalTok{, table_name, }\DataTypeTok{id}\NormalTok{):}
        \KeywordTok{for} \NormalTok{row in }\OtherTok{self}\NormalTok{.writer.get_rows(table_name, }\StringTok{"id=?"}\NormalTok{, [}\DataTypeTok{id}\NormalTok{]):}
            \KeywordTok{return} \OtherTok{self}\NormalTok{.row_to_object(table_name, row)}

    \KeywordTok{def} \NormalTok{count(}\OtherTok{self}\NormalTok{, table_name, sql = }\StringTok{"1"}\NormalTok{, replace=}\OtherTok{None}\NormalTok{):}
        \KeywordTok{return} \OtherTok{self}\NormalTok{.writer.get_count(table_name, sql, replace }\KeywordTok{if} \NormalTok{replace is not }\OtherTok{None} \KeywordTok{else} \NormalTok{[])}

    \KeywordTok{def} \NormalTok{find(}\OtherTok{self}\NormalTok{, table_name, sql = }\StringTok{"1"}\NormalTok{, replace=}\OtherTok{None}\NormalTok{):}
        \KeywordTok{for} \NormalTok{row in }\OtherTok{self}\NormalTok{.writer.get_rows(table_name, sql, replace }\KeywordTok{if} \NormalTok{replace is not }\OtherTok{None} \KeywordTok{else} \NormalTok{[]):}
            \KeywordTok{yield} \OtherTok{self}\NormalTok{.row_to_object(table_name, row)}

    \KeywordTok{def} \NormalTok{find_one(}\OtherTok{self}\NormalTok{, table_name, sql = }\StringTok{"1"}\NormalTok{, replace=}\OtherTok{None}\NormalTok{):}
        \KeywordTok{try}\NormalTok{:}
            \KeywordTok{return} \DataTypeTok{next}\NormalTok{(}\OtherTok{self}\NormalTok{.find(table_name, sql, replace))}
        \KeywordTok{except} \OtherTok{StopIteration}\NormalTok{:}
            \KeywordTok{return} \OtherTok{None}

    \KeywordTok{def} \NormalTok{delete(}\OtherTok{self}\NormalTok{, bean):}
        \OtherTok{self}\NormalTok{.writer.delete(bean)}
    
    \KeywordTok{def} \NormalTok{link(}\OtherTok{self}\NormalTok{, bean_a, bean_b):}
        \OtherTok{self}\NormalTok{.writer.link(bean_a, bean_b)}
    
    \KeywordTok{def} \NormalTok{unlink(}\OtherTok{self}\NormalTok{, bean_a, bean_b):}
        \OtherTok{self}\NormalTok{.writer.unlink(bean_a, bean_b)}
    
    \KeywordTok{def} \NormalTok{get_linked(}\OtherTok{self}\NormalTok{, bean, table_name):}
        \KeywordTok{for} \NormalTok{row in }\OtherTok{self}\NormalTok{.writer.get_linked_rows(bean, table_name):}
            \KeywordTok{yield} \OtherTok{self}\NormalTok{.row_to_object(table_name, row)}

    \KeywordTok{def} \NormalTok{delete_all(}\OtherTok{self}\NormalTok{, table_name, sql = }\StringTok{"1"}\NormalTok{, replace=}\OtherTok{None}\NormalTok{):}
        \KeywordTok{return} \OtherTok{self}\NormalTok{.writer.delete_all(table_name, sql, replace }\KeywordTok{if} \NormalTok{replace is not }\OtherTok{None} \KeywordTok{else} \NormalTok{[])}

    \KeywordTok{def} \NormalTok{row_to_object(}\OtherTok{self}\NormalTok{, table_name, row):}
        \NormalTok{new_object = }\DataTypeTok{type}\NormalTok{(table_name,(}\DataTypeTok{object}\NormalTok{,),\{\})()}
        \KeywordTok{for} \NormalTok{key in row.keys():}
            \NormalTok{new_object.__dict__[key] = row[key]}
        \KeywordTok{return} \NormalTok{new_object}

    \KeywordTok{def} \NormalTok{commit(}\OtherTok{self}\NormalTok{):}
        \OtherTok{self}\NormalTok{.writer.commit()}
\end{Highlighting}
\end{Shaded}

http://inversionconsulting.blogspot.in/2008/06/proe-vb-api-not-just-for-visual-basic.html

'Code for Auto Assembly Program Imports pfcls Imports System Imports
System.IO

Public Class Form1 Private Sub Button1\_Click(ByVal sender As
System.Object, ByVal e As System.EventArgs\_ )Handles Button1.Click Dim
conn As IpfcAsyncConnection = Nothing Dim session As IpfcBaseSession Dim
pat, pat1 As String Dim model As IpfcModel Dim modelDesc As
IpfcModelDescriptor Dim modelDesc1 As IpfcModelDescriptor Dim i, k As
Integer Dim win As IpfcWindow Dim loc As String Dim modelnames As
String() = Nothing Dim asmModels As Cstringseq Dim drawingPath As String
Dim drawingName, drawingName1 As String Dim drawingName2() As String Dim
errMsg As String = ``''

'For Assembly constraints Dim components As IpfcFeatures Dim component
As IpfcComponentFeat Dim compConstraints As IpfcComponentConstraints Dim
compConstraint As IpfcComponentConstraint Dim Constraint1 As
IpfcComponentConstraint Dim Constraints1 As ipfcComponentConstraints Dim
assembly As IpfcAssembly Dim assemblyDatums(2) As String Dim
componentDatums(2) As String Dim asmReference As IpfcSelection Dim
compReference As IpfcSelection Dim constraintType As String Dim
componentModel As IpfcSolid Dim asmcomp As IpfcComponentFeat

If Not System.Diagnostics.Process.GetProcessesByName(``nmsd'').Length
\textgreater{} 0 Then errMsg =\_ ``Name service is not running on the
system.'' End If

If System.Environment.GetEnvironmentVariable(``PRO\_COMM\_MSG\_EXE'') =
``'' Then If errMsg = ``'' Then errMsg = ``Environment variable
`PRO\_COMM\_MSG\_EXE' has not been\_ set.'' Else

errMsg = errMsg + Chr(13).ToString + ``Environment variable\_
`PRO\_COMM\_MSG\_EXE' has not been set.''

End If End If

If System.Environment.GetEnvironmentVariable(``PRO\_DIRECTORY'') Is
Nothing Then

If errMsg = ``'' Then errMsg = ``Environment variable `PRO\_DIRECTORY'
has not been set.'' Else

errMsg = errMsg + Chr(13).ToString + ``Environment variable
`PRO\_DIRECTORY' has not been set.''

End If

End If 'If Services are not running Exit Application

If Not errMsg = ``'' Then errMsg = errMsg + Chr(13).ToString + ``These
may lead to errors in running the application.'' MsgBox(errMsg,
MsgBoxStyle.Critical)

Else 'Connect to Pro/Engineer

If conn Is Nothing OrElse Not conn.IsRunning Then conn = (New
CCpfcAsyncConnection).Connect(Nothing, Nothing, Nothing, Nothing)
session = conn.Session

If session Is Nothing Then Throw New Exception(``Session does not
exist'') Else session.EraseUndisplayedModels()

'Directory for Pro/E Models

loc = ``d:\proemodels".ToLower asmModels = session.ListFiles(''*.asm``,
EpfcFileListOpt.EpfcFILE\_LIST\_LATEST, loc)

If asmModels.Count \textless{}\textgreater{} 0 Then

k = 0

For k = 0 To asmModels.Count - 1 drawingPath = asmModels.Item(k)
drawingName = drawingPath.Substring(loc.Length) drawingName1 =
drawingName.Substring(0, drawingName.Length - 4) drawingName2 =
drawingName1.Split(``-'') pat1 = drawingName.Substring(0, 1).ToUpper \&
``SAMPLE'' \& drawingName2(1).ToUpper \& ``.prt'' drawingName1 =
drawingName1.Substring(drawingName1.Length - 7, 7).ToUpper

modelDesc1 = (New CCpfcModelDescriptor).Create(0, drawingName, Nothing)
session.RetrieveModel(modelDesc1)
session.OpenFile(modelDesc1).Activate() model = session.CurrentModel
assembly = CType(model, IpfcAssembly) modelDesc = (New
CCpfcModelDescriptor).Create(1, pat1, Nothing) componentModel =
session.RetrieveModel(modelDesc) pat1 = pat1.Substring(0, pat1.Length -
4) components = assembly.ListFeaturesByType(False,
EpfcFeatureType.EpfcFEATTYPE\_COMPONENT)

'Search through the models in the current assembly

For i = 0 To components.Count - 1 component = components.Item(i)
modelDesc = component.ModelDescr

If modelDesc.InstanceName = pat1 Then compConstraints =
component.GetConstraints() Constraints1 = New CpfcComponentConstraints
asmcomp = assembly.AssembleComponent(componentModel, Nothing)

For j = 0 To compConstraints.Count - 1 compConstraint =
compConstraints.Item(j) constraintType =
constraintTypeToString(compConstraint.Type) asmReference =
compConstraint.AssemblyReference compReference =
compConstraint.ComponentReference

Select Case (constraintType)

Case ``Mate'' Constraint1 = (New
CCpfcComponentConstraint).Create\_(EpfcComponentConstraintType.EpfcASM\_CONSTRAINT\_MATE)
Constraint1.AssemblyReference = asmReference
Constraint1.ComponentReference = compReference
Constraints1.Insert(Constraints1.Count, Constraint1)

Case ``Align'' Constraint1 = (New
CCpfcComponentConstraint).Create\_(EpfcComponentConstraintType.EpfcASM\_CONSTRAINT\_ALIGN\_OFF)
Constraint1.AssemblyReference = asmReference
Constraint1.ComponentReference = compReference
Constraints1.Insert(Constraints1.Count, Constraint1)

Case ``Align Offset'' Constraint1 = (New
CCpfcComponentConstraint).Create\_(EpfcComponentConstraintType.EpfcASM\_CONSTRAINT\_ALIGN)
Constraint1.AssemblyReference = asmReference
Constraint1.ComponentReference = compReference Constraint1.Offset = 10
Constraints1.Insert(Constraints1.Count, Constraint1)

End Select

Next

asmcomp.SetConstraints(Constraints1, Nothing)
assembly.Regenerate(Nothing) session.GetModelWindow(assembly).Repaint()
model.Save()

End If

Next

session.CurrentWindow.Activate() win = session.CurrentWindow Next
conn.Disconnect(2)

End If

End If

End If

End If

End Sub

Private Function constraintTypeToString(ByVal type As Integer) As String

Select Case (type)

Case EpfcComponentConstraintType.EpfcASM\_CONSTRAINT\_MATE Return
(``Mate'')

Case EpfcComponentConstraintType.EpfcASM\_CONSTRAINT\_MATE\_OFF Return
(``Mate Offset'')

Case EpfcComponentConstraintType.EpfcASM\_CONSTRAINT\_ALIGN Return
(``Align'')

Case EpfcComponentConstraintType.EpfcASM\_CONSTRAINT\_ALIGN\_OFF Return
(``Align Offset'')

Case EpfcComponentConstraintType.EpfcASM\_CONSTRAINT\_INSERT Return
(``Insert'')

Case EpfcComponentConstraintType.EpfcASM\_CONSTRAINT\_ORIENT Return
(``Orient'')

Case EpfcComponentConstraintType.EpfcASM\_CONSTRAINT\_CSYS Return
(``Csys'')

Case EpfcComponentConstraintType.EpfcASM\_CONSTRAINT\_TANGENT Return
(``Tangent'')

Case EpfcComponentConstraintType.EpfcASM\_CONSTRAINT\_PNT\_ON\_SRF
Return (``Point on Surf'')

Case EpfcComponentConstraintType.EpfcASM\_CONSTRAINT\_EDGE\_ON\_SRF
Return (``Edge on Surf'')

Case EpfcComponentConstraintType.EpfcASM\_CONSTRAINT\_DEF\_PLACEMENT
Return (``Default'')

Case EpfcComponentConstraintType.EpfcASM\_CONSTRAINT\_SUBSTITUTE Return
(``Substitute'')

Case EpfcComponentConstraintType.EpfcASM\_CONSTRAINT\_PNT\_ON\_LINE
Return (``Point on Line'')

Case EpfcComponentConstraintType.EpfcASM\_CONSTRAINT\_FIX Return
(``Fix'')

Case EpfcComponentConstraintType.EpfcASM\_CONSTRAINT\_AUTO Return
(``Auto'')

End Select

Return (``Unrecognized Type'') End Function End Class

\begin{Shaded}
\begin{Highlighting}[]
\CommentTok{/*}
\CommentTok{   HISTORY}
\CommentTok{   }
\CommentTok{14-NOV-02   J-03-38   $$1   JCN      Adapted from J-Link examples.}
\CommentTok{07-MAR-03   K-01-03   $$2   JCN      UNIX support}

\CommentTok{*/}

\KeywordTok{function} \FunctionTok{replaceBoltsInAssembly} \NormalTok{()}
\NormalTok{\{}
  \KeywordTok{var} \NormalTok{oldInstance = }\StringTok{"PHILLIPS7_8"}\NormalTok{;}
  \KeywordTok{var} \NormalTok{newInstance = }\StringTok{"SLOT7_8"}\NormalTok{;}
  
  \KeywordTok{if} \NormalTok{(!}\FunctionTok{pfcIsWindows}\NormalTok{())}
    \OtherTok{netscape}\NormalTok{.}\OtherTok{security}\NormalTok{.}\OtherTok{PrivilegeManager}\NormalTok{.}\FunctionTok{enablePrivilege}\NormalTok{(}\StringTok{"UniversalXPConnect"}\NormalTok{);}

 \CommentTok{/*--------------------------------------------------------------------*\textbackslash{} }
\CommentTok{   Get the current assembly }
\CommentTok{ \textbackslash{}*--------------------------------------------------------------------*/}  
  \KeywordTok{var} \NormalTok{session = }\FunctionTok{pfcGetProESession} \NormalTok{();}
  \KeywordTok{var} \NormalTok{assembly = }\OtherTok{session}\NormalTok{.}\FunctionTok{CurrentModel}\NormalTok{;}
  
  \KeywordTok{if} \NormalTok{(}\OtherTok{assembly}\NormalTok{.}\FunctionTok{Type} \NormalTok{!= }\FunctionTok{pfcCreate} \NormalTok{(}\StringTok{"pfcModelType"}\NormalTok{).}\FunctionTok{MDL_ASSEMBLY}\NormalTok{)}
    \KeywordTok{throw} \KeywordTok{new} \FunctionTok{Error} \NormalTok{(}\DecValTok{0}\NormalTok{, }\StringTok{"Current model is not an assembly"}\NormalTok{);}
  
 \CommentTok{/*--------------------------------------------------------------------*\textbackslash{} }
\CommentTok{   Get the new instance model for use in replacement}
\CommentTok{ \textbackslash{}*--------------------------------------------------------------------*/}  
  \KeywordTok{var} \NormalTok{bolt = }\OtherTok{session}\NormalTok{.}\FunctionTok{GetModel} \NormalTok{(}\StringTok{"BOLT"}\NormalTok{, }\FunctionTok{pfcCreate} \NormalTok{(}\StringTok{"pfcModelType"}\NormalTok{).}\FunctionTok{MDL_PART}\NormalTok{);}
  
  \KeywordTok{var} \NormalTok{row = }\OtherTok{bolt}\NormalTok{.}\FunctionTok{GetRow} \NormalTok{(newInstance);  }
  
  \KeywordTok{var} \NormalTok{newBolt = }\OtherTok{row}\NormalTok{.}\FunctionTok{CreateInstance}\NormalTok{();}
  
  \KeywordTok{var} \NormalTok{replaceOps = }\FunctionTok{pfcCreate} \NormalTok{(}\StringTok{"pfcFeatureOperations"}\NormalTok{);}
 
 \CommentTok{/*--------------------------------------------------------------------*\textbackslash{} }
\CommentTok{   Visit the assembly components}
\CommentTok{ \textbackslash{}*--------------------------------------------------------------------*/}               
  \KeywordTok{var} \NormalTok{components = }\OtherTok{assembly}\NormalTok{.}\FunctionTok{ListFeaturesByType} \NormalTok{(}\KeywordTok{false}\NormalTok{,}
                        \FunctionTok{pfcCreate} \NormalTok{(}\StringTok{"pfcFeatureType"}\NormalTok{).}\FunctionTok{FEATTYPE_COMPONENT}\NormalTok{);}
  
  \KeywordTok{for} \NormalTok{(ii = }\DecValTok{0}\NormalTok{; ii < }\OtherTok{components}\NormalTok{.}\FunctionTok{Count}\NormalTok{; ii++)}
    \NormalTok{\{}
      \KeywordTok{var} \NormalTok{component = }\OtherTok{components}\NormalTok{.}\FunctionTok{Item}\NormalTok{(ii);}
      
      \KeywordTok{var} \NormalTok{desc = }\OtherTok{component}\NormalTok{.}\FunctionTok{ModelDescr}\NormalTok{;}
      
      \KeywordTok{if} \NormalTok{(}\OtherTok{desc}\NormalTok{.}\FunctionTok{InstanceName} \NormalTok{== oldInstance)}
    \NormalTok{\{}
      \KeywordTok{var} \NormalTok{replace = }\OtherTok{component}\NormalTok{.}\FunctionTok{CreateReplaceOp} \NormalTok{(newBolt);}
      
      \OtherTok{replaceOps}\NormalTok{.}\FunctionTok{Append} \NormalTok{(replace);}
    \NormalTok{\}}
    \NormalTok{\}}
  
 \CommentTok{/*--------------------------------------------------------------------*\textbackslash{} }
\CommentTok{   Carry out the replacements}
\CommentTok{ \textbackslash{}*--------------------------------------------------------------------*/}  
  \OtherTok{assembly}\NormalTok{.}\FunctionTok{ExecuteFeatureOps} \NormalTok{(replaceOps, }\KeywordTok{null}\NormalTok{);}
  
  \KeywordTok{return}\NormalTok{;}
\NormalTok{\}}


\CommentTok{/*=====================================================================*\textbackslash{}}
 \NormalTok{This }\KeywordTok{function} \NormalTok{displays each constraint of the component visually on }
 \NormalTok{the screen, and includes a text explanation }\KeywordTok{for} \NormalTok{each }\OtherTok{constraint}\NormalTok{.}
\NormalTok{\textbackslash{}*=====================================================================*}\OtherTok{/}

\CommentTok{/*=====================================================================*\textbackslash{}}
\NormalTok{FUNCTION: highlightConstraints}
\NormalTok{PURPOSE:  Highlights and labels a component}\StringTok{'s constraints}
\NormalTok{\textbackslash{}*=====================================================================*}\OtherTok{/}
\KeywordTok{function} \FunctionTok{highlightConstraints} \NormalTok{()}
\NormalTok{\{}
  
  \KeywordTok{if} \NormalTok{(!}\FunctionTok{pfcIsWindows}\NormalTok{())}
    \OtherTok{netscape}\NormalTok{.}\OtherTok{security}\NormalTok{.}\OtherTok{PrivilegeManager}\NormalTok{.}\FunctionTok{enablePrivilege}\NormalTok{(}\StringTok{"UniversalXPConnect"}\NormalTok{);   }
\CommentTok{/*---------------------------------------------------------------------*\textbackslash{}}
  \NormalTok{Get the constraints }\KeywordTok{for} \NormalTok{the }\OtherTok{component}\NormalTok{.}
\NormalTok{\textbackslash{}*---------------------------------------------------------------------*}\OtherTok{/}
  \KeywordTok{var} \NormalTok{session = }\FunctionTok{pfcGetProESession} \NormalTok{();}
  
  \OtherTok{session}\NormalTok{.}\OtherTok{CurrentWindow}\NormalTok{.}\FunctionTok{SetBrowserSize} \NormalTok{(}\FloatTok{0.0}\NormalTok{);}
  
  \KeywordTok{var} \NormalTok{options = }\FunctionTok{pfcCreate} \NormalTok{(}\StringTok{"pfcSelectionOptions"}\NormalTok{).}\FunctionTok{Create} \NormalTok{(}\StringTok{"membfeat"}\NormalTok{);}
  \OtherTok{options}\NormalTok{.}\FunctionTok{MaxNumSels}  \NormalTok{= }\DecValTok{1}\NormalTok{;}
  \KeywordTok{var} \NormalTok{selections = }\OtherTok{session}\NormalTok{.}\FunctionTok{Select} \NormalTok{(options, }\KeywordTok{void} \KeywordTok{null}\NormalTok{);}
  \KeywordTok{if} \NormalTok{(selections == }\KeywordTok{void} \KeywordTok{null} \NormalTok{|| }\OtherTok{selections}\NormalTok{.}\FunctionTok{Count} \NormalTok{== }\DecValTok{0}\NormalTok{)}
    \KeywordTok{return}\NormalTok{;}
  
  \OtherTok{selections}\NormalTok{.}\FunctionTok{Item}\NormalTok{(}\DecValTok{0}\NormalTok{).}\FunctionTok{UnHighlight}\NormalTok{();}
  
  \KeywordTok{var} \NormalTok{feature = }\OtherTok{selections}\NormalTok{.}\FunctionTok{Item} \NormalTok{(}\DecValTok{0}\NormalTok{).}\FunctionTok{SelItem}\NormalTok{;}
  
  \KeywordTok{if} \NormalTok{(}\OtherTok{feature}\NormalTok{.}\FunctionTok{FeatType} \NormalTok{!= }\FunctionTok{pfcCreate} \NormalTok{(}\StringTok{"pfcFeatureType"}\NormalTok{).}\FunctionTok{FEATTYPE_COMPONENT}\NormalTok{)}
    \KeywordTok{return}\NormalTok{;}
  
  \KeywordTok{var} \NormalTok{asmcomp = feature;}
  
  \KeywordTok{var} \NormalTok{constrs = }\OtherTok{asmcomp}\NormalTok{.}\FunctionTok{GetConstraints} \NormalTok{();}
  
  \KeywordTok{if} \NormalTok{(constrs == }\KeywordTok{void} \KeywordTok{null} \NormalTok{|| }\OtherTok{constrs}\NormalTok{.}\FunctionTok{Count} \NormalTok{== }\DecValTok{0}\NormalTok{)}
    \KeywordTok{return}\NormalTok{;}
  
  \KeywordTok{for} \NormalTok{(}\KeywordTok{var} \NormalTok{i = }\DecValTok{0}\NormalTok{; i < }\OtherTok{constrs}\NormalTok{.}\FunctionTok{Count}\NormalTok{; i++)}
    \NormalTok{\{}
\CommentTok{/*---------------------------------------------------------------------*\textbackslash{}}
  \NormalTok{Highlight the assembly reference geometry}
\NormalTok{\textbackslash{}*---------------------------------------------------------------------*}\OtherTok{/}
      \KeywordTok{var} \NormalTok{c = }\OtherTok{constrs}\NormalTok{.}\FunctionTok{Item} \NormalTok{(i);}
      
      \KeywordTok{var} \NormalTok{asmRef = }\OtherTok{c}\NormalTok{.}\FunctionTok{AssemblyReference}\NormalTok{;}
      
      \KeywordTok{if} \NormalTok{(asmRef != }\KeywordTok{void} \KeywordTok{null}\NormalTok{)}
    \OtherTok{asmRef}\NormalTok{.}\FunctionTok{Highlight} \NormalTok{(}\FunctionTok{pfcCreate} \NormalTok{(}\StringTok{"pfcStdColor"}\NormalTok{).}\FunctionTok{COLOR_ERROR}\NormalTok{);}

\CommentTok{/*---------------------------------------------------------------------*\textbackslash{}}
  \NormalTok{Highlight the component reference geometry}
\NormalTok{\textbackslash{}*---------------------------------------------------------------------*}\OtherTok{/}
      \KeywordTok{var} \NormalTok{compRef = }\OtherTok{c}\NormalTok{.}\FunctionTok{ComponentReference}\NormalTok{;}
      
      \KeywordTok{if} \NormalTok{(compRef != }\KeywordTok{void} \KeywordTok{null}\NormalTok{)}
    \OtherTok{compRef}\NormalTok{.}\FunctionTok{Highlight} \NormalTok{(}\FunctionTok{pfcCreate} \NormalTok{(}\StringTok{"pfcStdColor"}\NormalTok{).}\FunctionTok{COLOR_WARNING}\NormalTok{);}
      
\CommentTok{/*---------------------------------------------------------------------*\textbackslash{}}
  \NormalTok{Prepare and display the message }\OtherTok{text}\NormalTok{.}
\NormalTok{\textbackslash{}*---------------------------------------------------------------------*}\OtherTok{/}
      \KeywordTok{var} \NormalTok{offset = }\OtherTok{c}\NormalTok{.}\FunctionTok{Offset}\NormalTok{;}
      \KeywordTok{var} \NormalTok{offsetString = }\StringTok{""}\NormalTok{;}
      \KeywordTok{if} \NormalTok{(offset != }\KeywordTok{void} \KeywordTok{null}\NormalTok{)}
    \NormalTok{offsetString = }\StringTok{", offset of "}\NormalTok{+offset;}
      
      \KeywordTok{var} \NormalTok{cType  = }\OtherTok{c}\NormalTok{.}\FunctionTok{Type}\NormalTok{;}
      \KeywordTok{var} \NormalTok{cTypeString = }\FunctionTok{constraintTypeToString} \NormalTok{(cType);}
      
      \FunctionTok{alert}  \NormalTok{(}\StringTok{"Showing constraint "} \NormalTok{+ (i}\DecValTok{+1}\NormalTok{) +}\StringTok{" of "} \NormalTok{+ }\OtherTok{constrs}\NormalTok{.}\FunctionTok{Count} \NormalTok{+ }\StringTok{"}\CharTok{\textbackslash{}n}\StringTok{"} \NormalTok{+ }
          \NormalTok{cTypeString + offsetString + }\StringTok{"."}\NormalTok{);}
    
\CommentTok{/*---------------------------------------------------------------------*\textbackslash{}}
  \NormalTok{Clean up the UI }\KeywordTok{for} \NormalTok{the next constraint}
\NormalTok{\textbackslash{}*---------------------------------------------------------------------*}\OtherTok{/}
      \KeywordTok{if} \NormalTok{(asmRef != }\KeywordTok{void} \KeywordTok{null}\NormalTok{)}
    \NormalTok{\{}
      \OtherTok{asmRef}\NormalTok{.}\FunctionTok{UnHighlight} \NormalTok{();}
    \NormalTok{\}}
      
      \KeywordTok{if} \NormalTok{(compRef != }\KeywordTok{void} \KeywordTok{null}\NormalTok{)}
    \NormalTok{\{}
      \OtherTok{compRef}\NormalTok{.}\FunctionTok{UnHighlight} \NormalTok{();}
    \NormalTok{\}}
    \NormalTok{\}}
\NormalTok{\}}
    
\CommentTok{/*=====================================================================*\textbackslash{}}
\NormalTok{FUNCTION: constraintTypeToString}
\NormalTok{PURPOSE:  Utility: convert the constraint type to a string }\KeywordTok{for} \NormalTok{printing}
\NormalTok{\textbackslash{}*=====================================================================*}\OtherTok{/}
\KeywordTok{function} \FunctionTok{constraintTypeToString} \NormalTok{(type }\CommentTok{/* pfcComponentConstraintType */}\NormalTok{)}
\NormalTok{\{   }
  \KeywordTok{var} \NormalTok{constrTypeClass = }\FunctionTok{pfcCreate} \NormalTok{(}\StringTok{"pfcComponentConstraintType"}\NormalTok{);}
  \KeywordTok{switch} \NormalTok{(type)}
    \NormalTok{\{}
    \KeywordTok{case} \OtherTok{constrTypeClass}\NormalTok{.}\FunctionTok{ASM_CONSTRAINT_MATE}\NormalTok{:}
      \KeywordTok{return} \NormalTok{(}\StringTok{"(Mate)"}\NormalTok{);        }
    \KeywordTok{case} \OtherTok{constrTypeClass}\NormalTok{.}\FunctionTok{ASM_CONSTRAINT_MATE_OFF}\NormalTok{:}
      \KeywordTok{return} \NormalTok{(}\StringTok{"(Mate Offset)"}\NormalTok{);     }
    \KeywordTok{case} \OtherTok{constrTypeClass}\NormalTok{.}\FunctionTok{ASM_CONSTRAINT_ALIGN}\NormalTok{:}
      \KeywordTok{return} \NormalTok{(}\StringTok{"(Align)"}\NormalTok{);       }
    \KeywordTok{case} \OtherTok{constrTypeClass}\NormalTok{.}\FunctionTok{ASM_CONSTRAINT_ALIGN_OFF}\NormalTok{:}
      \KeywordTok{return} \NormalTok{(}\StringTok{"(Align Offset)"}\NormalTok{);        }
    \KeywordTok{case} \OtherTok{constrTypeClass}\NormalTok{.}\FunctionTok{ASM_CONSTRAINT_INSERT}\NormalTok{:}
      \KeywordTok{return} \NormalTok{(}\StringTok{"(Insert)"}\NormalTok{);      }
    \KeywordTok{case} \OtherTok{constrTypeClass}\NormalTok{.}\FunctionTok{ASM_CONSTRAINT_ORIENT}\NormalTok{:}
      \KeywordTok{return} \NormalTok{(}\StringTok{"(Orient)"}\NormalTok{);      }
    \KeywordTok{case} \OtherTok{constrTypeClass}\NormalTok{.}\FunctionTok{ASM_CONSTRAINT_CSYS}\NormalTok{:}
      \KeywordTok{return} \NormalTok{(}\StringTok{"(Csys)"}\NormalTok{);        }
    \KeywordTok{case} \OtherTok{constrTypeClass}\NormalTok{.}\FunctionTok{ASM_CONSTRAINT_TANGENT}\NormalTok{:}
      \KeywordTok{return} \NormalTok{(}\StringTok{"(Tangent)"}\NormalTok{);     }
    \KeywordTok{case} \OtherTok{constrTypeClass}\NormalTok{.}\FunctionTok{ASM_CONSTRAINT_PNT_ON_SRF}\NormalTok{:}
      \KeywordTok{return} \NormalTok{(}\StringTok{"(Point on Surf)"}\NormalTok{);       }
    \KeywordTok{case} \OtherTok{constrTypeClass}\NormalTok{.}\FunctionTok{ASM_CONSTRAINT_EDGE_ON_SRF}\NormalTok{:}
      \KeywordTok{return} \NormalTok{(}\StringTok{"(Edge on Surf)"}\NormalTok{);        }
    \KeywordTok{case} \OtherTok{constrTypeClass}\NormalTok{.}\FunctionTok{ASM_CONSTRAINT_DEF_PLACEMENT}\NormalTok{:}
      \KeywordTok{return} \NormalTok{(}\StringTok{"(Default)"}\NormalTok{);     }
    \KeywordTok{case} \OtherTok{constrTypeClass}\NormalTok{.}\FunctionTok{ASM_CONSTRAINT_SUBSTITUTE}\NormalTok{:}
      \KeywordTok{return} \NormalTok{(}\StringTok{"(Substitute)"}\NormalTok{);      }
    \KeywordTok{case} \OtherTok{constrTypeClass}\NormalTok{.}\FunctionTok{ASM_CONSTRAINT_PNT_ON_LINE}\NormalTok{:}
      \KeywordTok{return} \NormalTok{(}\StringTok{"(Point on Line)"}\NormalTok{);       }
    \KeywordTok{case} \OtherTok{constrTypeClass}\NormalTok{.}\FunctionTok{ASM_CONSTRAINT_FIX}\NormalTok{:}
      \KeywordTok{return} \NormalTok{(}\StringTok{"(Fix)"}\NormalTok{);     }
    \KeywordTok{case} \OtherTok{constrTypeClass}\NormalTok{.}\FunctionTok{ASM_CONSTRAINT_AUTO}\NormalTok{:}
      \KeywordTok{return} \NormalTok{(}\StringTok{"(Auto)"}\NormalTok{);        }
    \KeywordTok{default}\NormalTok{:}
      \KeywordTok{return} \NormalTok{(}\StringTok{"(Unrecognized Type)"}\NormalTok{);       }
    \NormalTok{\}}
\NormalTok{\}}

\CommentTok{/* }
\CommentTok{   The following example demonstrates how to assemble a component into an}
\CommentTok{   assembly, and how to constrain the component by aligning datum planes.}
\CommentTok{   If the complete set of datum planes is not found, the function will show}
\CommentTok{   the component constraint dialog to the user to allow them to adjust the}
\CommentTok{   placement as they wish.}
\CommentTok{*/}

\CommentTok{/*=====================================================================*\textbackslash{}}
\NormalTok{FUNCTION: UserAssembleByDatums}
\NormalTok{PURPOSE:  Assemble a component by aligning named }\OtherTok{datums}\NormalTok{.         }
\NormalTok{\textbackslash{}*=====================================================================*}\OtherTok{/}
\KeywordTok{function} \FunctionTok{assembleByDatums} \NormalTok{(componentFilename }\CommentTok{/* string as ?????.??? */}\NormalTok{)}
\NormalTok{\{}
  \KeywordTok{if} \NormalTok{(!}\FunctionTok{pfcIsWindows}\NormalTok{())}
    \OtherTok{netscape}\NormalTok{.}\OtherTok{security}\NormalTok{.}\OtherTok{PrivilegeManager}\NormalTok{.}\FunctionTok{enablePrivilege}\NormalTok{(}\StringTok{"UniversalXPConnect"}\NormalTok{);}
  \KeywordTok{var} \NormalTok{interactFlag = }\KeywordTok{false}\NormalTok{;}
  \KeywordTok{var} \NormalTok{identityMatrix = }\FunctionTok{pfcCreate} \NormalTok{(}\StringTok{"pfcMatrix3D"}\NormalTok{);}
  \KeywordTok{for} \NormalTok{(}\KeywordTok{var} \NormalTok{x = }\DecValTok{0}\NormalTok{; x < }\DecValTok{4}\NormalTok{; x++)}
    \KeywordTok{for} \NormalTok{(}\KeywordTok{var} \NormalTok{y = }\DecValTok{0}\NormalTok{; y < }\DecValTok{4}\NormalTok{; y++)}
      \NormalTok{\{}
    \KeywordTok{if} \NormalTok{(x == y)}
      \OtherTok{identityMatrix}\NormalTok{.}\FunctionTok{Set} \NormalTok{(x, y, }\FloatTok{1.0}\NormalTok{);}
    \KeywordTok{else}
      \OtherTok{identityMatrix}\NormalTok{.}\FunctionTok{Set} \NormalTok{(x, y, }\FloatTok{0.0}\NormalTok{);}
      \NormalTok{\}}
  \KeywordTok{var} \NormalTok{transf = }\FunctionTok{pfcCreate} \NormalTok{(}\StringTok{"pfcTransform3D"}\NormalTok{).}\FunctionTok{Create} \NormalTok{(identityMatrix);}
\CommentTok{/*-----------------------------------------------------------------*\textbackslash{}}
  \NormalTok{Get the current assembly}
\NormalTok{\textbackslash{}*-----------------------------------------------------------------*}\OtherTok{/  }
\OtherTok{  var session = pfcGetProESession }\FloatTok{()}\OtherTok{;}
\OtherTok{  var model = session.CurrentModel;}
\OtherTok{  if }\FloatTok{(}\OtherTok{model == void null }\FloatTok{||}\OtherTok{ model.Type != pfcCreate }\FloatTok{(}\OtherTok{"pfcModelType"}\FloatTok{)}\OtherTok{.MDL_ASSEMBLY}\FloatTok{)}
\OtherTok{    throw new Error }\FloatTok{(}\OtherTok{0, "Current model is not an assembly."}\FloatTok{)}\OtherTok{;}
\OtherTok{  }
\OtherTok{  var assembly = model;}
\OtherTok{  }
\OtherTok{  var descr = }
\OtherTok{    pfcCreate }\FloatTok{(}\OtherTok{"pfcModelDescriptor"}\FloatTok{)}\OtherTok{.CreateFromFileName }\FloatTok{(}\OtherTok{componentFilename}\FloatTok{)}\OtherTok{;}
\OtherTok{  var componentModel = session.GetModelFromDescr }\FloatTok{(}\OtherTok{descr}\FloatTok{)}\OtherTok{;}
\OtherTok{  }
\OtherTok{  if }\FloatTok{(}\OtherTok{componentModel == void null}\FloatTok{)}
\OtherTok{    \{}
\OtherTok{      componentModel = session.RetrieveModel }\FloatTok{(}\OtherTok{descr}\FloatTok{)}\OtherTok{;}
\OtherTok{    \}}
\OtherTok{  }
\OtherTok{/}\NormalTok{*-----------------------------------------------------------------*\textbackslash{}}
  \NormalTok{Set up the arrays of datum names}
\NormalTok{\textbackslash{}*-----------------------------------------------------------------*}\OtherTok{/}
  \KeywordTok{var} \NormalTok{asmDatums = }\KeywordTok{new} \FunctionTok{Array} \NormalTok{(}\StringTok{"ASM_D_FRONT"}\NormalTok{, }\StringTok{"ASM_D_TOP"}\NormalTok{, }\StringTok{"ASM_D_RIGHT"}\NormalTok{);}
  \KeywordTok{var} \NormalTok{compDatums = }\KeywordTok{new} \FunctionTok{Array} \NormalTok{(}\StringTok{"COMP_D_FRONT"}\NormalTok{, }
                  \StringTok{"COMP_D_TOP"}\NormalTok{, }
                  \StringTok{"COMP_D_RIGHT"}\NormalTok{);}
      
\CommentTok{/*-----------------------------------------------------------------*\textbackslash{}}
  \NormalTok{Package the component initially}
\NormalTok{\textbackslash{}*-----------------------------------------------------------------*}\OtherTok{/}
  \KeywordTok{var} \NormalTok{asmcomp = }\OtherTok{assembly}\NormalTok{.}\FunctionTok{AssembleComponent} \NormalTok{(componentModel,}
                        \NormalTok{transf);}
  
\CommentTok{/*-----------------------------------------------------------------*\textbackslash{}}
  \NormalTok{Prepare the constraints array}
\NormalTok{\textbackslash{}*-----------------------------------------------------------------*}\OtherTok{/}
  \KeywordTok{var} \NormalTok{constrs = }\FunctionTok{pfcCreate} \NormalTok{(}\StringTok{"pfcComponentConstraints"}\NormalTok{);}
  
  \KeywordTok{for} \NormalTok{(}\KeywordTok{var} \NormalTok{i = }\DecValTok{0}\NormalTok{; i < }\DecValTok{3}\NormalTok{; i++)}
    \NormalTok{\{}
\CommentTok{/*-----------------------------------------------------------------*\textbackslash{}}
  \NormalTok{Find the assembly datum }
\NormalTok{\textbackslash{}*-----------------------------------------------------------------*}\OtherTok{/}
      \KeywordTok{var} \NormalTok{asmItem = }
    \OtherTok{assembly}\NormalTok{.}\FunctionTok{GetItemByName} \NormalTok{(}\FunctionTok{pfcCreate} \NormalTok{(}\StringTok{"pfcModelItemType"}\NormalTok{).}\FunctionTok{ITEM_SURFACE}\NormalTok{, }
                \NormalTok{asmDatums [i]);}
      
      \KeywordTok{if} \NormalTok{(asmItem == }\KeywordTok{void} \KeywordTok{null}\NormalTok{) }
    \NormalTok{\{}
      \NormalTok{interactFlag = }\KeywordTok{true}\NormalTok{;}
      \KeywordTok{continue}\NormalTok{;}
    \NormalTok{\}}
      
\CommentTok{/*-----------------------------------------------------------------*\textbackslash{}}
  \NormalTok{Find the component datum}
\NormalTok{\textbackslash{}*-----------------------------------------------------------------*}\OtherTok{/}
      \KeywordTok{var} \NormalTok{compItem = }
    \OtherTok{componentModel}\NormalTok{.}\FunctionTok{GetItemByName} \NormalTok{(}\FunctionTok{pfcCreate} \NormalTok{(}\StringTok{"pfcModelItemType"}\NormalTok{).}\FunctionTok{ITEM_SURFACE}\NormalTok{, }
                      \NormalTok{compDatums [i]);}
      
      \KeywordTok{if} \NormalTok{(compItem == }\KeywordTok{void} \KeywordTok{null}\NormalTok{) }
    \NormalTok{\{}
      \NormalTok{interactFlag = }\KeywordTok{true}\NormalTok{;}
      \KeywordTok{continue}\NormalTok{;}
    \NormalTok{\}}

\CommentTok{/*-----------------------------------------------------------------*\textbackslash{}}
  \NormalTok{For the assembly reference, initialize a component }\OtherTok{path}\NormalTok{.}
  \NormalTok{This is necessary even }\KeywordTok{if} \NormalTok{the reference geometry is }\KeywordTok{in} \NormalTok{the }\OtherTok{assembly}\NormalTok{.}
\NormalTok{\textbackslash{}*-----------------------------------------------------------------*}\OtherTok{/}
      \KeywordTok{var} \NormalTok{ids = }\FunctionTok{pfcCreate} \NormalTok{(}\StringTok{"intseq"}\NormalTok{);}
      
      \KeywordTok{var} \NormalTok{path = }\FunctionTok{pfcCreate} \NormalTok{(}\StringTok{"MpfcAssembly"}\NormalTok{).}\FunctionTok{CreateComponentPath} \NormalTok{(assembly,}
                                 \NormalTok{ids);}
      
\CommentTok{/*-----------------------------------------------------------------*\textbackslash{}}
  \NormalTok{Allocate the references}
\NormalTok{\textbackslash{}*-----------------------------------------------------------------*}\OtherTok{/}
      \KeywordTok{var} \NormalTok{MpfcSelect = }\FunctionTok{pfcCreate} \NormalTok{(}\StringTok{"MpfcSelect"}\NormalTok{);}
      \KeywordTok{var} \NormalTok{asmSel = }\OtherTok{MpfcSelect}\NormalTok{.}\FunctionTok{CreateModelItemSelection} \NormalTok{(asmItem, path);}
      \KeywordTok{var} \NormalTok{compSel = }\OtherTok{MpfcSelect}\NormalTok{.}\FunctionTok{CreateModelItemSelection} \NormalTok{(compItem, }\KeywordTok{void} \KeywordTok{null}\NormalTok{);}
      
\CommentTok{/*-----------------------------------------------------------------*\textbackslash{}}
  \NormalTok{Allocate and fill the }\OtherTok{constraint}\NormalTok{.}
\NormalTok{\textbackslash{}*-----------------------------------------------------------------*}\OtherTok{/}
      \KeywordTok{var} \NormalTok{constr = }\FunctionTok{pfcCreate} \NormalTok{(}\StringTok{"pfcComponentConstraint"}\NormalTok{).}\FunctionTok{Create} \NormalTok{(}
                                \FunctionTok{pfcCreate} \NormalTok{(}\StringTok{"pfcComponentConstraintType"}\NormalTok{).}\FunctionTok{ASM_CONSTRAINT_ALIGN}\NormalTok{);}
      
      \OtherTok{constr}\NormalTok{.}\FunctionTok{AssemblyReference}  \NormalTok{= asmSel;}
      \OtherTok{constr}\NormalTok{.}\FunctionTok{ComponentReference} \NormalTok{= compSel;}
      
      \OtherTok{constr}\NormalTok{.}\FunctionTok{Attributes} \NormalTok{= }\FunctionTok{pfcCreate} \NormalTok{(}\StringTok{"pfcConstraintAttributes"}\NormalTok{).}\FunctionTok{Create} \NormalTok{(}\KeywordTok{false}\NormalTok{, }\KeywordTok{false}\NormalTok{);}
      
      \OtherTok{constrs}\NormalTok{.}\FunctionTok{Append} \NormalTok{(constr);}
    \NormalTok{\}}

\CommentTok{/*-----------------------------------------------------------------*\textbackslash{}}
  \NormalTok{Set the assembly component constraints and regenerate the }\OtherTok{assembly}\NormalTok{.}
\NormalTok{\textbackslash{}*-----------------------------------------------------------------*}\OtherTok{/}
  \OtherTok{asmcomp}\NormalTok{.}\FunctionTok{SetConstraints} \NormalTok{(constrs, }\KeywordTok{void} \KeywordTok{null}\NormalTok{);}
  
  \OtherTok{assembly}\NormalTok{.}\FunctionTok{Regenerate} \NormalTok{(}\KeywordTok{void} \KeywordTok{null}\NormalTok{);}
  
  \OtherTok{session}\NormalTok{.}\FunctionTok{GetModelWindow} \NormalTok{(assembly).}\FunctionTok{Repaint}\NormalTok{();}

\CommentTok{/*-----------------------------------------------------------------*\textbackslash{}}
  \NormalTok{If any of the expect datums was not found, prompt the user to constrain}
  \NormalTok{the }\KeywordTok{new} \OtherTok{component}\NormalTok{.}
\NormalTok{\textbackslash{}*-----------------------------------------------------------------*}\OtherTok{/}
  \KeywordTok{if} \NormalTok{(interactFlag)}
    \NormalTok{\{}
      \FunctionTok{alert} \NormalTok{(}\StringTok{"Unable to locate all required datum references.  New component is packaged."}\NormalTok{);}
      \OtherTok{asmcomp}\NormalTok{.}\FunctionTok{RedefineThroughUI}\NormalTok{();}
    \NormalTok{\}}
\NormalTok{\}}

\end{Highlighting}
\end{Shaded}

\begin{Shaded}
\begin{Highlighting}[]
\CommentTok{/*}
\CommentTok{   HISTORY}

\CommentTok{14-NOV-02   J-03-38   $$1   JCN      Adapted from J-Link examples.}
\CommentTok{07-MAR-03   K-01-03   $$2   JCN      UNIX support}

\CommentTok{*/}

\CommentTok{/*}
\CommentTok{  The following example code shows a utility function that sets angular }
\CommentTok{  tolerances to a specified range.   For each angular dimension in the current }
\CommentTok{  model the function gets the dimension value and adds or subtracts the range }
\CommentTok{  to it to get the upper and lower limits.  The function then initializes a }
\CommentTok{  pfcDimTolLimits tolerance object and assigns it to the dimension.   The }
\CommentTok{  function displays each shown dimension.}
\CommentTok{*/}

\KeywordTok{function} \FunctionTok{setAngularToleranceToLimits} \NormalTok{(range }\CommentTok{/* number */}\NormalTok{)}
\NormalTok{\{}
  
  \KeywordTok{if} \NormalTok{(!}\FunctionTok{pfcIsWindows}\NormalTok{())}
    \OtherTok{netscape}\NormalTok{.}\OtherTok{security}\NormalTok{.}\OtherTok{PrivilegeManager}\NormalTok{.}\FunctionTok{enablePrivilege}\NormalTok{(}\StringTok{"UniversalXPConnect"}\NormalTok{);}
 \CommentTok{/*--------------------------------------------------------------------*\textbackslash{} }
\CommentTok{   Get the current solid model }
\CommentTok{ \textbackslash{}*--------------------------------------------------------------------*/}  
  \KeywordTok{var} \NormalTok{session = }\FunctionTok{pfcGetProESession}\NormalTok{();}
  \KeywordTok{var} \NormalTok{model = }\OtherTok{session}\NormalTok{.}\FunctionTok{CurrentModel}\NormalTok{;}
  
  \KeywordTok{if} \NormalTok{(model == }\KeywordTok{void} \KeywordTok{null} \NormalTok{|| (}\OtherTok{model}\NormalTok{.}\FunctionTok{Type} \NormalTok{!= }\FunctionTok{pfcCreate} \NormalTok{(}\StringTok{"pfcModelType"}\NormalTok{).}\FunctionTok{MDL_PART} \NormalTok{&& }
                 \OtherTok{model}\NormalTok{.}\FunctionTok{Type} \NormalTok{!= }\FunctionTok{pfcCreate} \NormalTok{(}\StringTok{"pfcModelType"}\NormalTok{).}\FunctionTok{MDL_ASSEMBLY}\NormalTok{))}
    \KeywordTok{throw} \KeywordTok{new} \FunctionTok{Error} \NormalTok{(}\DecValTok{0}\NormalTok{, }\StringTok{"Current model is not a part or assembly."}\NormalTok{);}
  
            
 \CommentTok{/*--------------------------------------------------------------------*\textbackslash{} }
\CommentTok{   Ensure that dimensions will be shown with tolerances }
\CommentTok{ \textbackslash{}*--------------------------------------------------------------------*/}  
  \OtherTok{session}\NormalTok{.}\FunctionTok{SetConfigOption} \NormalTok{(}\StringTok{"tol_display"}\NormalTok{, }\StringTok{"yes"}\NormalTok{);}
    
 \CommentTok{/*--------------------------------------------------------------------*\textbackslash{} }
\CommentTok{   List all model dimensions}
\CommentTok{ \textbackslash{}*--------------------------------------------------------------------*/}  
  \KeywordTok{var} \NormalTok{dimensions = }\OtherTok{model}\NormalTok{.}\FunctionTok{ListItems} \NormalTok{(}\FunctionTok{pfcCreate} \NormalTok{(}\StringTok{"pfcModelItemType"}\NormalTok{).}\FunctionTok{ITEM_DIMENSION}\NormalTok{);}
  
  \KeywordTok{for} \NormalTok{(}\KeywordTok{var} \NormalTok{i = }\DecValTok{0}\NormalTok{;  i < }\OtherTok{dimensions}\NormalTok{.}\FunctionTok{Count}\NormalTok{; i++)}
    \NormalTok{\{}
      \KeywordTok{var} \NormalTok{dimension = }\OtherTok{dimensions}\NormalTok{.}\FunctionTok{Item} \NormalTok{(i);}

 \CommentTok{/*--------------------------------------------------------------------*\textbackslash{} }
\CommentTok{   Check for angular dimensions}
\CommentTok{ \textbackslash{}*--------------------------------------------------------------------*/}  
      \KeywordTok{var} \NormalTok{dType = }\OtherTok{dimension}\NormalTok{.}\FunctionTok{DimType}\NormalTok{;  }\CommentTok{// from class pfcBaseDimension}
      
      
      \KeywordTok{if} \NormalTok{(dType == }\FunctionTok{pfcCreate} \NormalTok{(}\StringTok{"pfcDimensionType"}\NormalTok{).}\FunctionTok{DIM_ANGULAR}\NormalTok{)}
    \NormalTok{\{}
      
 \CommentTok{/*--------------------------------------------------------------------*\textbackslash{} }
\CommentTok{   Assign the limits tolerance }
\CommentTok{ \textbackslash{}*--------------------------------------------------------------------*/}  
      \KeywordTok{var} \NormalTok{dvalue = }\OtherTok{dimension}\NormalTok{.}\FunctionTok{DimValue}\NormalTok{;  }\CommentTok{//from class pfcBaseDimension}
      
      \KeywordTok{var} \NormalTok{upper = dvalue + range/}\FloatTok{2.0}\NormalTok{;}
      \KeywordTok{var} \NormalTok{lower = dvalue - range/}\FloatTok{2.0}\NormalTok{;}
      
      \NormalTok{limits =  }\FunctionTok{pfcCreate} \NormalTok{(}\StringTok{"pfcDimTolLimits"}\NormalTok{).}\FunctionTok{Create}\NormalTok{(upper, lower);}
      
      \OtherTok{dimension}\NormalTok{.}\FunctionTok{Tolerance} \NormalTok{= limits;  }\CommentTok{// from class pfcDimension}
        
 \CommentTok{/*--------------------------------------------------------------------*\textbackslash{} }
\CommentTok{   Display the modified dimension}
\CommentTok{ \textbackslash{}*--------------------------------------------------------------------*/}  
      \KeywordTok{var} \NormalTok{showInstrs = }
        \FunctionTok{pfcCreate} \NormalTok{(}\StringTok{"pfcComponentDimensionShowInstructions"}\NormalTok{).}\FunctionTok{Create} \NormalTok{(}\KeywordTok{void} \KeywordTok{null}\NormalTok{);}
      
      \OtherTok{dimension}\NormalTok{.}\FunctionTok{Show} \NormalTok{(showInstrs);}
      
    \NormalTok{\}}
    \NormalTok{\}}
\NormalTok{\}}
\end{Highlighting}
\end{Shaded}

\begin{Shaded}
\begin{Highlighting}[]
\CommentTok{/*}

\CommentTok{  HISTORY:}
\CommentTok{ }
\CommentTok{ 14-NOV-02   J-03-38   $$1   JCN      Adapted from J-Link examples.}
\CommentTok{ 07-MAR-03   K-01-03   $$2   JCN      UNIX support}
\CommentTok{ 02-Jun-03   K-01-08   $$3   JCN      Fix comparison for mouse button}
\CommentTok{*/}

\CommentTok{/* }
\CommentTok{   This example demonstrates the use of mouse tracking methods to draw graphics}
\CommentTok{   on the screen.  The static method DrawRubberbandLine prompts the user to}
\CommentTok{   pick a screen point.  The example uses 'complement mode' to cause the line }
\CommentTok{   to display and erase as the user moves the mouse around the window.  }
\CommentTok{   }
\CommentTok{   NOTE:  This example uses the method transformPosition to convert the }
\CommentTok{   coordinates into the 3D coordinate system of a solid model, if one is }
\CommentTok{   displayed.}
\CommentTok{*/}
\KeywordTok{function} \FunctionTok{drawRubberbandLine} \NormalTok{() }
\NormalTok{\{}
  
  \KeywordTok{if} \NormalTok{(!}\FunctionTok{pfcIsWindows}\NormalTok{())}
    \OtherTok{netscape}\NormalTok{.}\OtherTok{security}\NormalTok{.}\OtherTok{PrivilegeManager}\NormalTok{.}\FunctionTok{enablePrivilege}\NormalTok{(}\StringTok{"UniversalXPConnect"}\NormalTok{);   }
\CommentTok{/*--------------------------------------------------------------------*\textbackslash{} }
\CommentTok{  Select the first end of the rubber band line.  Expect the user to pick with }
\CommentTok{  the left mouse button.}
\CommentTok{\textbackslash{}*--------------------------------------------------------------------*/}      
  \KeywordTok{var} \NormalTok{session = }\FunctionTok{pfcGetProESession}\NormalTok{();}
  
  \OtherTok{session}\NormalTok{.}\OtherTok{CurrentWindow}\NormalTok{.}\FunctionTok{SetBrowserSize} \NormalTok{(}\FloatTok{0.0}\NormalTok{);}
  
  \NormalTok{mouse = }\OtherTok{session}\NormalTok{.}\FunctionTok{UIGetNextMousePick} \NormalTok{(}\FunctionTok{pfcCreate} \NormalTok{(}\StringTok{"pfcMouseButton"}\NormalTok{).}\FunctionTok{MOUSE_BTN_LEFT}\NormalTok{);}
  
\CommentTok{/*--------------------------------------------------------------------*\textbackslash{} }
\CommentTok{  Transform screen point -> model location, if necessary}
\CommentTok{\textbackslash{}*--------------------------------------------------------------------*/}  
  \KeywordTok{var} \NormalTok{firstPos = }\FunctionTok{transformPosition} \NormalTok{(session, }\OtherTok{mouse}\NormalTok{.}\FunctionTok{Position}\NormalTok{);}
  
\CommentTok{/*--------------------------------------------------------------------*\textbackslash{} }
\CommentTok{  Set graphics mode to complement, so that graphics erase after use. }
\CommentTok{\textbackslash{}*--------------------------------------------------------------------*/} 
  \KeywordTok{var} \NormalTok{currentMode = }\OtherTok{session}\NormalTok{.}\FunctionTok{CurrentGraphicsMode}\NormalTok{;}
  \OtherTok{session}\NormalTok{.}\FunctionTok{CurrentGraphicsMode} \NormalTok{= }\FunctionTok{pfcCreate} \NormalTok{(}\StringTok{"pfcGraphicsMode"}\NormalTok{).}\FunctionTok{DRAW_GRAPHICS_COMPLEMENT}\NormalTok{;}
  
\CommentTok{/*--------------------------------------------------------------------*\textbackslash{} }
\CommentTok{  Get current mouse position.}
\CommentTok{\textbackslash{}*--------------------------------------------------------------------*/} 
  \KeywordTok{var} \NormalTok{mouse = }\OtherTok{session}\NormalTok{.}\FunctionTok{UIGetCurrentMouseStatus} \NormalTok{(}\KeywordTok{false}\NormalTok{);}
  \KeywordTok{while} \NormalTok{(}\OtherTok{mouse}\NormalTok{.}\FunctionTok{SelectedButton} \NormalTok{== }\FunctionTok{pfcCreate} \NormalTok{(}\StringTok{"pfcMouseButton"}\NormalTok{).}\FunctionTok{MouseButton_nil}\NormalTok{)}
    \NormalTok{\{  }
      \OtherTok{session}\NormalTok{.}\FunctionTok{SetPenPosition} \NormalTok{(firstPos);}
      \KeywordTok{var} \NormalTok{secondPos = }\FunctionTok{transformPosition} \NormalTok{(session, }\OtherTok{mouse}\NormalTok{.}\FunctionTok{Position}\NormalTok{);}

\CommentTok{/*--------------------------------------------------------------------*\textbackslash{} }
\CommentTok{  Draw rubberband line }
\CommentTok{\textbackslash{}*--------------------------------------------------------------------*/} 
      \OtherTok{session}\NormalTok{.}\FunctionTok{DrawLine} \NormalTok{(secondPos);}
      
      \NormalTok{mouse = }\OtherTok{session}\NormalTok{.}\FunctionTok{UIGetCurrentMouseStatus} \NormalTok{(}\KeywordTok{false}\NormalTok{);}
      
\CommentTok{/*--------------------------------------------------------------------*\textbackslash{} }
\CommentTok{  Erase previously drawn line}
\CommentTok{\textbackslash{}*--------------------------------------------------------------------*/} 
      \OtherTok{session}\NormalTok{.}\FunctionTok{SetPenPosition} \NormalTok{(firstPos);}
      \OtherTok{session}\NormalTok{.}\FunctionTok{DrawLine} \NormalTok{(secondPos);    }
    \NormalTok{\}}
  
  \OtherTok{session}\NormalTok{.}\FunctionTok{CurrentGraphicsMode} \NormalTok{=  currentMode;}
  
  \KeywordTok{return}\NormalTok{;}
\NormalTok{\}}


\CommentTok{/* This method transforms the 2D screen coordinates into }
\CommentTok{   3D model coordinates - if necessary. */}
\KeywordTok{function} \FunctionTok{transformPosition} \NormalTok{(s }\CommentTok{/* pfcSession */}\NormalTok{, inPnt }\CommentTok{/* pfcPoint3D */}\NormalTok{)}
\NormalTok{\{}
  \KeywordTok{var} \NormalTok{mdl = }\OtherTok{s}\NormalTok{.}\FunctionTok{CurrentModel}\NormalTok{;}
  
\CommentTok{/*--------------------------------------------------------------------*\textbackslash{} }
\CommentTok{  Skip transform if not in 3D model }
\CommentTok{\textbackslash{}*--------------------------------------------------------------------*/} 
  \KeywordTok{if} \NormalTok{(mdl == }\KeywordTok{void} \KeywordTok{null}\NormalTok{)}
    \KeywordTok{return} \NormalTok{inPnt;}
  
  \KeywordTok{var} \NormalTok{type = }\OtherTok{mdl}\NormalTok{.}\FunctionTok{Type}\NormalTok{;}
  \KeywordTok{var} \NormalTok{modelTypeClass =  }\FunctionTok{pfcCreate} \NormalTok{(}\StringTok{"pfcModelType"}\NormalTok{);}
  \KeywordTok{var} \NormalTok{isSolid = ((type == }\OtherTok{modelTypeClass}\NormalTok{.}\FunctionTok{MDL_PART}\NormalTok{) || }
         \NormalTok{(type == }\OtherTok{modelTypeClass}\NormalTok{.}\FunctionTok{MDL_ASSEMBLY}\NormalTok{) || }
         \NormalTok{(type == }\OtherTok{modelTypeClass}\NormalTok{.}\FunctionTok{MDL_MFG}\NormalTok{));}
  \KeywordTok{if} \NormalTok{(!isSolid)}
    \KeywordTok{return} \NormalTok{inPnt;}
  
\CommentTok{/*--------------------------------------------------------------------*\textbackslash{} }
\CommentTok{  Get current view's orientation and invert it}
\CommentTok{\textbackslash{}*--------------------------------------------------------------------*/} 
  \KeywordTok{var} \NormalTok{currView = }\OtherTok{mdl}\NormalTok{.}\FunctionTok{GetCurrentView}\NormalTok{();}
  \KeywordTok{var} \NormalTok{invOrient = }\OtherTok{currView}\NormalTok{.}\FunctionTok{Transform}\NormalTok{;}
  \OtherTok{invOrient}\NormalTok{.}\FunctionTok{Invert}\NormalTok{();}
  
\CommentTok{/*--------------------------------------------------------------------*\textbackslash{} }
\CommentTok{  Get the point in the model csys}
\CommentTok{\textbackslash{}*--------------------------------------------------------------------*/} 
  \KeywordTok{var} \NormalTok{outPnt = }\OtherTok{invOrient}\NormalTok{.}\FunctionTok{TransformPoint} \NormalTok{(inPnt);}
  
  \KeywordTok{return} \NormalTok{outPnt;}
\NormalTok{\}}
\end{Highlighting}
\end{Shaded}

\begin{Shaded}
\begin{Highlighting}[]
\CommentTok{/*  }
\CommentTok{   HISTORY}

\CommentTok{23-JUL-07   L-01-35   $$1  SNV     Created and submitted}

\CommentTok{*/}

\CommentTok{/*====================================================================*\textbackslash{}}
 \NormalTok{FUNCTION : }\FunctionTok{placeDetailSymbol}\NormalTok{() }
 \NormalTok{PURPOSE  : This }\KeywordTok{function} \NormalTok{creates a free instance of a symbol }
            \NormalTok{definition }\KeywordTok{with} \NormalTok{drawing unit heights, variable text and }
            \OtherTok{groups}\NormalTok{. }\FunctionTok{A} \FunctionTok{symbol} \FunctionTok{is} \FunctionTok{placed} \FunctionTok{with} \FunctionTok{no} \FunctionTok{leaders} \FunctionTok{at} \FunctionTok{a} \FunctionTok{specified} 
            \OtherTok{location}\NormalTok{.  }
\NormalTok{\textbackslash{}*====================================================================*}\OtherTok{/}
\KeywordTok{function} \FunctionTok{placeDetailSymbol}\NormalTok{(groupName , variableText, symHeight)}
\NormalTok{\{}
  \KeywordTok{if} \NormalTok{(!}\FunctionTok{pfcIsWindows}\NormalTok{())}
    \OtherTok{netscape}\NormalTok{.}\OtherTok{security}\NormalTok{.}\OtherTok{PrivilegeManager}\NormalTok{.}\FunctionTok{enablePrivilege}\NormalTok{(}\StringTok{"UniversalXPConnect"}\NormalTok{); }

 \CommentTok{/*--------------------------------------------------------------------*\textbackslash{} }
\CommentTok{   Get the current drawing}
\CommentTok{ \textbackslash{}*--------------------------------------------------------------------*/}
  \KeywordTok{var} \NormalTok{session = }\FunctionTok{pfcGetProESession} \NormalTok{();}
  \KeywordTok{var} \NormalTok{drawing = }\OtherTok{session}\NormalTok{.}\FunctionTok{CurrentModel}\NormalTok{;}

  
  \KeywordTok{if} \NormalTok{(}\OtherTok{drawing}\NormalTok{.}\FunctionTok{Type} \NormalTok{!= }\FunctionTok{pfcCreate} \NormalTok{(}\StringTok{"pfcModelType"}\NormalTok{).}\FunctionTok{MDL_DRAWING}\NormalTok{)}
    \KeywordTok{throw} \KeywordTok{new} \FunctionTok{Error} \NormalTok{(}\DecValTok{0}\NormalTok{, }\StringTok{"Current model is not a drawing"}\NormalTok{);}
      
\CommentTok{/*--------------------------------------------------------------------*\textbackslash{}  }
\CommentTok{  Retrieve the symbol definition from the system}
\CommentTok{\textbackslash{}*--------------------------------------------------------------------*/}    
  \KeywordTok{var} \NormalTok{symDef = }\OtherTok{drawing}\NormalTok{.}\FunctionTok{RetrieveSymbolDefinition} \NormalTok{(}\StringTok{"detail_symbol_example"}\NormalTok{, }
                         \StringTok{"./"}\NormalTok{, }\KeywordTok{void} \KeywordTok{null}\NormalTok{, }\KeywordTok{void} \KeywordTok{null}\NormalTok{);}
  
\CommentTok{/*--------------------------------------------------------------------*\textbackslash{} }
\CommentTok{  Select the locations for the symbol}
\CommentTok{\textbackslash{}*--------------------------------------------------------------------*/}
  \KeywordTok{var} \NormalTok{browserSize = }\OtherTok{session}\NormalTok{.}\OtherTok{CurrentWindow}\NormalTok{.}\FunctionTok{GetBrowserSize}\NormalTok{();}
  \OtherTok{session}\NormalTok{.}\OtherTok{CurrentWindow}\NormalTok{.}\FunctionTok{SetBrowserSize} \NormalTok{(}\FloatTok{0.0}\NormalTok{);}
  
  \KeywordTok{var} \NormalTok{stop = }\KeywordTok{false}\NormalTok{;}
  \KeywordTok{var} \NormalTok{point}
  \KeywordTok{while} \NormalTok{(!stop)}
    \NormalTok{\{}
        \KeywordTok{var} \NormalTok{mouse = }
        \OtherTok{session}\NormalTok{.}\FunctionTok{UIGetNextMousePick} \NormalTok{(}\FunctionTok{pfcCreate} \NormalTok{(}\StringTok{"pfcMouseButton"}\NormalTok{).}\FunctionTok{MouseButton_nil}\NormalTok{);}
          
        \KeywordTok{if} \NormalTok{(}\OtherTok{mouse}\NormalTok{.}\FunctionTok{SelectedButton} \NormalTok{== }\FunctionTok{pfcCreate} \NormalTok{(}\StringTok{"pfcMouseButton"}\NormalTok{).}\FunctionTok{MOUSE_BTN_LEFT}\NormalTok{)}
        \NormalTok{\{}
          \NormalTok{point = }\OtherTok{mouse}\NormalTok{.}\FunctionTok{Position}\NormalTok{;}
        \NormalTok{\}}
        \KeywordTok{else}
            \NormalTok{stop = }\KeywordTok{true}\NormalTok{;}
    \NormalTok{\}}
  
    \OtherTok{session}\NormalTok{.}\OtherTok{CurrentWindow}\NormalTok{.}\FunctionTok{SetBrowserSize} \NormalTok{(browserSize);}
 
\CommentTok{/*--------------------------------------------------------------------*\textbackslash{}  }
\CommentTok{  Allocate the symbol instance instructions }
\CommentTok{\textbackslash{}*--------------------------------------------------------------------*/} 
  \KeywordTok{var} \NormalTok{instrs = }
    \FunctionTok{pfcCreate} \NormalTok{(}\StringTok{"pfcDetailSymbolInstInstructions"}\NormalTok{).}\FunctionTok{Create} \NormalTok{(symDef); }
    
\CommentTok{/*--------------------------------------------------------------------*\textbackslash{}    }
\CommentTok{ Set the symbol height in drawing units }
\CommentTok{\textbackslash{}*--------------------------------------------------------------------*/} 
  \KeywordTok{if} \NormalTok{(symHeight > }\DecValTok{0}\NormalTok{)}
  \NormalTok{\{}
      \OtherTok{instrs}\NormalTok{.}\FunctionTok{ScaledHeight} \NormalTok{= symHeight;}
  \NormalTok{\}    }

\CommentTok{/*--------------------------------------------------------------------*\textbackslash{}    }
\CommentTok{ Set text to the variable text in the symbol. This will be displayed }
\CommentTok{ instead of the text defined when creating the symbol}
\CommentTok{\textbackslash{}*--------------------------------------------------------------------*/}        
  \KeywordTok{if} \NormalTok{(variableText != }\KeywordTok{void} \KeywordTok{null}\NormalTok{)}
  \NormalTok{\{}
      \KeywordTok{var} \NormalTok{varText = }\FunctionTok{pfcCreate} \NormalTok{(}\StringTok{"pfcDetailVariantText"}\NormalTok{).}\FunctionTok{Create}\NormalTok{(}\StringTok{"VAR_TEXT"} \NormalTok{, variableText);}
      \KeywordTok{var} \NormalTok{varTexts = }\FunctionTok{pfcCreate}\NormalTok{(}\StringTok{"pfcDetailVariantTexts"}\NormalTok{);}
      \OtherTok{varTexts}\NormalTok{.}\FunctionTok{Append}\NormalTok{(varText);}
      
      \OtherTok{instrs}\NormalTok{.}\FunctionTok{TextValues} \NormalTok{= varTexts;}
  \NormalTok{\}}

\CommentTok{/*--------------------------------------------------------------------*\textbackslash{}    }
\CommentTok{ Display the groups in symbol depending on group name}
\CommentTok{\textbackslash{}*--------------------------------------------------------------------*/}        
  \KeywordTok{if} \NormalTok{(groupName == }\StringTok{"ALL"}\NormalTok{)}
    \OtherTok{instrs}\NormalTok{.}\FunctionTok{SetGroups}\NormalTok{(}\FunctionTok{pfcCreate}\NormalTok{(}\StringTok{"pfcDetailSymbolGroupOption"}\NormalTok{).}\FunctionTok{DETAIL_SYMBOL_GROUP_ALL} \NormalTok{, }\KeywordTok{null}\NormalTok{);}
  \KeywordTok{else} \KeywordTok{if} \NormalTok{(groupName == }\StringTok{"NONE"}\NormalTok{)}
    \OtherTok{instrs}\NormalTok{.}\FunctionTok{SetGroups}\NormalTok{(}\FunctionTok{pfcCreate}\NormalTok{(}\StringTok{"pfcDetailSymbolGroupOption"}\NormalTok{).}\FunctionTok{DETAIL_SYMBOL_GROUP_NONE} \NormalTok{, }\KeywordTok{null}\NormalTok{);}
  \KeywordTok{else}
  \NormalTok{\{}
    \KeywordTok{var} \NormalTok{allGroups = }\OtherTok{instrs}\NormalTok{.}\OtherTok{SymbolDef}\NormalTok{.}\FunctionTok{ListSubgroups}\NormalTok{();}
    \NormalTok{group = }\FunctionTok{getGroup}\NormalTok{(allGroups , groupName );}
    \KeywordTok{if} \NormalTok{(group != }\KeywordTok{void} \KeywordTok{null}\NormalTok{)}
    \NormalTok{\{}
        \NormalTok{groups = }\FunctionTok{pfcCreate}\NormalTok{(}\StringTok{"pfcDetailSymbolGroups"}\NormalTok{);}
        \OtherTok{groups}\NormalTok{.}\FunctionTok{Append}\NormalTok{(group);}
        \OtherTok{instrs}\NormalTok{.}\FunctionTok{SetGroups} \NormalTok{(}\FunctionTok{pfcCreate}\NormalTok{(}\StringTok{"pfcDetailSymbolGroupOption"}\NormalTok{).}\FunctionTok{DETAIL_SYMBOL_GROUP_CUSTOM} \NormalTok{, groups);}
    \NormalTok{\}           }
  \NormalTok{\}}
                
\CommentTok{/*--------------------------------------------------------------------*\textbackslash{}    }
\CommentTok{  Set the attachment structure}
\CommentTok{\textbackslash{}*--------------------------------------------------------------------*/} 
  \KeywordTok{var} \NormalTok{position = }\FunctionTok{pfcCreate} \NormalTok{(}\StringTok{"pfcFreeAttachment"}\NormalTok{).}\FunctionTok{Create} \NormalTok{(point);}
  \KeywordTok{var} \NormalTok{allAttachments = }\FunctionTok{pfcCreate} \NormalTok{(}\StringTok{"pfcDetailLeaders"}\NormalTok{).}\FunctionTok{Create} \NormalTok{();    }
  \OtherTok{allAttachments}\NormalTok{.}\FunctionTok{ItemAttachment} \NormalTok{= position;}
  
  \OtherTok{instrs}\NormalTok{.}\FunctionTok{InstAttachment} \NormalTok{= allAttachments;}

\CommentTok{/*--------------------------------------------------------------------*\textbackslash{}    }
\CommentTok{  Create and display the symbol }
\CommentTok{\textbackslash{}*--------------------------------------------------------------------*/}  
  \KeywordTok{var} \NormalTok{symInst = }\OtherTok{drawing}\NormalTok{.}\FunctionTok{CreateDetailItem} \NormalTok{(instrs);}
  \OtherTok{symInst}\NormalTok{.}\FunctionTok{Show}\NormalTok{();}
       
\NormalTok{\}}

\CommentTok{/*====================================================================*\textbackslash{}}
 \NormalTok{FUNCTION : }\FunctionTok{getGroup}\NormalTok{() }
 \NormalTok{PURPOSE  : Return the specific group depending on the group }\OtherTok{name}\NormalTok{. }
\NormalTok{\textbackslash{}*====================================================================*}\OtherTok{/    }
\OtherTok{function getGroup}\FloatTok{(}\OtherTok{groups, groupName}\FloatTok{)}
\OtherTok{\{}
\OtherTok{    var group;}
\OtherTok{    var groupInstrs;}
\OtherTok{    }
\OtherTok{    if }\FloatTok{(}\OtherTok{groups.Count <=0 }\FloatTok{)}
\OtherTok{    \{}
\OtherTok{        return null;}
\OtherTok{    \}}
\OtherTok{    }
\OtherTok{/}\NormalTok{*--------------------------------------------------------------------*\textbackslash{}    }
   \NormalTok{Loop through all the groups }\KeywordTok{in} \NormalTok{the symbol and }\KeywordTok{return} \NormalTok{the group }\KeywordTok{with} 
   \NormalTok{the selected name}
\NormalTok{\textbackslash{}*--------------------------------------------------------------------*}\OtherTok{/  }
\OtherTok{    for}\FloatTok{(}\OtherTok{var i=0;i<groups.Count;i}\FloatTok{++)}
\OtherTok{    \{}
\OtherTok{        group = groups.Item}\FloatTok{(}\OtherTok{i}\FloatTok{)}\OtherTok{;}
\OtherTok{        groupInstrs = group.GetInstructions}\FloatTok{()}\OtherTok{;}
\OtherTok{        }
\OtherTok{        if }\FloatTok{(}\OtherTok{groupInstrs.Name == groupName}\FloatTok{)}
\OtherTok{            return group;           }
\OtherTok{    \}}
\OtherTok{    return null;        }
\OtherTok{\}}
\end{Highlighting}
\end{Shaded}

\begin{Shaded}
\begin{Highlighting}[]
\CommentTok{/*}
\CommentTok{   HISTORY}
\CommentTok{   }
\CommentTok{14-NOV-02   J-03-38   $$1   JCN      Adapted from J-Link examples.}
\CommentTok{07-MAR-03   K-01-03   $$2   JCN      UNIX support}
\CommentTok{08-JAN-08   L-01-42   $$3   SNV      Fix for SPR 1430116}
\CommentTok{*/}

\CommentTok{/*====================================================================*\textbackslash{}}
\NormalTok{FUNCTION: createDrawingFromTemplate}
\NormalTok{PURPOSE:  Create a }\KeywordTok{new} \NormalTok{drawing using a predefined }\OtherTok{template}\NormalTok{.}
\NormalTok{\textbackslash{}*====================================================================*}\OtherTok{/}
\KeywordTok{function} \FunctionTok{createDrawingFromTemplate} \NormalTok{(newDrawingName }\CommentTok{/* string */}\NormalTok{)}
\NormalTok{\{}
  \KeywordTok{if} \NormalTok{(!}\FunctionTok{pfcIsWindows}\NormalTok{())}
    \OtherTok{netscape}\NormalTok{.}\OtherTok{security}\NormalTok{.}\OtherTok{PrivilegeManager}\NormalTok{.}\FunctionTok{enablePrivilege}\NormalTok{(}\StringTok{"UniversalXPConnect"}\NormalTok{);   }
  
  \KeywordTok{var} \NormalTok{predefinedTemplate = }\StringTok{"c_drawing"}\NormalTok{;}
  
  \KeywordTok{if} \NormalTok{(newDrawingName == }\StringTok{""}\NormalTok{)}
    \NormalTok{\{}
      \KeywordTok{throw} \KeywordTok{new} \FunctionTok{Error} \NormalTok{(}\StringTok{"Please supply a drawing name.  Aborting..."}\NormalTok{);}
    \NormalTok{\}}
  
\CommentTok{/*------------------------------------------------------------------*\textbackslash{}}
  \NormalTok{Use the current model to create the }\OtherTok{drawing}\NormalTok{.}
\NormalTok{\textbackslash{}*------------------------------------------------------------------*}\OtherTok{/}
  \KeywordTok{var} \NormalTok{session = }\FunctionTok{pfcGetProESession} \NormalTok{();}
  \KeywordTok{var} \NormalTok{solid = }\OtherTok{session}\NormalTok{.}\FunctionTok{CurrentModel}\NormalTok{;}
  \NormalTok{modelTypeClass = }\FunctionTok{pfcCreate} \NormalTok{(}\StringTok{"pfcModelType"}\NormalTok{);}
  
  \KeywordTok{if} \NormalTok{(solid == }\KeywordTok{void} \KeywordTok{null} \NormalTok{|| (}\OtherTok{solid}\NormalTok{.}\FunctionTok{Type} \NormalTok{!= }\OtherTok{modelTypeClass}\NormalTok{.}\FunctionTok{MDL_PART} \NormalTok{&& }
                 \OtherTok{solid}\NormalTok{.}\FunctionTok{Type} \NormalTok{!= }\OtherTok{modelTypeClass}\NormalTok{.}\FunctionTok{MDL_ASSEMBLY}\NormalTok{))}
    \NormalTok{\{}
      \KeywordTok{throw} \KeywordTok{new} \FunctionTok{Error} \NormalTok{(}\StringTok{"Current model is not usable for new drawing.  Aborting..."}\NormalTok{);}
    \NormalTok{\}}
  
  \KeywordTok{var} \NormalTok{options = }\FunctionTok{pfcCreate} \NormalTok{(}\StringTok{"pfcDrawingCreateOptions"}\NormalTok{);}
  \OtherTok{options}\NormalTok{.}\FunctionTok{Append} \NormalTok{(}\FunctionTok{pfcCreate} \NormalTok{(}\StringTok{"pfcDrawingCreateOption"}\NormalTok{).}\FunctionTok{DRAWINGCREATE_DISPLAY_DRAWING}\NormalTok{);}
  
\CommentTok{/*------------------------------------------------------------------*\textbackslash{}}
  \NormalTok{Create the required }\OtherTok{drawing}\NormalTok{.}
\NormalTok{\textbackslash{}*------------------------------------------------------------------*}\OtherTok{/}
  \KeywordTok{var} \NormalTok{drw = }\OtherTok{session}\NormalTok{.}\FunctionTok{CreateDrawingFromTemplate}  \NormalTok{(newDrawingName, }
                        \NormalTok{predefinedTemplate,}
                        \OtherTok{solid}\NormalTok{.}\FunctionTok{Descr}\NormalTok{, options);}
\NormalTok{\}}


\CommentTok{/*====================================================================*\textbackslash{}}
\NormalTok{FUNCTION : }\FunctionTok{listSheets}\NormalTok{() }
\NormalTok{PURPOSE  : Command to list drawing sheet info }\KeywordTok{in} \NormalTok{an information window}
\NormalTok{\textbackslash{}*====================================================================*}\OtherTok{/}
\KeywordTok{function} \FunctionTok{listSheets}\NormalTok{()}
\NormalTok{\{ }
  \KeywordTok{if} \NormalTok{(!}\FunctionTok{pfcIsWindows}\NormalTok{())}
    \OtherTok{netscape}\NormalTok{.}\OtherTok{security}\NormalTok{.}\OtherTok{PrivilegeManager}\NormalTok{.}\FunctionTok{enablePrivilege}\NormalTok{(}\StringTok{"UniversalXPConnect"}\NormalTok{);  }

\CommentTok{/*--------------------------------------------------------------------*\textbackslash{}}
  \NormalTok{Open a browser window to contain the information to be displayed}
\NormalTok{\textbackslash{}*--------------------------------------------------------------------*}\OtherTok{/ }

\OtherTok{  var newWin = window.open }\FloatTok{(}\OtherTok{'', "_LS", "scrollbars"}\FloatTok{)}\OtherTok{;}
\OtherTok{  if }\FloatTok{(}\OtherTok{pfcIsWindows}\FloatTok{())}
\OtherTok{    \{}
\OtherTok{      newWin.resizeTo }\FloatTok{(}\OtherTok{300, screen.height/2}\FloatTok{.0}\NormalTok{);}
      \OtherTok{newWin}\NormalTok{.}\FunctionTok{moveTo} \NormalTok{(}\OtherTok{screen}\NormalTok{.}\FunctionTok{width}\DecValTok{-300}\NormalTok{, }\DecValTok{0}\NormalTok{);}
    \NormalTok{\}}
  \OtherTok{newWin}\NormalTok{.}\OtherTok{document}\NormalTok{.}\FunctionTok{writeln} \NormalTok{(}\StringTok{"<html><head></head><body>"}\NormalTok{);}
   
\CommentTok{/*--------------------------------------------------------------------*\textbackslash{} }
\CommentTok{  Get the current drawing}
\CommentTok{\textbackslash{}*--------------------------------------------------------------------*/}
  \KeywordTok{var} \NormalTok{session = }\FunctionTok{pfcGetProESession} \NormalTok{();}
  \KeywordTok{var} \NormalTok{drawing = }\OtherTok{session}\NormalTok{.}\FunctionTok{CurrentModel}\NormalTok{;}
  
  \KeywordTok{if} \NormalTok{(}\OtherTok{drawing}\NormalTok{.}\FunctionTok{Type} \NormalTok{!= }\FunctionTok{pfcCreate} \NormalTok{(}\StringTok{"pfcModelType"}\NormalTok{).}\FunctionTok{MDL_DRAWING}\NormalTok{)}
    \KeywordTok{throw} \KeywordTok{new} \FunctionTok{Error} \NormalTok{(}\DecValTok{0}\NormalTok{, }\StringTok{"Current model is not a drawing"}\NormalTok{);}
  
\CommentTok{/*--------------------------------------------------------------------*\textbackslash{} }
\CommentTok{  Get the number of sheets}
\CommentTok{\textbackslash{}*--------------------------------------------------------------------*/}
  \KeywordTok{var} \NormalTok{sheets = }\OtherTok{drawing}\NormalTok{.}\FunctionTok{NumberOfSheets}\NormalTok{;}
  
  \KeywordTok{for} \NormalTok{(i = }\DecValTok{1}\NormalTok{; i <= sheets; i++)}
    \NormalTok{\{}
\CommentTok{/*--------------------------------------------------------------------*\textbackslash{}}
  \NormalTok{Get the drawing sheet size }\OtherTok{etc}\NormalTok{.}
\NormalTok{\textbackslash{}*--------------------------------------------------------------------*}\OtherTok{/}
      \KeywordTok{var} \NormalTok{info = }\OtherTok{drawing}\NormalTok{.}\FunctionTok{GetSheetData} \NormalTok{(i);}
      
      \KeywordTok{var} \NormalTok{format = }\OtherTok{drawing}\NormalTok{.}\FunctionTok{GetSheetFormat} \NormalTok{(i);}
      
\CommentTok{/*--------------------------------------------------------------------*\textbackslash{} }
\CommentTok{  Print the information to the window}
\CommentTok{\textbackslash{}*--------------------------------------------------------------------*/} 

      \KeywordTok{var} \NormalTok{unit = }\StringTok{"unknown"}\NormalTok{;}
      \KeywordTok{var} \NormalTok{lengthUnitClass = }\FunctionTok{pfcCreate} \NormalTok{(}\StringTok{"pfcLengthUnitType"}\NormalTok{);}
      
      \KeywordTok{switch} \NormalTok{(}\OtherTok{info}\NormalTok{.}\OtherTok{Units}\NormalTok{.}\FunctionTok{GetType}\NormalTok{())}
    \NormalTok{\{}
    \KeywordTok{case} \OtherTok{lengthUnitClass}\NormalTok{.}\FunctionTok{LENGTHUNIT_INCH}\NormalTok{:}
      \NormalTok{unit = }\StringTok{"inches"}\NormalTok{;}
      \KeywordTok{break}\NormalTok{;}
    \KeywordTok{case} \OtherTok{lengthUnitClass}\NormalTok{.}\FunctionTok{LENGTHUNIT_FOOT}\NormalTok{:}
      \NormalTok{unit = }\StringTok{"feet"}\NormalTok{;}
      \KeywordTok{break}\NormalTok{;}
    \KeywordTok{case} \OtherTok{lengthUnitClass}\NormalTok{.}\FunctionTok{LENGTHUNIT_MM}\NormalTok{:}
      \NormalTok{unit = }\StringTok{"mm"}\NormalTok{;}
      \KeywordTok{break}\NormalTok{;}
    \KeywordTok{case} \OtherTok{lengthUnitClass}\NormalTok{.}\FunctionTok{LENGTHUNIT_CM}\NormalTok{:}
      \NormalTok{unit = }\StringTok{"cm"}\NormalTok{;}
      \KeywordTok{break}\NormalTok{;}
    \KeywordTok{case} \OtherTok{lengthUnitClass}\NormalTok{.}\FunctionTok{LENGTHUNIT_M}\NormalTok{:}
      \NormalTok{unit = }\StringTok{"m"}\NormalTok{;}
      \KeywordTok{break}\NormalTok{;}
    \KeywordTok{case} \OtherTok{lengthUnitClass}\NormalTok{.}\FunctionTok{LENGTHUNIT_MCM}\NormalTok{:}
      \NormalTok{unit = }\StringTok{"mcm"}\NormalTok{;}
      \KeywordTok{break}\NormalTok{;}
    \NormalTok{\}}
      
      
      
      \OtherTok{newWin}\NormalTok{.}\OtherTok{document}\NormalTok{.}\FunctionTok{writeln} \NormalTok{(}\StringTok{"<h2>Sheet "}\NormalTok{+ i + }\StringTok{"</h2>"}\NormalTok{);}
      \OtherTok{newWin}\NormalTok{.}\OtherTok{document}\NormalTok{.}\FunctionTok{writeln} \NormalTok{(}\StringTok{"<table>"}\NormalTok{);}
      \OtherTok{newWin}\NormalTok{.}\OtherTok{document}\NormalTok{.}\FunctionTok{writeln} \NormalTok{(}\StringTok{" <tr><td> Width </td><td> "}\NormalTok{+ }
                   \OtherTok{info}\NormalTok{.}\FunctionTok{Width} \NormalTok{+ }\StringTok{" </td></tr> "}\NormalTok{);}
      \OtherTok{newWin}\NormalTok{.}\OtherTok{document}\NormalTok{.}\FunctionTok{writeln} \NormalTok{(}\StringTok{"  <tr><td> Height </td><td> "}\NormalTok{+ }
                   \OtherTok{info}\NormalTok{.}\FunctionTok{Height} \NormalTok{+ }\StringTok{" </td></tr> "}\NormalTok{);}
      \OtherTok{newWin}\NormalTok{.}\OtherTok{document}\NormalTok{.}\FunctionTok{writeln} \NormalTok{(}\StringTok{" <tr><td> Units </td><td> "}\NormalTok{+ }
                   \NormalTok{unit + }\StringTok{" </td></tr> "}\NormalTok{);}
      \KeywordTok{var} \NormalTok{formatName;}
      \KeywordTok{if} \NormalTok{(format == }\KeywordTok{void} \KeywordTok{null}\NormalTok{)}
    \NormalTok{formatName = }\StringTok{"none"}\NormalTok{;}
      \KeywordTok{else}
    \NormalTok{formatName = }\OtherTok{format}\NormalTok{.}\FunctionTok{FullName}\NormalTok{;}
      \OtherTok{newWin}\NormalTok{.}\OtherTok{document}\NormalTok{.}\FunctionTok{writeln} \NormalTok{(}\StringTok{" <tr><td> Format </td><td> "}\NormalTok{+ }
                   \NormalTok{formatName + }\StringTok{" </td></tr> "}\NormalTok{);}
      \OtherTok{newWin}\NormalTok{.}\OtherTok{document}\NormalTok{.}\FunctionTok{writeln} \NormalTok{(}\StringTok{"</table>"}\NormalTok{);}
      \OtherTok{newWin}\NormalTok{.}\OtherTok{document}\NormalTok{.}\FunctionTok{writeln} \NormalTok{(}\StringTok{"<br>"}\NormalTok{);}
      
    \NormalTok{\}}
  \OtherTok{newWin}\NormalTok{.}\OtherTok{document}\NormalTok{.}\FunctionTok{writeln} \NormalTok{(}\StringTok{"</body></html>"}\NormalTok{);}
\NormalTok{\} }

\CommentTok{/*====================================================================*\textbackslash{} }
\CommentTok{FUNCTION : listViews()}
\CommentTok{PURPOSE  : Command to list view info in an information window }
\CommentTok{\textbackslash{}*====================================================================*/}
\KeywordTok{function} \FunctionTok{listViews}\NormalTok{() }
\NormalTok{\{}
  
  \KeywordTok{if} \NormalTok{(!}\FunctionTok{pfcIsWindows}\NormalTok{())}
    \OtherTok{netscape}\NormalTok{.}\OtherTok{security}\NormalTok{.}\OtherTok{PrivilegeManager}\NormalTok{.}\FunctionTok{enablePrivilege}\NormalTok{(}\StringTok{"UniversalXPConnect"}\NormalTok{); }

\CommentTok{/*--------------------------------------------------------------------*\textbackslash{}}
     \NormalTok{Open a browser window to contain the information to be displayed}
\NormalTok{\textbackslash{}*--------------------------------------------------------------------*}\OtherTok{/ }
\OtherTok{  }
\OtherTok{  var newWin = window.open }\FloatTok{(}\OtherTok{'', "_LV", "scrollbars"}\FloatTok{)}\OtherTok{;}
\OtherTok{  if }\FloatTok{(}\OtherTok{pfcIsWindows}\FloatTok{())}
\OtherTok{    \{}
\OtherTok{      newWin.resizeTo }\FloatTok{(}\OtherTok{300, screen.height/2}\FloatTok{.0}\NormalTok{);}
      \OtherTok{newWin}\NormalTok{.}\FunctionTok{moveTo} \NormalTok{(}\OtherTok{screen}\NormalTok{.}\FunctionTok{width}\DecValTok{-300}\NormalTok{, }\DecValTok{0}\NormalTok{);}
    \NormalTok{\}}
  \OtherTok{newWin}\NormalTok{.}\OtherTok{document}\NormalTok{.}\FunctionTok{writeln} \NormalTok{(}\StringTok{"<html><head></head><body>"}\NormalTok{);}

\CommentTok{/*--------------------------------------------------------------------*\textbackslash{} }
\CommentTok{   Get the current drawing}
\CommentTok{\textbackslash{}*--------------------------------------------------------------------*/}
  \KeywordTok{var} \NormalTok{session = }\FunctionTok{pfcGetProESession} \NormalTok{();}
  \KeywordTok{var} \NormalTok{drawing = }\OtherTok{session}\NormalTok{.}\FunctionTok{CurrentModel}\NormalTok{;}
  
  \KeywordTok{if} \NormalTok{(}\OtherTok{drawing}\NormalTok{.}\FunctionTok{Type} \NormalTok{!= }\FunctionTok{pfcCreate} \NormalTok{(}\StringTok{"pfcModelType"}\NormalTok{).}\FunctionTok{MDL_DRAWING}\NormalTok{)}
    \KeywordTok{throw} \KeywordTok{new} \FunctionTok{Error} \NormalTok{(}\DecValTok{0}\NormalTok{, }\StringTok{"Current model is not a drawing"}\NormalTok{);}
  
\CommentTok{/*--------------------------------------------------------------------*\textbackslash{}  }
\CommentTok{  Collect the views }
\CommentTok{\textbackslash{}*--------------------------------------------------------------------*/}

  \KeywordTok{var} \NormalTok{views = }\OtherTok{drawing}\NormalTok{.}\FunctionTok{List2DViews} \NormalTok{();}
  
  \KeywordTok{for}\NormalTok{(i=}\DecValTok{0}\NormalTok{; i<}\OtherTok{views}\NormalTok{.}\FunctionTok{Count}\NormalTok{; i++)}
    \NormalTok{\{ }
      
      \KeywordTok{var} \NormalTok{view = }\OtherTok{views}\NormalTok{.}\FunctionTok{Item} \NormalTok{(i);}

\CommentTok{/*--------------------------------------------------------------------*\textbackslash{} }
\CommentTok{  Get the name & sheet number for this view}
\CommentTok{\textbackslash{}*--------------------------------------------------------------------*/}
      \KeywordTok{var} \NormalTok{viewName = }\OtherTok{view}\NormalTok{.}\FunctionTok{Name}\NormalTok{;}
      \KeywordTok{var} \NormalTok{sheetNo = }\OtherTok{view}\NormalTok{.}\FunctionTok{GetSheetNumber} \NormalTok{();}

\CommentTok{/*--------------------------------------------------------------------*\textbackslash{} }
\CommentTok{  Get the name of the solid that the view contains}
\CommentTok{\textbackslash{}*--------------------------------------------------------------------*/} 
      \KeywordTok{var} \NormalTok{solid  = }\OtherTok{view}\NormalTok{.}\FunctionTok{GetModel} \NormalTok{();}
      \KeywordTok{var} \NormalTok{descr = }\OtherTok{solid}\NormalTok{.}\FunctionTok{Descr}\NormalTok{;}
      
\CommentTok{/*--------------------------------------------------------------------*\textbackslash{}  }
\CommentTok{  Get the outline, scale, and display state}
\CommentTok{\textbackslash{}*--------------------------------------------------------------------*/} 
      \KeywordTok{var} \NormalTok{outline = }\OtherTok{view}\NormalTok{.}\FunctionTok{Outline}\NormalTok{;}
      \KeywordTok{var} \NormalTok{scale = }\OtherTok{view}\NormalTok{.}\FunctionTok{Scale}\NormalTok{;}
      \KeywordTok{var} \NormalTok{display = }\OtherTok{view}\NormalTok{.}\FunctionTok{Display}\NormalTok{;}
    
\CommentTok{/*--------------------------------------------------------------------*\textbackslash{} }
\CommentTok{  Write the information to the browser window file}
\CommentTok{\textbackslash{}*--------------------------------------------------------------------*/}
      \NormalTok{displayStyleClass = }\FunctionTok{pfcCreate} \NormalTok{(}\StringTok{"pfcDisplayStyle"}\NormalTok{); }
      \KeywordTok{var} \NormalTok{dispStyle;}
      \KeywordTok{switch}\NormalTok{(}\OtherTok{display}\NormalTok{.}\FunctionTok{Style}\NormalTok{)}
    \NormalTok{\{         }
    \KeywordTok{case} \OtherTok{displayStyleClass}\NormalTok{.}\FunctionTok{DISPSTYLE_DEFAULT}\NormalTok{: }
      \NormalTok{dispStyle = }\StringTok{"default"}\NormalTok{; }
      \KeywordTok{break}\NormalTok{;}
    \KeywordTok{case} \OtherTok{displayStyleClass}\NormalTok{.}\FunctionTok{DISPSTYLE_WIREFRAME}\NormalTok{: }
      \NormalTok{dispStyle = }\StringTok{"wireframe"}\NormalTok{; }
      \KeywordTok{break}\NormalTok{;}
    \KeywordTok{case} \OtherTok{displayStyleClass}\NormalTok{.}\FunctionTok{DISPSTYLE_HIDDEN_LINE}\NormalTok{: }
      \NormalTok{dispStyle = }\StringTok{"hidden line"}\NormalTok{; }
      \KeywordTok{break}\NormalTok{;}
    \KeywordTok{case} \OtherTok{displayStyleClass}\NormalTok{.}\FunctionTok{DISPSTYLE_NO_HIDDEN}\NormalTok{: }
      \NormalTok{dispStyle = }\StringTok{"no hidden"}\NormalTok{;}
      \KeywordTok{break}\NormalTok{;}
    \KeywordTok{case} \OtherTok{displayStyleClass}\NormalTok{.}\FunctionTok{DISPSTYLE_SHADED}\NormalTok{: }
      \NormalTok{dispStyle = }\StringTok{"shaded"}\NormalTok{; }
      \KeywordTok{break}\NormalTok{;         }
    \NormalTok{\}}
      
      \OtherTok{newWin}\NormalTok{.}\OtherTok{document}\NormalTok{.}\FunctionTok{writeln} \NormalTok{(}\StringTok{"<h2>View "}\NormalTok{+ viewName + }\StringTok{"</h2>"}\NormalTok{);}
      \OtherTok{newWin}\NormalTok{.}\OtherTok{document}\NormalTok{.}\FunctionTok{writeln} \NormalTok{(}\StringTok{"<table>"}\NormalTok{);}
      \OtherTok{newWin}\NormalTok{.}\OtherTok{document}\NormalTok{.}\FunctionTok{writeln} \NormalTok{(}\StringTok{" <tr><td> Sheet </td><td> "}\NormalTok{+ }
                   \NormalTok{sheetNo + }\StringTok{" </td></tr> "}\NormalTok{);}
      \OtherTok{newWin}\NormalTok{.}\OtherTok{document}\NormalTok{.}\FunctionTok{writeln} \NormalTok{(}\StringTok{"  <tr><td> Model </td><td> "}\NormalTok{+ }
                   \OtherTok{descr}\NormalTok{.}\FunctionTok{GetFullName}\NormalTok{() + }\StringTok{" </td></tr> "}\NormalTok{);}
      \OtherTok{newWin}\NormalTok{.}\OtherTok{document}\NormalTok{.}\FunctionTok{writeln} \NormalTok{(}\StringTok{" <tr><td> Outline </td><td> "}\NormalTok{);}
      \OtherTok{newWin}\NormalTok{.}\OtherTok{document}\NormalTok{.}\FunctionTok{writeln} \NormalTok{(}\StringTok{"<table><tr><td> <i>Lower left:</i> </td><td>"}\NormalTok{);}
      \OtherTok{newWin}\NormalTok{.}\OtherTok{document}\NormalTok{.}\FunctionTok{writeln} \NormalTok{(}\OtherTok{outline}\NormalTok{.}\FunctionTok{Item} \NormalTok{(}\DecValTok{0}\NormalTok{).}\FunctionTok{Item} \NormalTok{(}\DecValTok{0}\NormalTok{) + }\StringTok{", "} \NormalTok{+ }
                   \OtherTok{outline}\NormalTok{.}\FunctionTok{Item} \NormalTok{(}\DecValTok{0}\NormalTok{).}\FunctionTok{Item}\NormalTok{(}\DecValTok{1}\NormalTok{) + }\StringTok{", "} \NormalTok{+ }
                   \OtherTok{outline}\NormalTok{.}\FunctionTok{Item} \NormalTok{(}\DecValTok{0}\NormalTok{).}\FunctionTok{Item}\NormalTok{(}\DecValTok{2}\NormalTok{));}
      \OtherTok{newWin}\NormalTok{.}\OtherTok{document}\NormalTok{.}\FunctionTok{writeln} \NormalTok{(}\StringTok{"</td></tr><tr><td> <iUpper right:</i></td><td>"}\NormalTok{);}
      \OtherTok{newWin}\NormalTok{.}\OtherTok{document}\NormalTok{.}\FunctionTok{writeln} \NormalTok{(}\OtherTok{outline}\NormalTok{.}\FunctionTok{Item} \NormalTok{(}\DecValTok{1}\NormalTok{).}\FunctionTok{Item} \NormalTok{(}\DecValTok{0}\NormalTok{) + }\StringTok{", "} \NormalTok{+ }
                   \OtherTok{outline}\NormalTok{.}\FunctionTok{Item} \NormalTok{(}\DecValTok{1}\NormalTok{).}\FunctionTok{Item}\NormalTok{(}\DecValTok{1}\NormalTok{) + }\StringTok{", "} \NormalTok{+ }
                   \OtherTok{outline}\NormalTok{.}\FunctionTok{Item} \NormalTok{(}\DecValTok{1}\NormalTok{).}\FunctionTok{Item}\NormalTok{(}\DecValTok{2}\NormalTok{));}
      \OtherTok{newWin}\NormalTok{.}\OtherTok{document}\NormalTok{.}\FunctionTok{writeln} \NormalTok{(}\StringTok{"</td></tr></table></td>"}\NormalTok{);}
      \OtherTok{newWin}\NormalTok{.}\OtherTok{document}\NormalTok{.}\FunctionTok{writeln} \NormalTok{(}\StringTok{" <tr><td> Scale </td><td> "}\NormalTok{+ scale + }
                   \StringTok{" </td></tr> "}\NormalTok{);}
      \OtherTok{newWin}\NormalTok{.}\OtherTok{document}\NormalTok{.}\FunctionTok{writeln} \NormalTok{(}\StringTok{" <tr><td> Display style </td><td> "}\NormalTok{+ }
                   \NormalTok{dispStyle + }\StringTok{" </td></tr>"}\NormalTok{);}
      \OtherTok{newWin}\NormalTok{.}\OtherTok{document}\NormalTok{.}\FunctionTok{writeln} \NormalTok{(}\StringTok{"</table>"}\NormalTok{);}
      \OtherTok{newWin}\NormalTok{.}\OtherTok{document}\NormalTok{.}\FunctionTok{writeln} \NormalTok{(}\StringTok{"<br>"}\NormalTok{);}
    \NormalTok{\}}
  \OtherTok{newWin}\NormalTok{.}\OtherTok{document}\NormalTok{.}\FunctionTok{writeln} \NormalTok{(}\StringTok{"</body></html>"}\NormalTok{);}
\NormalTok{\}}

\CommentTok{/*==================================================================*\textbackslash{}}
\NormalTok{FUNCTION: }\FunctionTok{drawingSolidReplace}\NormalTok{()}
\NormalTok{PURPOSE:  Replaces all instance solid models }\KeywordTok{in} \NormalTok{a drawing }\KeywordTok{with} \NormalTok{their}
          \OtherTok{generic}\NormalTok{.  }\FunctionTok{Similar} \FunctionTok{to} \FunctionTok{the} \FunctionTok{Pro}\NormalTok{/ENGINEER behavior,  }
          \NormalTok{the }\KeywordTok{function} \NormalTok{will not replace models }\KeywordTok{if} \NormalTok{the target generic }
          \NormalTok{model is already present }\KeywordTok{in} \NormalTok{the }\OtherTok{drawing}\NormalTok{.}
\NormalTok{\textbackslash{}*==================================================================*}\OtherTok{/}
\KeywordTok{function} \FunctionTok{replaceModels}\NormalTok{()}
\NormalTok{\{}
  \KeywordTok{if} \NormalTok{(!}\FunctionTok{pfcIsWindows}\NormalTok{())}
    \OtherTok{netscape}\NormalTok{.}\OtherTok{security}\NormalTok{.}\OtherTok{PrivilegeManager}\NormalTok{.}\FunctionTok{enablePrivilege}\NormalTok{(}\StringTok{"UniversalXPConnect"}\NormalTok{); }
\CommentTok{/*--------------------------------------------------------------------*\textbackslash{} }
\CommentTok{  Get the current drawing}
\CommentTok{\textbackslash{}*--------------------------------------------------------------------*/}
  \KeywordTok{var} \NormalTok{session = }\FunctionTok{pfcGetProESession} \NormalTok{();}
  \KeywordTok{var} \NormalTok{drawing = }\OtherTok{session}\NormalTok{.}\FunctionTok{CurrentModel}\NormalTok{;}
  
  \KeywordTok{if} \NormalTok{(}\OtherTok{drawing}\NormalTok{.}\FunctionTok{Type} \NormalTok{!= }\FunctionTok{pfcCreate} \NormalTok{(}\StringTok{"pfcModelType"}\NormalTok{).}\FunctionTok{MDL_DRAWING}\NormalTok{)}
    \KeywordTok{throw} \KeywordTok{new} \FunctionTok{Error} \NormalTok{(}\DecValTok{0}\NormalTok{, }\StringTok{"Current model is not a drawing"}\NormalTok{);}
    
\CommentTok{/*------------------------------------------------------------------*\textbackslash{}}
  \NormalTok{Visit the drawing }\OtherTok{models}\NormalTok{.}
\NormalTok{\textbackslash{}*------------------------------------------------------------------*}\OtherTok{/}
  \KeywordTok{var} \NormalTok{solids = }\OtherTok{drawing}\NormalTok{.}\FunctionTok{ListModels} \NormalTok{();}
  
\CommentTok{/*------------------------------------------------------------------*\textbackslash{}}
  \NormalTok{Loop on all of the drawing }\OtherTok{models}\NormalTok{.}
\NormalTok{\textbackslash{}*------------------------------------------------------------------*}\OtherTok{/  }
\OtherTok{  for }\FloatTok{(}\OtherTok{i = 0; i < solids.Count; i}\FloatTok{++)}
\OtherTok{    \{}
\OtherTok{      var solid = solids.Item }\FloatTok{(}\OtherTok{i}\FloatTok{)}\OtherTok{;}
\OtherTok{/}\NormalTok{*------------------------------------------------------------------*\textbackslash{}}
  \NormalTok{If the generic is not an instance, }\KeywordTok{continue} \NormalTok{(Parent property }
  \NormalTok{from }\KeywordTok{class} \NormalTok{pfcFamilyMember)}
\NormalTok{\textbackslash{}*------------------------------------------------------------------*}\OtherTok{/ }
\OtherTok{      var generic = solid.Parent;}
\OtherTok{      }
\OtherTok{      if }\FloatTok{(}\OtherTok{generic == void null}\FloatTok{)}
\OtherTok{        continue;}
\OtherTok{      }
\OtherTok{/}\NormalTok{*------------------------------------------------------------------*\textbackslash{}}
  \NormalTok{Replace all instances }\KeywordTok{with} \FunctionTok{their} \NormalTok{(top-level) }\OtherTok{generic}\NormalTok{.}
\NormalTok{\textbackslash{}*------------------------------------------------------------------*}\OtherTok{/}
      \KeywordTok{try}
    \NormalTok{\{}
      \OtherTok{drawing}\NormalTok{.}\FunctionTok{ReplaceModel} \NormalTok{(solid, generic, }\KeywordTok{true}\NormalTok{);}
    \NormalTok{\}}
      \KeywordTok{catch} \NormalTok{(err)}
    \NormalTok{\{}
      \KeywordTok{if} \NormalTok{(}\FunctionTok{pfcGetExceptionType} \NormalTok{(err) == }\StringTok{"pfcXToolkitFound"}\NormalTok{)}
        \NormalTok{; }\CommentTok{// Target generic is already in drawing; do nothing}
      \KeywordTok{else}
        \KeywordTok{throw} \NormalTok{err;}
    \NormalTok{\}}
    \NormalTok{\}}
\NormalTok{\}}

\CommentTok{/*====================================================================*\textbackslash{} }
\CommentTok{FUNCTION : util_solidFind()}
\CommentTok{PURPOSE  : Utility to select a solid using the file browser and retrieve}
\CommentTok{           it if it is not already in session.}
\CommentTok{\textbackslash{}*====================================================================*/} 
\KeywordTok{function} \FunctionTok{util_solidFind}\NormalTok{(filename }\CommentTok{/* string "????.???" format */}\NormalTok{)}
\NormalTok{\{}
\CommentTok{/*--------------------------------------------------------------------*\textbackslash{} }
\CommentTok{   Find it, if its in session and return it}
\CommentTok{\textbackslash{}*--------------------------------------------------------------------*/}
  \KeywordTok{var} \NormalTok{session = }\FunctionTok{pfcGetProESession} \NormalTok{();}
  \KeywordTok{var} \NormalTok{mdlDescr = }
    \FunctionTok{pfcCreate} \NormalTok{(}\StringTok{"pfcModelDescriptor"}\NormalTok{).}\FunctionTok{CreateFromFileName} \NormalTok{(filename);}
  \KeywordTok{var} \NormalTok{mdl = }\OtherTok{session}\NormalTok{.}\FunctionTok{GetModelFromDescr} \NormalTok{(mdlDescr);}
  
  \KeywordTok{if} \NormalTok{(mdl != }\KeywordTok{void} \KeywordTok{null}\NormalTok{)}
    \KeywordTok{return} \NormalTok{mdl;}
  
\CommentTok{/*--------------------------------------------------------------------*\textbackslash{}     }
\CommentTok{  Try to retieve the solid }
\CommentTok{\textbackslash{}*--------------------------------------------------------------------*/} 
  \NormalTok{mdl = }\OtherTok{session}\NormalTok{.}\FunctionTok{RetrieveModel} \NormalTok{(mdlDescr);}
  
  \KeywordTok{if} \NormalTok{(mdl != }\KeywordTok{void} \KeywordTok{null}\NormalTok{)}
    \KeywordTok{return} \NormalTok{mdl;}
  
  \KeywordTok{throw} \KeywordTok{new} \FunctionTok{Error} \NormalTok{(}\DecValTok{0}\NormalTok{, }
           \StringTok{"Model "}\NormalTok{+filename+}\StringTok{" cannot be found or retrieved."}\NormalTok{);}
  
  \KeywordTok{return} \KeywordTok{void} \KeywordTok{null}\NormalTok{;}
\NormalTok{\}}


\CommentTok{/*====================================================================*\textbackslash{} }
\CommentTok{FUNCTION : createSheetAndViews() }
\CommentTok{PURPOSE  : Create a new drawing sheet with a general, and two}
\CommentTok{           projected,views of a selected solid }
\CommentTok{\textbackslash{}*====================================================================*/} 
\KeywordTok{function} \FunctionTok{createSheetAndViews}\NormalTok{(solidName }\CommentTok{/* string as ????.??? */}\NormalTok{)}
\NormalTok{\{}
  \KeywordTok{if} \NormalTok{(!}\FunctionTok{pfcIsWindows}\NormalTok{())}
    \OtherTok{netscape}\NormalTok{.}\OtherTok{security}\NormalTok{.}\OtherTok{PrivilegeManager}\NormalTok{.}\FunctionTok{enablePrivilege}\NormalTok{(}\StringTok{"UniversalXPConnect"}\NormalTok{); }
\CommentTok{/*--------------------------------------------------------------------*\textbackslash{} }
\CommentTok{  Get the current drawing, create a new sheet}
\CommentTok{\textbackslash{}*--------------------------------------------------------------------*/}
  \KeywordTok{var} \NormalTok{session = }\FunctionTok{pfcGetProESession} \NormalTok{();}
  \KeywordTok{var} \NormalTok{drawing = }\OtherTok{session}\NormalTok{.}\FunctionTok{CurrentModel}\NormalTok{;}
  
  \KeywordTok{if} \NormalTok{(}\OtherTok{drawing}\NormalTok{.}\FunctionTok{Type} \NormalTok{!= }\FunctionTok{pfcCreate} \NormalTok{(}\StringTok{"pfcModelType"}\NormalTok{).}\FunctionTok{MDL_DRAWING}\NormalTok{)}
    \KeywordTok{throw} \KeywordTok{new} \FunctionTok{Error} \NormalTok{(}\DecValTok{0}\NormalTok{, }\StringTok{"Current model is not a drawing"}\NormalTok{);}
  
  \KeywordTok{var} \NormalTok{sheetNo = }\OtherTok{drawing}\NormalTok{.}\FunctionTok{AddSheet} \NormalTok{();}
  \OtherTok{drawing}\NormalTok{.}\FunctionTok{CurrentSheetNumber} \NormalTok{= sheetNo;}
  
\CommentTok{/*--------------------------------------------------------------------*\textbackslash{} }
\CommentTok{  Find the solid model, if its in session}
\CommentTok{\textbackslash{}*--------------------------------------------------------------------*/}
  \KeywordTok{var} \NormalTok{mdlDescr = }
    \FunctionTok{pfcCreate} \NormalTok{(}\StringTok{"pfcModelDescriptor"}\NormalTok{).}\FunctionTok{CreateFromFileName} \NormalTok{(solidName);}
  \KeywordTok{var} \NormalTok{solidMdl = }\OtherTok{session}\NormalTok{.}\FunctionTok{GetModelFromDescr} \NormalTok{(mdlDescr);}
  
  \KeywordTok{if} \NormalTok{(solidMdl == }\KeywordTok{void} \KeywordTok{null}\NormalTok{)}
    \NormalTok{\{}
\CommentTok{/*--------------------------------------------------------------------*\textbackslash{}     }
\CommentTok{  If its not found, try to retieve the solid model}
\CommentTok{\textbackslash{}*--------------------------------------------------------------------*/} 
      \NormalTok{solidMdl = }\OtherTok{session}\NormalTok{.}\FunctionTok{RetrieveModel} \NormalTok{(mdlDescr);}
      
      \KeywordTok{if} \NormalTok{(solidMdl == }\KeywordTok{void} \KeywordTok{null}\NormalTok{)}
    \KeywordTok{throw} \KeywordTok{new} \FunctionTok{Error} \NormalTok{(}\DecValTok{0}\NormalTok{, }
             \StringTok{"Model "}\NormalTok{+solidName+}\StringTok{" cannot be found or retrieved."}\NormalTok{);}
    \NormalTok{\}}
        
\CommentTok{/*--------------------------------------------------------------------*\textbackslash{}     }
\CommentTok{  Try to add it to the drawing}
\CommentTok{\textbackslash{}*--------------------------------------------------------------------*/}    
  \KeywordTok{try}
    \NormalTok{\{ }
      \OtherTok{drawing}\NormalTok{.}\FunctionTok{AddModel} \NormalTok{(solidMdl);}
    \NormalTok{\}}
  \KeywordTok{catch} \NormalTok{(err)}
    \NormalTok{\{}
      \KeywordTok{if} \NormalTok{(}\FunctionTok{pfcGetExceptionType} \NormalTok{(err) == }\StringTok{"pfcXToolkitInUse"}\NormalTok{)}
    \NormalTok{; }\CommentTok{// model is already in this drawing, nothing to do}
      \KeywordTok{else}
    \KeywordTok{throw} \NormalTok{err;}
    \NormalTok{\}}
  
\CommentTok{/*--------------------------------------------------------------------*\textbackslash{} }
\CommentTok{  Create a general view from the Z axis direction at a predefined location }
\CommentTok{\textbackslash{}*--------------------------------------------------------------------*/} 
  \KeywordTok{var} \NormalTok{matrix = }\FunctionTok{pfcCreate} \NormalTok{(}\StringTok{"pfcMatrix3D"}\NormalTok{);}
  \KeywordTok{for} \NormalTok{(i = }\DecValTok{0}\NormalTok{; i < }\DecValTok{4}\NormalTok{; i++)}
    \KeywordTok{for} \NormalTok{(j = }\DecValTok{0}\NormalTok{; j < }\DecValTok{4}\NormalTok{; j++)}
      \NormalTok{\{}
    \KeywordTok{if} \NormalTok{(i == j)}
      \OtherTok{matrix}\NormalTok{.}\FunctionTok{Set} \NormalTok{(i, j, }\FloatTok{1.0}\NormalTok{);}
    \KeywordTok{else}
      \OtherTok{matrix}\NormalTok{.}\FunctionTok{Set} \NormalTok{(i, j, }\FloatTok{0.0}\NormalTok{);}
      \NormalTok{\}}
  
  \KeywordTok{var} \NormalTok{transf = }\FunctionTok{pfcCreate} \NormalTok{(}\StringTok{"pfcTransform3D"}\NormalTok{).}\FunctionTok{Create} \NormalTok{(matrix);}
  
  \KeywordTok{var} \NormalTok{pos = }\FunctionTok{pfcCreate} \NormalTok{(}\StringTok{"pfcPoint3D"}\NormalTok{);}
  \OtherTok{pos}\NormalTok{.}\FunctionTok{Set} \NormalTok{(}\DecValTok{0}\NormalTok{, }\FloatTok{200.0}\NormalTok{);}
  \OtherTok{pos}\NormalTok{.}\FunctionTok{Set} \NormalTok{(}\DecValTok{1}\NormalTok{, }\FloatTok{600.0}\NormalTok{);}
  \OtherTok{pos}\NormalTok{.}\FunctionTok{Set} \NormalTok{(}\DecValTok{2}\NormalTok{, }\FloatTok{0.0}\NormalTok{);}
  
  \KeywordTok{var} \NormalTok{instrs = }
    \FunctionTok{pfcCreate} \NormalTok{(}\StringTok{"pfcGeneralViewCreateInstructions"}\NormalTok{).}\FunctionTok{Create} \NormalTok{(solidMdl,}
                               \NormalTok{sheetNo, pos, transf);}
  
  \KeywordTok{var} \NormalTok{genView = }\OtherTok{drawing}\NormalTok{.}\FunctionTok{CreateView} \NormalTok{(instrs);}
    
\CommentTok{/*--------------------------------------------------------------------*\textbackslash{}     }
\CommentTok{  Get the position and size of the new view }
\CommentTok{\textbackslash{}*--------------------------------------------------------------------*/} 
  \KeywordTok{var} \NormalTok{outline = }\OtherTok{genView}\NormalTok{.}\FunctionTok{Outline}\NormalTok{;}

\CommentTok{/*--------------------------------------------------------------------*\textbackslash{}     }
\CommentTok{  Create a projected view to the right of the general view }
\CommentTok{\textbackslash{}*--------------------------------------------------------------------*/} 
  \OtherTok{pos}\NormalTok{.}\FunctionTok{Set} \NormalTok{(}\DecValTok{0}\NormalTok{, }\OtherTok{outline}\NormalTok{.}\FunctionTok{Item} \NormalTok{(}\DecValTok{1}\NormalTok{).}\FunctionTok{Item} \NormalTok{(}\DecValTok{0}\NormalTok{) + (}\OtherTok{outline}\NormalTok{.}\FunctionTok{Item}\NormalTok{(}\DecValTok{1}\NormalTok{).}\FunctionTok{Item}\NormalTok{(}\DecValTok{0}\NormalTok{) - }
                       \OtherTok{outline}\NormalTok{.}\FunctionTok{Item} \NormalTok{(}\DecValTok{0}\NormalTok{).}\FunctionTok{Item} \NormalTok{(}\DecValTok{0}\NormalTok{)));}
  \OtherTok{pos}\NormalTok{.}\FunctionTok{Set} \NormalTok{(}\DecValTok{1}\NormalTok{, (}\OtherTok{outline}\NormalTok{.}\FunctionTok{Item} \NormalTok{(}\DecValTok{0}\NormalTok{).}\FunctionTok{Item}\NormalTok{(}\DecValTok{1}\NormalTok{) + }\OtherTok{outline}\NormalTok{.}\FunctionTok{Item} \NormalTok{(}\DecValTok{1}\NormalTok{).}\FunctionTok{Item}\NormalTok{(}\DecValTok{1}\NormalTok{))/}\DecValTok{2}\NormalTok{);}
  
  \NormalTok{instrs = }
    \FunctionTok{pfcCreate} \NormalTok{(}\StringTok{"pfcProjectionViewCreateInstructions"}\NormalTok{).}\FunctionTok{Create} \NormalTok{(genView, }
                                  \NormalTok{pos);}
  
  \OtherTok{drawing}\NormalTok{.}\FunctionTok{CreateView} \NormalTok{(instrs);}
    
\CommentTok{/*--------------------------------------------------------------------*\textbackslash{}     }
\CommentTok{  Create a projected view below the general view }
\CommentTok{\textbackslash{}*--------------------------------------------------------------------*/} 
  \OtherTok{pos}\NormalTok{.}\FunctionTok{Set} \NormalTok{(}\DecValTok{0}\NormalTok{, (}\OtherTok{outline}\NormalTok{.}\FunctionTok{Item} \NormalTok{(}\DecValTok{0}\NormalTok{).}\FunctionTok{Item}\NormalTok{(}\DecValTok{0}\NormalTok{) + }\OtherTok{outline}\NormalTok{.}\FunctionTok{Item} \NormalTok{(}\DecValTok{1}\NormalTok{).}\FunctionTok{Item}\NormalTok{(}\DecValTok{0}\NormalTok{))/}\DecValTok{2}\NormalTok{);}
  \OtherTok{pos}\NormalTok{.}\FunctionTok{Set} \NormalTok{(}\DecValTok{1}\NormalTok{, }\OtherTok{outline}\NormalTok{.}\FunctionTok{Item} \NormalTok{(}\DecValTok{0}\NormalTok{).}\FunctionTok{Item} \NormalTok{(}\DecValTok{1}\NormalTok{) - (}\OtherTok{outline}\NormalTok{.}\FunctionTok{Item}\NormalTok{(}\DecValTok{1}\NormalTok{).}\FunctionTok{Item}\NormalTok{(}\DecValTok{1}\NormalTok{) - }
                       \OtherTok{outline}\NormalTok{.}\FunctionTok{Item} \NormalTok{(}\DecValTok{0}\NormalTok{).}\FunctionTok{Item} \NormalTok{(}\DecValTok{1}\NormalTok{)));}
  
  \NormalTok{instrs = }
    \FunctionTok{pfcCreate} \NormalTok{(}\StringTok{"pfcProjectionViewCreateInstructions"}\NormalTok{).}\FunctionTok{Create} \NormalTok{(genView, }
                                  \NormalTok{pos);}
    
  \OtherTok{drawing}\NormalTok{.}\FunctionTok{CreateView} \NormalTok{(instrs);}
\NormalTok{\}}


\CommentTok{/*====================================================================*\textbackslash{} }
\CommentTok{FUNCTION : lineEntityCreate() }
\CommentTok{PURPOSE  : Utility to create a line entity between specified points }
\CommentTok{\textbackslash{}*====================================================================*/} 
\KeywordTok{function} \FunctionTok{lineEntityCreate}\NormalTok{()}
\NormalTok{\{}
  \KeywordTok{if} \NormalTok{(!}\FunctionTok{pfcIsWindows}\NormalTok{())}
    \OtherTok{netscape}\NormalTok{.}\OtherTok{security}\NormalTok{.}\OtherTok{PrivilegeManager}\NormalTok{.}\FunctionTok{enablePrivilege}\NormalTok{(}\StringTok{"UniversalXPConnect"}\NormalTok{); }
  
  \KeywordTok{var} \NormalTok{color = }\FunctionTok{pfcCreate} \NormalTok{(}\StringTok{"pfcStdColor"}\NormalTok{).}\FunctionTok{COLOR_QUILT}\NormalTok{;}
  \KeywordTok{var} \NormalTok{session = }\FunctionTok{pfcGetProESession} \NormalTok{();}
  
\CommentTok{/*--------------------------------------------------------------------*\textbackslash{} }
\CommentTok{  Get the current drawing & its background view}
\CommentTok{\textbackslash{}*--------------------------------------------------------------------*/}  
  \KeywordTok{var} \NormalTok{drawing = }\OtherTok{session}\NormalTok{.}\FunctionTok{CurrentModel}\NormalTok{;}
  
  \KeywordTok{if} \NormalTok{(}\OtherTok{drawing}\NormalTok{.}\FunctionTok{Type} \NormalTok{!= }\FunctionTok{pfcCreate} \NormalTok{(}\StringTok{"pfcModelType"}\NormalTok{).}\FunctionTok{MDL_DRAWING}\NormalTok{)}
    \KeywordTok{throw} \KeywordTok{new} \FunctionTok{Error} \NormalTok{(}\DecValTok{0}\NormalTok{, }\StringTok{"Current model is not a drawing"}\NormalTok{);}
  
  \KeywordTok{var} \NormalTok{currSheet = }\OtherTok{drawing}\NormalTok{.}\FunctionTok{CurrentSheetNumber}\NormalTok{;}
  \KeywordTok{var} \NormalTok{view = }\OtherTok{drawing}\NormalTok{.}\FunctionTok{GetSheetBackgroundView} \NormalTok{(currSheet);}
 
\CommentTok{/*--------------------------------------------------------------------*\textbackslash{} }
\CommentTok{   Select the endpoints of the line}
\CommentTok{\textbackslash{}*--------------------------------------------------------------------*/}
  \OtherTok{session}\NormalTok{.}\OtherTok{CurrentWindow}\NormalTok{.}\FunctionTok{SetBrowserSize} \NormalTok{(}\FloatTok{0.0}\NormalTok{);}
  
  \KeywordTok{var} \NormalTok{left = }\FunctionTok{pfcCreate} \NormalTok{(}\StringTok{"pfcMouseButton"}\NormalTok{).}\FunctionTok{MOUSE_BTN_LEFT}\NormalTok{;}
  \KeywordTok{var} \NormalTok{mouse1 = }\OtherTok{session}\NormalTok{.}\FunctionTok{UIGetNextMousePick} \NormalTok{(left);}
  \KeywordTok{var} \NormalTok{start = }\OtherTok{mouse1}\NormalTok{.}\FunctionTok{Position}\NormalTok{;}
  \KeywordTok{var} \NormalTok{mouse2 = }\OtherTok{session}\NormalTok{.}\FunctionTok{UIGetNextMousePick} \NormalTok{(left);}
  \KeywordTok{var} \NormalTok{end = }\OtherTok{mouse2}\NormalTok{.}\FunctionTok{Position}\NormalTok{;}
  
\CommentTok{/*--------------------------------------------------------------------*\textbackslash{} }
\CommentTok{  Allocate and initialize a curve descriptor }
\CommentTok{\textbackslash{}*--------------------------------------------------------------------*/}     
  \KeywordTok{var} \NormalTok{geom = }\FunctionTok{pfcCreate} \NormalTok{(}\StringTok{"pfcLineDescriptor"}\NormalTok{).}\FunctionTok{Create} \NormalTok{(start, end);}
  
\CommentTok{/*--------------------------------------------------------------------*\textbackslash{} }
\CommentTok{  Allocate data for the draft entity }
\CommentTok{\textbackslash{}*--------------------------------------------------------------------*/}     
  \KeywordTok{var} \NormalTok{instrs = }\FunctionTok{pfcCreate} \NormalTok{(}\StringTok{"pfcDetailEntityInstructions"}\NormalTok{).}\FunctionTok{Create} \NormalTok{(geom, }
                                 \NormalTok{view);}

\CommentTok{/*--------------------------------------------------------------------*\textbackslash{} }
\CommentTok{  Set the color to the specified Pro/ENGINEER predefined color }
\CommentTok{\textbackslash{}*--------------------------------------------------------------------*/}
  \KeywordTok{var} \NormalTok{rgb = }\OtherTok{session}\NormalTok{.}\FunctionTok{GetRGBFromStdColor} \NormalTok{(color);}
  \OtherTok{instrs}\NormalTok{.}\FunctionTok{Color} \NormalTok{= rgb;     }
  
\CommentTok{/*--------------------------------------------------------------------*\textbackslash{} }
\CommentTok{  Create the entity }
\CommentTok{\textbackslash{}*--------------------------------------------------------------------*/}     
  \OtherTok{drawing}\NormalTok{.}\FunctionTok{CreateDetailItem} \NormalTok{(instrs);}

\CommentTok{/*--------------------------------------------------------------------*\textbackslash{} }
\CommentTok{  Display the entity }
\CommentTok{\textbackslash{}*--------------------------------------------------------------------*/}     
  \OtherTok{session}\NormalTok{.}\OtherTok{CurrentWindow}\NormalTok{.}\FunctionTok{Repaint}\NormalTok{(); }
\NormalTok{\}}
 
 
\CommentTok{/*====================================================================*\textbackslash{}}
\NormalTok{FUNCTION : }\FunctionTok{createSurfNote}\NormalTok{() }
\NormalTok{PURPOSE  : Utility to create a note that documents the surface name or }\OtherTok{id}\NormalTok{.}
\NormalTok{The note text will be placed at the upper right corner of the selected }\OtherTok{view}\NormalTok{.}
\NormalTok{\textbackslash{}*====================================================================*}\OtherTok{/}
\KeywordTok{function} \FunctionTok{createSurfNote}\NormalTok{()}
\NormalTok{\{}
  \KeywordTok{if} \NormalTok{(!}\FunctionTok{pfcIsWindows}\NormalTok{())}
    \OtherTok{netscape}\NormalTok{.}\OtherTok{security}\NormalTok{.}\OtherTok{PrivilegeManager}\NormalTok{.}\FunctionTok{enablePrivilege}\NormalTok{(}\StringTok{"UniversalXPConnect"}\NormalTok{); }
\CommentTok{/*--------------------------------------------------------------------*\textbackslash{} }
\CommentTok{  Get the current drawing & its background view}
\CommentTok{\textbackslash{}*--------------------------------------------------------------------*/}  
  \KeywordTok{var} \NormalTok{session = }\FunctionTok{pfcGetProESession} \NormalTok{();}
  \KeywordTok{var} \NormalTok{drawing = }\OtherTok{session}\NormalTok{.}\FunctionTok{CurrentModel}\NormalTok{;}
  
  \KeywordTok{if} \NormalTok{(}\OtherTok{drawing}\NormalTok{.}\FunctionTok{Type} \NormalTok{!= }\FunctionTok{pfcCreate} \NormalTok{(}\StringTok{"pfcModelType"}\NormalTok{).}\FunctionTok{MDL_DRAWING}\NormalTok{)}
    \KeywordTok{throw} \KeywordTok{new} \FunctionTok{Error} \NormalTok{(}\DecValTok{0}\NormalTok{, }\StringTok{"Current model is not a drawing"}\NormalTok{);}
  
\CommentTok{/*--------------------------------------------------------------------*\textbackslash{}  }
\CommentTok{  Interactively select a surface in a drawing view}
\CommentTok{\textbackslash{}*--------------------------------------------------------------------*/} 
  \KeywordTok{var} \NormalTok{browserSize = }\OtherTok{session}\NormalTok{.}\OtherTok{CurrentWindow}\NormalTok{.}\FunctionTok{GetBrowserSize}\NormalTok{();}
  \OtherTok{session}\NormalTok{.}\OtherTok{CurrentWindow}\NormalTok{.}\FunctionTok{SetBrowserSize}\NormalTok{(}\FloatTok{0.0}\NormalTok{);}
  
  \KeywordTok{var} \NormalTok{options = }\FunctionTok{pfcCreate} \NormalTok{(}\StringTok{"pfcSelectionOptions"}\NormalTok{).}\FunctionTok{Create} \NormalTok{(}\StringTok{"surface"}\NormalTok{);}
  \OtherTok{options}\NormalTok{.}\FunctionTok{MaxNumSels} \NormalTok{= }\DecValTok{1}\NormalTok{;}
  
  \KeywordTok{var} \NormalTok{sels = }\OtherTok{session}\NormalTok{.}\FunctionTok{Select} \NormalTok{(options, }\KeywordTok{void} \KeywordTok{null}\NormalTok{);}
  \KeywordTok{var} \NormalTok{selSurf = }\OtherTok{sels}\NormalTok{.}\FunctionTok{Item} \NormalTok{(}\DecValTok{0}\NormalTok{);}
  \KeywordTok{var} \NormalTok{item = }\OtherTok{selSurf}\NormalTok{.}\FunctionTok{SelItem}\NormalTok{;}
  
  \KeywordTok{var} \NormalTok{name = }\OtherTok{item}\NormalTok{.}\FunctionTok{GetName}\NormalTok{();}
  \KeywordTok{if} \NormalTok{(name == }\KeywordTok{void} \KeywordTok{null}\NormalTok{)}
    \NormalTok{name = }\KeywordTok{new} \FunctionTok{String} \NormalTok{(}\StringTok{"Surface ID "}\NormalTok{+}\OtherTok{item}\NormalTok{.}\FunctionTok{Id}\NormalTok{);}
  
  \OtherTok{session}\NormalTok{.}\OtherTok{CurrentWindow}\NormalTok{.}\FunctionTok{SetBrowserSize}\NormalTok{(browserSize);}
    
\CommentTok{/*--------------------------------------------------------------------*\textbackslash{}  }
\CommentTok{  Allocate a text item, and set its properties}
\CommentTok{\textbackslash{}*--------------------------------------------------------------------*/}   
  \KeywordTok{var} \NormalTok{text = }\FunctionTok{pfcCreate} \NormalTok{(}\StringTok{"pfcDetailText"}\NormalTok{).}\FunctionTok{Create} \NormalTok{(name);}
 
\CommentTok{/*--------------------------------------------------------------------*\textbackslash{} }
\CommentTok{  Allocate a new text line, and add the text item to it}
\CommentTok{/*--------------------------------------------------------------------*/} 
  \KeywordTok{var} \NormalTok{texts = }\FunctionTok{pfcCreate} \NormalTok{(}\StringTok{"pfcDetailTexts"}\NormalTok{);}
  \OtherTok{texts}\NormalTok{.}\FunctionTok{Append} \NormalTok{(text);}
  
  \KeywordTok{var} \NormalTok{textLine = }\FunctionTok{pfcCreate} \NormalTok{(}\StringTok{"pfcDetailTextLine"}\NormalTok{).}\FunctionTok{Create} \NormalTok{(texts);}
  
  \KeywordTok{var} \NormalTok{textLines = }\FunctionTok{pfcCreate} \NormalTok{(}\StringTok{"pfcDetailTextLines"}\NormalTok{);}
  \OtherTok{textLines}\NormalTok{.}\FunctionTok{Append} \NormalTok{(textLine);}

\CommentTok{/*--------------------------------------------------------------------*\textbackslash{}    }
\CommentTok{  Set the location of the note text }
\CommentTok{\textbackslash{}*--------------------------------------------------------------------*/} 
  \KeywordTok{var} \NormalTok{dwgView = }\OtherTok{selSurf}\NormalTok{.}\FunctionTok{SelView2D}\NormalTok{;}
  \KeywordTok{var} \NormalTok{outline = }\OtherTok{dwgView}\NormalTok{.}\FunctionTok{Outline}\NormalTok{;}
  \KeywordTok{var} \NormalTok{textPos = }\OtherTok{outline}\NormalTok{.}\FunctionTok{Item} \NormalTok{(}\DecValTok{1}\NormalTok{);}
  
  \CommentTok{// Force the note to be slightly beyond the view outline boundary}
 \OtherTok{textPos}\NormalTok{.}\FunctionTok{Set} \NormalTok{(}\DecValTok{0}\NormalTok{, }\OtherTok{textPos}\NormalTok{.}\FunctionTok{Item} \NormalTok{(}\DecValTok{0}\NormalTok{) + }\FloatTok{0.25} \NormalTok{* (}\OtherTok{textPos}\NormalTok{.}\FunctionTok{Item} \NormalTok{(}\DecValTok{0}\NormalTok{) - }
                        \OtherTok{outline}\NormalTok{.}\FunctionTok{Item} \NormalTok{(}\DecValTok{0}\NormalTok{).}\FunctionTok{Item}\NormalTok{(}\DecValTok{0}\NormalTok{)));}
 \OtherTok{textPos}\NormalTok{.}\FunctionTok{Set} \NormalTok{(}\DecValTok{1}\NormalTok{, }\OtherTok{textPos}\NormalTok{.}\FunctionTok{Item} \NormalTok{(}\DecValTok{1}\NormalTok{) + }\FloatTok{0.25} \NormalTok{* (}\OtherTok{textPos}\NormalTok{.}\FunctionTok{Item} \NormalTok{(}\DecValTok{1}\NormalTok{) - }
                        \OtherTok{outline}\NormalTok{.}\FunctionTok{Item} \NormalTok{(}\DecValTok{0}\NormalTok{).}\FunctionTok{Item}\NormalTok{(}\DecValTok{1}\NormalTok{)));}
 
 \KeywordTok{var} \NormalTok{position = }\FunctionTok{pfcCreate} \NormalTok{(}\StringTok{"pfcFreeAttachment"}\NormalTok{).}\FunctionTok{Create} \NormalTok{(textPos);}
 \OtherTok{position}\NormalTok{.}\FunctionTok{View} \NormalTok{= dwgView;}
 
\CommentTok{/*--------------------------------------------------------------------*\textbackslash{}    }
\CommentTok{  Set the attachment for the note leader}
\CommentTok{\textbackslash{}*--------------------------------------------------------------------*/} 
 \KeywordTok{var} \NormalTok{leaderToSurf = }\FunctionTok{pfcCreate} \NormalTok{(}\StringTok{"pfcParametricAttachment"}\NormalTok{).}\FunctionTok{Create} \NormalTok{(selSurf);}

\CommentTok{/*--------------------------------------------------------------------*\textbackslash{}    }
\CommentTok{  Set the attachment structure}
\CommentTok{\textbackslash{}*--------------------------------------------------------------------*/} 
 \KeywordTok{var} \NormalTok{allAttachments = }\FunctionTok{pfcCreate} \NormalTok{(}\StringTok{"pfcDetailLeaders"}\NormalTok{).}\FunctionTok{Create} \NormalTok{();}
 \OtherTok{allAttachments}\NormalTok{.}\FunctionTok{ItemAttachment} \NormalTok{= position;}
 \OtherTok{allAttachments}\NormalTok{.}\FunctionTok{Leaders} \NormalTok{= }\FunctionTok{pfcCreate} \NormalTok{(}\StringTok{"pfcAttachments"}\NormalTok{);}
 \OtherTok{allAttachments}\NormalTok{.}\OtherTok{Leaders}\NormalTok{.}\FunctionTok{Append} \NormalTok{(leaderToSurf);}
    
\CommentTok{/*--------------------------------------------------------------------*\textbackslash{} }
\CommentTok{  Allocate a note description, and set its properties}
\CommentTok{\textbackslash{}*--------------------------------------------------------------------*/}    
 \KeywordTok{var} \NormalTok{instrs = }\FunctionTok{pfcCreate} \NormalTok{(}\StringTok{"pfcDetailNoteInstructions"}\NormalTok{).}\FunctionTok{Create} \NormalTok{(textLines);}
 
 \OtherTok{instrs}\NormalTok{.}\FunctionTok{Leader} \NormalTok{= allAttachments;}
   
\CommentTok{/*--------------------------------------------------------------------*\textbackslash{}    }
\CommentTok{  Create the note}
\CommentTok{\textbackslash{}*--------------------------------------------------------------------*/}    
 \KeywordTok{var} \NormalTok{note = }\OtherTok{drawing}\NormalTok{.}\FunctionTok{CreateDetailItem} \NormalTok{(instrs);}
 
\CommentTok{/*--------------------------------------------------------------------*\textbackslash{}    }
\CommentTok{  Display the note}
\CommentTok{\textbackslash{}*--------------------------------------------------------------------*/}  
 \OtherTok{note}\NormalTok{.}\FunctionTok{Show} \NormalTok{(); }
\NormalTok{\}}

\CommentTok{/*====================================================================*\textbackslash{}}
 \NormalTok{FUNCTION : }\FunctionTok{placeSymInst}\NormalTok{() }
 \NormalTok{PURPOSE  :  Place a CG symbol }\KeywordTok{with} \NormalTok{no leaders at a specified location  }
\NormalTok{\textbackslash{}*====================================================================*}\OtherTok{/}
\KeywordTok{function} \FunctionTok{placeSymInst}\NormalTok{()}
\NormalTok{\{}
  \KeywordTok{if} \NormalTok{(!}\FunctionTok{pfcIsWindows}\NormalTok{())}
    \OtherTok{netscape}\NormalTok{.}\OtherTok{security}\NormalTok{.}\OtherTok{PrivilegeManager}\NormalTok{.}\FunctionTok{enablePrivilege}\NormalTok{(}\StringTok{"UniversalXPConnect"}\NormalTok{); }
 \CommentTok{/*--------------------------------------------------------------------*\textbackslash{} }
\CommentTok{   Get the current drawing}
\CommentTok{ \textbackslash{}*--------------------------------------------------------------------*/}
  \KeywordTok{var} \NormalTok{session = }\FunctionTok{pfcGetProESession} \NormalTok{();}
  \KeywordTok{var} \NormalTok{drawing = }\OtherTok{session}\NormalTok{.}\FunctionTok{CurrentModel}\NormalTok{;}
  
  \KeywordTok{if} \NormalTok{(}\OtherTok{drawing}\NormalTok{.}\FunctionTok{Type} \NormalTok{!= }\FunctionTok{pfcCreate} \NormalTok{(}\StringTok{"pfcModelType"}\NormalTok{).}\FunctionTok{MDL_DRAWING}\NormalTok{)}
    \KeywordTok{throw} \KeywordTok{new} \FunctionTok{Error} \NormalTok{(}\DecValTok{0}\NormalTok{, }\StringTok{"Current model is not a drawing"}\NormalTok{);}
  
\CommentTok{/*--------------------------------------------------------------------*\textbackslash{}  }
\CommentTok{  Retrieve the symbol definition from the system}
\CommentTok{\textbackslash{}*--------------------------------------------------------------------*/}    
  \KeywordTok{var} \NormalTok{symDef = }\OtherTok{drawing}\NormalTok{.}\FunctionTok{RetrieveSymbolDefinition} \NormalTok{(}\StringTok{"CG"}\NormalTok{, }
                         \KeywordTok{void} \KeywordTok{null}\NormalTok{, }\KeywordTok{void} \KeywordTok{null}\NormalTok{, }\KeywordTok{void} \KeywordTok{null}\NormalTok{);}
  
\CommentTok{/*--------------------------------------------------------------------*\textbackslash{} }
\CommentTok{  Select the locations for the symbol}
\CommentTok{\textbackslash{}*--------------------------------------------------------------------*/}
  \KeywordTok{var} \NormalTok{browserSize = }\OtherTok{session}\NormalTok{.}\OtherTok{CurrentWindow}\NormalTok{.}\FunctionTok{GetBrowserSize}\NormalTok{();}
  \OtherTok{session}\NormalTok{.}\OtherTok{CurrentWindow}\NormalTok{.}\FunctionTok{SetBrowserSize} \NormalTok{(}\FloatTok{0.0}\NormalTok{);}
  
  \KeywordTok{var} \NormalTok{stop = }\KeywordTok{false}\NormalTok{;}
  \KeywordTok{var} \NormalTok{points = }\FunctionTok{pfcCreate} \NormalTok{(}\StringTok{"pfcPoint3Ds"}\NormalTok{);}
  \KeywordTok{while} \NormalTok{(!stop)}
    \NormalTok{\{}
      \KeywordTok{var} \NormalTok{mouse = }
    \OtherTok{session}\NormalTok{.}\FunctionTok{UIGetNextMousePick} \NormalTok{(}\FunctionTok{pfcCreate} \NormalTok{(}\StringTok{"pfcMouseButton"}\NormalTok{).}\FunctionTok{MouseButton_nil}\NormalTok{);}
      
      \KeywordTok{if} \NormalTok{(}\OtherTok{mouse}\NormalTok{.}\FunctionTok{SelectedButton} \NormalTok{== }
      \FunctionTok{pfcCreate} \NormalTok{(}\StringTok{"pfcMouseButton"}\NormalTok{).}\FunctionTok{MOUSE_BTN_LEFT}\NormalTok{)}
    \NormalTok{\{}
      \OtherTok{points}\NormalTok{.}\FunctionTok{Append} \NormalTok{(}\OtherTok{mouse}\NormalTok{.}\FunctionTok{Position}\NormalTok{);}
    \NormalTok{\}}
      \KeywordTok{else}
    \NormalTok{stop = }\KeywordTok{true}\NormalTok{;}
    \NormalTok{\}}
  
  \OtherTok{session}\NormalTok{.}\OtherTok{CurrentWindow}\NormalTok{.}\FunctionTok{SetBrowserSize} \NormalTok{(browserSize);}
 
\CommentTok{/*--------------------------------------------------------------------*\textbackslash{}  }
\CommentTok{  Allocate the symbol instance instructions }
\CommentTok{\textbackslash{}*--------------------------------------------------------------------*/} 
  \KeywordTok{var} \NormalTok{instrs = }
    \FunctionTok{pfcCreate} \NormalTok{(}\StringTok{"pfcDetailSymbolInstInstructions"}\NormalTok{).}\FunctionTok{Create} \NormalTok{(symDef); }
  \KeywordTok{var} \NormalTok{position = }\FunctionTok{pfcCreate} \NormalTok{(}\StringTok{"pfcFreeAttachment"}\NormalTok{).}\FunctionTok{Create} \NormalTok{(}\OtherTok{points}\NormalTok{.}\FunctionTok{Item} \NormalTok{(}\DecValTok{0}\NormalTok{));}
  \KeywordTok{var} \NormalTok{allAttachments = }\FunctionTok{pfcCreate} \NormalTok{(}\StringTok{"pfcDetailLeaders"}\NormalTok{).}\FunctionTok{Create} \NormalTok{();    }
  \KeywordTok{for} \NormalTok{(i = }\DecValTok{0}\NormalTok{; i < }\OtherTok{points}\NormalTok{.}\FunctionTok{Count}\NormalTok{; i++)}
    \NormalTok{\{}
      
\CommentTok{/*--------------------------------------------------------------------*\textbackslash{}    }
\CommentTok{  Set the location of the note text }
\CommentTok{\textbackslash{}*--------------------------------------------------------------------*/} 
      \OtherTok{position}\NormalTok{.}\FunctionTok{AttachmentPoint} \NormalTok{= }\OtherTok{points}\NormalTok{.}\FunctionTok{Item} \NormalTok{(i);}
      
\CommentTok{/*--------------------------------------------------------------------*\textbackslash{}    }
\CommentTok{  Set the attachment structure}
\CommentTok{\textbackslash{}*--------------------------------------------------------------------*/} 
      \OtherTok{allAttachments}\NormalTok{.}\FunctionTok{ItemAttachment} \NormalTok{= position;}
      
      \OtherTok{instrs}\NormalTok{.}\FunctionTok{InstAttachment} \NormalTok{= allAttachments;}

\CommentTok{/*--------------------------------------------------------------------*\textbackslash{}    }
\CommentTok{  Create and display the symbol }
\CommentTok{\textbackslash{}*--------------------------------------------------------------------*/}  
      \KeywordTok{var} \NormalTok{symInst = }\OtherTok{drawing}\NormalTok{.}\FunctionTok{CreateDetailItem} \NormalTok{(instrs);}
      \OtherTok{symInst}\NormalTok{.}\FunctionTok{Show}\NormalTok{();}
    \NormalTok{\}   }
\NormalTok{\}}
 
 
\CommentTok{/*====================================================================*\textbackslash{}}
 \NormalTok{FUNCTION : }\FunctionTok{createGroup}\NormalTok{() }
 \NormalTok{PURPOSE  : Command to create a }\KeywordTok{new} \NormalTok{group }\KeywordTok{with} \NormalTok{selected items }
\NormalTok{\textbackslash{}*====================================================================*}\OtherTok{/}
\KeywordTok{function} \FunctionTok{createGroup} \NormalTok{(groupName }\CommentTok{/* string */}\NormalTok{)}
\NormalTok{\{ }
  \KeywordTok{if} \NormalTok{(!}\FunctionTok{pfcIsWindows}\NormalTok{())}
    \OtherTok{netscape}\NormalTok{.}\OtherTok{security}\NormalTok{.}\OtherTok{PrivilegeManager}\NormalTok{.}\FunctionTok{enablePrivilege}\NormalTok{(}\StringTok{"UniversalXPConnect"}\NormalTok{);    }
\CommentTok{/*--------------------------------------------------------------------*\textbackslash{}    }
\CommentTok{  Select notes, draft entities, symbol instances }
\CommentTok{\textbackslash{}*--------------------------------------------------------------------*/}     
  \KeywordTok{var} \NormalTok{session = }\FunctionTok{pfcGetProESession} \NormalTok{();}
  \KeywordTok{var} \NormalTok{selOptions = }
    \FunctionTok{pfcCreate} \NormalTok{(}\StringTok{"pfcSelectionOptions"}\NormalTok{).}\FunctionTok{Create} \NormalTok{(}\StringTok{"any_note,draft_ent,dtl_symbol"}\NormalTok{);}
  \KeywordTok{var} \NormalTok{selections = }\OtherTok{session}\NormalTok{.}\FunctionTok{Select} \NormalTok{(selOptions, }\KeywordTok{void} \KeywordTok{null}\NormalTok{);}
  
  \KeywordTok{if} \NormalTok{(selections == }\KeywordTok{void} \KeywordTok{null} \NormalTok{|| }\OtherTok{selections}\NormalTok{.}\FunctionTok{Count} \NormalTok{== }\DecValTok{0}\NormalTok{)}
    \KeywordTok{return}\NormalTok{;}
         
\CommentTok{/*--------------------------------------------------------------------*\textbackslash{}    }
\CommentTok{  Allocate and fill a sequence with the detail item handles }
\CommentTok{\textbackslash{}*--------------------------------------------------------------------*/}     
  \KeywordTok{var} \NormalTok{items = }\FunctionTok{pfcCreate} \NormalTok{(}\StringTok{"pfcDetailItems"}\NormalTok{);}
  
  \KeywordTok{for} \NormalTok{(i = }\DecValTok{0}\NormalTok{; i < }\OtherTok{selections}\NormalTok{.}\FunctionTok{Count}\NormalTok{; i ++)}
    \NormalTok{\{}
      \OtherTok{items}\NormalTok{.}\FunctionTok{Append} \NormalTok{(}\OtherTok{selections}\NormalTok{.}\FunctionTok{Item} \NormalTok{(i).}\FunctionTok{SelItem}\NormalTok{);}
    \NormalTok{\}}
    
\CommentTok{/*--------------------------------------------------------------------*\textbackslash{}    }
\CommentTok{  Get the drawing which will own the group}
\CommentTok{\textbackslash{}*--------------------------------------------------------------------*/}     
  \KeywordTok{var} \NormalTok{drawing = }\OtherTok{items}\NormalTok{.}\FunctionTok{Item} \NormalTok{(}\DecValTok{0}\NormalTok{).}\FunctionTok{DBParent}\NormalTok{;}

\CommentTok{/*--------------------------------------------------------------------*\textbackslash{}    }
\CommentTok{  Allocate group data and set the group items}
\CommentTok{\textbackslash{}*--------------------------------------------------------------------*/}
  \KeywordTok{var} \NormalTok{instrs = }
    \FunctionTok{pfcCreate} \NormalTok{(}\StringTok{"pfcDetailGroupInstructions"}\NormalTok{).}\FunctionTok{Create} \NormalTok{(groupName, items);}
  
\CommentTok{/*--------------------------------------------------------------------*\textbackslash{}    }
\CommentTok{  Create the group}
\CommentTok{\textbackslash{}*--------------------------------------------------------------------*/}     
  \OtherTok{drawing}\NormalTok{.}\FunctionTok{CreateDetailItem} \NormalTok{(instrs);}
\NormalTok{\}}

\CommentTok{/*====================================================================*\textbackslash{}}
\NormalTok{FUNCTION : }\FunctionTok{createTableOfPoints}\NormalTok{() }
\NormalTok{PURPOSE  : Command to create a table of points}
\NormalTok{\textbackslash{}*====================================================================*}\OtherTok{/}
\KeywordTok{function} \FunctionTok{createTableOfPoints}\NormalTok{() }
\NormalTok{\{}
  \KeywordTok{if} \NormalTok{(!}\FunctionTok{pfcIsWindows}\NormalTok{())}
    \OtherTok{netscape}\NormalTok{.}\OtherTok{security}\NormalTok{.}\OtherTok{PrivilegeManager}\NormalTok{.}\FunctionTok{enablePrivilege}\NormalTok{(}\StringTok{"UniversalXPConnect"}\NormalTok{);   }
  
  \KeywordTok{var} \NormalTok{widths = }\KeywordTok{new} \FunctionTok{Array} \NormalTok{();}
  \NormalTok{widths [}\DecValTok{0}\NormalTok{] = }\FloatTok{8.0}\NormalTok{;}
  \NormalTok{widths [}\DecValTok{1}\NormalTok{] = }\FloatTok{10.0}\NormalTok{;}
  \NormalTok{widths [}\DecValTok{2}\NormalTok{] = }\FloatTok{10.0}\NormalTok{;}
  \NormalTok{widths [}\DecValTok{3}\NormalTok{] = }\FloatTok{10.0}\NormalTok{;}
   
\CommentTok{/*--------------------------------------------------------------------*\textbackslash{}}
  \NormalTok{Select a coordinate }\OtherTok{system}\NormalTok{. }\FunctionTok{This} \FunctionTok{defines} \FunctionTok{the} \FunctionTok{model} \NormalTok{(the top one}
  \KeywordTok{in} \NormalTok{that view), and the reference }\KeywordTok{for} \NormalTok{the datum point }\OtherTok{positions}\NormalTok{.}
\NormalTok{\textbackslash{}*--------------------------------------------------------------------*}\OtherTok{/}
  \KeywordTok{var} \NormalTok{session = }\FunctionTok{pfcGetProESession} \NormalTok{();}
  
  \OtherTok{session}\NormalTok{.}\OtherTok{CurrentWindow}\NormalTok{.}\FunctionTok{SetBrowserSize} \NormalTok{(}\FloatTok{0.0}\NormalTok{);}
  
  \KeywordTok{var} \NormalTok{selOptions = }\FunctionTok{pfcCreate} \NormalTok{(}\StringTok{"pfcSelectionOptions"}\NormalTok{).}\FunctionTok{Create}\NormalTok{(}\StringTok{"csys"}\NormalTok{);}
  \OtherTok{selOptions}\NormalTok{.}\FunctionTok{MaxNumSels} \NormalTok{= }\DecValTok{1}\NormalTok{;}
  \KeywordTok{var} \NormalTok{selections = }\OtherTok{session}\NormalTok{.}\FunctionTok{Select} \NormalTok{(selOptions, }\KeywordTok{void} \KeywordTok{null}\NormalTok{);}
  
  \KeywordTok{if} \NormalTok{(selections == }\KeywordTok{void} \KeywordTok{null} \NormalTok{|| }\OtherTok{selections}\NormalTok{.}\FunctionTok{Count} \NormalTok{== }\DecValTok{0}\NormalTok{)}
    \KeywordTok{return}\NormalTok{;}
  
\CommentTok{/*--------------------------------------------------------------------*\textbackslash{}    }
\CommentTok{  Extract the csys handle, and assembly path, and view handle.}
\CommentTok{\textbackslash{}*--------------------------------------------------------------------*/} 
  \KeywordTok{var} \NormalTok{selItem = }\OtherTok{selections}\NormalTok{.}\FunctionTok{Item} \NormalTok{(}\DecValTok{0}\NormalTok{).}\FunctionTok{SelItem}\NormalTok{;}
  \KeywordTok{var} \NormalTok{selPath = }\OtherTok{selections}\NormalTok{.}\FunctionTok{Item} \NormalTok{(}\DecValTok{0}\NormalTok{).}\FunctionTok{Path}\NormalTok{;  }
  \KeywordTok{var} \NormalTok{selView = }\OtherTok{selections}\NormalTok{.}\FunctionTok{Item} \NormalTok{(}\DecValTok{0}\NormalTok{).}\FunctionTok{SelView2D}\NormalTok{;}
  
  \KeywordTok{if} \NormalTok{(selView == }\KeywordTok{void} \KeywordTok{null}\NormalTok{)}
    \KeywordTok{throw} \KeywordTok{new} \FunctionTok{Error} \NormalTok{(}\DecValTok{0}\NormalTok{, }\StringTok{"Must select coordinate system from a drawing view."}\NormalTok{);}
  
  \KeywordTok{var} \NormalTok{drawing = }\OtherTok{selView}\NormalTok{.}\FunctionTok{DBParent}\NormalTok{;}
 
\CommentTok{/*--------------------------------------------------------------------*\textbackslash{}    }
\CommentTok{  Extract the csys location (property CoordSys from class pfcCoordSystem)}
\CommentTok{\textbackslash{}*--------------------------------------------------------------------*/}
  \KeywordTok{var} \NormalTok{csysTransf = }\OtherTok{selItem}\NormalTok{.}\FunctionTok{CoordSys}\NormalTok{;}
  \OtherTok{csysTransf}\NormalTok{.}\FunctionTok{Invert} \NormalTok{();}
  
\CommentTok{/*--------------------------------------------------------------------*\textbackslash{}    }
\CommentTok{  Extract the cys name}
\CommentTok{\textbackslash{}*--------------------------------------------------------------------*/}
  \KeywordTok{var} \NormalTok{csysName = }\OtherTok{selItem}\NormalTok{.}\FunctionTok{GetName}\NormalTok{();    }
 
\CommentTok{/*--------------------------------------------------------------------*\textbackslash{}    }
\CommentTok{  Get the root solid, and the transform from the root to the}
\CommentTok{  component owning the csys}
\CommentTok{\textbackslash{}*--------------------------------------------------------------------*/}
  
  \KeywordTok{var} \NormalTok{asmTransf = }\KeywordTok{void} \KeywordTok{null}\NormalTok{;}
  \KeywordTok{var} \NormalTok{rootSolid = }\OtherTok{selItem}\NormalTok{.}\FunctionTok{DBParent}\NormalTok{;}
  \KeywordTok{if} \NormalTok{(selPath != }\KeywordTok{void} \KeywordTok{null}\NormalTok{)}
    \NormalTok{\{}
      \NormalTok{rootSolid = }\OtherTok{selPath}\NormalTok{.}\FunctionTok{Root}\NormalTok{;}
      \NormalTok{asmTransf = }\OtherTok{selPath}\NormalTok{.}\FunctionTok{GetTransform}\NormalTok{(}\KeywordTok{false}\NormalTok{);   }
    \NormalTok{\}}

\CommentTok{/*--------------------------------------------------------------------*\textbackslash{} }
\CommentTok{  Get a list of datum points in the model}
\CommentTok{\textbackslash{}*--------------------------------------------------------------------*/}
  
  \KeywordTok{var} \NormalTok{points = }\OtherTok{rootSolid}\NormalTok{.}\FunctionTok{ListItems} \NormalTok{(}
                    \FunctionTok{pfcCreate} \NormalTok{(}\StringTok{"pfcModelItemType"}\NormalTok{).}\FunctionTok{ITEM_POINT}\NormalTok{);}
  
  \KeywordTok{if} \NormalTok{(points == }\KeywordTok{void} \KeywordTok{null} \NormalTok{|| }\OtherTok{points}\NormalTok{.}\FunctionTok{Count} \NormalTok{== }\DecValTok{0}\NormalTok{)}
    \KeywordTok{return}\NormalTok{;}

\CommentTok{/*--------------------------------------------------------------------*\textbackslash{}    }
\CommentTok{  Set the table position}
\CommentTok{\textbackslash{}*--------------------------------------------------------------------*/} 
  \KeywordTok{var} \NormalTok{location = }\FunctionTok{pfcCreate} \NormalTok{(}\StringTok{"pfcPoint3D"}\NormalTok{);}
  \OtherTok{location}\NormalTok{.}\FunctionTok{Set} \NormalTok{(}\DecValTok{0}\NormalTok{, }\FloatTok{200.0}\NormalTok{);}
  \OtherTok{location}\NormalTok{.}\FunctionTok{Set} \NormalTok{(}\DecValTok{1}\NormalTok{, }\FloatTok{600.0}\NormalTok{);}
  \OtherTok{location}\NormalTok{.}\FunctionTok{Set} \NormalTok{(}\DecValTok{2}\NormalTok{, }\FloatTok{0.0}\NormalTok{);}
       
\CommentTok{/*--------------------------------------------------------------------*\textbackslash{}    }
\CommentTok{  Setup the table creation instructions}
\CommentTok{\textbackslash{}*--------------------------------------------------------------------*/} 
  \KeywordTok{var} \NormalTok{instrs = }
    \FunctionTok{pfcCreate} \NormalTok{(}\StringTok{"pfcTableCreateInstructions"}\NormalTok{).}\FunctionTok{Create} \NormalTok{(location);   }
  
  \OtherTok{instrs}\NormalTok{.}\FunctionTok{SizeType} \NormalTok{=}
    \FunctionTok{pfcCreate} \NormalTok{(}\StringTok{"pfcTableSizeType"}\NormalTok{).}\FunctionTok{TABLESIZE_BY_NUM_CHARS}\NormalTok{;}
  
  \KeywordTok{var} \NormalTok{columnInfo = }\FunctionTok{pfcCreate} \NormalTok{(}\StringTok{"pfcColumnCreateOptions"}\NormalTok{);}
  
  \KeywordTok{for} \NormalTok{(i = }\DecValTok{0}\NormalTok{; i < }\OtherTok{widths}\NormalTok{.}\FunctionTok{length}\NormalTok{; i++)}
    \NormalTok{\{}
      \KeywordTok{var} \NormalTok{column = }\FunctionTok{pfcCreate} \NormalTok{(}\StringTok{"pfcColumnCreateOption"}\NormalTok{).}\FunctionTok{Create} \NormalTok{(}
                  \FunctionTok{pfcCreate} \NormalTok{(}\StringTok{"pfcColumnJustification"}\NormalTok{).}\FunctionTok{COL_JUSTIFY_LEFT}\NormalTok{,}
          \NormalTok{widths [i]);}
      \OtherTok{columnInfo}\NormalTok{.}\FunctionTok{Append} \NormalTok{(column);}
    \NormalTok{\}}
  \OtherTok{instrs}\NormalTok{.}\FunctionTok{ColumnData} \NormalTok{= columnInfo;}
  
  \KeywordTok{var} \NormalTok{rowInfo = }\FunctionTok{pfcCreate} \NormalTok{(}\StringTok{"realseq"}\NormalTok{);}
  
  \KeywordTok{for} \NormalTok{(i = }\DecValTok{0}\NormalTok{; i < }\OtherTok{points}\NormalTok{.}\FunctionTok{Count} \NormalTok{+ }\DecValTok{2}\NormalTok{; i++)}
    \NormalTok{\{}
      \OtherTok{rowInfo}\NormalTok{.}\FunctionTok{Append} \NormalTok{(}\FloatTok{1.0}\NormalTok{);}
    \NormalTok{\}}
  \OtherTok{instrs}\NormalTok{.}\FunctionTok{RowHeights} \NormalTok{= rowInfo;}

\CommentTok{/*--------------------------------------------------------------------*\textbackslash{}    }
\CommentTok{  Create the table}
\CommentTok{\textbackslash{}*--------------------------------------------------------------------*/}    
  \KeywordTok{var} \NormalTok{dwgTable = }\OtherTok{drawing}\NormalTok{.}\FunctionTok{CreateTable} \NormalTok{(instrs);}
  
\CommentTok{/*--------------------------------------------------------------------*\textbackslash{}    }
\CommentTok{  Merge the top row cells to form the header}
\CommentTok{\textbackslash{}*--------------------------------------------------------------------*/} 
  \KeywordTok{var} \NormalTok{topLeft = }\FunctionTok{pfcCreate} \NormalTok{(}\StringTok{"pfcTableCell"}\NormalTok{).}\FunctionTok{Create} \NormalTok{(}\DecValTok{1}\NormalTok{, }\DecValTok{1}\NormalTok{);}
  \KeywordTok{var} \NormalTok{bottomRight = }\FunctionTok{pfcCreate} \NormalTok{(}\StringTok{"pfcTableCell"}\NormalTok{).}\FunctionTok{Create} \NormalTok{(}\DecValTok{1}\NormalTok{, }\DecValTok{4}\NormalTok{);}
  \OtherTok{dwgTable}\NormalTok{.}\FunctionTok{MergeRegion} \NormalTok{(topLeft, bottomRight, }\KeywordTok{void} \KeywordTok{null}\NormalTok{);}

\CommentTok{/*--------------------------------------------------------------------*\textbackslash{}    }
\CommentTok{  Write header text specifying model and csys}
\CommentTok{\textbackslash{}*--------------------------------------------------------------------*/}
  \FunctionTok{writeTextInCell} \NormalTok{(dwgTable, }\DecValTok{1}\NormalTok{, }\DecValTok{1}\NormalTok{, }
           \StringTok{"Datum points for "}\NormalTok{+}\OtherTok{rootSolid}\NormalTok{.}\FunctionTok{FileName} \NormalTok{+ }\StringTok{" w.r.t. csys "}\NormalTok{+csysName);    }

\CommentTok{/*--------------------------------------------------------------------*\textbackslash{}    }
\CommentTok{  Add subheadings to columns}
\CommentTok{\textbackslash{}*--------------------------------------------------------------------*/}
  \FunctionTok{writeTextInCell} \NormalTok{(dwgTable, }\DecValTok{2}\NormalTok{, }\DecValTok{1}\NormalTok{, }\StringTok{"Point"}\NormalTok{);}
  \FunctionTok{writeTextInCell} \NormalTok{(dwgTable, }\DecValTok{2}\NormalTok{, }\DecValTok{2}\NormalTok{, }\StringTok{"X"}\NormalTok{);}
  \FunctionTok{writeTextInCell} \NormalTok{(dwgTable, }\DecValTok{2}\NormalTok{, }\DecValTok{3}\NormalTok{, }\StringTok{"Y"}\NormalTok{);}
  \FunctionTok{writeTextInCell} \NormalTok{(dwgTable, }\DecValTok{2}\NormalTok{, }\DecValTok{4}\NormalTok{, }\StringTok{"Z"}\NormalTok{);}

\CommentTok{/*--------------------------------------------------------------------*\textbackslash{} }
\CommentTok{  For each datum point...}
\CommentTok{\textbackslash{}*--------------------------------------------------------------------*/}    
  \KeywordTok{for}\NormalTok{(p=}\DecValTok{0}\NormalTok{; p<}\OtherTok{points}\NormalTok{.}\FunctionTok{Count}\NormalTok{; p++)}
    \NormalTok{\{ }
      \KeywordTok{var} \NormalTok{point = }\OtherTok{points}\NormalTok{.}\FunctionTok{Item} \NormalTok{(p);}
\CommentTok{/*--------------------------------------------------------------------*\textbackslash{}    }
\CommentTok{  Add the point name to column 1}
\CommentTok{\textbackslash{}*--------------------------------------------------------------------*/}   
      \FunctionTok{writeTextInCell} \NormalTok{(dwgTable, p}\DecValTok{+3}\NormalTok{, }\DecValTok{1}\NormalTok{, }\OtherTok{point}\NormalTok{.}\FunctionTok{GetName}\NormalTok{());}
      
\CommentTok{/*--------------------------------------------------------------------*\textbackslash{}   }
\CommentTok{ Transform the location w.r.t to the csys}
\CommentTok{\textbackslash{}*--------------------------------------------------------------------*/}
      \KeywordTok{var} \NormalTok{trfPoint = }\OtherTok{point}\NormalTok{.}\FunctionTok{Point}\NormalTok{;}
      \KeywordTok{if} \NormalTok{(asmTransf != }\KeywordTok{void} \KeywordTok{null}\NormalTok{)}
    \NormalTok{trfPoint = }\OtherTok{asmTransf}\NormalTok{.}\FunctionTok{TransformPoint} \NormalTok{(}\OtherTok{point}\NormalTok{.}\FunctionTok{Point}\NormalTok{);}
      \NormalTok{trfPoint = }\OtherTok{csysTransf}\NormalTok{.}\FunctionTok{TransformPoint} \NormalTok{(trfPoint);}

\CommentTok{/*--------------------------------------------------------------------*\textbackslash{}   }
\CommentTok{  Add the XYZ to column 2,3,4}
\CommentTok{\textbackslash{}*--------------------------------------------------------------------*/}   
      \FunctionTok{writeTextInCell} \NormalTok{(dwgTable, p}\DecValTok{+3}\NormalTok{, }\DecValTok{2}\NormalTok{, }\OtherTok{trfPoint}\NormalTok{.}\FunctionTok{Item} \NormalTok{(}\DecValTok{0}\NormalTok{));}
      \FunctionTok{writeTextInCell} \NormalTok{(dwgTable, p}\DecValTok{+3}\NormalTok{, }\DecValTok{3}\NormalTok{, }\OtherTok{trfPoint}\NormalTok{.}\FunctionTok{Item} \NormalTok{(}\DecValTok{1}\NormalTok{));}
      \FunctionTok{writeTextInCell} \NormalTok{(dwgTable, p}\DecValTok{+3}\NormalTok{, }\DecValTok{4}\NormalTok{, }\OtherTok{trfPoint}\NormalTok{.}\FunctionTok{Item} \NormalTok{(}\DecValTok{2}\NormalTok{));   }
    \NormalTok{\}}
\CommentTok{/*--------------------------------------------------------------------*\textbackslash{}  }
\CommentTok{  Display the table}
\CommentTok{\textbackslash{}*--------------------------------------------------------------------*/}
  \OtherTok{dwgTable}\NormalTok{.}\FunctionTok{Display} \NormalTok{();}
\NormalTok{\}}
 
\CommentTok{/*====================================================================*\textbackslash{}}
\NormalTok{FUNCTION : }\FunctionTok{writeTextInCell}\NormalTok{() }
\NormalTok{PURPOSE  : Utility to add one text line to a table cell}
\NormalTok{\textbackslash{}*====================================================================*}\OtherTok{/}
\KeywordTok{function} \FunctionTok{writeTextInCell}\NormalTok{(table }\CommentTok{/* pfcTable */}\NormalTok{, row }\CommentTok{/* integer */}\NormalTok{,}
             \NormalTok{col }\CommentTok{/* integer */}\NormalTok{, text }\CommentTok{/* string */}\NormalTok{)}
\NormalTok{\{}
  \KeywordTok{var} \NormalTok{cell = }\FunctionTok{pfcCreate} \NormalTok{(}\StringTok{"pfcTableCell"}\NormalTok{).}\FunctionTok{Create} \NormalTok{(row, col);}
  \KeywordTok{var} \NormalTok{lines = }\FunctionTok{pfcCreate} \NormalTok{(}\StringTok{"stringseq"}\NormalTok{);}
  \OtherTok{lines}\NormalTok{.}\FunctionTok{Append} \NormalTok{(text);}
  \OtherTok{table}\NormalTok{.}\FunctionTok{SetText} \NormalTok{(cell, lines);}
\NormalTok{\}}

\CommentTok{/*====================================================================*\textbackslash{}}
\NormalTok{FUNCTION: }\FunctionTok{createPointDims}\NormalTok{() }
\NormalTok{PURPOSE  : Command to create dimensions to each of the models}\StringTok{' datum points}
\NormalTok{\textbackslash{}*====================================================================*}\OtherTok{/}
\KeywordTok{function} \FunctionTok{createPointDims}\NormalTok{()}
\NormalTok{\{}
  \KeywordTok{if} \NormalTok{(!}\FunctionTok{pfcIsWindows}\NormalTok{())}
    \OtherTok{netscape}\NormalTok{.}\OtherTok{security}\NormalTok{.}\OtherTok{PrivilegeManager}\NormalTok{.}\FunctionTok{enablePrivilege}\NormalTok{(}\StringTok{"UniversalXPConnect"}\NormalTok{);   }
  \KeywordTok{var} \NormalTok{hBaseline = }\KeywordTok{void} \KeywordTok{null}\NormalTok{;}
  \KeywordTok{var} \NormalTok{vBaseline = }\KeywordTok{void} \KeywordTok{null}\NormalTok{;}

\CommentTok{/*--------------------------------------------------------------------*\textbackslash{}}
  \NormalTok{Select a coordinate }\OtherTok{system}\NormalTok{. }\FunctionTok{This} \FunctionTok{defines} \FunctionTok{the} \FunctionTok{model} \NormalTok{(the top one}
  \KeywordTok{in} \NormalTok{that view), and the common attachments }\KeywordTok{for} \NormalTok{the dimensions}
\NormalTok{\textbackslash{}*--------------------------------------------------------------------*}\OtherTok{/}
  \KeywordTok{var} \NormalTok{session = }\FunctionTok{pfcGetProESession} \NormalTok{();}
  
  \OtherTok{session}\NormalTok{.}\OtherTok{CurrentWindow}\NormalTok{.}\FunctionTok{SetBrowserSize} \NormalTok{(}\FloatTok{0.0}\NormalTok{);}
  
  \KeywordTok{var} \NormalTok{selOptions = }\FunctionTok{pfcCreate} \NormalTok{(}\StringTok{"pfcSelectionOptions"}\NormalTok{).}\FunctionTok{Create}\NormalTok{(}\StringTok{"csys"}\NormalTok{);}
  \OtherTok{selOptions}\NormalTok{.}\FunctionTok{MaxNumSels} \NormalTok{= }\DecValTok{1}\NormalTok{;}
  \KeywordTok{var} \NormalTok{selections = }\OtherTok{session}\NormalTok{.}\FunctionTok{Select} \NormalTok{(selOptions, }\KeywordTok{void} \KeywordTok{null}\NormalTok{);}
  
  \KeywordTok{if} \NormalTok{(selections == }\KeywordTok{void} \KeywordTok{null} \NormalTok{|| }\OtherTok{selections}\NormalTok{.}\FunctionTok{Count} \NormalTok{== }\DecValTok{0}\NormalTok{)}
    \KeywordTok{return}\NormalTok{;}
        
\CommentTok{/*--------------------------------------------------------------------*\textbackslash{}    }
\CommentTok{  Extract the csys handle, and assembly path, and view handle.}
\CommentTok{\textbackslash{}*--------------------------------------------------------------------*/}
  \KeywordTok{var} \NormalTok{csysSel = }\OtherTok{selections}\NormalTok{.}\FunctionTok{Item} \NormalTok{(}\DecValTok{0}\NormalTok{);}
  \KeywordTok{var} \NormalTok{selItem = }\OtherTok{csysSel}\NormalTok{.}\FunctionTok{SelItem}\NormalTok{;}
  \KeywordTok{var} \NormalTok{selPath = }\OtherTok{csysSel}\NormalTok{.}\FunctionTok{Path}\NormalTok{;  }
  \KeywordTok{var} \NormalTok{selView = }\OtherTok{csysSel}\NormalTok{.}\FunctionTok{SelView2D}\NormalTok{;}
  \KeywordTok{var} \NormalTok{selPos = }\OtherTok{csysSel}\NormalTok{.}\FunctionTok{Point}\NormalTok{;}
  
  \KeywordTok{if} \NormalTok{(selView == }\KeywordTok{void} \KeywordTok{null}\NormalTok{)}
    \KeywordTok{throw} \KeywordTok{new} \FunctionTok{Error} \NormalTok{(}\DecValTok{0}\NormalTok{, }\StringTok{"Must select coordinate system from a drawing view."}\NormalTok{);}
  
  \KeywordTok{var} \NormalTok{drawing = }\OtherTok{selView}\NormalTok{.}\FunctionTok{DBParent}\NormalTok{;}
    
\CommentTok{/*--------------------------------------------------------------------*\textbackslash{}    }
\CommentTok{  Get the root solid, and the transform from the root to the}
\CommentTok{  component owning the csys}
\CommentTok{\textbackslash{}*--------------------------------------------------------------------*/}

  \KeywordTok{var} \NormalTok{asmTransf = }\KeywordTok{void} \KeywordTok{null}\NormalTok{;}
  \KeywordTok{var} \NormalTok{rootSolid = }\OtherTok{selItem}\NormalTok{.}\FunctionTok{DBParent}\NormalTok{;}
  \KeywordTok{if} \NormalTok{(selPath != }\KeywordTok{null}\NormalTok{)}
    \NormalTok{\{}
      \NormalTok{rootSolid = }\OtherTok{selPath}\NormalTok{.}\FunctionTok{Root}\NormalTok{;}
      \NormalTok{asmTransf = }\OtherTok{selPath}\NormalTok{.}\FunctionTok{GetTransform}\NormalTok{(}\KeywordTok{true}\NormalTok{);   }
    \NormalTok{\}}
  
\CommentTok{/*--------------------------------------------------------------------*\textbackslash{} }
\CommentTok{  Get a list of datum points in the model}
\CommentTok{\textbackslash{}*--------------------------------------------------------------------*/}
  
  \KeywordTok{var} \NormalTok{points = }
    \OtherTok{rootSolid}\NormalTok{.}\FunctionTok{ListItems} \NormalTok{(}\FunctionTok{pfcCreate} \NormalTok{(}\StringTok{"pfcModelItemType"}\NormalTok{).}\FunctionTok{ITEM_POINT}\NormalTok{);}
  
  \KeywordTok{if} \NormalTok{(points == }\KeywordTok{void} \KeywordTok{null} \NormalTok{|| }\OtherTok{points}\NormalTok{.}\FunctionTok{Count} \NormalTok{== }\DecValTok{0}\NormalTok{)}
    \KeywordTok{return}\NormalTok{;}
  
\CommentTok{/*--------------------------------------------------------------------*\textbackslash{}}
  \NormalTok{Calculate where the csys is located on the drawing}
\NormalTok{\textbackslash{}*--------------------------------------------------------------------*}\OtherTok{/ }
\OtherTok{  }
\OtherTok{  var csysPos = selPos;}
\OtherTok{  if }\FloatTok{(}\OtherTok{asmTransf != void null}\FloatTok{)}
\OtherTok{    \{}
\OtherTok{      csysPos = asmTransf.TransformPoint }\FloatTok{(}\OtherTok{selPos}\FloatTok{)}\OtherTok{;}
\OtherTok{    \}}
\OtherTok{  var viewTransf = selView.GetTransform}\FloatTok{()}\OtherTok{;}
\OtherTok{  csysPos = viewTransf.TransformPoint }\FloatTok{(}\OtherTok{csysPos}\FloatTok{)}\OtherTok{;}
\OtherTok{  }
\OtherTok{  var csys3DPos = pfcCreate }\FloatTok{(}\OtherTok{"pfcVector2D"}\FloatTok{)}\OtherTok{;}
\OtherTok{  }
\OtherTok{  csys3DPos.Set }\FloatTok{(}\OtherTok{0, csysPos.Item }\FloatTok{(}\OtherTok{0}\FloatTok{))}\OtherTok{;}
\OtherTok{  csys3DPos.Set }\FloatTok{(}\OtherTok{1, csysPos.Item }\FloatTok{(}\OtherTok{1}\FloatTok{))}\OtherTok{;  }
\OtherTok{  }
\OtherTok{/}\NormalTok{*--------------------------------------------------------------------*\textbackslash{}    }
  \NormalTok{Get the view outline}
\NormalTok{\textbackslash{}*--------------------------------------------------------------------*}\OtherTok{/    }
\OtherTok{  var outline = selView.Outline;}
\OtherTok{  }
\OtherTok{/}\NormalTok{*--------------------------------------------------------------------*\textbackslash{}    }
  \NormalTok{Allocate the attachment arrays}
\NormalTok{\textbackslash{}*--------------------------------------------------------------------*}\OtherTok{/    }
\OtherTok{  var senses = pfcCreate }\FloatTok{(}\OtherTok{"pfcDimensionSenses"}\FloatTok{)}\OtherTok{;}
\OtherTok{  var attachments = pfcCreate }\FloatTok{(}\OtherTok{"pfcSelections"}\FloatTok{)}\OtherTok{;}
\OtherTok{    }
\OtherTok{/}\NormalTok{*--------------------------------------------------------------------*\textbackslash{}    }
  \NormalTok{For each datum }\OtherTok{point}\NormalTok{...}
\NormalTok{\textbackslash{}*--------------------------------------------------------------------*}\OtherTok{/    }
\OtherTok{  for}\FloatTok{(}\OtherTok{var p=0; p <points.Count; p}\FloatTok{++)}
\OtherTok{    \{}
\OtherTok{      }
\OtherTok{/}\NormalTok{*--------------------------------------------------------------------*\textbackslash{}  }
  \NormalTok{Calculate the position of the point on the drawing}
\NormalTok{\textbackslash{}*--------------------------------------------------------------------*}\OtherTok{/ }
\OtherTok{      var point = points.Item }\FloatTok{(}\OtherTok{p}\FloatTok{)}\OtherTok{;}
\OtherTok{      var pntPos = point.Point;}
\OtherTok{      }
\OtherTok{      pntPos = viewTransf.TransformPoint }\FloatTok{(}\OtherTok{pntPos}\FloatTok{)}\OtherTok{;}
\OtherTok{      }
\OtherTok{/}\NormalTok{*--------------------------------------------------------------------*\textbackslash{}  }
  \NormalTok{Set up the }\StringTok{"sense"} \NormalTok{information}
\NormalTok{\textbackslash{}*--------------------------------------------------------------------*}\OtherTok{/  }
\OtherTok{      var sense1 = pfcCreate }\FloatTok{(}\OtherTok{"pfcPointDimensionSense"}\FloatTok{)}\OtherTok{.Create }\FloatTok{(}
\OtherTok{            pfcCreate }\FloatTok{(}\OtherTok{"pfcDimensionPointType"}\FloatTok{)}\OtherTok{.DIMPOINT_CENTER}\FloatTok{)}\OtherTok{;}
\OtherTok{      senses.Set }\FloatTok{(}\OtherTok{0, sense1}\FloatTok{)}\OtherTok{;}
\OtherTok{      var sense2 = pfcCreate }\FloatTok{(}\OtherTok{"pfcPointDimensionSense"}\FloatTok{)}\OtherTok{.Create }\FloatTok{(}
\OtherTok{            pfcCreate }\FloatTok{(}\OtherTok{"pfcDimensionPointType"}\FloatTok{)}\OtherTok{.DIMPOINT_CENTER}\FloatTok{)}\OtherTok{;}
\OtherTok{      senses.Set }\FloatTok{(}\OtherTok{1, sense2}\FloatTok{)}\OtherTok{;}
\OtherTok{        }
\OtherTok{/}\NormalTok{*--------------------------------------------------------------------*\textbackslash{} }
  \NormalTok{Set the attachment information}
\NormalTok{\textbackslash{}*--------------------------------------------------------------------*}\OtherTok{/}
      \KeywordTok{var} \NormalTok{pntSel = }
    \FunctionTok{pfcCreate} \NormalTok{(}\StringTok{"MpfcSelect"}\NormalTok{).}\FunctionTok{CreateModelItemSelection} \NormalTok{(point, }
                               \KeywordTok{void} \KeywordTok{null}\NormalTok{);}
      \OtherTok{pntSel}\NormalTok{.}\FunctionTok{SelView2D} \NormalTok{= selView;}
      \OtherTok{attachments}\NormalTok{.}\FunctionTok{Set} \NormalTok{(}\DecValTok{0}\NormalTok{, pntSel);}
      \OtherTok{attachments}\NormalTok{.}\FunctionTok{Set} \NormalTok{(}\DecValTok{1}\NormalTok{, csysSel);}
        
\CommentTok{/*--------------------------------------------------------------------*\textbackslash{}  }
\CommentTok{  Calculate the dim position to be just to the left of the}
\CommentTok{  drawing view, midway between the point and csys}
\CommentTok{\textbackslash{}*--------------------------------------------------------------------*/} 
      \KeywordTok{var} \NormalTok{dimPos = }\FunctionTok{pfcCreate} \NormalTok{(}\StringTok{"pfcVector2D"}\NormalTok{);}
      \OtherTok{dimPos}\NormalTok{.}\FunctionTok{Set} \NormalTok{(}\DecValTok{0}\NormalTok{, }\OtherTok{outline}\NormalTok{.}\FunctionTok{Item} \NormalTok{(}\DecValTok{0}\NormalTok{).}\FunctionTok{Item} \NormalTok{(}\DecValTok{0}\NormalTok{) - }\FloatTok{20.0}\NormalTok{);}
      \OtherTok{dimPos}\NormalTok{.}\FunctionTok{Set} \NormalTok{(}\DecValTok{1}\NormalTok{, (}\OtherTok{csysPos}\NormalTok{.}\FunctionTok{Item} \NormalTok{(}\DecValTok{1}\NormalTok{) + }\OtherTok{pntPos}\NormalTok{.}\FunctionTok{Item} \NormalTok{(}\DecValTok{1}\NormalTok{))/}\FloatTok{2.0}\NormalTok{);}
        
\CommentTok{/*--------------------------------------------------------------------*\textbackslash{}  }
\CommentTok{  Create and display the dimension}
\CommentTok{\textbackslash{}*--------------------------------------------------------------------*/}
      \KeywordTok{var} \NormalTok{createInstrs = }
    \FunctionTok{pfcCreate} \NormalTok{(}\StringTok{"pfcDrawingDimCreateInstructions"}\NormalTok{).}\FunctionTok{Create} \NormalTok{(attachments, }
                                  \NormalTok{senses,}
                                  \NormalTok{dimPos, }
                                  \FunctionTok{pfcCreate} \NormalTok{(}\StringTok{"pfcOrientationHint"}\NormalTok{).}\FunctionTok{ORIENTHINT_VERTICAL}\NormalTok{);}
      \KeywordTok{var} \NormalTok{dim = }\OtherTok{drawing}\NormalTok{.}\FunctionTok{CreateDrawingDimension} \NormalTok{(createInstrs);}
      
      \KeywordTok{var} \NormalTok{showInstrs =}
    \FunctionTok{pfcCreate} \NormalTok{(}\StringTok{"pfcDrawingDimensionShowInstructions"}\NormalTok{).}\FunctionTok{Create} \NormalTok{(selView, }
                                                                    \KeywordTok{void} \KeywordTok{null}\NormalTok{);}
      \OtherTok{dim}\NormalTok{.}\FunctionTok{Show} \NormalTok{(showInstrs);}
      
\CommentTok{/*--------------------------------------------------------------------*\textbackslash{}    }
\CommentTok{  If this is the first vertical dim, create an ordinate base}
\CommentTok{  line from it, else just convert it to ordinate}
\CommentTok{\textbackslash{}*--------------------------------------------------------------------*/}  
      \KeywordTok{if}\NormalTok{(p==}\DecValTok{0}\NormalTok{)}
    \NormalTok{\{}
      \NormalTok{vBaseline = }\OtherTok{dim}\NormalTok{.}\FunctionTok{ConvertToBaseline} \NormalTok{(csys3DPos);}
    \NormalTok{\}}
      
      \KeywordTok{else}
    \OtherTok{dim}\NormalTok{.}\FunctionTok{ConvertToOrdinate} \NormalTok{(vBaseline);}
            
\CommentTok{/*--------------------------------------------------------------------*\textbackslash{}   }
\CommentTok{  Set this dimension to be horizontal}
\CommentTok{\textbackslash{}*--------------------------------------------------------------------*/}   
      \OtherTok{createInstrs}\NormalTok{.}\FunctionTok{OrientationHint} \NormalTok{=}
    \FunctionTok{pfcCreate} \NormalTok{(}\StringTok{"pfcOrientationHint"}\NormalTok{).}\FunctionTok{ORIENTHINT_HORIZONTAL}\NormalTok{;}
\CommentTok{/*--------------------------------------------------------------------*\textbackslash{}    }
\CommentTok{  Calculate the dim position to be just to the bottom of the}
\CommentTok{  drawing view, midway between the point and csys}
\CommentTok{\textbackslash{}*--------------------------------------------------------------------*/}   
      \OtherTok{dimPos}\NormalTok{.}\FunctionTok{Set} \NormalTok{(}\DecValTok{0}\NormalTok{, (}\OtherTok{csysPos}\NormalTok{.}\FunctionTok{Item} \NormalTok{(}\DecValTok{0}\NormalTok{) + }\OtherTok{pntPos}\NormalTok{.}\FunctionTok{Item} \NormalTok{(}\DecValTok{0}\NormalTok{))/}\FloatTok{2.0}\NormalTok{);}
      \OtherTok{dimPos}\NormalTok{.}\FunctionTok{Set} \NormalTok{(}\DecValTok{1}\NormalTok{, }\OtherTok{outline}\NormalTok{.}\FunctionTok{Item} \NormalTok{(}\DecValTok{1}\NormalTok{).}\FunctionTok{Item} \NormalTok{(}\DecValTok{1}\NormalTok{) - }\FloatTok{20.0}\NormalTok{);}
      
      \OtherTok{createInstrs}\NormalTok{.}\FunctionTok{TextLocation} \NormalTok{= dimPos;}
      
\CommentTok{/*--------------------------------------------------------------------*\textbackslash{}    }
\CommentTok{  Create and display the dimension}
\CommentTok{\textbackslash{}*--------------------------------------------------------------------*/} 
      \NormalTok{dim = }\OtherTok{drawing}\NormalTok{.}\FunctionTok{CreateDrawingDimension} \NormalTok{(createInstrs);}
      \OtherTok{dim}\NormalTok{.}\FunctionTok{Show} \NormalTok{(showInstrs);}
      
\CommentTok{/*--------------------------------------------------------------------*\textbackslash{}     }
\CommentTok{  If this is the first horizontal dim, create an ordinate base line}
\CommentTok{  from it, else just convert it to ordinate}
\CommentTok{\textbackslash{}*--------------------------------------------------------------------*/}   
      \KeywordTok{if}\NormalTok{(p==}\DecValTok{0}\NormalTok{)}
    \NormalTok{\{}
      \NormalTok{hBaseline = }\OtherTok{dim}\NormalTok{.}\FunctionTok{ConvertToBaseline} \NormalTok{(csys3DPos);}
    \NormalTok{\}}
      
      \KeywordTok{else}
    \OtherTok{dim}\NormalTok{.}\FunctionTok{ConvertToOrdinate} \NormalTok{(hBaseline);}
      
    \NormalTok{\}}
\NormalTok{\}}
\end{Highlighting}
\end{Shaded}

\begin{Shaded}
\begin{Highlighting}[]
\CommentTok{/*}
\CommentTok{   HISTORY}
\CommentTok{   }
\CommentTok{14-NOV-02   J-03-38   $$1   JCN      Adapted from J-Link examples.}
\CommentTok{07-MAR-03   K-01-03   $$2   JCN      UNIX support}
\CommentTok{*/}

\CommentTok{/*====================================================================*\textbackslash{}}
\NormalTok{FUNCTION: addHoleDiameterColumns}
\NormalTok{PURPOSE:  Add all hole diameters to the family table of a }\OtherTok{model}\NormalTok{.}
\NormalTok{\textbackslash{}*====================================================================*}\OtherTok{/}
\KeywordTok{function} \FunctionTok{addHoleDiameterColumns} \NormalTok{()}
\NormalTok{\{}
  \KeywordTok{if} \NormalTok{(!}\FunctionTok{pfcIsWindows}\NormalTok{())}
    \OtherTok{netscape}\NormalTok{.}\OtherTok{security}\NormalTok{.}\OtherTok{PrivilegeManager}\NormalTok{.}\FunctionTok{enablePrivilege}\NormalTok{(}\StringTok{"UniversalXPConnect"}\NormalTok{);}
  
\CommentTok{/*------------------------------------------------------------------*\textbackslash{}}
  \NormalTok{Use the current solid }\OtherTok{model}\NormalTok{.}
\NormalTok{\textbackslash{}*------------------------------------------------------------------*}\OtherTok{/}
  \KeywordTok{var} \NormalTok{session = }\FunctionTok{pfcGetProESession} \NormalTok{();}
  \KeywordTok{var} \NormalTok{solid = }\OtherTok{session}\NormalTok{.}\FunctionTok{CurrentModel}\NormalTok{;}
  \NormalTok{modelTypeClass = }\FunctionTok{pfcCreate} \NormalTok{(}\StringTok{"pfcModelType"}\NormalTok{);}
  
  \KeywordTok{if} \NormalTok{(solid == }\KeywordTok{void} \KeywordTok{null} \NormalTok{|| (}\OtherTok{solid}\NormalTok{.}\FunctionTok{Type} \NormalTok{!= }\OtherTok{modelTypeClass}\NormalTok{.}\FunctionTok{MDL_PART} \NormalTok{&& }
                 \OtherTok{solid}\NormalTok{.}\FunctionTok{Type} \NormalTok{!= }\OtherTok{modelTypeClass}\NormalTok{.}\FunctionTok{MDL_ASSEMBLY}\NormalTok{))}
    \NormalTok{\{}
      \KeywordTok{throw} \KeywordTok{new} \FunctionTok{Error}  \NormalTok{(}\DecValTok{0}\NormalTok{, }\StringTok{"Current model is not a part or assembly."}\NormalTok{);}
    \NormalTok{\}}
     
\CommentTok{/*------------------------------------------------------------------*\textbackslash{}}
  \NormalTok{List all holes }\KeywordTok{in} \NormalTok{the solid model}
\NormalTok{\textbackslash{}*------------------------------------------------------------------*}\OtherTok{/ }
\OtherTok{  var holeFeatures = solid.ListFeaturesByType }\FloatTok{(}\OtherTok{true, }
\OtherTok{         pfcCreate }\FloatTok{(}\OtherTok{"pfcFeatureType"}\FloatTok{)}\OtherTok{.FEATTYPE_HOLE}\FloatTok{)}\OtherTok{;}
\OtherTok{  for }\FloatTok{(}\OtherTok{var ii =0; ii < holeFeatures.Count; ii}\FloatTok{++)}
\OtherTok{    \{}
\OtherTok{      var holeFeat = holeFeatures.Item}\FloatTok{(}\OtherTok{ii}\FloatTok{)}\OtherTok{;}
\OtherTok{      }
\OtherTok{/}\NormalTok{*------------------------------------------------------------------*\textbackslash{}}
  \NormalTok{List all dimensions }\KeywordTok{in} \NormalTok{the feature }
\NormalTok{\textbackslash{}*------------------------------------------------------------------*}\OtherTok{/}
      \NormalTok{dimensions = }
    \OtherTok{holeFeat}\NormalTok{.}\FunctionTok{ListSubItems}\NormalTok{(}\FunctionTok{pfcCreate} \NormalTok{(}\StringTok{"pfcModelItemType"}\NormalTok{).}\FunctionTok{ITEM_DIMENSION}\NormalTok{);}
      
      \KeywordTok{for} \NormalTok{(}\KeywordTok{var} \NormalTok{jj = }\DecValTok{0}\NormalTok{; jj < }\OtherTok{dimensions}\NormalTok{.}\FunctionTok{Count}\NormalTok{; jj++)}
    \NormalTok{\{}
      \KeywordTok{var} \NormalTok{dim = }\OtherTok{dimensions}\NormalTok{.}\FunctionTok{Item}\NormalTok{(jj);}
      
\CommentTok{/*------------------------------------------------------------------*\textbackslash{}}
  \NormalTok{Determine }\KeywordTok{if} \NormalTok{the dimension is a diameter dimension}
\NormalTok{\textbackslash{}*------------------------------------------------------------------*}\OtherTok{/          }
\OtherTok{      if }\FloatTok{(}\OtherTok{dim.DimType == pfcCreate }\FloatTok{(}\OtherTok{"pfcDimensionType"}\FloatTok{)}\OtherTok{.DIM_DIAMETER}\FloatTok{)}\OtherTok{ }
\OtherTok{        \{}
\OtherTok{/}\NormalTok{*------------------------------------------------------------------*\textbackslash{}}
  \NormalTok{Create the family table }\FunctionTok{column} \NormalTok{(from pfcFamilyMember }\KeywordTok{class}\NormalTok{)}
\NormalTok{\textbackslash{}*------------------------------------------------------------------*}\OtherTok{/            }
\OtherTok{          var dimColumn = solid.CreateDimensionColumn}\FloatTok{(}\OtherTok{dim}\FloatTok{)}\OtherTok{;}
\OtherTok{/}\NormalTok{*------------------------------------------------------------------*\textbackslash{}}
  \NormalTok{Add the column to the family }\OtherTok{table}\NormalTok{.  }\FunctionTok{Second} \FunctionTok{argument} \FunctionTok{could} \FunctionTok{be}
  \NormalTok{a sequence of pfcParamValues to use }\KeywordTok{for} \NormalTok{each family table }\OtherTok{instance}\NormalTok{.}
\NormalTok{\textbackslash{}*------------------------------------------------------------------*}\OtherTok{/        }
\OtherTok{          solid.AddColumn}\FloatTok{(}\OtherTok{dimColumn, void null}\FloatTok{)}\OtherTok{;}
\OtherTok{        \}}
\OtherTok{    \}}
\OtherTok{    \}}
\OtherTok{\}}
\end{Highlighting}
\end{Shaded}

\begin{Shaded}
\begin{Highlighting}[]
\CommentTok{/*}
\CommentTok{  HISTORY}
\CommentTok{  }
\CommentTok{  14-NOV-02   J-03-38   $$1   JCN      Adapted from J-Link examples.}
\CommentTok{  07-MAR-03   K-01-03   $$2   JCN      UNIX support}
\CommentTok{  09-APR-09   L-03-29   $$3   SRV      Remvng Enum:pfcIntfCATIA }
\CommentTok{  02-Oct-09   L-05-10   $$4 tshmeleva  Removed pfcIntfPDGS}
\CommentTok{*/}

\CommentTok{/*}
\CommentTok{  This function will return a Feature object when provided with a solid}
\CommentTok{  a coordinate system name and an import feature's filename. The method}
\CommentTok{  find the coordinate system in the model, sets the Import Feature Attributes,}
\CommentTok{  and creates the Import Feature. Then the Feature is returned.}
\CommentTok{*/}
\KeywordTok{function} \FunctionTok{createImportFeatureFromDataFile} \NormalTok{(fileType }\CommentTok{/* string */}\NormalTok{, }
                      \NormalTok{fileName }\CommentTok{/* string */}\NormalTok{,}
                      \NormalTok{csysName }\CommentTok{/* string */}\NormalTok{) }
\NormalTok{\{}
  \KeywordTok{if} \NormalTok{(!}\FunctionTok{pfcIsWindows}\NormalTok{())}
    \OtherTok{netscape}\NormalTok{.}\OtherTok{security}\NormalTok{.}\OtherTok{PrivilegeManager}\NormalTok{.}\FunctionTok{enablePrivilege}\NormalTok{(}\StringTok{"UniversalXPConnect"}\NormalTok{);}
  
  \KeywordTok{var} \NormalTok{fileClass = }\KeywordTok{void} \KeywordTok{null}\NormalTok{;}
  
  \KeywordTok{if} \NormalTok{(fileType == }\StringTok{"Neutral"}\NormalTok{)}
    \NormalTok{fileClass = }\FunctionTok{pfcCreate} \NormalTok{(}\StringTok{"pfcIntfNeutralFile"}\NormalTok{);}
  \KeywordTok{else} \KeywordTok{if} \NormalTok{(fileType == }\StringTok{"IGES"}\NormalTok{)}
    \NormalTok{fileClass = }\FunctionTok{pfcCreate} \NormalTok{(}\StringTok{"pfcIntfIges"}\NormalTok{);}
  \KeywordTok{else} \KeywordTok{if} \NormalTok{(fileType == }\StringTok{"SET"}\NormalTok{)}
    \NormalTok{fileClass = }\FunctionTok{pfcCreate} \NormalTok{(}\StringTok{"pfcIntfSet"}\NormalTok{);}
  \KeywordTok{else} \KeywordTok{if} \NormalTok{(fileType == }\StringTok{"STEP"}\NormalTok{)}
    \NormalTok{fileClass = }\FunctionTok{pfcCreate} \NormalTok{(}\StringTok{"pfcIntfStep"}\NormalTok{);}
  \KeywordTok{else} \KeywordTok{if} \NormalTok{(fileType == }\StringTok{"VDA"}\NormalTok{)}
    \NormalTok{fileClass = }\FunctionTok{pfcCreate} \NormalTok{(}\StringTok{"pfcIntfVDA"}\NormalTok{);}
  \KeywordTok{else}
    \KeywordTok{throw} \KeywordTok{new} \FunctionTok{Error} \NormalTok{(}\DecValTok{0}\NormalTok{, }\StringTok{"Unrecognized file type"}\NormalTok{);}
  
\CommentTok{/*--------------------------------------------------------------------*\textbackslash{} }
\CommentTok{  Get the current part }
\CommentTok{\textbackslash{}*--------------------------------------------------------------------*/}  
  \KeywordTok{var} \NormalTok{session = }\FunctionTok{pfcGetProESession} \NormalTok{();}
  \KeywordTok{var} \NormalTok{solid = }\OtherTok{session}\NormalTok{.}\FunctionTok{CurrentModel}\NormalTok{;}
  
  \KeywordTok{if} \NormalTok{(}\OtherTok{solid}\NormalTok{.}\FunctionTok{Type} \NormalTok{!= }\FunctionTok{pfcCreate} \NormalTok{(}\StringTok{"pfcModelType"}\NormalTok{).}\FunctionTok{MDL_PART}\NormalTok{)}
    \KeywordTok{throw} \KeywordTok{new} \FunctionTok{Error} \NormalTok{(}\DecValTok{0}\NormalTok{, }\StringTok{"Current model is not an assembly"}\NormalTok{);}

\CommentTok{/*--------------------------------------------------------------------*\textbackslash{} }
\CommentTok{  Find the indicated coordinate system}
\CommentTok{\textbackslash{}*--------------------------------------------------------------------*/} 
  \KeywordTok{var} \NormalTok{cSystem = }\OtherTok{solid}\NormalTok{.}\FunctionTok{GetItemByName} \NormalTok{(}\FunctionTok{pfcCreate} \NormalTok{(}\StringTok{"pfcModelItemType"}\NormalTok{).}\FunctionTok{ITEM_COORD_SYS}\NormalTok{, }
                     \NormalTok{csysName);}
  \KeywordTok{if} \NormalTok{(cSystem == }\KeywordTok{void} \KeywordTok{null}\NormalTok{)}
    \KeywordTok{throw} \KeywordTok{new} \FunctionTok{Error} \NormalTok{(}\DecValTok{0}\NormalTok{, }\StringTok{"Couldn't find named coordinate system."}\NormalTok{);}

\CommentTok{/*--------------------------------------------------------------------*\textbackslash{} }
\CommentTok{  Prepare the import feature instructions classes and create the feature}
\CommentTok{\textbackslash{}*--------------------------------------------------------------------*/} 
  \KeywordTok{var} \NormalTok{dataSource = }\OtherTok{fileClass}\NormalTok{.}\FunctionTok{Create}\NormalTok{(fileName);}
  \KeywordTok{var} \NormalTok{featAttr = }\FunctionTok{pfcCreate} \NormalTok{(}\StringTok{"pfcImportFeatAttr"}\NormalTok{).}\FunctionTok{Create}\NormalTok{();}
  \OtherTok{featAttr}\NormalTok{.}\FunctionTok{JoinSurfs} \NormalTok{= }\KeywordTok{true}\NormalTok{;}
  \OtherTok{featAttr}\NormalTok{.}\FunctionTok{MakeSolid} \NormalTok{= }\KeywordTok{true}\NormalTok{;}
  \OtherTok{featAttr}\NormalTok{.}\FunctionTok{Operation} \NormalTok{= }\FunctionTok{pfcCreate}\NormalTok{(}\StringTok{"pfcOperationType"}\NormalTok{).}\FunctionTok{ADD_OPERATION}\NormalTok{;}
  
  \KeywordTok{var} \NormalTok{importFeature = }\OtherTok{solid}\NormalTok{.}\FunctionTok{CreateImportFeat}\NormalTok{(dataSource, cSystem, featAttr);}
  
  \KeywordTok{return} \NormalTok{importFeature;}
\NormalTok{\}}

\end{Highlighting}
\end{Shaded}

\begin{Shaded}
\begin{Highlighting}[]
\CommentTok{/*}
\CommentTok{   HISTORY}

\CommentTok{14-NOV-02   J-03-38   $$1   JCN      Adapted from J-Link examples.}
\CommentTok{07-MAR-03   K-01-03   $$2   JCN      UNIX support}
\CommentTok{*/}

\CommentTok{/*}
\CommentTok{   This method allows a user to evaluate the assembly for a presence of any}
\CommentTok{   interferences. Upon finding one, this method will highlight the interfering}
\CommentTok{   surfaces, compute and highlight the interference volume.}
\CommentTok{*/}
\KeywordTok{function} \FunctionTok{showInterferences}\NormalTok{()}
\NormalTok{\{}
  \KeywordTok{if} \NormalTok{(!}\FunctionTok{pfcIsWindows}\NormalTok{())}
    \OtherTok{netscape}\NormalTok{.}\OtherTok{security}\NormalTok{.}\OtherTok{PrivilegeManager}\NormalTok{.}\FunctionTok{enablePrivilege}\NormalTok{(}\StringTok{"UniversalXPConnect"}\NormalTok{);}

\CommentTok{/*--------------------------------------------------------------------*\textbackslash{} }
\CommentTok{  Get the current assembly }
\CommentTok{\textbackslash{}*--------------------------------------------------------------------*/}  
  \KeywordTok{var} \NormalTok{session = }\FunctionTok{pfcGetProESession} \NormalTok{();}
  \KeywordTok{var} \NormalTok{assembly = }\OtherTok{session}\NormalTok{.}\FunctionTok{CurrentModel}\NormalTok{;}
  
  \KeywordTok{if} \NormalTok{(}\OtherTok{assembly}\NormalTok{.}\FunctionTok{Type} \NormalTok{!= }\FunctionTok{pfcCreate} \NormalTok{(}\StringTok{"pfcModelType"}\NormalTok{).}\FunctionTok{MDL_ASSEMBLY}\NormalTok{)}
    \KeywordTok{throw} \KeywordTok{new} \FunctionTok{Error} \NormalTok{(}\DecValTok{0}\NormalTok{, }\StringTok{"Current model is not an assembly"}\NormalTok{);}
  
\CommentTok{/*--------------------------------------------------------------------*\textbackslash{} }
\CommentTok{  Calculate the assembly interference}
\CommentTok{\textbackslash{}*--------------------------------------------------------------------*/}
  \KeywordTok{var} \NormalTok{gblEval = }
    \FunctionTok{pfcCreate} \NormalTok{(}\StringTok{"MpfcInterference"}\NormalTok{).}\FunctionTok{CreateGlobalEvaluator}\NormalTok{(assembly);}
  
  \KeywordTok{var} \NormalTok{gblInters = }\OtherTok{gblEval}\NormalTok{.}\FunctionTok{ComputeGlobalInterference}\NormalTok{(}\KeywordTok{true}\NormalTok{);}
  
  \KeywordTok{if} \NormalTok{(gblInters != }\KeywordTok{void} \KeywordTok{null}\NormalTok{)}
    \NormalTok{\{}
      \KeywordTok{var} \NormalTok{size = }\OtherTok{gblInters}\NormalTok{.}\FunctionTok{Count}\NormalTok{;}
      
\CommentTok{/*--------------------------------------------------------------------*\textbackslash{} }
\CommentTok{  For each interference object display the interfering surfaces}
\CommentTok{  and compute the interference volume}
\CommentTok{\textbackslash{}*--------------------------------------------------------------------*/}
      \OtherTok{session}\NormalTok{.}\OtherTok{CurrentWindow}\NormalTok{.}\FunctionTok{SetBrowserSize} \NormalTok{(}\FloatTok{0.0}\NormalTok{);}
      \OtherTok{session}\NormalTok{.}\OtherTok{CurrentWindow}\NormalTok{.}\FunctionTok{Repaint}\NormalTok{();}
      \FunctionTok{alert} \NormalTok{(}\StringTok{"Interferences detected, highlighting each instance."}\NormalTok{);}
      \KeywordTok{for} \NormalTok{(}\KeywordTok{var} \NormalTok{i = }\DecValTok{0}\NormalTok{; i < size; i++)}
    \NormalTok{\{}
      \KeywordTok{var} \NormalTok{gblInter = }\OtherTok{gblInters}\NormalTok{.}\FunctionTok{Item} \NormalTok{(i);}
      
      \KeywordTok{var} \NormalTok{selectPair = }\OtherTok{gblInter}\NormalTok{.}\FunctionTok{SelParts}\NormalTok{;}
      \KeywordTok{var} \NormalTok{sel1 = }\OtherTok{selectPair}\NormalTok{.}\FunctionTok{Sel1}\NormalTok{;}
      \KeywordTok{var} \NormalTok{sel2 = }\OtherTok{selectPair}\NormalTok{.}\FunctionTok{Sel2}\NormalTok{;}
      \OtherTok{sel1}\NormalTok{.}\FunctionTok{Highlight}\NormalTok{(}\FunctionTok{pfcCreate} \NormalTok{(}\StringTok{"pfcStdColor"}\NormalTok{).}\FunctionTok{COLOR_HIGHLIGHT}\NormalTok{);}
      \OtherTok{sel2}\NormalTok{.}\FunctionTok{Highlight}\NormalTok{(}\FunctionTok{pfcCreate} \NormalTok{(}\StringTok{"pfcStdColor"}\NormalTok{).}\FunctionTok{COLOR_HIGHLIGHT}\NormalTok{);}
      
      \KeywordTok{var} \NormalTok{vol = }\OtherTok{gblInter}\NormalTok{.}\FunctionTok{Volume}\NormalTok{;}
      \KeywordTok{var} \NormalTok{totalVolume = }\OtherTok{vol}\NormalTok{.}\FunctionTok{ComputeVolume}\NormalTok{();}
      \OtherTok{vol}\NormalTok{.}\FunctionTok{Highlight}\NormalTok{(}\FunctionTok{pfcCreate} \NormalTok{(}\StringTok{"pfcStdColor"}\NormalTok{).}\FunctionTok{COLOR_PREHIGHLIGHT}\NormalTok{);}
      \FunctionTok{alert} \NormalTok{(}\StringTok{"Interference "} \NormalTok{+ (i + }\DecValTok{1}\NormalTok{) + }\StringTok{" = "} \NormalTok{+ totalVolume);}
      
      \OtherTok{sel1}\NormalTok{.}\FunctionTok{UnHighlight}\NormalTok{();}
      \OtherTok{sel2}\NormalTok{.}\FunctionTok{UnHighlight}\NormalTok{();           }
    \NormalTok{\}}
    \NormalTok{\}}
\NormalTok{\}}
\end{Highlighting}
\end{Shaded}

\begin{Shaded}
\begin{Highlighting}[]
\CommentTok{/*}
\CommentTok{   HISTORY}

\CommentTok{14-NOV-02   J-03-38   $$1   JCN      Adapted from J-Link examples.}
\CommentTok{07-MAR-03   K-01-02   $$3   JCN      UNIX support}
\CommentTok{*/}

\CommentTok{/*====================================================================*\textbackslash{}}
\NormalTok{FUNCTION: createParametersFromArguments}
\NormalTok{PURPOSE:  Create/modify parameters }\KeywordTok{in} \NormalTok{the model based on name-value pairs }
            \KeywordTok{in} \NormalTok{the page URL}
\NormalTok{\textbackslash{}*====================================================================*}\OtherTok{/}
\KeywordTok{function} \FunctionTok{createParametersFromArguments} \NormalTok{()  }
\NormalTok{\{}
  \KeywordTok{if} \NormalTok{(!}\FunctionTok{pfcIsWindows}\NormalTok{())}
    \OtherTok{netscape}\NormalTok{.}\OtherTok{security}\NormalTok{.}\OtherTok{PrivilegeManager}\NormalTok{.}\FunctionTok{enablePrivilege}\NormalTok{(}\StringTok{"UniversalXPConnect"}\NormalTok{);}
  
  \KeywordTok{var} \NormalTok{propValue;}
  \KeywordTok{var} \NormalTok{propsfile = }\StringTok{"params.properties"}\NormalTok{;}
  \KeywordTok{var} \NormalTok{p;}

  \KeywordTok{var} \NormalTok{args = }\FunctionTok{getArgs} \NormalTok{();}
  
\CommentTok{/*------------------------------------------------------------------*\textbackslash{}}
  \NormalTok{Use the current model as the parameter }\OtherTok{owner}\NormalTok{.}
\NormalTok{\textbackslash{}*------------------------------------------------------------------*}\OtherTok{/}
  \KeywordTok{var} \NormalTok{session = }\FunctionTok{pfcGetProESession} \NormalTok{();}
  \KeywordTok{var} \NormalTok{pOwner = }\OtherTok{session}\NormalTok{.}\FunctionTok{CurrentModel}\NormalTok{;}
  
  \KeywordTok{if} \NormalTok{(pOwner == }\KeywordTok{void} \KeywordTok{null}\NormalTok{)}
    \KeywordTok{throw} \KeywordTok{new} \FunctionTok{Error} \NormalTok{(}\DecValTok{0}\NormalTok{, }\StringTok{"No current model."}\NormalTok{);}

\CommentTok{/*------------------------------------------------------------------*\textbackslash{}}
  \NormalTok{Process each name/value pair as a Pro/E }\OtherTok{parameter}\NormalTok{.}
\NormalTok{\textbackslash{}*------------------------------------------------------------------*}\OtherTok{/}
  \KeywordTok{for} \NormalTok{(}\KeywordTok{var} \NormalTok{i = }\DecValTok{0}\NormalTok{; i < }\OtherTok{args}\NormalTok{.}\FunctionTok{length}\NormalTok{; i++)}
    \NormalTok{\{}
      \KeywordTok{var} \NormalTok{pName = args[i].}\FunctionTok{Name}\NormalTok{;}
      \KeywordTok{var} \NormalTok{pv = }\FunctionTok{createParamValueFromString}\NormalTok{(args[i].}\FunctionTok{Value}\NormalTok{);}
      \NormalTok{p = }\OtherTok{pOwner}\NormalTok{.}\FunctionTok{GetParam}\NormalTok{(pName);}
\CommentTok{/*------------------------------------------------------------------*\textbackslash{}}
  \NormalTok{GetParam returns }\KeywordTok{null} \KeywordTok{if} \NormalTok{it can}\StringTok{'t find the param.  Create it.}
\NormalTok{\textbackslash{}*------------------------------------------------------------------*}\OtherTok{/}
      \KeywordTok{if} \NormalTok{(p == }\KeywordTok{void} \KeywordTok{null}\NormalTok{) }
    \NormalTok{\{}
      \OtherTok{pOwner}\NormalTok{.}\FunctionTok{CreateParam} \NormalTok{(pName, pv);}
    \NormalTok{\}}
      \KeywordTok{else}
    \NormalTok{\{}
      \OtherTok{p}\NormalTok{.}\FunctionTok{Value} \NormalTok{= pv;}
    \NormalTok{\}}
    \NormalTok{\}}
  
  \OtherTok{session}\NormalTok{.}\FunctionTok{RunMacro} \NormalTok{(}\StringTok{"~ Select `main_dlg_cur` `MenuBar1` `Utilities`;~ Close `main_dlg_cur` `MenuBar1`;~ Activate `main_dlg_cur` `Utilities.psh_params`"}\NormalTok{);}
\NormalTok{\}}

\CommentTok{/*====================================================================*\textbackslash{}}
\NormalTok{FUNCTION: getArgs}
\NormalTok{PURPOSE:  Parse arguments passed via the URL}
\NormalTok{\textbackslash{}*====================================================================*}\OtherTok{/}
\KeywordTok{function} \FunctionTok{getArgs} \NormalTok{()}
\NormalTok{\{}
  \KeywordTok{var} \NormalTok{args = }\KeywordTok{new} \FunctionTok{Array} \NormalTok{();}
  
  \KeywordTok{var} \NormalTok{query = }\OtherTok{location}\NormalTok{.}\OtherTok{search}\NormalTok{.}\FunctionTok{substring} \NormalTok{(}\DecValTok{1}\NormalTok{);}
  
  \KeywordTok{var} \NormalTok{pairs = }\OtherTok{query}\NormalTok{.}\FunctionTok{split} \NormalTok{(}\StringTok{"&"}\NormalTok{);}
  \KeywordTok{for} \NormalTok{(}\KeywordTok{var} \NormalTok{i = }\DecValTok{0}\NormalTok{; i < }\OtherTok{pairs}\NormalTok{.}\FunctionTok{length}\NormalTok{; i++)}
    \NormalTok{\{   }
      \KeywordTok{var} \NormalTok{pos = pairs [i].}\FunctionTok{indexOf} \NormalTok{(}\StringTok{'='}\NormalTok{);}
      \KeywordTok{if} \NormalTok{(pos == -}\DecValTok{1}\NormalTok{) }\KeywordTok{continue}\NormalTok{;}
      \KeywordTok{var} \NormalTok{argname = pairs[i].}\FunctionTok{substring} \NormalTok{(}\DecValTok{0}\NormalTok{, pos);}
      \KeywordTok{var} \NormalTok{value = pairs[i].}\FunctionTok{substring} \NormalTok{(pos}\DecValTok{+1}\NormalTok{);}
      \KeywordTok{var} \NormalTok{argPair = }\KeywordTok{new} \FunctionTok{Object} \NormalTok{();}
      \OtherTok{argPair}\NormalTok{.}\FunctionTok{Name} \NormalTok{= argname;}
      \OtherTok{argPair}\NormalTok{.}\FunctionTok{Value} \NormalTok{= }\FunctionTok{unescape} \NormalTok{(value);}
      \OtherTok{args}\NormalTok{.}\FunctionTok{push} \NormalTok{(argPair);}
    \NormalTok{\}}
  
  \KeywordTok{return} \NormalTok{args;}
\NormalTok{\}}

\CommentTok{/*====================================================================*\textbackslash{}}
\NormalTok{FUNCTION: createParamValueFromString}
\NormalTok{PURPOSE:  Parses a string into a pfcParamValue object, checking }\KeywordTok{for} \NormalTok{most }
        \NormalTok{restrictive possible type to }\OtherTok{use}\NormalTok{.}
\NormalTok{\textbackslash{}*====================================================================*}\OtherTok{/    }
\OtherTok{function createParamValueFromString }\FloatTok{(}\OtherTok{s /}\NormalTok{* string *}\OtherTok{/}\FloatTok{)}
\OtherTok{\{}
\OtherTok{  if }\FloatTok{(}\OtherTok{s.indexOf }\FloatTok{(}\OtherTok{"."}\FloatTok{)}\OtherTok{ == -1}\FloatTok{)}
\OtherTok{    \{}
\OtherTok{      var i = parseInt }\FloatTok{(}\OtherTok{s}\FloatTok{)}\OtherTok{;}
\OtherTok{      if }\FloatTok{(}\OtherTok{!isNaN}\FloatTok{(}\OtherTok{i}\FloatTok{))}
\OtherTok{    return pfcCreate }\FloatTok{(}\OtherTok{"MpfcModelItem"}\FloatTok{)}\OtherTok{.CreateIntParamValue}\FloatTok{(}\OtherTok{i}\FloatTok{)}\OtherTok{;}
\OtherTok{    \}}
\OtherTok{  else}
\OtherTok{    \{}
\OtherTok{      var d = parseFloat }\FloatTok{(}\OtherTok{s}\FloatTok{)}\OtherTok{;}
\OtherTok{      if }\FloatTok{(}\OtherTok{!isNaN}\FloatTok{(}\OtherTok{d}\FloatTok{))}
\OtherTok{    return pfcCreate }\FloatTok{(}\OtherTok{"MpfcModelItem"}\FloatTok{)}\OtherTok{.CreateDoubleParamValue}\FloatTok{(}\OtherTok{d}\FloatTok{)}\OtherTok{;}
\OtherTok{    \}}
\OtherTok{  if }\FloatTok{(}\OtherTok{s.toUpperCase}\FloatTok{()}\OtherTok{ == "Y" }\FloatTok{||}\OtherTok{ s.toUpperCase }\FloatTok{()}\OtherTok{== "TRUE"}\FloatTok{)}
\OtherTok{    return pfcCreate }\FloatTok{(}\OtherTok{"MpfcModelItem"}\FloatTok{)}\OtherTok{.CreateBoolParamValue}\FloatTok{(}\OtherTok{true}\FloatTok{)}\OtherTok{;}
\OtherTok{  }
\OtherTok{  if }\FloatTok{(}\OtherTok{s.toUpperCase}\FloatTok{()}\OtherTok{ == "N" }\FloatTok{||}\OtherTok{ s.toUpperCase }\FloatTok{()}\OtherTok{== "FALSE"}\FloatTok{)}
\OtherTok{    return pfcCreate }\FloatTok{(}\OtherTok{"MpfcModelItem"}\FloatTok{)}\OtherTok{.CreateBoolParamValue}\FloatTok{(}\OtherTok{false}\FloatTok{)}\OtherTok{;}
\OtherTok{  }
\OtherTok{  return pfcCreate }\FloatTok{(}\OtherTok{"MpfcModelItem"}\FloatTok{)}\OtherTok{.CreateStringParamValue}\FloatTok{(}\OtherTok{s}\FloatTok{)}\OtherTok{;}
\OtherTok{\}}

\OtherTok{                    }
\OtherTok{    }
\end{Highlighting}
\end{Shaded}

\begin{Shaded}
\begin{Highlighting}[]
\CommentTok{/*}
\CommentTok{   HISTORY}
\CommentTok{   }
\CommentTok{14-NOV-02   J-03-38   $$1   JCN      Adapted from J-Link examples.}
\CommentTok{07-MAR-03   K-01-03   $$2   JCN      UNIX support}

\CommentTok{ */}
 
\CommentTok{/* }
\CommentTok{   A sample that shows the use of the GetProEArguments}
\CommentTok{   method to access the Pro/ENGINEER command line arguments. }
\CommentTok{   The first argument is always the full path to the Pro/E executable. }
\CommentTok{   For this application the next two arguments can be either ("+runtime" or }
\CommentTok{   "+development") or ("-Unix" or "-NT"). Based on these values 2 boolean }
\CommentTok{   variables are set and passed on to another function which makes use of }
\CommentTok{   this information.   }
\CommentTok{*/}    

\KeywordTok{var} \NormalTok{runtime = }\KeywordTok{true}\NormalTok{;}
\KeywordTok{var} \NormalTok{unix = }\KeywordTok{false}\NormalTok{;}

\KeywordTok{function} \FunctionTok{getArguments} \NormalTok{()}
\NormalTok{\{}
  \KeywordTok{if} \NormalTok{(!}\FunctionTok{pfcIsWindows}\NormalTok{())}
    \OtherTok{netscape}\NormalTok{.}\OtherTok{security}\NormalTok{.}\OtherTok{PrivilegeManager}\NormalTok{.}\FunctionTok{enablePrivilege}\NormalTok{(}\StringTok{"UniversalXPConnect"}\NormalTok{);}
  
  \KeywordTok{var} \NormalTok{argseq = }\FunctionTok{pfcCreate} \NormalTok{(}\StringTok{"MpfcCOMGlobal"}\NormalTok{).}\FunctionTok{GetProEArguments}\NormalTok{();         }
  
\CommentTok{/*------------------------------------------------------------------*\textbackslash{}}
  \NormalTok{Making sure that there are three }\OtherTok{arguments}\NormalTok{.}
\NormalTok{\textbackslash{}*------------------------------------------------------------------*}\OtherTok{/}
  \KeywordTok{if} \NormalTok{(}\OtherTok{argseq}\NormalTok{.}\FunctionTok{Count} \NormalTok{== }\DecValTok{3}\NormalTok{)}
    \NormalTok{\{}
\CommentTok{/*------------------------------------------------------------------*\textbackslash{}}
  \NormalTok{First argument is Pro/E executable  - skip it}
\NormalTok{\textbackslash{}*------------------------------------------------------------------*}\OtherTok{/}
\CommentTok{/*------------------------------------------------------------------*\textbackslash{}}
  \NormalTok{Set flags based on Pro/E input arguments}
\NormalTok{\textbackslash{}*------------------------------------------------------------------*}\OtherTok{/}
      \KeywordTok{var} \NormalTok{val=}\OtherTok{argseq}\NormalTok{.}\FunctionTok{Item}\NormalTok{(}\DecValTok{1}\NormalTok{);}
      
      \FunctionTok{setFlags} \NormalTok{(val);}
     
\CommentTok{/*------------------------------------------------------------------*\textbackslash{}}
  \NormalTok{Set third flag based on Pro/E input argument}
\NormalTok{\textbackslash{}*------------------------------------------------------------------*}\OtherTok{/}
      \NormalTok{val=}\OtherTok{argseq}\NormalTok{.}\FunctionTok{Item}\NormalTok{(}\DecValTok{2}\NormalTok{);}
      
      \FunctionTok{setFlags} \NormalTok{(val);}
    \NormalTok{\}}
\CommentTok{/*------------------------------------------------------------------*\textbackslash{}}
  \NormalTok{Pass the boolean values to another }\KeywordTok{function}
\NormalTok{\textbackslash{}*------------------------------------------------------------------*}\OtherTok{/}
  \CommentTok{//runApplication(runtime,unix);}
\NormalTok{\}}
    
    
\KeywordTok{function} \FunctionTok{setFlags} \NormalTok{(val }\CommentTok{/* string */}\NormalTok{)}
\NormalTok{\{}
  \KeywordTok{if} \NormalTok{(val == }\StringTok{"+runtime"}\NormalTok{)}
    \NormalTok{\{}
      \NormalTok{runtime=}\KeywordTok{true}\NormalTok{;}
    \NormalTok{\}}
  \KeywordTok{else} \KeywordTok{if} \NormalTok{(val == }\StringTok{"+development"}\NormalTok{)}
    \NormalTok{\{                   }
      \NormalTok{runtime=}\KeywordTok{false}\NormalTok{;}
    \NormalTok{\}      }
  \KeywordTok{else} \KeywordTok{if} \NormalTok{(val == }\StringTok{"-Unix"}\NormalTok{)}
    \NormalTok{\{}
      \NormalTok{unix=}\KeywordTok{true}\NormalTok{;}
    \NormalTok{\}}
  \KeywordTok{else} \KeywordTok{if} \NormalTok{(val == }\StringTok{"-NT"}\NormalTok{)}
    \NormalTok{\{}
      \NormalTok{unix=}\KeywordTok{false}\NormalTok{;}
    \NormalTok{\}}
\NormalTok{\}}
           


\end{Highlighting}
\end{Shaded}

\begin{Shaded}
\begin{Highlighting}[]
\CommentTok{/*  }
\CommentTok{   HISTORY}

\CommentTok{23-JUL-07   L-01-35   $$1  SNV     Created and submitted}

\CommentTok{*/}

\CommentTok{/*====================================================================*\textbackslash{}}
\NormalTok{FUNCTION: createParamDimRelation_script_wrapper}
\NormalTok{PURPOSE:  Wrapper }\KeywordTok{function} \KeywordTok{for} \OtherTok{createParamDimRelation}\NormalTok{.}
\NormalTok{\textbackslash{}*====================================================================*}\OtherTok{/}
\KeywordTok{function} \FunctionTok{createParamDimRelation_script_wrapper}\NormalTok{()}
\NormalTok{\{   }
    \KeywordTok{var} \NormalTok{i;}
    \KeywordTok{var} \NormalTok{selections;}
    \KeywordTok{var} \NormalTok{options ;}
    
    \KeywordTok{var} \NormalTok{session = }\FunctionTok{pfcGetProESession} \NormalTok{();}
    \CommentTok{/*=====================================================================*\textbackslash{}}
    \NormalTok{Get the current part model}
    \NormalTok{\textbackslash{}*=====================================================================*}\OtherTok{/      }
\OtherTok{    var solid = session.CurrentModel;}
\OtherTok{    modelTypeClass = pfcCreate }\FloatTok{(}\OtherTok{"pfcModelType"}\FloatTok{)}\OtherTok{;}
\OtherTok{    }
\OtherTok{    if }\FloatTok{(}\OtherTok{solid == void null }\FloatTok{||}\OtherTok{ }\FloatTok{(}\OtherTok{solid.Type != modelTypeClass.MDL_PART}\FloatTok{))}
\OtherTok{    \{}
\OtherTok{        throw new Error  }\FloatTok{(}\OtherTok{0, "Current model is not a part."}\FloatTok{)}\OtherTok{;}
\OtherTok{    \}}
\OtherTok{      }
\OtherTok{    /}\NormalTok{*=====================================================================*\textbackslash{}}
        \NormalTok{Get selected components}
    \NormalTok{\textbackslash{}*=====================================================================*}\OtherTok{/               }
\OtherTok{    var browserSize = session.CurrentWindow.GetBrowserSize}\FloatTok{()}\OtherTok{;}
\OtherTok{    session.CurrentWindow.SetBrowserSize }\FloatTok{(}\OtherTok{0.0}\FloatTok{)}\OtherTok{;}
\OtherTok{    }
\OtherTok{    options = pfcCreate}\FloatTok{(}\OtherTok{"pfcSelectionOptions"}\FloatTok{)}\OtherTok{.Create}\FloatTok{(}\OtherTok{"feature"}\FloatTok{)}\OtherTok{;}
\OtherTok{    selections = session.Select }\FloatTok{(}\OtherTok{options, void null}\FloatTok{)}\OtherTok{;}
\OtherTok{    }
\OtherTok{    session.CurrentWindow.SetBrowserSize }\FloatTok{(}\OtherTok{browserSize}\FloatTok{)}\OtherTok{;}
\OtherTok{    if }\FloatTok{(}\OtherTok{selections == void null }\FloatTok{||}\OtherTok{ selections.Count == 0}\FloatTok{)}
\OtherTok{    \{}
\OtherTok{        throw new Error  }\FloatTok{(}\OtherTok{0, "Nothing selected"}\FloatTok{)}\OtherTok{;}
\OtherTok{    \}}
\OtherTok{    }
\OtherTok{    var features = pfcCreate}\FloatTok{(}\OtherTok{"pfcFeatures"}\FloatTok{)}\OtherTok{;}
\OtherTok{    for }\FloatTok{(}\OtherTok{i =0; i < selections.Count ; i}\FloatTok{++)}
\OtherTok{    \{}
\OtherTok{        features.Append}\FloatTok{(}\OtherTok{selections.Item}\FloatTok{(}\OtherTok{i}\FloatTok{)}\OtherTok{.SelItem}\FloatTok{)}\OtherTok{;                        }
\OtherTok{    \}}
\OtherTok{    }
\OtherTok{    createParamDimRelation}\FloatTok{(}\OtherTok{features}\FloatTok{)}\OtherTok{;}
\OtherTok{    }
\OtherTok{\}}


\OtherTok{/}\NormalTok{*====================================================================*\textbackslash{}}
\DataTypeTok{FUNCTION}\NormalTok{: createParamDimRelation}
\DataTypeTok{PURPOSE}\NormalTok{:  This }\KeywordTok{function} \NormalTok{creates parameters }\KeywordTok{for} \NormalTok{all dimensions }\KeywordTok{in} \NormalTok{input }
          \NormalTok{features of a part model and adds relation between }\OtherTok{them}\NormalTok{.}
\NormalTok{\textbackslash{}*====================================================================*}\OtherTok{/}
\KeywordTok{function} \FunctionTok{createParamDimRelation} \NormalTok{(features)}
\NormalTok{\{}
  \KeywordTok{if} \NormalTok{(!}\FunctionTok{pfcIsWindows}\NormalTok{())}
    \OtherTok{netscape}\NormalTok{.}\OtherTok{security}\NormalTok{.}\OtherTok{PrivilegeManager}\NormalTok{.}\FunctionTok{enablePrivilege}\NormalTok{(}\StringTok{"UniversalXPConnect"}\NormalTok{);}
    
  \KeywordTok{var} \NormalTok{i,j;}
  \KeywordTok{var} \NormalTok{relations;}
  \KeywordTok{var} \NormalTok{items;  }
  \KeywordTok{var} \NormalTok{dimName , paramName;}
  \KeywordTok{var} \NormalTok{dimValue;}
  \KeywordTok{var} \NormalTok{paramAdded;}
  \KeywordTok{var} \NormalTok{param ;}
  \KeywordTok{var} \NormalTok{paramValue;}
                   
  \KeywordTok{for} \NormalTok{(i =}\DecValTok{0}\NormalTok{; i < }\OtherTok{features}\NormalTok{.}\FunctionTok{Count} \NormalTok{; i++)}
  \NormalTok{\{   }
      \CommentTok{/*=====================================================================*\textbackslash{}}
        \NormalTok{Get the selected feature}
      \NormalTok{\textbackslash{}*=====================================================================*}\OtherTok{/   }
\OtherTok{      var feature = features.Item}\FloatTok{(}\OtherTok{i}\FloatTok{)}\OtherTok{;}
\OtherTok{      if }\FloatTok{(}\OtherTok{feature == void null}\FloatTok{)}
\OtherTok{      \{}
\OtherTok{          continue;}
\OtherTok{      \}}
\OtherTok{      }
\OtherTok{      /}\NormalTok{*=====================================================================*\textbackslash{}}
        \NormalTok{Get the dimensions }\KeywordTok{in} \NormalTok{the current feature}
      \NormalTok{\textbackslash{}*=====================================================================*}\OtherTok{/}
      \NormalTok{items = }\OtherTok{feature}\NormalTok{.}\FunctionTok{ListSubItems}\NormalTok{(}\FunctionTok{pfcCreate} \NormalTok{(}\StringTok{"pfcModelItemType"}\NormalTok{).}\FunctionTok{ITEM_DIMENSION}\NormalTok{);}

      \KeywordTok{if} \NormalTok{((items == }\KeywordTok{void} \KeywordTok{null}\NormalTok{) || (}\OtherTok{items}\NormalTok{.}\FunctionTok{Count} \NormalTok{== }\DecValTok{0} \NormalTok{))}
      \NormalTok{\{}
          \KeywordTok{continue}\NormalTok{;}
      \NormalTok{\}}
              
      \NormalTok{relations = }\FunctionTok{pfcCreate}\NormalTok{(}\StringTok{"stringseq"}\NormalTok{);}
            
      \CommentTok{/*=====================================================================*\textbackslash{}}
        \NormalTok{Loop through all the dimensions and create relations}
      \NormalTok{\textbackslash{}*=====================================================================*}\OtherTok{/}
      \KeywordTok{for} \NormalTok{(j = }\DecValTok{0}\NormalTok{; j < }\OtherTok{items}\NormalTok{.}\FunctionTok{Count}\NormalTok{; j++)}
      \NormalTok{\{       }
          \KeywordTok{var} \NormalTok{item = }\OtherTok{items}\NormalTok{.}\FunctionTok{Item}\NormalTok{(j);       }
          \NormalTok{dimName = }\OtherTok{item}\NormalTok{.}\FunctionTok{GetName}\NormalTok{();       }
          \NormalTok{paramName = paramName = }\StringTok{"PARAM_"} \NormalTok{+ dimName;         }

          \NormalTok{dimValue = }\OtherTok{item}\NormalTok{.}\FunctionTok{DimValue}\NormalTok{;}
                          
          \NormalTok{param = }\OtherTok{feature}\NormalTok{.}\FunctionTok{GetParam}\NormalTok{(paramName);        }
          \NormalTok{paramAdded = }\KeywordTok{false}\NormalTok{;}
          
          \KeywordTok{if} \NormalTok{(param == }\KeywordTok{void} \KeywordTok{null}\NormalTok{)}
          \NormalTok{\{}
              \NormalTok{paramValue = }\FunctionTok{pfcCreate} \NormalTok{(}\StringTok{"MpfcModelItem"}\NormalTok{).}\FunctionTok{CreateDoubleParamValue}\NormalTok{(dimValue);}
              \OtherTok{feature}\NormalTok{.}\FunctionTok{CreateParam} \NormalTok{(paramName, paramValue);}
              \NormalTok{paramAdded = }\KeywordTok{true}\NormalTok{;}
          \NormalTok{\}}
          \KeywordTok{else}
          \NormalTok{\{}
              \KeywordTok{if} \NormalTok{(}\OtherTok{param}\NormalTok{.}\OtherTok{Value}\NormalTok{.}\FunctionTok{discr} \NormalTok{== }\FunctionTok{pfcCreate} \NormalTok{(}\StringTok{"pfcParamValueType"}\NormalTok{).}\FunctionTok{PARAM_DOUBLE}\NormalTok{)}
              \NormalTok{\{}
                  \NormalTok{paramValue = }\FunctionTok{pfcCreate} \NormalTok{(}\StringTok{"MpfcModelItem"}\NormalTok{).}\FunctionTok{CreateDoubleParamValue}\NormalTok{(dimValue);}
                  \OtherTok{param}\NormalTok{.}\FunctionTok{Value} \NormalTok{= paramValue;}
                  \NormalTok{paramAdded = }\KeywordTok{true}\NormalTok{;}
              \NormalTok{\}}
          \NormalTok{\}}
          
          \KeywordTok{if} \NormalTok{(paramAdded == }\KeywordTok{true}\NormalTok{)}
          \NormalTok{\{}
              \OtherTok{relations}\NormalTok{.}\FunctionTok{Append}\NormalTok{(dimName + }\StringTok{" = "} \NormalTok{+ paramName);                  }
          \NormalTok{\}}
          \NormalTok{param = }\KeywordTok{void} \KeywordTok{null}\NormalTok{;}
          
      \NormalTok{\}}
     \OtherTok{feature}\NormalTok{.}\FunctionTok{Relations} \NormalTok{= relations;}
  \NormalTok{\}}
\NormalTok{\}         }
    
      
\end{Highlighting}
\end{Shaded}

\begin{Shaded}
\begin{Highlighting}[]
\CommentTok{/*}
\CommentTok{   HISTORY}
\CommentTok{   }
\CommentTok{14-NOV-02   J-03-38   $$1   JCN      Adapted from J-Link examples.}
\CommentTok{07-MAR-03   K-01-03   $$2   JCN      UNIX support}
\CommentTok{*/}
 
\CommentTok{/*}
\CommentTok{  This example code demonstrates how to invoke an interactive selection. }
\CommentTok{*/}
\KeywordTok{function} \FunctionTok{selectItems} \NormalTok{(options }\CommentTok{/* string[] */}\NormalTok{, max }\CommentTok{/* integer */}\NormalTok{)}
\NormalTok{\{}
  \KeywordTok{if} \NormalTok{(!}\FunctionTok{pfcIsWindows}\NormalTok{())}
    \OtherTok{netscape}\NormalTok{.}\OtherTok{security}\NormalTok{.}\OtherTok{PrivilegeManager}\NormalTok{.}\FunctionTok{enablePrivilege}\NormalTok{(}\StringTok{"UniversalXPConnect"}\NormalTok{);}
  
\CommentTok{/*--------------------------------------------------------------------*\textbackslash{} }
\CommentTok{  Get the session. If no model in present abort the operation. }
\CommentTok{\textbackslash{}*--------------------------------------------------------------------*/}  
  \KeywordTok{var} \NormalTok{session = }\FunctionTok{pfcGetProESession} \NormalTok{();}
  \KeywordTok{var} \NormalTok{model = }\OtherTok{session}\NormalTok{.}\FunctionTok{CurrentModel}\NormalTok{;}
  
  \KeywordTok{if} \NormalTok{(model == }\KeywordTok{void} \KeywordTok{null}\NormalTok{)}
    \KeywordTok{throw} \KeywordTok{new} \FunctionTok{Error} \NormalTok{(}\DecValTok{0}\NormalTok{, }\StringTok{"No current model."}\NormalTok{);}
  
\CommentTok{/*--------------------------------------------------------------------*\textbackslash{} }
\CommentTok{  Collect the options array into a comma delimited list}
\CommentTok{\textbackslash{}*--------------------------------------------------------------------*/}  
  \KeywordTok{var} \NormalTok{optString = }\StringTok{""}\NormalTok{;}
  \KeywordTok{for} \NormalTok{(}\KeywordTok{var} \NormalTok{i = }\DecValTok{0}\NormalTok{; i < }\OtherTok{options}\NormalTok{.}\FunctionTok{length}\NormalTok{; i++)}
    \NormalTok{\{}
      \NormalTok{optString += options [i];}
      \KeywordTok{if} \NormalTok{(i != }\OtherTok{options}\NormalTok{.}\FunctionTok{length} \NormalTok{-}\DecValTok{1}\NormalTok{)}
    \NormalTok{optString += }\StringTok{","}\NormalTok{;}
    \NormalTok{\}}
  
\CommentTok{/*--------------------------------------------------------------------*\textbackslash{} }
\CommentTok{  Prompt for selection.}
\CommentTok{\textbackslash{}*--------------------------------------------------------------------*/}  
  \NormalTok{selOptions = }\FunctionTok{pfcCreate} \NormalTok{(}\StringTok{"pfcSelectionOptions"}\NormalTok{).}\FunctionTok{Create} \NormalTok{(optString);}
  
  \KeywordTok{if} \NormalTok{(max != }\StringTok{"UNLIMITED"}\NormalTok{)}
    \NormalTok{\{}
      \OtherTok{selOptions}\NormalTok{.}\FunctionTok{MaxNumSels} \NormalTok{= }\FunctionTok{parseInt} \NormalTok{(max);}
    \NormalTok{\}}
  
  \OtherTok{session}\NormalTok{.}\OtherTok{CurrentWindow}\NormalTok{.}\FunctionTok{SetBrowserSize} \NormalTok{(}\FloatTok{0.0}\NormalTok{);}
  
  \KeywordTok{var} \NormalTok{selections = }\KeywordTok{void} \KeywordTok{null}\NormalTok{;}
  \KeywordTok{try} \NormalTok{\{}
    \NormalTok{selections = }\OtherTok{session}\NormalTok{.}\FunctionTok{Select} \NormalTok{(selOptions, }\KeywordTok{void} \KeywordTok{null}\NormalTok{);}
  \NormalTok{\}}
  \KeywordTok{catch} \NormalTok{(err) \{}
\CommentTok{/*--------------------------------------------------------------------*\textbackslash{} }
\CommentTok{  Handle the situation where the  user didn't make selections, but picked }
\CommentTok{  elsewhere instead.}
\CommentTok{  \textbackslash{}*--------------------------------------------------------------------*/}  
    \KeywordTok{if} \NormalTok{(}\FunctionTok{pfcGetExceptionType} \NormalTok{(err) == }\StringTok{"pfcXToolkitUserAbort"} \NormalTok{|| }
    \FunctionTok{pfcGetExceptionType} \NormalTok{(err) == }\StringTok{"pfcXToolkitPickAbove"}\NormalTok{)}
      \KeywordTok{return} \NormalTok{(}\KeywordTok{void} \KeywordTok{null}\NormalTok{);}
    \KeywordTok{else}
      \KeywordTok{throw} \NormalTok{err;}
  \NormalTok{\}}
  \KeywordTok{if} \NormalTok{(}\OtherTok{selections}\NormalTok{.}\FunctionTok{Count} \NormalTok{== }\DecValTok{0}\NormalTok{)}
    \KeywordTok{return} \NormalTok{(}\KeywordTok{void} \KeywordTok{null}\NormalTok{);}
  
\CommentTok{/*--------------------------------------------------------------------*\textbackslash{} }
\CommentTok{  Write selection info to the browser window}
\CommentTok{\textbackslash{}*--------------------------------------------------------------------*/}  
  \KeywordTok{var} \NormalTok{newWin = }\OtherTok{window}\NormalTok{.}\FunctionTok{open} \NormalTok{(}\StringTok{''}\NormalTok{, }\StringTok{"_IS"}\NormalTok{, }\StringTok{"scrollbars"}\NormalTok{);}
  \KeywordTok{if} \NormalTok{(}\FunctionTok{pfcIsWindows}\NormalTok{())}
    \NormalTok{\{}
      \OtherTok{newWin}\NormalTok{.}\FunctionTok{resizeTo} \NormalTok{(}\DecValTok{300}\NormalTok{, }\OtherTok{screen}\NormalTok{.}\FunctionTok{height}\NormalTok{/}\FloatTok{2.0}\NormalTok{);}
      \OtherTok{newWin}\NormalTok{.}\FunctionTok{moveTo} \NormalTok{(}\OtherTok{screen}\NormalTok{.}\FunctionTok{width}\DecValTok{-300}\NormalTok{, }\DecValTok{0}\NormalTok{);}
    \NormalTok{\}}
  \OtherTok{newWin}\NormalTok{.}\OtherTok{document}\NormalTok{.}\FunctionTok{writeln} \NormalTok{(}\StringTok{"<html><head></head><body>"}\NormalTok{);}
  
  \KeywordTok{for} \NormalTok{(}\KeywordTok{var} \NormalTok{i = }\DecValTok{0}\NormalTok{; i < }\OtherTok{selections}\NormalTok{.}\FunctionTok{Count}\NormalTok{; i ++)}
    \NormalTok{\{}
      \KeywordTok{var} \NormalTok{sel = }\OtherTok{selections}\NormalTok{.}\FunctionTok{Item} \NormalTok{(i);}
      \OtherTok{newWin}\NormalTok{.}\OtherTok{document}\NormalTok{.}\FunctionTok{writeln} \NormalTok{(}\StringTok{"<h2>Selection "}\NormalTok{+(i}\DecValTok{+1}\NormalTok{)+}\StringTok{": </h2>"}\NormalTok{);}
      \OtherTok{newWin}\NormalTok{.}\OtherTok{document}\NormalTok{.}\FunctionTok{writeln} \NormalTok{(}\StringTok{"<table>"}\NormalTok{);}
      
      \KeywordTok{var} \NormalTok{selModelName = }\StringTok{"N/A"}\NormalTok{;}
      \KeywordTok{if} \NormalTok{(}\OtherTok{sel}\NormalTok{.}\FunctionTok{SelModel} \NormalTok{!= }\KeywordTok{void} \KeywordTok{null}\NormalTok{)}
    \NormalTok{selModelName = }\OtherTok{sel}\NormalTok{.}\OtherTok{SelModel}\NormalTok{.}\FunctionTok{FullName}\NormalTok{;   }
      \OtherTok{newWin}\NormalTok{.}\OtherTok{document}\NormalTok{.}\FunctionTok{writeln} \NormalTok{(}\StringTok{"<tr><td>Sel model: </td><td>"}\NormalTok{+}
                   \NormalTok{selModelName+}\StringTok{"</td></tr>"}\NormalTok{);}
      
      \KeywordTok{var} \NormalTok{selItemInfo = }\StringTok{"N/A"}\NormalTok{;}
      \KeywordTok{if} \NormalTok{(}\OtherTok{sel}\NormalTok{.}\FunctionTok{SelItem} \NormalTok{!= }\KeywordTok{void} \KeywordTok{null}\NormalTok{)}
    \NormalTok{selItemInfo = }\StringTok{"Type: "}\NormalTok{+ }\OtherTok{sel}\NormalTok{.}\OtherTok{SelItem}\NormalTok{.}\OtherTok{Type}\NormalTok{.}\FunctionTok{toString}\NormalTok{() + }
      \StringTok{" id: "}\NormalTok{+}\OtherTok{sel}\NormalTok{.}\OtherTok{SelItem}\NormalTok{.}\FunctionTok{Id}\NormalTok{;}
      
      \OtherTok{newWin}\NormalTok{.}\OtherTok{document}\NormalTok{.}\FunctionTok{writeln} \NormalTok{(}\StringTok{"<tr><td>Sel item: </td><td>"} \NormalTok{+}
                   \NormalTok{selItemInfo + }\StringTok{"</td></tr>"}\NormalTok{);}
      \OtherTok{newWin}\NormalTok{.}\OtherTok{document}\NormalTok{.}\FunctionTok{writeln} \NormalTok{(}\StringTok{"</table>"}\NormalTok{);}
    \NormalTok{\}}
  
  \OtherTok{newWin}\NormalTok{.}\OtherTok{document}\NormalTok{.}\FunctionTok{writeln} \NormalTok{(}\StringTok{"<html><head></head><body>"}\NormalTok{);}
  
  \KeywordTok{return} \NormalTok{(selections);}
\NormalTok{\}}


\CommentTok{/* }
\CommentTok{   This method highlights all the features in all levels of an assembly that have a given name. }
\CommentTok{*/}
\KeywordTok{function} \FunctionTok{createAndHighlightSelections} \NormalTok{(featureName }\CommentTok{/* string */}\NormalTok{)}
\NormalTok{\{}
  \KeywordTok{if} \NormalTok{(!}\FunctionTok{pfcIsWindows}\NormalTok{())}
    \OtherTok{netscape}\NormalTok{.}\OtherTok{security}\NormalTok{.}\OtherTok{PrivilegeManager}\NormalTok{.}\FunctionTok{enablePrivilege}\NormalTok{(}\StringTok{"UniversalXPConnect"}\NormalTok{);}
  
\CommentTok{/*--------------------------------------------------------------------*\textbackslash{} }
\CommentTok{  Get the session. If no model in present abort the operation. }
\CommentTok{\textbackslash{}*--------------------------------------------------------------------*/}  
  \KeywordTok{var} \NormalTok{session = }\FunctionTok{pfcGetProESession} \NormalTok{();}
  \KeywordTok{var} \NormalTok{assem = }\OtherTok{session}\NormalTok{.}\FunctionTok{CurrentModel}\NormalTok{;}
  
  \KeywordTok{if} \NormalTok{(assem == }\KeywordTok{void} \KeywordTok{null} \NormalTok{|| }
      \OtherTok{assem}\NormalTok{.}\FunctionTok{Type} \NormalTok{!= }\FunctionTok{pfcCreate} \NormalTok{(}\StringTok{"pfcModelType"}\NormalTok{).}\FunctionTok{MDL_ASSEMBLY}\NormalTok{)}
    \KeywordTok{throw} \KeywordTok{new} \FunctionTok{Error} \NormalTok{(}\DecValTok{0}\NormalTok{, }\StringTok{"Current model is not an assembly."}\NormalTok{);}
  
\CommentTok{/*--------------------------------------------------------------------*\textbackslash{} }
\CommentTok{  Start a recursive traversal of the assembly structure. }
\CommentTok{\textbackslash{}*--------------------------------------------------------------------*/} 
  \NormalTok{intPath = }\FunctionTok{pfcCreate} \NormalTok{(}\StringTok{"intseq"}\NormalTok{);}
  
  \FunctionTok{highlightFeaturesRecursively} \NormalTok{(assem, intPath, featureName);}
\NormalTok{\}}

\KeywordTok{function} \FunctionTok{highlightFeaturesRecursively} \NormalTok{(assem }\CommentTok{/* pfcAssembly */}\NormalTok{, }
                       \NormalTok{intPath }\CommentTok{/* intseq */}\NormalTok{,}
                       \NormalTok{featureName }\CommentTok{/* string */}\NormalTok{)}
\NormalTok{\{}
\CommentTok{/*--------------------------------------------------------------------*\textbackslash{} }
\CommentTok{  Obtain the model at the current assembly level.}
\CommentTok{\textbackslash{}*--------------------------------------------------------------------*/}
  \KeywordTok{var} \NormalTok{subcomponent;}
  \KeywordTok{var} \NormalTok{cmpPath = }\KeywordTok{void} \KeywordTok{null}\NormalTok{;}
  \KeywordTok{if} \NormalTok{(}\OtherTok{intPath}\NormalTok{.}\FunctionTok{Count} \NormalTok{== }\DecValTok{0}\NormalTok{)}
    \NormalTok{subcomponent = assem;}
  \KeywordTok{else}
    \NormalTok{\{}
      \NormalTok{cmpPath = }
    \FunctionTok{pfcCreate} \NormalTok{(}\StringTok{"MpfcAssembly"}\NormalTok{).}\FunctionTok{CreateComponentPath}\NormalTok{( assem, intPath );}
      \NormalTok{subcomponent = }\OtherTok{cmpPath}\NormalTok{.}\FunctionTok{Leaf}\NormalTok{; }
    \NormalTok{\}}
   
\CommentTok{/*--------------------------------------------------------------------*\textbackslash{} }
\CommentTok{  Search for the desired feature.}
\CommentTok{\textbackslash{}*--------------------------------------------------------------------*/} 
  \KeywordTok{var} \NormalTok{theFeat = }\OtherTok{subcomponent}\NormalTok{.}\FunctionTok{GetFeatureByName} \NormalTok{(featureName);}
  \KeywordTok{if} \NormalTok{(theFeat != }\KeywordTok{void} \KeywordTok{null}\NormalTok{)}
    \NormalTok{\{}
      \KeywordTok{var} \NormalTok{cmpSelection = }
    \FunctionTok{pfcCreate} \NormalTok{(}\StringTok{"MpfcSelect"}\NormalTok{).}\FunctionTok{CreateModelItemSelection} \NormalTok{( theFeat, cmpPath );}
      
      \OtherTok{cmpSelection}\NormalTok{.}\FunctionTok{Highlight}\NormalTok{(}\FunctionTok{pfcCreate} \NormalTok{(}\StringTok{"pfcStdColor"}\NormalTok{).}\FunctionTok{COLOR_HIGHLIGHT}\NormalTok{);}
    \NormalTok{\}}
  
\CommentTok{/*--------------------------------------------------------------------*\textbackslash{} }
\CommentTok{  Search for subcomponents, and traverse each of them.}
\CommentTok{\textbackslash{}*--------------------------------------------------------------------*/} 
  \KeywordTok{var} \NormalTok{components = }\OtherTok{subcomponent}\NormalTok{.}\FunctionTok{ListFeaturesByType}\NormalTok{(}\KeywordTok{true}\NormalTok{, }
        \FunctionTok{pfcCreate} \NormalTok{(}\StringTok{"pfcFeatureType"}\NormalTok{).}\FunctionTok{FEATTYPE_COMPONENT}\NormalTok{);}
  \KeywordTok{for} \NormalTok{(}\KeywordTok{var} \NormalTok{i = }\DecValTok{0}\NormalTok{; i < }\OtherTok{components}\NormalTok{.}\FunctionTok{Count}\NormalTok{; i++)}
    \NormalTok{\{}
      \KeywordTok{var} \NormalTok{compFeat = }\OtherTok{components}\NormalTok{.}\FunctionTok{Item} \NormalTok{(i);}
      \KeywordTok{if} \NormalTok{(}\OtherTok{compFeat}\NormalTok{.}\FunctionTok{Status} \NormalTok{== }\FunctionTok{pfcCreate} \NormalTok{(}\StringTok{"pfcFeatureStatus"}\NormalTok{).}\FunctionTok{FEAT_ACTIVE}\NormalTok{)}
    \NormalTok{\{}
      \OtherTok{intPath}\NormalTok{.}\FunctionTok{Append} \NormalTok{(}\OtherTok{components}\NormalTok{.}\FunctionTok{Item} \NormalTok{(i).}\FunctionTok{Id}\NormalTok{);}
      \FunctionTok{highlightFeaturesRecursively} \NormalTok{(assem, intPath, featureName); }
    \NormalTok{\}}
    \NormalTok{\}}
    
\CommentTok{/*--------------------------------------------------------------------*\textbackslash{} }
\CommentTok{  Clean up the assembly ids at this level before returning.}
\CommentTok{\textbackslash{}*--------------------------------------------------------------------*/} 
  \KeywordTok{if} \NormalTok{(}\OtherTok{intPath}\NormalTok{.}\FunctionTok{Count} \NormalTok{> }\DecValTok{0}\NormalTok{)}
    \NormalTok{\{}
      \OtherTok{intPath}\NormalTok{.}\FunctionTok{Remove} \NormalTok{(}\OtherTok{intPath}\NormalTok{.}\FunctionTok{Count} \NormalTok{- }\DecValTok{1}\NormalTok{, }\OtherTok{intPath}\NormalTok{.}\FunctionTok{Count}\NormalTok{);}
    \NormalTok{\} }
\NormalTok{\}}
\end{Highlighting}
\end{Shaded}

\begin{Shaded}
\begin{Highlighting}[]
\CommentTok{/*}
\CommentTok{   HISTORY}
\CommentTok{   }
\CommentTok{02-Aug-10   L-05-28   $$1   pdeshmuk      Created.}

\CommentTok{*/}

\KeywordTok{function} \FunctionTok{deleteItemsInSimpRep} \NormalTok{()}
\NormalTok{\{  }
  \KeywordTok{if} \NormalTok{(!}\FunctionTok{pfcIsWindows}\NormalTok{())}
    \OtherTok{netscape}\NormalTok{.}\OtherTok{security}\NormalTok{.}\OtherTok{PrivilegeManager}\NormalTok{.}\FunctionTok{enablePrivilege}\NormalTok{(}\StringTok{"UniversalXPConnect"}\NormalTok{);}

 \CommentTok{/*--------------------------------------------------------------------*\textbackslash{} }
\CommentTok{   Get the current assembly }
\CommentTok{ \textbackslash{}*--------------------------------------------------------------------*/}  
  \KeywordTok{var} \NormalTok{session = }\FunctionTok{pfcGetProESession} \NormalTok{();}
  \KeywordTok{var} \NormalTok{assembly = }\OtherTok{session}\NormalTok{.}\FunctionTok{CurrentModel}\NormalTok{;}
  
  \KeywordTok{if} \NormalTok{(}\OtherTok{assembly}\NormalTok{.}\FunctionTok{Type} \NormalTok{!= }\FunctionTok{pfcCreate} \NormalTok{(}\StringTok{"pfcModelType"}\NormalTok{).}\FunctionTok{MDL_ASSEMBLY}\NormalTok{)}
    \KeywordTok{throw} \KeywordTok{new} \FunctionTok{Error} \NormalTok{(}\DecValTok{0}\NormalTok{, }\StringTok{"Current model is not an assembly"}\NormalTok{);}
  
 \CommentTok{/*--------------------------------------------------------------------*\textbackslash{} }
\CommentTok{   Get the current active simprep.}
\CommentTok{ \textbackslash{}*--------------------------------------------------------------------*/}  
  \KeywordTok{var} \NormalTok{simp_rep = }\OtherTok{assembly}\NormalTok{.}\FunctionTok{GetActiveSimpRep}\NormalTok{();}
 
 \CommentTok{/*--------------------------------------------------------------------*\textbackslash{} }
\CommentTok{   Get the current number of items}
\CommentTok{ \textbackslash{}*--------------------------------------------------------------------*/}  
  \KeywordTok{var} \NormalTok{simp_rep_instructions = }\OtherTok{simp_rep}\NormalTok{.}\FunctionTok{GetInstructions}\NormalTok{();}
    
  \KeywordTok{var} \NormalTok{number_items = }\OtherTok{simp_rep_instructions}\NormalTok{.}\OtherTok{Items}\NormalTok{.}\FunctionTok{Count}\NormalTok{;}
  \OtherTok{document}\NormalTok{.}\FunctionTok{getElementById}\NormalTok{(}\StringTok{"numItems"}\NormalTok{).}\FunctionTok{value} \NormalTok{= number_items;}

 
 \CommentTok{/*--------------------------------------------------------------------*\textbackslash{} }
\CommentTok{   Deleting items}
\CommentTok{ \textbackslash{}*--------------------------------------------------------------------*/}               
  \KeywordTok{var} \NormalTok{simp_rep_instructions_item = }\OtherTok{simp_rep_instructions}\NormalTok{.}\OtherTok{Items}\NormalTok{.}\FunctionTok{Item}\NormalTok{(number_items}\DecValTok{-1}\NormalTok{);}
  \OtherTok{simp_rep_instructions_item}\NormalTok{.}\FunctionTok{Action} \NormalTok{= }\KeywordTok{null}\NormalTok{;    }
  \OtherTok{simp_rep}\NormalTok{.}\FunctionTok{SetInstructions}\NormalTok{(simp_rep_instructions);  }
  \NormalTok{number_items = }\OtherTok{simp_rep}\NormalTok{.}\FunctionTok{GetInstructions}\NormalTok{().}\OtherTok{Items}\NormalTok{.}\FunctionTok{Count}\NormalTok{;}
  \OtherTok{document}\NormalTok{.}\FunctionTok{getElementById}\NormalTok{(}\StringTok{"numItems"}\NormalTok{).}\FunctionTok{value} \NormalTok{= number_items;}
  

  
  \KeywordTok{return}\NormalTok{;}
\NormalTok{\}}


    


\KeywordTok{function} \FunctionTok{addItemsInSimpRep} \NormalTok{()}
\NormalTok{\{  }
  \KeywordTok{if} \NormalTok{(!}\FunctionTok{pfcIsWindows}\NormalTok{())}
    \OtherTok{netscape}\NormalTok{.}\OtherTok{security}\NormalTok{.}\OtherTok{PrivilegeManager}\NormalTok{.}\FunctionTok{enablePrivilege}\NormalTok{(}\StringTok{"UniversalXPConnect"}\NormalTok{);}

 \CommentTok{/*--------------------------------------------------------------------*\textbackslash{} }
\CommentTok{   Get the current assembly }
\CommentTok{ \textbackslash{}*--------------------------------------------------------------------*/}  
  \KeywordTok{var} \NormalTok{session = }\FunctionTok{pfcGetProESession} \NormalTok{();}
  \KeywordTok{var} \NormalTok{assembly = }\OtherTok{session}\NormalTok{.}\FunctionTok{CurrentModel}\NormalTok{;}
  
  \KeywordTok{if} \NormalTok{(}\OtherTok{assembly}\NormalTok{.}\FunctionTok{Type} \NormalTok{!= }\FunctionTok{pfcCreate} \NormalTok{(}\StringTok{"pfcModelType"}\NormalTok{).}\FunctionTok{MDL_ASSEMBLY}\NormalTok{)}
    \KeywordTok{throw} \KeywordTok{new} \FunctionTok{Error} \NormalTok{(}\DecValTok{0}\NormalTok{, }\StringTok{"Current model is not an assembly"}\NormalTok{);}
  
 \CommentTok{/*--------------------------------------------------------------------*\textbackslash{} }
\CommentTok{   Get the current active simprep.}
\CommentTok{ \textbackslash{}*--------------------------------------------------------------------*/}  
  \KeywordTok{var} \NormalTok{simp_rep = }\OtherTok{assembly}\NormalTok{.}\FunctionTok{GetActiveSimpRep}\NormalTok{();}
 
 \CommentTok{/*--------------------------------------------------------------------*\textbackslash{} }
\CommentTok{   Get the current number of items}
\CommentTok{ \textbackslash{}*--------------------------------------------------------------------*/}  
  \KeywordTok{var} \NormalTok{simp_rep_instructions = }\OtherTok{simp_rep}\NormalTok{.}\FunctionTok{GetInstructions}\NormalTok{();}
    
  \KeywordTok{var} \NormalTok{number_items = }\OtherTok{simp_rep_instructions}\NormalTok{.}\OtherTok{Items}\NormalTok{.}\FunctionTok{Count}\NormalTok{;}
  \OtherTok{document}\NormalTok{.}\FunctionTok{getElementById}\NormalTok{(}\StringTok{"numItems"}\NormalTok{).}\FunctionTok{value} \NormalTok{= number_items;         }

  
 \CommentTok{/*--------------------------------------------------------------------*\textbackslash{} }
\CommentTok{   Add an item  }
\CommentTok{ \textbackslash{}*--------------------------------------------------------------------*/}  
  \KeywordTok{var} \NormalTok{item_path = }\FunctionTok{pfcCreate} \NormalTok{(}\StringTok{"intseq"}\NormalTok{);}
  
  \CommentTok{/*--------------------------------------------------------------------*\textbackslash{} }
\CommentTok{  Prompt for selection.}
\CommentTok{\textbackslash{}*--------------------------------------------------------------------*/}  
  \NormalTok{selOptions = }\FunctionTok{pfcCreate} \NormalTok{(}\StringTok{"pfcSelectionOptions"}\NormalTok{).}\FunctionTok{Create} \NormalTok{(}\StringTok{"feature"}\NormalTok{);}
  
  \OtherTok{selOptions}\NormalTok{.}\FunctionTok{MaxNumSels} \NormalTok{= }\FunctionTok{parseInt} \NormalTok{(}\DecValTok{1}\NormalTok{);}
  
  \KeywordTok{var} \NormalTok{selections = }\KeywordTok{void} \KeywordTok{null}\NormalTok{;}
  \KeywordTok{try} \NormalTok{\{}
    \NormalTok{selections = }\OtherTok{session}\NormalTok{.}\FunctionTok{Select} \NormalTok{(selOptions, }\KeywordTok{void} \KeywordTok{null}\NormalTok{);}
  \NormalTok{\}}
  \KeywordTok{catch} \NormalTok{(err) \{}
\CommentTok{/*--------------------------------------------------------------------*\textbackslash{} }
\CommentTok{  Handle the situation where the  user didn't make selections, but picked }
\CommentTok{  elsewhere instead.}
\CommentTok{  \textbackslash{}*--------------------------------------------------------------------*/}  
  \KeywordTok{if} \NormalTok{(}\FunctionTok{pfcGetExceptionType} \NormalTok{(err) == }\StringTok{"pfcXToolkitUserAbort"} \NormalTok{|| }
    \FunctionTok{pfcGetExceptionType} \NormalTok{(err) == }\StringTok{"pfcXToolkitPickAbove"}\NormalTok{)}
      \KeywordTok{return} \NormalTok{(}\KeywordTok{void} \KeywordTok{null}\NormalTok{);}
    \KeywordTok{else}
      \KeywordTok{throw} \NormalTok{err;}
  \NormalTok{\}}
  \KeywordTok{if} \NormalTok{(}\OtherTok{selections}\NormalTok{.}\FunctionTok{Count} \NormalTok{== }\DecValTok{0}\NormalTok{)}
    \KeywordTok{return} \NormalTok{(}\KeywordTok{void} \KeywordTok{null}\NormalTok{);}

  \KeywordTok{var} \NormalTok{selection = }\OtherTok{selections}\NormalTok{.}\FunctionTok{Item}\NormalTok{(}\DecValTok{0}\NormalTok{);}
  \KeywordTok{var} \NormalTok{componentpath = }\OtherTok{selection}\NormalTok{.}\FunctionTok{Path}\NormalTok{;}
  \KeywordTok{var} \NormalTok{intseqIds = }\OtherTok{componentpath}\NormalTok{.}\FunctionTok{ComponentIds}\NormalTok{;}
  \OtherTok{item_path}\NormalTok{.}\FunctionTok{Append}\NormalTok{(}\OtherTok{intseqIds}\NormalTok{.}\FunctionTok{Item}\NormalTok{(}\DecValTok{0}\NormalTok{));}
  \NormalTok{simp_rep_comp_item_path = }\FunctionTok{pfcCreate}\NormalTok{(}\StringTok{"pfcSimpRepCompItemPath"}\NormalTok{).}\FunctionTok{Create}\NormalTok{(item_path);}
  
  \NormalTok{simp_rep_item = }\FunctionTok{pfcCreate}\NormalTok{(}\StringTok{"pfcSimpRepItem"}\NormalTok{).}\FunctionTok{Create}\NormalTok{(simp_rep_comp_item_path);}
  
  \NormalTok{simp_rep_action = }\FunctionTok{pfcCreate}\NormalTok{(}\StringTok{"pfcSimpRepExclude"}\NormalTok{).}\FunctionTok{Create}\NormalTok{();}
  
  \OtherTok{simp_rep_item}\NormalTok{.}\FunctionTok{Action} \NormalTok{= simp_rep_action;}
  
  \OtherTok{simp_rep_instructions}\NormalTok{.}\OtherTok{Items}\NormalTok{.}\FunctionTok{Append}\NormalTok{(simp_rep_item);}
  
  \OtherTok{simp_rep}\NormalTok{.}\FunctionTok{SetInstructions}\NormalTok{(simp_rep_instructions);}
  
  \NormalTok{simp_rep_instructions = }\OtherTok{simp_rep}\NormalTok{.}\FunctionTok{GetInstructions}\NormalTok{();}
  
  \NormalTok{number_items = }\OtherTok{simp_rep_instructions}\NormalTok{.}\OtherTok{Items}\NormalTok{.}\FunctionTok{Count}\NormalTok{;}
  \OtherTok{document}\NormalTok{.}\FunctionTok{getElementById}\NormalTok{(}\StringTok{"numItems"}\NormalTok{).}\FunctionTok{value} \NormalTok{= number_items; }
  
  \KeywordTok{return}\NormalTok{;}
\NormalTok{\}}

\end{Highlighting}
\end{Shaded}

\begin{Shaded}
\begin{Highlighting}[]
\CommentTok{/*}
\CommentTok{   HISTORY}
\CommentTok{   }
\CommentTok{14-NOV-02   J-03-38   $$1   JCN      Adapted from J-Link examples.}
\CommentTok{07-MAR-03   K-01-03   $$2   JCN      UNIX support}
\CommentTok{*/}

\CommentTok{/* This method retrieves a MassProperty object from the provided solid}
\CommentTok{ * model. Then solid's mass, volume, and center of gravity point are printed}
\CommentTok{ */}
\KeywordTok{function} \FunctionTok{printMassProperties} \NormalTok{()}
\NormalTok{\{}
  \KeywordTok{if} \NormalTok{(!}\FunctionTok{pfcIsWindows}\NormalTok{())}
    \OtherTok{netscape}\NormalTok{.}\OtherTok{security}\NormalTok{.}\OtherTok{PrivilegeManager}\NormalTok{.}\FunctionTok{enablePrivilege}\NormalTok{(}\StringTok{"UniversalXPConnect"}\NormalTok{);}
  
\CommentTok{/*--------------------------------------------------------------------*\textbackslash{} }
\CommentTok{  Get the session. If no model in present abort the operation. }
\CommentTok{\textbackslash{}*--------------------------------------------------------------------*/}  
  \KeywordTok{var} \NormalTok{session = }\FunctionTok{pfcGetProESession} \NormalTok{();}
  \KeywordTok{var} \NormalTok{solid = }\OtherTok{session}\NormalTok{.}\FunctionTok{CurrentModel}\NormalTok{;}
  
  \KeywordTok{if} \NormalTok{(solid == }\KeywordTok{void} \KeywordTok{null} \NormalTok{|| (}\OtherTok{solid}\NormalTok{.}\FunctionTok{Type} \NormalTok{!= }\FunctionTok{pfcCreate} \NormalTok{(}\StringTok{"pfcModelType"}\NormalTok{).}\FunctionTok{MDL_PART} \NormalTok{&&}
                 \OtherTok{solid}\NormalTok{.}\FunctionTok{Type} \NormalTok{!= }\FunctionTok{pfcCreate} \NormalTok{(}\StringTok{"pfcModelType"}\NormalTok{).}\FunctionTok{MDL_ASSEMBLY}\NormalTok{))}
    \KeywordTok{throw} \KeywordTok{new} \FunctionTok{Error} \NormalTok{(}\DecValTok{0}\NormalTok{, }\StringTok{"Current model is not a part or assembly."}\NormalTok{);}
  
\CommentTok{/*--------------------------------------------------------------------*\textbackslash{} }
\CommentTok{  Calculate the mass properties.  Pass null to use the model }
\CommentTok{    coordinate system.}
\CommentTok{\textbackslash{}*--------------------------------------------------------------------*/}  
  \NormalTok{properties = }\OtherTok{solid}\NormalTok{.}\FunctionTok{GetMassProperty}\NormalTok{(}\KeywordTok{void} \KeywordTok{null}\NormalTok{);}
  
\CommentTok{/*--------------------------------------------------------------------*\textbackslash{} }
\CommentTok{  Display selected results.}
\CommentTok{\textbackslash{}*--------------------------------------------------------------------*/}  
  \KeywordTok{var} \NormalTok{newWin = }\OtherTok{window}\NormalTok{.}\FunctionTok{open} \NormalTok{(}\StringTok{''}\NormalTok{, }\StringTok{"_MP"}\NormalTok{, }\StringTok{"scrollbars"}\NormalTok{);}
  \KeywordTok{if} \NormalTok{(}\FunctionTok{pfcIsWindows}\NormalTok{())}
    \NormalTok{\{}
      \OtherTok{newWin}\NormalTok{.}\FunctionTok{resizeTo} \NormalTok{(}\DecValTok{300}\NormalTok{, }\OtherTok{screen}\NormalTok{.}\FunctionTok{height}\NormalTok{/}\FloatTok{2.0}\NormalTok{);}
      \OtherTok{newWin}\NormalTok{.}\FunctionTok{moveTo} \NormalTok{(}\OtherTok{screen}\NormalTok{.}\FunctionTok{width}\DecValTok{-300}\NormalTok{, }\DecValTok{0}\NormalTok{);}
    \NormalTok{\}}
  \OtherTok{newWin}\NormalTok{.}\OtherTok{document}\NormalTok{.}\FunctionTok{writeln} \NormalTok{(}\StringTok{"<html><head></head><body>"}\NormalTok{);}
  
  \OtherTok{newWin}\NormalTok{.}\OtherTok{document}\NormalTok{.}\FunctionTok{writeln} \NormalTok{(}\StringTok{"<p>The solid mass is: "} \NormalTok{+ }\OtherTok{properties}\NormalTok{.}\FunctionTok{Mass}\NormalTok{);}
  \OtherTok{newWin}\NormalTok{.}\OtherTok{document}\NormalTok{.}\FunctionTok{writeln} \NormalTok{(}\StringTok{"<p>The solid volume is: "} \NormalTok{+ }\OtherTok{properties}\NormalTok{.}\FunctionTok{Volume}\NormalTok{);}
  
  \NormalTok{COG = }\OtherTok{properties}\NormalTok{.}\FunctionTok{GravityCenter}\NormalTok{;}
  \OtherTok{newWin}\NormalTok{.}\OtherTok{document}\NormalTok{.}\FunctionTok{writeln} \NormalTok{(}\StringTok{"<hr><p>The Center Of Gravity is at "}\NormalTok{);}
  \OtherTok{newWin}\NormalTok{.}\OtherTok{document}\NormalTok{.}\FunctionTok{writeln} \NormalTok{(}\StringTok{"<table>"}\NormalTok{);}
  \OtherTok{newWin}\NormalTok{.}\OtherTok{document}\NormalTok{.}\FunctionTok{writeln} \NormalTok{(}\StringTok{"<tr><td>X</td><td>"}\NormalTok{+}\OtherTok{COG}\NormalTok{.}\FunctionTok{Item}\NormalTok{(}\DecValTok{0}\NormalTok{)+}\StringTok{"</td></tr>"}\NormalTok{);}
  \OtherTok{newWin}\NormalTok{.}\OtherTok{document}\NormalTok{.}\FunctionTok{writeln} \NormalTok{(}\StringTok{"<tr><td>Y</td><td>"}\NormalTok{+}\OtherTok{COG}\NormalTok{.}\FunctionTok{Item}\NormalTok{(}\DecValTok{1}\NormalTok{)+}\StringTok{"</td></tr>"}\NormalTok{);}
  \OtherTok{newWin}\NormalTok{.}\OtherTok{document}\NormalTok{.}\FunctionTok{writeln} \NormalTok{(}\StringTok{"<tr><td>Z</td><td>"}\NormalTok{+}\OtherTok{COG}\NormalTok{.}\FunctionTok{Item}\NormalTok{(}\DecValTok{2}\NormalTok{)+}\StringTok{"</td></tr>"}\NormalTok{);}
  \OtherTok{newWin}\NormalTok{.}\OtherTok{document}\NormalTok{.}\FunctionTok{writeln} \NormalTok{(}\StringTok{"</table>"}\NormalTok{); }
  \OtherTok{newWin}\NormalTok{.}\OtherTok{document}\NormalTok{.}\FunctionTok{writeln} \NormalTok{(}\StringTok{"<html><head></head><body>"}\NormalTok{);}
\NormalTok{\}}
\end{Highlighting}
\end{Shaded}

\begin{Shaded}
\begin{Highlighting}[]
\CommentTok{/*}
\CommentTok{   HISTORY}

\CommentTok{14-NOV-02   J-03-38   $$1   JCN      Adapted from J-Link examples.}
\CommentTok{07-MAR-03   K-01-03   $$2   JCN      UNIX support}
\CommentTok{ */}

\CommentTok{/*====================================================================*\textbackslash{}}
\NormalTok{FUNCTION: createNodeUDFInPart}
\NormalTok{PURPOSE:  Places copies of a node UDF at a particular coordinate system }
          \NormalTok{location }\KeywordTok{in} \NormalTok{a }\OtherTok{part}\NormalTok{.  }\FunctionTok{The} \FunctionTok{node} \FunctionTok{UDF} \FunctionTok{is} \FunctionTok{a} \FunctionTok{spherical} \FunctionTok{cut} \FunctionTok{centered} \FunctionTok{at} \FunctionTok{the} 
          \NormalTok{coordinate system whose diameter is driven by the }\StringTok{'diam'} \NormalTok{argument to the }
          \OtherTok{method}\NormalTok{.}
\NormalTok{\textbackslash{}*====================================================================*}\OtherTok{/}
\KeywordTok{function} \FunctionTok{createNodeUDFInPart} \NormalTok{(csysName }\CommentTok{/* string */}\NormalTok{, }
                  \NormalTok{diam }\CommentTok{/* number */}\NormalTok{) }
\NormalTok{\{}
  \KeywordTok{if} \NormalTok{(!}\FunctionTok{pfcIsWindows}\NormalTok{())}
    \OtherTok{netscape}\NormalTok{.}\OtherTok{security}\NormalTok{.}\OtherTok{PrivilegeManager}\NormalTok{.}\FunctionTok{enablePrivilege}\NormalTok{(}\StringTok{"UniversalXPConnect"}\NormalTok{);}
  
\CommentTok{/*------------------------------------------------------------------*\textbackslash{}}
  \NormalTok{Use the current model to place the }\OtherTok{UDF}\NormalTok{.}
\NormalTok{\textbackslash{}*------------------------------------------------------------------*}\OtherTok{/}
  \KeywordTok{var} \NormalTok{session = }\FunctionTok{pfcGetProESession} \NormalTok{();}
  \KeywordTok{var} \NormalTok{solid = }\OtherTok{session}\NormalTok{.}\FunctionTok{CurrentModel}\NormalTok{;}
  
  \KeywordTok{if} \NormalTok{(solid == }\KeywordTok{void} \KeywordTok{null} \NormalTok{|| }\OtherTok{solid}\NormalTok{.}\FunctionTok{Type} \NormalTok{!= }\FunctionTok{pfcCreate} \NormalTok{(}\StringTok{"pfcModelType"}\NormalTok{).}\FunctionTok{MDL_PART}\NormalTok{)}
    \KeywordTok{throw} \KeywordTok{new} \FunctionTok{Error} \NormalTok{(}\DecValTok{0}\NormalTok{, }\StringTok{"Current model is not a part.  Aborting..."}\NormalTok{);}
  
\CommentTok{/*------------------------------------------------------------------*\textbackslash{}}
  \NormalTok{The instructions }\KeywordTok{for} \NormalTok{the UDF }\OtherTok{creation}\NormalTok{.}
\NormalTok{\textbackslash{}*------------------------------------------------------------------*}\OtherTok{/  }
\OtherTok{  var instrs = }
\OtherTok{    pfcCreate }\FloatTok{(}\OtherTok{"pfcUDFCustomCreateInstructions"}\FloatTok{)}\OtherTok{.Create }\FloatTok{(}\OtherTok{"node"}\FloatTok{)}\OtherTok{;}
\OtherTok{    }
\OtherTok{/}\NormalTok{*------------------------------------------------------------------*\textbackslash{}}
  \NormalTok{Make non-variant dimensions blank so they cannot be }\OtherTok{changed}\NormalTok{.}
\NormalTok{\textbackslash{}*------------------------------------------------------------------*}\OtherTok{/  }
\OtherTok{  instrs.DimDisplayType = }
\OtherTok{    pfcCreate }\FloatTok{(}\OtherTok{"pfcUDFDimensionDisplayType"}\FloatTok{)}\OtherTok{.UDFDISPLAY_BLANK;}
\OtherTok{  }
\OtherTok{/}\NormalTok{*------------------------------------------------------------------*\textbackslash{}}
  \NormalTok{Initialize the UDF reference and assign it to the }\OtherTok{instructions}\NormalTok{.  }
  \NormalTok{The string argument is the reference prompt }\KeywordTok{for} \NormalTok{the particular }
  \OtherTok{reference}\NormalTok{.}
\NormalTok{\textbackslash{}*------------------------------------------------------------------*}\OtherTok{/       }
\OtherTok{  csys = }
\OtherTok{    solid.GetItemByName }\FloatTok{(}\OtherTok{pfcCreate }\FloatTok{(}\OtherTok{"pfcModelItemType"}\FloatTok{)}\OtherTok{.ITEM_COORD_SYS, }
\OtherTok{             csysName}\FloatTok{)}\OtherTok{;}
\OtherTok{  if }\FloatTok{(}\OtherTok{csys == void null}\FloatTok{)}
\OtherTok{    throw new Error }\FloatTok{(}\OtherTok{0, "Requested coordinate system "}\FloatTok{+}\OtherTok{csysName}\FloatTok{+}\OtherTok{" not found."}\FloatTok{)}\OtherTok{;}
\OtherTok{  csysSel = }
\OtherTok{    pfcCreate }\FloatTok{(}\OtherTok{"MpfcSelect"}\FloatTok{)}\OtherTok{.CreateModelItemSelection  }\FloatTok{(}\OtherTok{csys, void null}\FloatTok{)}\OtherTok{;}
\OtherTok{  }
\OtherTok{  var csysRef = }
\OtherTok{    pfcCreate }\FloatTok{(}\OtherTok{"pfcUDFReference"}\FloatTok{)}\OtherTok{.Create }\FloatTok{(}\OtherTok{"REF_CSYS", csysSel}\FloatTok{)}\OtherTok{;}
\OtherTok{  }
\OtherTok{  var refs = pfcCreate }\FloatTok{(}\OtherTok{"pfcUDFReferences"}\FloatTok{)}\OtherTok{;}
\OtherTok{  refs.Append }\FloatTok{(}\OtherTok{csysRef}\FloatTok{)}\OtherTok{;}
\OtherTok{  }
\OtherTok{  instrs.References = refs;}
\OtherTok{  }
\OtherTok{/}\NormalTok{*------------------------------------------------------------------*\textbackslash{}}
  \NormalTok{Initialize the variant dimension and assign it to the }\OtherTok{instructions}\NormalTok{.  }
  \NormalTok{The string argument is the dimension symbol }\KeywordTok{for} \NormalTok{the variant }
  \OtherTok{dimension}\NormalTok{.}
\NormalTok{\textbackslash{}*------------------------------------------------------------------*}\OtherTok{/  }
\OtherTok{   var varDiam = }
\OtherTok{     pfcCreate }\FloatTok{(}\OtherTok{"pfcUDFVariantDimension"}\FloatTok{)}\OtherTok{.Create }\FloatTok{(}\OtherTok{"d11", diam}\FloatTok{)}\OtherTok{;}
\OtherTok{   }
\OtherTok{   var vals = pfcCreate }\FloatTok{(}\OtherTok{"pfcUDFVariantValues"}\FloatTok{)}\OtherTok{;}
\OtherTok{   vals.Append }\FloatTok{(}\OtherTok{varDiam}\FloatTok{)}\OtherTok{;}
\OtherTok{   }
\OtherTok{   instrs.VariantValues = vals;}
\OtherTok{   }
\OtherTok{/}\NormalTok{*------------------------------------------------------------------*\textbackslash{}}
  \NormalTok{Create the }\KeywordTok{new} \NormalTok{UDF }\OtherTok{placement}\NormalTok{.}
\NormalTok{\textbackslash{}*------------------------------------------------------------------*}\OtherTok{/  }
\OtherTok{   var group = solid.CreateUDFGroup }\FloatTok{(}\OtherTok{instrs}\FloatTok{)}\OtherTok{;}
\OtherTok{    return }\FloatTok{(}\OtherTok{group}\FloatTok{)}\OtherTok{;}
\OtherTok{\}}
\end{Highlighting}
\end{Shaded}

\begin{Shaded}
\begin{Highlighting}[]
\CommentTok{/*}
\CommentTok{   HISTORY}

\CommentTok{14-NOV-02   J-03-38   $$1   JCN      Submitted.}
\CommentTok{07-MAR-03   K-01-03   $$2   JCN      UNIX support}
\CommentTok{01-MAY-07   L-01-31   $$3   JCN      New exception messaging.}
\CommentTok{ */}

\KeywordTok{function} \FunctionTok{isProEEmbeddedBrowser} \NormalTok{()}
\NormalTok{\{}
  \KeywordTok{if} \NormalTok{(}\OtherTok{top}\NormalTok{.}\FunctionTok{external} \NormalTok{&& }\OtherTok{top}\NormalTok{.}\OtherTok{external}\NormalTok{.}\FunctionTok{ptc}\NormalTok{)}
    \KeywordTok{return} \KeywordTok{true}\NormalTok{;}
  \KeywordTok{else}
    \KeywordTok{return} \KeywordTok{false}\NormalTok{;}
\NormalTok{\}}

\KeywordTok{function} \FunctionTok{pfcIsWindows} \NormalTok{()}
\NormalTok{\{}
  \KeywordTok{if} \NormalTok{(}\OtherTok{navigator}\NormalTok{.}\OtherTok{appName}\NormalTok{.}\FunctionTok{indexOf} \NormalTok{(}\StringTok{"Microsoft"}\NormalTok{) != -}\DecValTok{1}\NormalTok{)}
    \KeywordTok{return} \KeywordTok{true}\NormalTok{;}
  \KeywordTok{else}
    \KeywordTok{return} \KeywordTok{false}\NormalTok{;}
\NormalTok{\}}

\KeywordTok{function} \FunctionTok{pfcCreate} \NormalTok{(className)}
\NormalTok{\{}
  \KeywordTok{if} \NormalTok{(!}\FunctionTok{pfcIsWindows}\NormalTok{())  }
    \OtherTok{netscape}\NormalTok{.}\OtherTok{security}\NormalTok{.}\OtherTok{PrivilegeManager}\NormalTok{.}\FunctionTok{enablePrivilege}\NormalTok{(}\StringTok{"UniversalXPConnect"}\NormalTok{);}
  
  \KeywordTok{if} \NormalTok{(}\FunctionTok{pfcIsWindows}\NormalTok{())}
    \KeywordTok{return} \KeywordTok{new} \FunctionTok{ActiveXObject} \NormalTok{(}\StringTok{"pfc."}\NormalTok{+className);}
  \KeywordTok{else}
    \NormalTok{\{}
      \NormalTok{ret = }\OtherTok{Components}\NormalTok{.}\FunctionTok{classes} \NormalTok{[}\StringTok{"@ptc.com/pfc/"} \NormalTok{+ className + }\StringTok{";1"}\NormalTok{].}
    
    \FunctionTok{createInstance}\NormalTok{();}
      
      \KeywordTok{return} \NormalTok{ret;}
    \NormalTok{\}}
\NormalTok{\}}

\KeywordTok{function} \FunctionTok{pfcGetProESession} \NormalTok{()}
\NormalTok{\{}
  \KeywordTok{if} \NormalTok{(!}\FunctionTok{isProEEmbeddedBrowser} \NormalTok{())}
    \NormalTok{\{}
      \KeywordTok{throw} \KeywordTok{new} \FunctionTok{Error} \NormalTok{(}\StringTok{"Not in embedded browser.  Aborting..."}\NormalTok{);}
    \NormalTok{\}}
  
  \CommentTok{// Security code}
  \KeywordTok{if} \NormalTok{(!}\FunctionTok{pfcIsWindows}\NormalTok{())}
    \OtherTok{netscape}\NormalTok{.}\OtherTok{security}\NormalTok{.}\OtherTok{PrivilegeManager}\NormalTok{.}\FunctionTok{enablePrivilege}\NormalTok{(}\StringTok{"UniversalXPConnect"}\NormalTok{);}
  
  \KeywordTok{var} \NormalTok{glob = }\FunctionTok{pfcCreate} \NormalTok{(}\StringTok{"MpfcCOMGlobal"}\NormalTok{);}
  \KeywordTok{return} \OtherTok{glob}\NormalTok{.}\FunctionTok{GetProESession}\NormalTok{();}
\NormalTok{\}}

\KeywordTok{function} \FunctionTok{pfcGetScript} \NormalTok{()}
\NormalTok{\{  }
  \KeywordTok{if} \NormalTok{(!}\FunctionTok{isProEEmbeddedBrowser} \NormalTok{())}
    \NormalTok{\{}
      \KeywordTok{throw} \KeywordTok{new} \FunctionTok{Error} \NormalTok{(}\StringTok{"Not in embedded browser.  Aborting..."}\NormalTok{);}
    \NormalTok{\}}
  
  \CommentTok{// Security code}
  \KeywordTok{if} \NormalTok{(!}\FunctionTok{pfcIsWindows}\NormalTok{())}
    \OtherTok{netscape}\NormalTok{.}\OtherTok{security}\NormalTok{.}\OtherTok{PrivilegeManager}\NormalTok{.}\FunctionTok{enablePrivilege}\NormalTok{(}\StringTok{"UniversalXPConnect"}\NormalTok{);}
  
  \KeywordTok{var} \NormalTok{glob = }\FunctionTok{pfcCreate} \NormalTok{(}\StringTok{"MpfcCOMGlobal"}\NormalTok{);}
  \KeywordTok{return} \OtherTok{glob}\NormalTok{.}\FunctionTok{GetScript}\NormalTok{();}
\NormalTok{\}}


\KeywordTok{function} \FunctionTok{pfcGetExceptionDescription} \NormalTok{(err)}
\NormalTok{\{}
 \KeywordTok{if} \NormalTok{(}\FunctionTok{pfcIsWindows}\NormalTok{())}
    \NormalTok{errString = }\OtherTok{err}\NormalTok{.}\FunctionTok{description}\NormalTok{;}
 \KeywordTok{else}
      \NormalTok{errString = }\OtherTok{err}\NormalTok{.}\FunctionTok{message}\NormalTok{;}

 \KeywordTok{return} \NormalTok{errString;}
\NormalTok{\}}

\KeywordTok{function} \FunctionTok{pfcGetExceptionType} \NormalTok{(err)}
\NormalTok{\{}
  \NormalTok{errString = }\FunctionTok{pfcGetExceptionDescription} \NormalTok{(err);}

  \CommentTok{// This should remove the XPCOM prefix ("XPCR_C")}
  \KeywordTok{if} \NormalTok{(}\OtherTok{errString}\NormalTok{.}\FunctionTok{search} \NormalTok{(}\StringTok{"XPCR_C"}\NormalTok{) < }\DecValTok{0}\NormalTok{)}
  \NormalTok{\{}
    \NormalTok{errString = }\OtherTok{errString}\NormalTok{.}\FunctionTok{replace} \NormalTok{(}\StringTok{"Exceptions::"}\NormalTok{, }\StringTok{""}\NormalTok{);}
    \NormalTok{semicolonIndex = }\OtherTok{errString}\NormalTok{.}\FunctionTok{search} \NormalTok{(}\StringTok{";"}\NormalTok{);}
    \KeywordTok{if} \NormalTok{(semicolonIndex > }\DecValTok{0}\NormalTok{)}
        \NormalTok{errString = }\OtherTok{errString}\NormalTok{.}\FunctionTok{substring} \NormalTok{(}\DecValTok{0}\NormalTok{, semicolonIndex);}
    \KeywordTok{return} \NormalTok{(errString);}
  \NormalTok{\}}
  \KeywordTok{else}
      \KeywordTok{return} \NormalTok{(}\OtherTok{errString}\NormalTok{.}\FunctionTok{replace}\NormalTok{(}\StringTok{"XPCR_C"}\NormalTok{, }\StringTok{""}\NormalTok{));}
\NormalTok{\}}

      
\end{Highlighting}
\end{Shaded}

\begin{Shaded}
\begin{Highlighting}[]
\CommentTok{/*}
\CommentTok{   HISTORY}
\CommentTok{   }
\CommentTok{14-NOV-02   J-03-38   $$1   JCN      Adapted from J-Link examples.}
\CommentTok{07-MAR-03   K-01-03   $$2   JCN      UNIX support}
\CommentTok{ */}


\KeywordTok{try}
\NormalTok{\{}
  \KeywordTok{if} \NormalTok{(!}\FunctionTok{pfcIsWindows}\NormalTok{())}
    \OtherTok{netscape}\NormalTok{.}\OtherTok{security}\NormalTok{.}\OtherTok{PrivilegeManager}\NormalTok{.}\FunctionTok{enablePrivilege}\NormalTok{(}\StringTok{"UniversalXPConnect"}\NormalTok{);}
  
  \NormalTok{wpwl = }\FunctionTok{pfcGetScript} \NormalTok{();}
  \OtherTok{document}\NormalTok{.}\FunctionTok{pwl} \NormalTok{= wpwl;}
  \NormalTok{wpwlc = }\OtherTok{wpwl}\NormalTok{.}\FunctionTok{GetPWLConstants} \NormalTok{();}
  \OtherTok{document}\NormalTok{.}\FunctionTok{pwlc} \NormalTok{= wpwlc;}
  \NormalTok{wpwlf = }\OtherTok{wpwl}\NormalTok{.}\FunctionTok{GetPWLFeatureConstants} \NormalTok{();}
  \OtherTok{document}\NormalTok{.}\FunctionTok{pwlf} \NormalTok{= wpwlf;}
\NormalTok{\}}
\KeywordTok{catch} \NormalTok{(err)}
\NormalTok{\{}
  \FunctionTok{alert} \NormalTok{(}\StringTok{"Exception caught: "}\NormalTok{+}\FunctionTok{pfcGetExceptionType} \NormalTok{(err));}
\NormalTok{\}}

\KeywordTok{function} \FunctionTok{WlProEStart}\NormalTok{()}
\NormalTok{\{ }
  \KeywordTok{if} \NormalTok{(}\OtherTok{document}\NormalTok{.}\FunctionTok{pwl} \NormalTok{== }\KeywordTok{void} \KeywordTok{null}\NormalTok{)}
    \NormalTok{\{}
      \FunctionTok{alert}\NormalTok{(}\StringTok{"Connect failed."}\NormalTok{);}
      \KeywordTok{return} \NormalTok{;}
    \NormalTok{\}}
\NormalTok{\}}

\KeywordTok{function} \FunctionTok{WlProEConnect}\NormalTok{()}
\CommentTok{//  Connect to a running Pro/ENGINEER session.}
\NormalTok{\{}
  \FunctionTok{WlProEStart}\NormalTok{();}
\NormalTok{\}}

\KeywordTok{function} \FunctionTok{WlModelOpen}\NormalTok{()}
\CommentTok{//  Open a Pro/ENGINEER model.}
\NormalTok{\{}
  \KeywordTok{if} \NormalTok{(!}\FunctionTok{pfcIsWindows}\NormalTok{())}
    \OtherTok{netscape}\NormalTok{.}\OtherTok{security}\NormalTok{.}\OtherTok{PrivilegeManager}\NormalTok{.}\FunctionTok{enablePrivilege}\NormalTok{(}\StringTok{"UniversalXPConnect"}\NormalTok{);}
  \KeywordTok{if} \NormalTok{(}\OtherTok{document}\NormalTok{.}\OtherTok{main}\NormalTok{.}\OtherTok{ModelName}\NormalTok{.}\FunctionTok{value} \NormalTok{== }\StringTok{""}\NormalTok{)}
    \KeywordTok{return} \NormalTok{;}
  \KeywordTok{var} \NormalTok{ret = }\OtherTok{document}\NormalTok{.}\OtherTok{pwl}\NormalTok{.}\FunctionTok{pwlMdlOpen}\NormalTok{(}\OtherTok{document}\NormalTok{.}\OtherTok{main}\NormalTok{.}\OtherTok{ModelName}\NormalTok{.}\FunctionTok{value}\NormalTok{,}
                    \OtherTok{document}\NormalTok{.}\OtherTok{main}\NormalTok{.}\OtherTok{ModelPath}\NormalTok{.}\FunctionTok{value}\NormalTok{, }\KeywordTok{true}\NormalTok{);}
  \KeywordTok{if} \NormalTok{(!}\OtherTok{ret}\NormalTok{.}\FunctionTok{Status}\NormalTok{)}
    \NormalTok{\{}
      \KeywordTok{if} \NormalTok{(}\OtherTok{ret}\NormalTok{.}\FunctionTok{ErrorCode} \NormalTok{== -}\DecValTok{4} \NormalTok{&& }\OtherTok{document}\NormalTok{.}\OtherTok{main}\NormalTok{.}\OtherTok{ModelPath}\NormalTok{.}\FunctionTok{value} \NormalTok{== }\StringTok{""}\NormalTok{)}
    \KeywordTok{return} \NormalTok{;}
      \KeywordTok{else}
        \NormalTok{\{}
      \FunctionTok{alert}\NormalTok{(}\StringTok{"pwlMdlOpen failed ("} \NormalTok{+ }\OtherTok{ret}\NormalTok{.}\FunctionTok{ErrorCode} \NormalTok{+ }\StringTok{")"}\NormalTok{);}
      \KeywordTok{return} \NormalTok{;}
        \NormalTok{\}}
    \NormalTok{\}}
\NormalTok{\}}

\KeywordTok{function} \FunctionTok{WlModelRegenerate}\NormalTok{()}
\CommentTok{//  Regenerate the Pro/ENGINEER model.}
\NormalTok{\{}
  \KeywordTok{if} \NormalTok{(!}\FunctionTok{pfcIsWindows}\NormalTok{())}
    \OtherTok{netscape}\NormalTok{.}\OtherTok{security}\NormalTok{.}\OtherTok{PrivilegeManager}\NormalTok{.}\FunctionTok{enablePrivilege}\NormalTok{(}\StringTok{"UniversalXPConnect"}\NormalTok{);}
  \KeywordTok{var} \NormalTok{ret = }\OtherTok{document}\NormalTok{.}\OtherTok{pwl}\NormalTok{.}\FunctionTok{pwlMdlRegenerate}\NormalTok{(}\OtherTok{document}\NormalTok{.}\OtherTok{main}\NormalTok{.}\OtherTok{ModelNameExt}\NormalTok{.}\FunctionTok{value}\NormalTok{);}
  \KeywordTok{if} \NormalTok{(!}\OtherTok{ret}\NormalTok{.}\FunctionTok{Status}\NormalTok{)}
    \NormalTok{\{}
      \FunctionTok{alert}\NormalTok{(}\StringTok{"pwlMdlRegenerate failed ("} \NormalTok{+ }\OtherTok{ret}\NormalTok{.}\FunctionTok{ErrorCode} \NormalTok{+ }\StringTok{")"}\NormalTok{);}
      \KeywordTok{return} \NormalTok{;}
    \NormalTok{\}}
\NormalTok{\}}

\KeywordTok{function} \FunctionTok{WlModelSave}\NormalTok{()}
\CommentTok{//  Save a Pro/ENGINEER model.}
\NormalTok{\{ }
  \KeywordTok{if} \NormalTok{(!}\FunctionTok{pfcIsWindows}\NormalTok{())}
    \OtherTok{netscape}\NormalTok{.}\OtherTok{security}\NormalTok{.}\OtherTok{PrivilegeManager}\NormalTok{.}\FunctionTok{enablePrivilege}\NormalTok{(}\StringTok{"UniversalXPConnect"}\NormalTok{);}
  \KeywordTok{var} \NormalTok{ret = }\OtherTok{document}\NormalTok{.}\OtherTok{pwl}\NormalTok{.}\FunctionTok{pwlMdlSaveAs}\NormalTok{(}\OtherTok{document}\NormalTok{.}\OtherTok{main}\NormalTok{.}\OtherTok{ModelNameExt}\NormalTok{.}\FunctionTok{value}\NormalTok{, }\KeywordTok{void} \KeywordTok{null}\NormalTok{, }\KeywordTok{void} \KeywordTok{null}\NormalTok{);}
  \KeywordTok{if} \NormalTok{(!}\OtherTok{ret}\NormalTok{.}\FunctionTok{Status}\NormalTok{)}
    \NormalTok{\{}
      \FunctionTok{alert}\NormalTok{(}\StringTok{"pwlMdlSaveAs failed ("} \NormalTok{+ }\OtherTok{ret}\NormalTok{.}\FunctionTok{ErrorCode} \NormalTok{+ }\StringTok{")"}\NormalTok{);}
      \KeywordTok{return} \NormalTok{;}
    \NormalTok{\}}
\NormalTok{\}}

\KeywordTok{function} \FunctionTok{WlModelSaveAs}\NormalTok{()}
\CommentTok{//  Save a Pro/ENGINEER model under a new name.}
\NormalTok{\{}
  \KeywordTok{if} \NormalTok{(!}\FunctionTok{pfcIsWindows}\NormalTok{())}
    \OtherTok{netscape}\NormalTok{.}\OtherTok{security}\NormalTok{.}\OtherTok{PrivilegeManager}\NormalTok{.}\FunctionTok{enablePrivilege}\NormalTok{(}\StringTok{"UniversalXPConnect"}\NormalTok{);}
  \KeywordTok{var} \NormalTok{NewPath = }\OtherTok{document}\NormalTok{.}\OtherTok{main}\NormalTok{.}\OtherTok{NewPath}\NormalTok{.}\FunctionTok{value}\NormalTok{;}
  \KeywordTok{var} \NormalTok{NewName = }\OtherTok{document}\NormalTok{.}\OtherTok{main}\NormalTok{.}\OtherTok{NewName}\NormalTok{.}\FunctionTok{value}\NormalTok{;}
  \KeywordTok{if} \NormalTok{(NewPath == }\StringTok{""}\NormalTok{)}
    \NormalTok{\{}
      \NormalTok{NewPath = }\KeywordTok{void} \KeywordTok{null}\NormalTok{;}
    \NormalTok{\}}
  \KeywordTok{if} \NormalTok{(NewName == }\StringTok{""}\NormalTok{)}
    \NormalTok{\{}
      \NormalTok{NewName = }\KeywordTok{void} \KeywordTok{null}\NormalTok{;}
    \NormalTok{\}}
  \KeywordTok{var} \NormalTok{ret = }\OtherTok{document}\NormalTok{.}\OtherTok{pwl}\NormalTok{.}\FunctionTok{pwlMdlSaveAs}\NormalTok{(}\OtherTok{document}\NormalTok{.}\OtherTok{main}\NormalTok{.}\OtherTok{ModelNameExt}\NormalTok{.}\FunctionTok{value}\NormalTok{,}
                      \NormalTok{NewPath, NewName);}
  \KeywordTok{if} \NormalTok{(!}\OtherTok{ret}\NormalTok{.}\FunctionTok{Status}\NormalTok{)}
    \NormalTok{\{}
      \FunctionTok{alert}\NormalTok{(}\StringTok{"pwlMdlSaveAs failed ("} \NormalTok{+ }\OtherTok{ret}\NormalTok{.}\FunctionTok{ErrorCode} \NormalTok{+ }\StringTok{")"}\NormalTok{);}
      \KeywordTok{return} \NormalTok{;}
    \NormalTok{\}}
\NormalTok{\}}

\KeywordTok{function} \FunctionTok{WlWindowRepaint}\NormalTok{()}
\CommentTok{//  Repaint the active window.}
\NormalTok{\{}
  \KeywordTok{if} \NormalTok{(!}\FunctionTok{pfcIsWindows}\NormalTok{())}
    \OtherTok{netscape}\NormalTok{.}\OtherTok{security}\NormalTok{.}\OtherTok{PrivilegeManager}\NormalTok{.}\FunctionTok{enablePrivilege}\NormalTok{(}\StringTok{"UniversalXPConnect"}\NormalTok{);}
  \KeywordTok{var} \NormalTok{get_ret = }\OtherTok{document}\NormalTok{.}\OtherTok{pwl}\NormalTok{.}\FunctionTok{pwlWindowActiveGet}\NormalTok{();}
  \KeywordTok{if} \NormalTok{(!}\OtherTok{get_ret}\NormalTok{.}\FunctionTok{Status}\NormalTok{)}
    \NormalTok{\{}
      \FunctionTok{alert}\NormalTok{(}\StringTok{"pwlWindowActiveGet failed ("} \NormalTok{+ }\OtherTok{get_ret}\NormalTok{.}\FunctionTok{ErrorCode} \NormalTok{+ }\StringTok{")"}\NormalTok{);}
      \KeywordTok{return} \NormalTok{;}
    \NormalTok{\}}
  \CommentTok{/* You can also repaint the active window using -1 as the window}
\CommentTok{     identifier. */}
  \KeywordTok{var} \NormalTok{ret = }\OtherTok{document}\NormalTok{.}\OtherTok{pwl}\NormalTok{.}\FunctionTok{pwlWindowRepaint}\NormalTok{(}\FunctionTok{parseInt}\NormalTok{(}\OtherTok{get_ret}\NormalTok{.}\FunctionTok{WindowID}\NormalTok{));}
  \KeywordTok{if} \NormalTok{(!}\OtherTok{ret}\NormalTok{.}\FunctionTok{Status}\NormalTok{)}
    \NormalTok{\{}
      \FunctionTok{alert}\NormalTok{(}\StringTok{"pwlWindowRepaint failed ("} \NormalTok{+ }\OtherTok{ret}\NormalTok{.}\FunctionTok{ErrorCode} \NormalTok{+ }\StringTok{")"}\NormalTok{);}
      \KeywordTok{return} \NormalTok{;}
    \NormalTok{\}}
\NormalTok{\}}

\CommentTok{// Define the form with all the buttons to perform the above actions.}
\OtherTok{document}\NormalTok{.}\FunctionTok{writeln}\NormalTok{(}\StringTok{"<form name='main'>"}\NormalTok{);}

\OtherTok{document}\NormalTok{.}\FunctionTok{writeln}\NormalTok{(}\StringTok{"<hr>"}\NormalTok{);}
\OtherTok{document}\NormalTok{.}\FunctionTok{writeln}\NormalTok{(}\StringTok{"<h4>Main Controls</h4>"}\NormalTok{);}
\OtherTok{document}\NormalTok{.}\FunctionTok{writeln}\NormalTok{(}\StringTok{"<p>"}\NormalTok{);}
\OtherTok{document}\NormalTok{.}\FunctionTok{writeln}\NormalTok{(}\StringTok{"<center>"}\NormalTok{);}
\OtherTok{document}\NormalTok{.}\FunctionTok{writeln}\NormalTok{(}\StringTok{"<input type='button' value='Start Pro/E' onclick='WlProEStart()'>"}\NormalTok{);}
\OtherTok{document}\NormalTok{.}\FunctionTok{writeln}\NormalTok{(}\StringTok{"<input type='button' value='Connect to Pro/E' onclick='WlProEConnect()'>"}\NormalTok{);}
\OtherTok{document}\NormalTok{.}\FunctionTok{writeln}\NormalTok{(}\StringTok{"<p>"}\NormalTok{);}
\OtherTok{document}\NormalTok{.}\FunctionTok{writeln}\NormalTok{(}\StringTok{"Path: <input type='text' name='ModelPath' onchange='WlModelOpen()'>"}\NormalTok{);}
\OtherTok{document}\NormalTok{.}\FunctionTok{writeln}\NormalTok{(}\StringTok{"<spacer size=20>"}\NormalTok{);}
\OtherTok{document}\NormalTok{.}\FunctionTok{writeln}\NormalTok{(}\StringTok{"Model: <input type='text' name='ModelName' onchange='WlModelOpen()'>"}\NormalTok{);}
\OtherTok{document}\NormalTok{.}\FunctionTok{writeln}\NormalTok{(}\StringTok{"<spacer size=20>"}\NormalTok{);}
\OtherTok{document}\NormalTok{.}\FunctionTok{writeln}\NormalTok{(}\StringTok{"<input type='button' value='Open Model' onclick='WlModelOpen()'>"}\NormalTok{);}
\OtherTok{document}\NormalTok{.}\FunctionTok{writeln}\NormalTok{(}\StringTok{"<p>"}\NormalTok{);}
\OtherTok{document}\NormalTok{.}\FunctionTok{writeln}\NormalTok{(}\StringTok{"<table>"}\NormalTok{);}
\OtherTok{document}\NormalTok{.}\FunctionTok{writeln}\NormalTok{(}\StringTok{"<tr>"}\NormalTok{);}
\OtherTok{document}\NormalTok{.}\FunctionTok{writeln}\NormalTok{(}\StringTok{"<td><center>Model:</center></td>"}\NormalTok{);}
\OtherTok{document}\NormalTok{.}\FunctionTok{writeln}\NormalTok{(}\StringTok{"<td><center>New Path:</center></td>"}\NormalTok{);}
\OtherTok{document}\NormalTok{.}\FunctionTok{writeln}\NormalTok{(}\StringTok{"<td><center>New Name:</center></td></tr>"}\NormalTok{);}
\OtherTok{document}\NormalTok{.}\FunctionTok{writeln}\NormalTok{(}\StringTok{"<tr>"}\NormalTok{);}
\OtherTok{document}\NormalTok{.}\FunctionTok{writeln}\NormalTok{(}\StringTok{"<td><input type='text' name='ModelNameExt'></td>"}\NormalTok{);}
\OtherTok{document}\NormalTok{.}\FunctionTok{writeln}\NormalTok{(}\StringTok{"<td><input type='text' name='NewPath'></td>"}\NormalTok{);}
\OtherTok{document}\NormalTok{.}\FunctionTok{writeln}\NormalTok{(}\StringTok{"<td><input type='text' name='NewName'></td></tr>"}\NormalTok{);}
\OtherTok{document}\NormalTok{.}\FunctionTok{writeln}\NormalTok{(}\StringTok{"</table>"}\NormalTok{);}
\OtherTok{document}\NormalTok{.}\FunctionTok{writeln}\NormalTok{(}\StringTok{"<input type='button' value='Regenerate Model' onclick='WlModelRegenerate()'>"}\NormalTok{);}
\OtherTok{document}\NormalTok{.}\FunctionTok{writeln}\NormalTok{(}\StringTok{"<spacer size=10>"}\NormalTok{);}
\OtherTok{document}\NormalTok{.}\FunctionTok{writeln}\NormalTok{(}\StringTok{"<input type='button' value='Save Model' onclick='WlModelSave()'>"}\NormalTok{);}
\OtherTok{document}\NormalTok{.}\FunctionTok{writeln}\NormalTok{(}\StringTok{"<spacer size=10>"}\NormalTok{);}
\OtherTok{document}\NormalTok{.}\FunctionTok{writeln}\NormalTok{(}\StringTok{"<input type='button' value='Save Model As' onclick='WlModelSaveAs()'>"}\NormalTok{);}
\OtherTok{document}\NormalTok{.}\FunctionTok{writeln}\NormalTok{(}\StringTok{"<p>"}\NormalTok{);}
\OtherTok{document}\NormalTok{.}\FunctionTok{writeln}\NormalTok{(}\StringTok{"<input type='button' value='REPAINT  SCREEN' onclick='WlWindowRepaint()'>"}\NormalTok{);}
\OtherTok{document}\NormalTok{.}\FunctionTok{writeln}\NormalTok{(}\StringTok{"</center>"}\NormalTok{);}
\OtherTok{document}\NormalTok{.}\FunctionTok{writeln}\NormalTok{(}\StringTok{"<hr>"}\NormalTok{);}

\OtherTok{document}\NormalTok{.}\FunctionTok{writeln}\NormalTok{(}\StringTok{"</form>"}\NormalTok{);}
\end{Highlighting}
\end{Shaded}

\section{Collaborative Engine for Distributed Mechanical
Design}\label{collaborative-engine-for-distributed-mechanical-design}

分散式機械設計協同引擎

Qianfu Ni and Wen Feng Lu

\section{Abstract}\label{abstract}

摘要

Effective collaboration is essential for engineers at geographically
dispersed locations to accomplish good design with less iteration.

分散在世界各地的工程師若希望以較少的重複流程來完成好的設計,
需要仰賴有效率的協同.

Over the last several years, more and more efforts have been put into
such research as many industries have distributed their product
development to locations with knowledge force.

過去幾年,
有越來越多的業者將心力投注在利用各地的知識人力來進行分散式產品開發.

This paper presents a collaborative engine to facilitate collaborations
among distributed mechanical designs.

本論文提出一種協同引擎(設計), 可用來協助分散式機械設計有效進行協同.

Using component-based software technology, collaboration functionality
is developed into a set of groupware that makes the collaborative engine
applicable to develop new collaborative applications or integrate legacy
applications into collaborative environments.

(本研究)運用元件導向的軟體技術, 將協同功能置入組群組軟體中,
讓此一協同系統可以用來開發新的協同應用或將傳統應用程式與協同環境加以整合.

An XML-based information representation is developed to streamline the
information transmission within the distributed environment.

(本研究)採用以 XML 為基礎的資訊表示法,
以便讓分散式環境中的資訊傳遞得以流暢進行.

A case study is carried out to show how this engine facilitates
designers to collaboratively create a 3D solid model of a same part in
real time.

(本研究)利用案例研究來說明此一(協同)引擎如何有效協助設計者以即時協同模式建立
3D 實體模型.

Keywords: collaboration, distributed design, collaborative engine

關鍵字: 協同, 分散式設計, 協同引擎

人偶零件:
http://2014cda-mdenfu.rhcloud.com/cmsimply/download/?filepath=/var/lib/openshift/5356315e4382ec2b3f000557/app-root/data//downloads/lego\_man.7z

\begin{Shaded}
\begin{Highlighting}[]
\OtherTok{@language} \NormalTok{python}

\CharTok{import} \NormalTok{cherrypy}

\CommentTok{# 這是 MAN 類別的定義}
\CommentTok{'''}
\CommentTok{# 在 application 中導入子模組}
\CommentTok{import programs.cdag30.man as cdag30_man}
\CommentTok{# 加入 cdag30 模組下的 man.py 且以子模組 man 對應其 MAN() 類別}
\CommentTok{root.cdag30.man = cdag30_man.MAN()}

\CommentTok{# 完成設定後, 可以利用}
\CommentTok{/cdag30/man/assembly}
\CommentTok{# 呼叫 man.py 中 MAN 類別的 assembly 方法}
\CommentTok{'''}
\KeywordTok{class} \NormalTok{MAN(}\DataTypeTok{object}\NormalTok{):}
    \CommentTok{# 各組利用 index 引導隨後的程式執行}
    \OtherTok{@cherrypy.expose}
    \KeywordTok{def} \NormalTok{index(}\OtherTok{self}\NormalTok{, *args, **kwargs):}
        \NormalTok{outstring = }\StringTok{'''}
\StringTok{這是 2014CDA 協同專案下的 cdag30 模組下的 MAN 類別.<br /><br />}
\StringTok{<!-- 這裡採用相對連結, 而非網址的絕對連結 (這一段為 html 註解) -->}
\StringTok{<a href="assembly">執行  MAN 類別中的 assembly 方法</a><br /><br />}
\StringTok{請確定下列零件於 V:/home/lego/man 目錄中, 且開啟空白 Creo 組立檔案.<br />}
\StringTok{<a href="/static/lego_man.7z">lego_man.7z</a>(滑鼠右鍵存成 .7z 檔案)<br />}
\StringTok{'''}
        \KeywordTok{return} \NormalTok{outstring}

    \OtherTok{@cherrypy.expose}
    \KeywordTok{def} \NormalTok{assembly(}\OtherTok{self}\NormalTok{, *args, **kwargs):}
        \NormalTok{outstring = }\StringTok{'''}
\StringTok{<!DOCTYPE html> }
\StringTok{<html>}
\StringTok{<head>}
\StringTok{<meta http-equiv="content-type" content="text/html;charset=utf-8">}
\StringTok{<script type="text/javascript" src="/static/weblink/examples/jscript/pfcUtils.js"></script>}
\StringTok{</head>}
\StringTok{<body>}
\StringTok{</script><script language="JavaScript">}
\StringTok{/*設計一個零件組立函示*/}
\StringTok{// featID 為組立件第一個組立零件的編號}
\StringTok{// inc 則為 part1 的組立順序編號, 第一個入組立檔編號為 featID+0}
\StringTok{// part2 為外加的零件名稱}
\StringTok{function axis_plane_assembly(session, assembly, transf, featID, inc, part2, axis1, plane1, axis2, plane2)\{}
\StringTok{var descr = pfcCreate("pfcModelDescriptor").CreateFromFileName ("v:/home/lego/man/"+part2);}
\StringTok{var componentModel = session.GetModelFromDescr(descr);}
\StringTok{var componentModel = session.RetrieveModel(descr);}
\StringTok{if (componentModel != void null)}
\StringTok{\{}
\StringTok{    var asmcomp = assembly.AssembleComponent (componentModel, transf);}
\StringTok{\}}
\StringTok{var ids = pfcCreate("intseq");}
\StringTok{ids.Append(featID+inc);}
\StringTok{var subPath = pfcCreate("MpfcAssembly").CreateComponentPath(assembly, ids);}
\StringTok{subassembly = subPath.Leaf;}
\StringTok{var asmDatums = new Array(axis1, plane1);}
\StringTok{var compDatums = new Array(axis2, plane2);}
\StringTok{var relation = new Array (pfcCreate("pfcComponentConstraintType").ASM_CONSTRAINT_ALIGN, pfcCreate("pfcComponentConstraintType").ASM_CONSTRAINT_MATE);}
\StringTok{var relationItem = new Array(pfcCreate("pfcModelItemType").ITEM_AXIS, pfcCreate("pfcModelItemType").ITEM_SURFACE);}
\StringTok{var constrs = pfcCreate("pfcComponentConstraints");}
\StringTok{    for (var i = 0; i < 2; i++)}
\StringTok{    \{}
\StringTok{        var asmItem = subassembly.GetItemByName (relationItem[i], asmDatums [i]);}
\StringTok{        if (asmItem == void null)}
\StringTok{        \{}
\StringTok{            interactFlag = true;}
\StringTok{            continue;}
\StringTok{        \}}
\StringTok{        var compItem = componentModel.GetItemByName (relationItem[i], compDatums [i]);}
\StringTok{        if (compItem == void null)}
\StringTok{        \{}
\StringTok{            interactFlag = true;}
\StringTok{            continue;}
\StringTok{        \}}
\StringTok{        var MpfcSelect = pfcCreate ("MpfcSelect");}
\StringTok{        var asmSel = MpfcSelect.CreateModelItemSelection (asmItem, subPath);}
\StringTok{        var compSel = MpfcSelect.CreateModelItemSelection (compItem, void null);}
\StringTok{        var constr = pfcCreate("pfcComponentConstraint").Create (relation[i]);}
\StringTok{        constr.AssemblyReference  = asmSel;}
\StringTok{        constr.ComponentReference = compSel;}
\StringTok{        constr.Attributes = pfcCreate("pfcConstraintAttributes").Create (true, false);}
\StringTok{        constrs.Append(constr);}
\StringTok{    \}}
\StringTok{asmcomp.SetConstraints(constrs, void null);}
\StringTok{\}}
\StringTok{// 以上為 axis_plane_assembly() 函式}
\StringTok{//}
\StringTok{function three_plane_assembly(session, assembly, transf, featID, inc, part2, plane1, plane2, plane3, plane4, plane5, plane6)\{}
\StringTok{var descr = pfcCreate("pfcModelDescriptor").CreateFromFileName ("v:/home/lego/man/"+part2);}
\StringTok{var componentModel = session.GetModelFromDescr(descr);}
\StringTok{var componentModel = session.RetrieveModel(descr);}
\StringTok{if (componentModel != void null)}
\StringTok{\{}
\StringTok{    var asmcomp = assembly.AssembleComponent (componentModel, transf);}
\StringTok{\}}
\StringTok{var ids = pfcCreate("intseq");}
\StringTok{ids.Append(featID+inc);}
\StringTok{var subPath = pfcCreate("MpfcAssembly").CreateComponentPath(assembly, ids);}
\StringTok{subassembly = subPath.Leaf;}
\StringTok{var constrs = pfcCreate("pfcComponentConstraints");}
\StringTok{var asmDatums = new Array(plane1, plane2, plane3);}
\StringTok{var compDatums = new Array(plane4, plane5, plane6);}
\StringTok{var MpfcSelect = pfcCreate("MpfcSelect");}
\StringTok{for (var i = 0; i < 3; i++)}
\StringTok{\{}
\StringTok{    var asmItem = subassembly.GetItemByName(pfcCreate("pfcModelItemType").ITEM_SURFACE, asmDatums[i]);}
\StringTok{    }
\StringTok{    if (asmItem == void null)}
\StringTok{    \{}
\StringTok{        interactFlag = true;}
\StringTok{        continue;}
\StringTok{    \}}
\StringTok{    var compItem = componentModel.GetItemByName(pfcCreate("pfcModelItemType").ITEM_SURFACE, compDatums[i]);}
\StringTok{    if (compItem == void null)}
\StringTok{    \{}
\StringTok{        interactFlag = true;}
\StringTok{        continue;}
\StringTok{    \}}
\StringTok{    var asmSel = MpfcSelect.CreateModelItemSelection(asmItem, subPath);}
\StringTok{    var compSel = MpfcSelect.CreateModelItemSelection(compItem, void null);}
\StringTok{    var constr = pfcCreate("pfcComponentConstraint").Create(pfcCreate("pfcComponentConstraintType").ASM_CONSTRAINT_MATE);}
\StringTok{    constr.AssemblyReference = asmSel;}
\StringTok{    constr.ComponentReference = compSel;}
\StringTok{    constr.Attributes = pfcCreate("pfcConstraintAttributes").Create (false, false);}
\StringTok{    constrs.Append(constr);}
\StringTok{\}}
\StringTok{asmcomp.SetConstraints(constrs, void null);}
\StringTok{\}}
\StringTok{// 以上為 three_plane_assembly() 函式}
\StringTok{//}
\StringTok{// 假如 Creo 所在的操作系統不是 Windows 環境}
\StringTok{if (!pfcIsWindows())}
\StringTok{// 則啟動對應的 UniversalXPConnect 執行權限 (等同 Windows 下的 ActiveX)}
\StringTok{netscape.security.PrivilegeManager.enablePrivilege("UniversalXPConnect");}
\StringTok{// pfcGetProESession() 是位於 pfcUtils.js 中的函式, 確定此 JavaScript 是在嵌入式瀏覽器中執行}
\StringTok{var session = pfcGetProESession();}
\StringTok{// 設定 config option, 不要使用元件組立流程中內建的假設約束條件}
\StringTok{session.SetConfigOption("comp_placement_assumptions","no");}
\StringTok{// 建立擺放零件的位置矩陣, Pro/Web.Link 中的變數無法直接建立, 必須透過 pfcCreate() 建立}
\StringTok{var identityMatrix = pfcCreate("pfcMatrix3D");}
\StringTok{// 建立 identity 位置矩陣}
\StringTok{for (var x = 0; x < 4; x++)}
\StringTok{for (var y = 0; y < 4; y++)}
\StringTok{\{}
\StringTok{    if (x == y)}
\StringTok{        identityMatrix.Set(x, y, 1.0);}
\StringTok{    else}
\StringTok{        identityMatrix.Set(x, y, 0.0);}
\StringTok{\}}
\StringTok{// 利用 identityMatrix 建立 transf 座標轉換矩陣}
\StringTok{var transf = pfcCreate("pfcTransform3D").Create(identityMatrix);}
\StringTok{// 取得目前的工作目錄}
\StringTok{var currentDir = session.getCurrentDirectory();}
\StringTok{// 以目前已開檔的空白組立檔案, 作為 model}
\StringTok{var model = session.CurrentModel;}
\StringTok{// 查驗有無 model, 或 model 類別是否為組立件, 若不符合條件則丟出錯誤訊息}
\StringTok{if (model == void null || model.Type != pfcCreate("pfcModelType").MDL_ASSEMBLY)}
\StringTok{throw new Error (0, "Current model is not an assembly.");}
\StringTok{// 將此模型設為組立物件}
\StringTok{var assembly = model;}

\StringTok{/**---------------------- LEGO_BODY--------------------**/}
\StringTok{// 設定零件的 descriptor 物件變數}
\StringTok{var descr = pfcCreate("pfcModelDescriptor").CreateFromFileName("v:/home/lego/man/LEGO_BODY.prt");}
\StringTok{// 若零件在 session 則直接取用}
\StringTok{var componentModel = session.GetModelFromDescr(descr);}
\StringTok{// 若零件不在 session 則從工作目錄中載入 session}
\StringTok{var componentModel = session.RetrieveModel(descr);}
\StringTok{// 若零件已經在 session 中則放入組立檔中}
\StringTok{if (componentModel != void null)}
\StringTok{\{}
\StringTok{    // 注意這個 asmcomp 即為設定約束條件的本體}
\StringTok{    // asmcomp 為特徵物件, 直接將零件, 以 transf 座標轉換矩陣方位放入組立檔案中}
\StringTok{    var asmcomp = assembly.AssembleComponent(componentModel, transf);}
\StringTok{\}}

\StringTok{// 建立約束條件變數}
\StringTok{var constrs = pfcCreate("pfcComponentConstraints");}
\StringTok{// 設定組立檔中的三個定位面, 注意內定名稱與 Pro/E WF 中的 ASM_D_FRONT 不同, 而是 ASM_FRONT, 可在組立件->info->model 中查詢定位面名稱}
\StringTok{// 組立檔案中的 Datum 名稱也可以利用 View->plane tag display 查詢名稱}
\StringTok{// 建立組立參考面所組成的陣列}
\StringTok{var asmDatums = new Array("ASM_FRONT", "ASM_TOP", "ASM_RIGHT");}
\StringTok{// 設定零件檔中的三個定位面, 名稱與 Pro/E WF 中相同}
\StringTok{var compDatums = new Array("FRONT", "TOP", "RIGHT");}
\StringTok{// 建立 ids 變數, intseq 為 sequence of integers 為資料類別, 使用者可以經由整數索引擷取此資料類別的元件, 第一個索引為 0}
\StringTok{       // intseq 等同 Python 的數列資料?}
\StringTok{var ids = pfcCreate("intseq");}
\StringTok{// 利用 assembly 物件模型, 建立路徑變數}
\StringTok{var path = pfcCreate("MpfcAssembly").CreateComponentPath(assembly, ids);}
\StringTok{// 採用互動式設定相關的變數, MpfcSelect 為 Module level class 中的一種}
\StringTok{var MpfcSelect = pfcCreate("MpfcSelect");}
\StringTok{// 利用迴圈分別約束組立與零件檔中的三個定位平面}
\StringTok{for (var i = 0; i < 3; i++)}
\StringTok{\{}
\StringTok{// 設定組立參考面, 也就是 "ASM_FRONT", "ASM_TOP", "ASM_RIGHT" 等三個 datum planes}
\StringTok{var asmItem = assembly.GetItemByName (pfcCreate("pfcModelItemType").ITEM_SURFACE, asmDatums[i]);}
\StringTok{// 若無對應的組立參考面, 則啟用互動式平面選擇表單 flag}
\StringTok{if (asmItem == void null)}
\StringTok{\{}
\StringTok{    interactFlag = true;}
\StringTok{    continue;}
\StringTok{\}}
\StringTok{// 設定零件參考面, 也就是 "FRONT", "TOP", "RIGHT" 等三個 datum planes}
\StringTok{var compItem = componentModel.GetItemByName (pfcCreate ("pfcModelItemType").ITEM_SURFACE, compDatums[i]);}
\StringTok{// 若無對應的零件參考面, 則啟用互動式平面選擇表單 flag}
\StringTok{if (compItem == void null)}
\StringTok{\{}
\StringTok{    interactFlag = true;}
\StringTok{    continue;}
\StringTok{\}}
\StringTok{        // 因為 asmItem 為組立件中的定位特徵, 必須透過 path 才能取得}
\StringTok{var asmSel = MpfcSelect.CreateModelItemSelection(asmItem, path);}
\StringTok{        // 而 compItem 則為零件, 沒有 path 路徑, 因此第二變數為 null}
\StringTok{var compSel = MpfcSelect.CreateModelItemSelection(compItem, void null);}
\StringTok{        // 利用 ASM_CONSTRAINT_ALIGN 對齊組立約束建立約束變數}
\StringTok{var constr = pfcCreate("pfcComponentConstraint").Create (pfcCreate ("pfcComponentConstraintType").ASM_CONSTRAINT_ALIGN);}
\StringTok{        // 設定約束條件的組立參考與元件參考選擇}
\StringTok{constr.AssemblyReference = asmSel;}
\StringTok{constr.ComponentReference = compSel;}
\StringTok{       // 第一個變數為強制變數, 第二個為忽略變數}
\StringTok{       // 強制變數為 false, 表示不強制約束, 只有透過點與線對齊時需設為 true}
\StringTok{       // 忽略變數為 false, 約束條件在更新模型時是否忽略, 設為 false 表示不忽略}
\StringTok{       // 通常在組立 closed chain 機構時,  忽略變數必須設為 true, 才能完成約束}
\StringTok{       // 因為三個面絕對約束, 因此輸入變數為 false, false}
\StringTok{constr.Attributes = pfcCreate("pfcConstraintAttributes").Create (false, false);}
\StringTok{// 將互動選擇相關資料, 附加在程式約束變數之後}
\StringTok{constrs.Append(constr);}
\StringTok{\}}

\StringTok{// 設定組立約束條件}
\StringTok{asmcomp.SetConstraints (constrs, void null);}
\StringTok{/**---------------------- LEGO_ARM_RT 右手上臂--------------------**/}
\StringTok{var descr = pfcCreate ("pfcModelDescriptor").CreateFromFileName ("v:/home/lego/man/LEGO_ARM_RT.prt");}
\StringTok{var componentModel = session.GetModelFromDescr(descr);}
\StringTok{var componentModel = session.RetrieveModel(descr);}
\StringTok{if (componentModel != void null)}
\StringTok{\{}
\StringTok{        // 注意這個 asmcomp 即為設定約束條件的本體}
\StringTok{        // asmcomp 為特徵物件,直接將零件, 以 transf 座標轉換放入組立檔案中}
\StringTok{var asmcomp = assembly.AssembleComponent (componentModel, transf);}
\StringTok{\}}
\StringTok{// 取得 assembly 項下的元件 id, 因為只有一個零件, 採用 index 0 取出其 featID}
\StringTok{var components = assembly.ListFeaturesByType(true, pfcCreate ("pfcFeatureType").FEATTYPE_COMPONENT);}
\StringTok{// 此一 featID 為組立件中的第一個零件編號, 也就是樂高人偶的 body}
\StringTok{var featID = components.Item(0).Id;}

\StringTok{ids.Append(featID);}
\StringTok{// 在 assembly 模型中建立子零件所對應的路徑}
\StringTok{var subPath = pfcCreate("MpfcAssembly").CreateComponentPath(assembly, ids);}
\StringTok{subassembly = subPath.Leaf;}
\StringTok{// 以下針對 body 的 A_13 軸與 DTM1 基準面及右臂的  A_4 軸線與 DTM1 進行對齊與面接約束}
\StringTok{var asmDatums = new Array("A_13", "DTM1");}
\StringTok{var compDatums = new Array("A_4", "DTM1");}
\StringTok{// 組立的關係變數為對齊與面接}
\StringTok{var relation = new Array (pfcCreate ("pfcComponentConstraintType").ASM_CONSTRAINT_ALIGN, pfcCreate ("pfcComponentConstraintType").ASM_CONSTRAINT_MATE);}
\StringTok{// 組立元件則為軸與平面}
\StringTok{var relationItem = new Array(pfcCreate("pfcModelItemType").ITEM_AXIS, pfcCreate("pfcModelItemType").ITEM_SURFACE);}
\StringTok{// 建立約束條件變數, 軸採對齊而基準面則以面接進行約束}
\StringTok{var constrs = pfcCreate ("pfcComponentConstraints");}
\StringTok{for (var i = 0; i < 2; i++)}
\StringTok{\{}
\StringTok{                  // 設定組立參考面, asmItem 為 model item}
\StringTok{    var asmItem = subassembly.GetItemByName (relationItem[i], asmDatums [i]);}
\StringTok{                  // 若無對應的組立參考面, 則啟用互動式平面選擇表單 flag}
\StringTok{    if (asmItem == void null)}
\StringTok{    \{}
\StringTok{        interactFlag = true;}
\StringTok{        continue;}
\StringTok{    \}}
\StringTok{                  // 設定零件參考面, compItem 為 model item}
\StringTok{    var compItem = componentModel.GetItemByName (relationItem[i], compDatums[i]);}
\StringTok{    if (compItem == void null)}
\StringTok{    \{}
\StringTok{        interactFlag = true;}
\StringTok{        continue;}
\StringTok{    \}}
\StringTok{                  // 採用互動式設定相關的變數}
\StringTok{    var MpfcSelect = pfcCreate ("MpfcSelect");}
\StringTok{    var asmSel = MpfcSelect.CreateModelItemSelection (asmItem, subPath);}
\StringTok{    var compSel = MpfcSelect.CreateModelItemSelection (compItem, void null);}
\StringTok{    var constr = pfcCreate("pfcComponentConstraint").Create (relation[i]);}
\StringTok{    constr.AssemblyReference  = asmSel;}
\StringTok{    constr.ComponentReference = compSel;}
\StringTok{                  // 因為透過軸線對齊, 第一 force 變數需設為 true}
\StringTok{    constr.Attributes = pfcCreate("pfcConstraintAttributes").Create (true, false);}
\StringTok{                  // 將互動選擇相關資料, 附加在程式約束變數之後}
\StringTok{    constrs.Append(constr);}
\StringTok{\}}
\StringTok{// 設定組立約束條件, 以 asmcomp 特徵進行約束條件設定}
\StringTok{// 請注意, 第二個變數必須為 void null 表示零件對零件進行約束, 若為 subPath, 則零件會與原始零件的平面進行約束}
\StringTok{asmcomp.SetConstraints (constrs, void null);}
\StringTok{/**---------------------- LEGO_ARM_LT 左手上臂--------------------**/}
\StringTok{var descr = pfcCreate ("pfcModelDescriptor").CreateFromFileName ("v:/home/lego/man/LEGO_ARM_LT.prt");}
\StringTok{var componentModel = session.GetModelFromDescr(descr);}
\StringTok{var componentModel = session.RetrieveModel(descr);}
\StringTok{if (componentModel != void null)}
\StringTok{\{}
\StringTok{        // 注意這個 asmcomp 即為設定約束條件的本體}
\StringTok{        // asmcomp 為特徵物件,直接將零件, 以 transf 座標轉換放入組立檔案中}
\StringTok{var asmcomp = assembly.AssembleComponent(componentModel, transf);}
\StringTok{\}}
\StringTok{// 取得 assembly 項下的元件 id, 因為只有一個零件, 採用 index 0 取出其 featID}
\StringTok{var components = assembly.ListFeaturesByType(true, pfcCreate ("pfcFeatureType").FEATTYPE_COMPONENT);}
\StringTok{var ids = pfcCreate ("intseq");}
\StringTok{// 因為左臂也是與 body 進行約束條件組立,  因此取 body 的 featID}
\StringTok{// 至此右臂 id 應該是 featID+1, 而左臂則是 featID+2}
\StringTok{ids.Append(featID);}
\StringTok{// 在 assembly 模型中建立子零件所對應的路徑}
\StringTok{var subPath = pfcCreate("MpfcAssembly").CreateComponentPath(assembly, ids);}
\StringTok{subassembly = subPath.Leaf;}
\StringTok{var asmDatums = new Array("A_9", "DTM2");}
\StringTok{var compDatums = new Array("A_4", "DTM1");}
\StringTok{var relation = new Array (pfcCreate ("pfcComponentConstraintType").ASM_CONSTRAINT_ALIGN, pfcCreate ("pfcComponentConstraintType").ASM_CONSTRAINT_MATE);}
\StringTok{var relationItem = new Array(pfcCreate("pfcModelItemType").ITEM_AXIS, pfcCreate("pfcModelItemType").ITEM_SURFACE);}
\StringTok{// 建立約束條件變數}
\StringTok{var constrs = pfcCreate ("pfcComponentConstraints");}
\StringTok{for (var i = 0; i < 2; i++)}
\StringTok{\{}
\StringTok{                  // 設定組立參考面, asmItem 為 model item}
\StringTok{    var asmItem = subassembly.GetItemByName (relationItem[i], asmDatums [i]);}
\StringTok{                  // 若無對應的組立參考面, 則啟用互動式平面選擇表單 flag}
\StringTok{    if (asmItem == void null)}
\StringTok{    \{}
\StringTok{        interactFlag = true;}
\StringTok{        continue;}
\StringTok{    \}}
\StringTok{                  // 設定零件參考面, compItem 為 model item}
\StringTok{    var compItem = componentModel.GetItemByName (relationItem[i], compDatums [i]);}
\StringTok{    if (compItem == void null)}
\StringTok{    \{}
\StringTok{        interactFlag = true;}
\StringTok{        continue;}
\StringTok{    \}}
\StringTok{                  // 採用互動式設定相關的變數}
\StringTok{    var MpfcSelect = pfcCreate ("MpfcSelect");}
\StringTok{    var asmSel = MpfcSelect.CreateModelItemSelection (asmItem, subPath);}
\StringTok{    var compSel = MpfcSelect.CreateModelItemSelection (compItem, void null);}
\StringTok{    var constr = pfcCreate("pfcComponentConstraint").Create (relation[i]);}
\StringTok{    constr.AssemblyReference  = asmSel;}
\StringTok{    constr.ComponentReference = compSel;}
\StringTok{    constr.Attributes = pfcCreate("pfcConstraintAttributes").Create (true, false);}
\StringTok{                  // 將互動選擇相關資料, 附加在程式約束變數之後}
\StringTok{    constrs.Append(constr);}
\StringTok{\}}
\StringTok{// 設定組立約束條件, 以 asmcomp 特徵進行約束條件設定}
\StringTok{// 請注意, 第二個變數必須為 void null 表示零件對零件進行約束, 若為 subPath, 則零件會與原始零件的平面進行約束}
\StringTok{asmcomp.SetConstraints (constrs, void null);}
\StringTok{/**---------------------- LEGO_HAND 右手手腕--------------------**/}
\StringTok{// 右手臂 LEGO_ARM_RT.prt 基準  A_2, DTM2}
\StringTok{// 右手腕 LEGO_HAND.prt 基準 A_1, DTM3}
\StringTok{var descr = pfcCreate ("pfcModelDescriptor").CreateFromFileName ("v:/home/lego/man/LEGO_HAND.prt");}
\StringTok{var componentModel = session.GetModelFromDescr(descr);}
\StringTok{var componentModel = session.RetrieveModel(descr);}
\StringTok{if (componentModel != void null)}
\StringTok{\{}
\StringTok{        // 注意這個 asmcomp 即為設定約束條件的本體}
\StringTok{        // asmcomp 為特徵物件,直接將零件, 以 transf 座標轉換放入組立檔案中}
\StringTok{var asmcomp = assembly.AssembleComponent (componentModel, transf);}
\StringTok{\}}
\StringTok{// 取得 assembly 項下的元件 id, 因為只有一個零件, 採用 index 0 取出其 featID}
\StringTok{var components = assembly.ListFeaturesByType(true, pfcCreate ("pfcFeatureType").FEATTYPE_COMPONENT);}
\StringTok{var ids = pfcCreate ("intseq");}

\StringTok{// 組立件中 LEGO_BODY.prt 編號為 featID}
\StringTok{// LEGO_ARM_RT.prt 則是組立件第二個置入的零件,  編號為 featID+1}
\StringTok{ids.Append(featID+1);}
\StringTok{// 在 assembly 模型中, 根據子零件的編號, 建立子零件所對應的路徑}
\StringTok{var subPath = pfcCreate("MpfcAssembly").CreateComponentPath(assembly, ids);}
\StringTok{subassembly = subPath.Leaf;}
\StringTok{// 以下針對 LEGO_ARM_RT 的 A_2 軸與 DTM2 基準面及 HAND 的  A_1 軸線與 DTM3 進行對齊與面接約束}
\StringTok{var asmDatums = new Array("A_2", "DTM2");}
\StringTok{var compDatums = new Array("A_1", "DTM3");}
\StringTok{// 組立的關係變數為對齊與面接}
\StringTok{var relation = new Array (pfcCreate ("pfcComponentConstraintType").ASM_CONSTRAINT_ALIGN, pfcCreate ("pfcComponentConstraintType").ASM_CONSTRAINT_MATE);}
\StringTok{// 組立元件則為軸與平面}
\StringTok{var relationItem = new Array(pfcCreate("pfcModelItemType").ITEM_AXIS, pfcCreate("pfcModelItemType").ITEM_SURFACE);}
\StringTok{// 建立約束條件變數, 軸採對齊而基準面則以面接進行約束}
\StringTok{var constrs = pfcCreate ("pfcComponentConstraints");}
\StringTok{for (var i = 0; i < 2; i++)}
\StringTok{\{}
\StringTok{                  // 設定組立參考面, asmItem 為 model item}
\StringTok{    var asmItem = subassembly.GetItemByName (relationItem[i], asmDatums [i]);}
\StringTok{                  // 若無對應的組立參考面, 則啟用互動式平面選擇表單 flag}
\StringTok{    if (asmItem == void null)}
\StringTok{    \{}
\StringTok{        interactFlag = true;}
\StringTok{        continue;}
\StringTok{    \}}
\StringTok{                  // 設定零件參考面, compItem 為 model item}
\StringTok{    var compItem = componentModel.GetItemByName (relationItem[i], compDatums [i]);}
\StringTok{    if (compItem == void null)}
\StringTok{    \{}
\StringTok{        interactFlag = true;}
\StringTok{        continue;}
\StringTok{    \}}
\StringTok{                  // 採用互動式設定相關的變數}
\StringTok{    var MpfcSelect = pfcCreate("MpfcSelect");}
\StringTok{    var asmSel = MpfcSelect.CreateModelItemSelection(asmItem, subPath);}
\StringTok{    var compSel = MpfcSelect.CreateModelItemSelection (compItem, void null);}
\StringTok{    var constr = pfcCreate("pfcComponentConstraint").Create (relation[i]);}
\StringTok{    constr.AssemblyReference  = asmSel;}
\StringTok{    constr.ComponentReference = compSel;}
\StringTok{                  // 因為透過軸線對齊, 第一 force 變數需設為 true}
\StringTok{    constr.Attributes = pfcCreate("pfcConstraintAttributes").Create (true, false);}
\StringTok{                  // 將互動選擇相關資料, 附加在程式約束變數之後}
\StringTok{    constrs.Append(constr);}
\StringTok{\}}
\StringTok{// 設定組立約束條件, 以 asmcomp 特徵進行約束條件設定}
\StringTok{// 請注意, 第二個變數必須為 void null 表示零件對零件進行約束, 若為 subPath, 則零件會與原始零件的平面進行約束}
\StringTok{asmcomp.SetConstraints (constrs, void null);}
\StringTok{// 利用函式呼叫組立左手 HAND}
\StringTok{axis_plane_assembly(session, assembly, transf, featID, 2, }
\StringTok{                              "LEGO_HAND.prt", "A_2", "DTM2", "A_1", "DTM3");}
\StringTok{// 利用函式呼叫組立人偶頭部 HEAD}
\StringTok{// BODY id 為 featID+0, 以 A_2 及  DTM3 約束}
\StringTok{// HEAD 則直接呼叫檔案名稱, 以 A_2, DTM2 約束}
\StringTok{axis_plane_assembly(session, assembly, transf, featID, 0, }
\StringTok{                              "LEGO_HEAD.prt", "A_2", "DTM3", "A_2", "DTM2");}
\StringTok{// Body 與 WAIST 採三個平面約束組立}
\StringTok{// Body 組立面為 DTM4, DTM5, DTM6}
\StringTok{// WAIST 組立面為 DTM1, DTM2, DTM3}
\StringTok{three_plane_assembly(session, assembly, transf, featID, 0, "LEGO_WAIST.prt", "DTM4", "DTM5", "DTM6", "DTM1", "DTM2", "DTM3"); }
\StringTok{// 右腳}
\StringTok{axis_plane_assembly(session, assembly, transf, featID, 6, }
\StringTok{                              "LEGO_LEG_RT.prt", "A_8", "DTM4", "A_10", "DTM1");}
\StringTok{// 左腳}
\StringTok{axis_plane_assembly(session, assembly, transf, featID, 6, }
\StringTok{                              "LEGO_LEG_LT.prt", "A_8", "DTM5", "A_10", "DTM1");}
\StringTok{// 紅帽}
\StringTok{axis_plane_assembly(session, assembly, transf, featID, 5, }
\StringTok{                              "LEGO_HAT.prt", "A_2", "TOP", "A_2", "FRONT");     }
\StringTok{</script>}
\StringTok{</body>}
\StringTok{</html>}
\StringTok{'''}
        \KeywordTok{return} \NormalTok{outstring}
\end{Highlighting}
\end{Shaded}

\section{cd2ag10報告(2ag10)}\label{cd2ag10ux5831ux544a2ag10}

以下為各週報告

\section{第十組組員}\label{ux7b2cux5341ux7d44ux7d44ux54e1}

小組網站:http://goo.gl/aztMHi

40123116-吳羽閔

40123118-吳謦麟

40123153-戴志軒

\section{第八週考試摘要(2ag10)}\label{ux7b2cux516bux9031ux8003ux8a66ux6458ux89812ag10}

小組網站:http://goo.gl/yi3qJ5

2014S CD Week8

課程教材:

A collaborative writing approach to wikis

Collabrative engine for distributed mechanical design

Web-based collaborative engineering support system

第八週考試題目

下列題目完成後, 必須(1)將程式碼送到個人的 Bitbucket repository
下(2)程式可在近端與雲端部署執行(3)在 wiki.mde.tw 個人第八週心得中留下
Bitbucket 與雲端執行連結(4)整理出一份小考第一(二, 或三)題的 PDF 檔案,
寄到 course@mde.tw, 標題為: 學號-小考第一(二, 或三)題,
內容必須包含程式碼, 解題過程, 解題心得, Bitbucket 連結, 雲端網址等資料.

(第一題) 請寫一個執行時可以列出 9×9 乘法表的網際 Python 程式, 然後 Push
到個人 bitbucket 空間, 而且同步指到 OpenShift 個人帳號上執行.

(第二題) 請將上述執行過程錄為 flv 後, 上傳到個人的 Vemeo 空間中,
並將網址回報到各組網站 (dokuwiki 與 CMSimply)與報告中,
並且將相關心得與報告連結登錄到 wiki.mde.tw 第八週的分組頁面中.

(第三題) (協同計分, 分組進行) 請在各組的雲端 dowiki 中,根據下列 40
個帳號與密碼, 新增對應的使用者帳號與密碼後, 將雲端網址登錄在 wiki.mde.tw
各組第八週頁面中, 並說明操作過程與心得後, 將心得整理成 pdf 後繳交到
course@mde.tw.

40 個帳號與密碼

58B39 , 9J35UAVM

3624D , QANF34CW

7345B , 3PAFXKWZ

358DA , E6RJFKW4

3BC7B , HG2ASNYH

CA55C , XUZTHWQK

D2756 , UHK2W3D2

368B8 , A5QGYA6W

7948A , PF278WDQ

C65C4 , GT4KBCXU

A8964 , 3RPQSW2U

22422 , 7E57K7F3

9A5B4 , 4MVKRE5Z

B5A76 , 3DTAAHUF

57388 , NP39FGXR

8A833 , C7DNBHCQ

29AB8 , 6KMGK73Z

5ABD2 , PV5FH722

86293 , PJ69FBMS

9DCBC , U5HR6QR8

276DD , URE9FNWD

599AA , X2P6CTXF

9C449 , DKRN3V59

64236 , 86UWN3E9

43AAC , MNCJZCAX

73B93 , QX945VJJ

36283 , M3MQGUXD

7447C , TQZVDKPT

C73AB , MSP4GPPX

8284B , XMT8W9RD

62454 , SD4C7V89

44B3C , 636DBRJC

C75CB , M66RMMQ2

4A3CB , FF485EQ4

7D248 , FGJHQDAS

A7CC4 , R47AHA4Y

99BCC , RNQYZGQZ

6DA77 , HRCDP9D8

357AB , CKB4Q2EC

A9525 , JDVX75ST

第八週協同設計練習題目:

建立各組 OpenShift 上的 dokuwiki

http://ethercalc.tw/ Openshift 上的 ethercalc

近端的 ethercalc

用 markdown 編輯電子書: https://github.com/progit/progit ,

http://johnmacfarlane.net/pandoc/epub.html

期中考試與分組報告必選題

請以三組共 9 個人的情況下(座位為 3×3 配置),
分析2014s\_week3的協同設計題目, 如何配置座位可以得到團隊的最大配分.
假如將組別擴大為五組共 15 人的情況下(座位為 4×4 配置,
則團隊座位安排後的最大配合又是多少?

上課影片 http://vimeo.com/user24079973/videos

\section{第八週報告(2ag10)}\label{ux7b2cux516bux9031ux5831ux544a2ag10}

小組網站:http://goo.gl/YCNd7p

\begin{Shaded}
\begin{Highlighting}[]
\CharTok{import} \NormalTok{cherrypy}
\KeywordTok{class} \NormalTok{HelloWorld(}\DataTypeTok{object}\NormalTok{):}
    \OtherTok{@cherrypy.expose}
    \KeywordTok{def} \NormalTok{index(}\OtherTok{self}\NormalTok{, var1=}\DecValTok{9}\NormalTok{, var2=}\DecValTok{9}\NormalTok{):}
        \CommentTok{# initialize outstring}
        \NormalTok{outstring = }\StringTok{""}
        \CommentTok{# initialize count}
        \NormalTok{count = }\DecValTok{0}
        \NormalTok{d = }\DataTypeTok{int}\NormalTok{(var1)}
        \NormalTok{e = }\DataTypeTok{int}\NormalTok{(var2)+}\DecValTok{1}
        \KeywordTok{for} \NormalTok{i in }\DataTypeTok{range}\NormalTok{(}\DecValTok{1}\NormalTok{, d):}
            \KeywordTok{for} \NormalTok{j in }\DataTypeTok{range}\NormalTok{(}\DecValTok{1}\NormalTok{, e):}
                \NormalTok{count += }\DecValTok{1}
                \CommentTok{#print(count)}
                \KeywordTok{if} \NormalTok{count%(}\DataTypeTok{int}\NormalTok{(var2)) == }\DecValTok{0}\NormalTok{:}
                    \NormalTok{outstring += }\StringTok{"<td>"}\NormalTok{+}\DataTypeTok{str}\NormalTok{(i) + }\StringTok{"*"} \NormalTok{+ }\DataTypeTok{str}\NormalTok{(j) + }\StringTok{"="} \NormalTok{+ }\DataTypeTok{str}\NormalTok{(i*j) +}\StringTok{"</td>"}\NormalTok{+}\StringTok{"</tr>"} \NormalTok{+ }\StringTok{"<br />"}
                \KeywordTok{else}\NormalTok{:}
                    \NormalTok{outstring += }\StringTok{"<td>"}\NormalTok{+}\DataTypeTok{str}\NormalTok{(i) + }\StringTok{"x"} \NormalTok{+ }\DataTypeTok{str}\NormalTok{(j) + }\StringTok{"="} \NormalTok{+ }\DataTypeTok{str}\NormalTok{(i*j) + }\StringTok{"</td>"}
        \KeywordTok{return} \StringTok{"<table border=2><tr><td>99乘法表</td></tr><tr>"}\NormalTok{+outstring}
\CommentTok{#http://127.0.0.1:8080/index?var1=10&var2=20}
\CommentTok{#"&nbsp;"*4}
 
 
\NormalTok{cherrypy.quickstart(HelloWorld())}
\CommentTok{#application=cherrypy.Application(HelloWorld()) #將符號打開上傳openshift即可使用,並刪掉上列}
\end{Highlighting}
\end{Shaded}

以上程式碼即可跑出99乘法表.

9x9openshift: http://9x9-cadp13ag8.rhcloud.com/
(因openshift空間不足,故利用第九週程式,程式碼略有差別,主要為多新增兩個輸入表單,其餘相同)

同步bitbucket: https://bitbucket.org/40123153/input9x9 (同上)

------------我是分隔線------------

影片(由於網路超慢,怕錄製影片過大所以只錄製最後結果)

https://vimeo.com/92003966

P.S 由於題目理解錯誤,所以上述影片為失效。

\section{第九週考試摘要(2ag10)}\label{ux7b2cux4e5dux9031ux8003ux8a66ux6458ux89812ag10}

小組網站:http://goo.gl/8nPx2Y

2014S CD Week9

課程教材:

A collaborative writing approach to wikis

Collabrative engine for distributed mechanical design

Web-based collaborative engineering support system

期中考試題目

下列題目完成後, 必須(1)將程式碼送到個人的 Bitbucket repository
下(2)程式可在近端與雲端部署執行(3)在 wiki.mde.tw 個人第九週心得中留下
Bitbucket 與雲端執行連結以及其它參考連結(4)整理出一份期中考第一(二,
或三)題的 PDF 檔案, 寄到 course@mde.tw, 標題為:
cda\_學號\_姓名\_期中考第一(二, 或三)題(乙班將 cda 改為 cdb),
內容必須包含程式碼, 解題過程, 解題心得, Bitbucket 連結, 雲端網址等資料.

(第一題 30\%) 請寫一個執行時可以列出以十為底對數表的網際 Python 程式,
然後 Push 到個人 bitbucket 空間, 而且同步指到 OpenShift 個人帳號上執行.

(第二題 40\%) 請在個人的 OpenShift
平台上建立一個能夠列印出與九九乘法表結果完全相同的網際程式,
接著在乘法表上端加上兩個輸入表單, 讓使用者輸入兩個整數, 按下送出鍵後,
程式會列出以此兩個整數為基底的乘法表, 例如: 若兩個欄位都輸入: 9,
則列出九九乘法表, 若輸入 9, 20, 則列出 9×20 的乘法表.

(第三題 30\%) 請在各組的雲端 dokuwiki 中, 新增帳號與密碼都是由 abc001
\textasciitilde{} abc399 字串所組成的 399 名用戶登入對應資料,
並將製作過程與驗證流程拍成 flv 後上傳到個人的 Vimeo 資料區,
並將連結放在個人第九週頁面.

第一題參考資料: 對數表與應用, 當 x=10, 對應到 0 行的值, 表示要對 1.00
取以十為底的對數, 所得到的值為 math.log(1.00, 10)=0, 而 x=10 對應到 1
行的值, 表示要對 1.01 取以十為底的對數, 所得到的值為 math.log(1.01,
10)=0.004321373782642578 然後再乘上 10000, 並且只取整數,
所以對應表的值為 43, 也就是表中的 0043, 當 x=20, 而且對應到第 9 行的值,
則為 math.log(2.09, 10)=0.32014628611105395, 然後再乘上 10000, 只取整數,
所以對應表上的值為 3201.

第一題參考: Python 中 str() 可以將整數或浮點數轉為字串, int()
則可以將字串轉為整數, round(1.0123, 2) 表示只取小數點後兩位, math.log(x,
10) 表示對 x 取以 10 為底的 log 值, html 表格請參考.

期中成績評量

請各組依據2014s\_week7中的說明完成期中報告.(最後繳交期限為 2014.4.26
晚上 12:00)

成績評量時, 將依據各組在 wiki.mde.tw 中的頁面進行評量,
期中成績包括平時成績(參考個人自評成績)、第八週考試與期中考試成績、期中報告成績等.

上課影片 http://vimeo.com/user24079973/videos

\section{第九週報告(2ag10)}\label{ux7b2cux4e5dux9031ux5831ux544a2ag10}

小組網站:http://goo.gl/Hy7Ktn

第一題:

Log以10為底之表單openshift: http://log-cadp13ag8.rhcloud.com/

bitbucket: https://bitbucket.org/40123153/log10-1.00-1.99

因為迴圈關係,多了一個21\ldots{}還在更新中。(解決)

P.S已利用elif判斷式將最後的21給剃除!

當在最後20的地方下elif將

給剃除,故不在往下做表格及列印出21數字。

\begin{Shaded}
\begin{Highlighting}[]
\CharTok{import} \NormalTok{cherrypy}
\CharTok{import} \NormalTok{os}
\CharTok{import} \NormalTok{math}
\CommentTok{# 1. 導入所需模組}

\CommentTok{# 2. 設定近端與遠端目錄}
\CommentTok{# 確定程式檔案所在目錄, 在 Windows 有最後的反斜線}
\NormalTok{_curdir = os.path.join(os.getcwd(), os.path.dirname(}\OtherTok{__file__}\NormalTok{))}
\CommentTok{# 設定在雲端與近端的資料儲存目錄}
\KeywordTok{if} \StringTok{'OPENSHIFT_REPO_DIR'} \NormalTok{in os.environ.keys():}
    \CommentTok{# 表示程式在雲端執行}
    \NormalTok{download_root_dir = os.environ[}\StringTok{'OPENSHIFT_DATA_DIR'}\NormalTok{]}
    \NormalTok{data_dir = os.environ[}\StringTok{'OPENSHIFT_DATA_DIR'}\NormalTok{]}
\KeywordTok{else}\NormalTok{:}
    \CommentTok{# 表示程式在近端執行}
    \NormalTok{download_root_dir = _curdir + }\StringTok{"/local_data/"}
    \NormalTok{data_dir = _curdir + }\StringTok{"/local_data/"}
 
\CommentTok{# 3. 建立主物件}
\KeywordTok{class} \NormalTok{HelloWorld(}\DataTypeTok{object}\NormalTok{):}
    \OtherTok{@cherrypy.expose}
    \KeywordTok{def} \NormalTok{index2(}\OtherTok{self}\NormalTok{, input1=}\OtherTok{None}\NormalTok{, input2=}\OtherTok{None}\NormalTok{):}
        \KeywordTok{return} \StringTok{"Hello world!"}\NormalTok{+}\DataTypeTok{str}\NormalTok{(input1)}
    \OtherTok{@cherrypy.expose}
    \KeywordTok{def} \NormalTok{inputform(}\OtherTok{self}\NormalTok{, input1=}\OtherTok{None}\NormalTok{, input2=}\OtherTok{None}\NormalTok{):}
        \KeywordTok{return} \StringTok{"input form"}\NormalTok{+}\DataTypeTok{str}\NormalTok{(input1)}
    \CommentTok{#index.exposed = True}
    \OtherTok{@cherrypy.expose}
    \KeywordTok{def} \NormalTok{index(}\OtherTok{self}\NormalTok{):}
        \NormalTok{out=}\StringTok{""}
        \NormalTok{c = }\DecValTok{0}
        \NormalTok{k = }\DecValTok{0}
        \KeywordTok{for} \NormalTok{i in }\DataTypeTok{range}\NormalTok{(}\DecValTok{0}\NormalTok{,}\DecValTok{11}\NormalTok{):}
            \KeywordTok{for} \NormalTok{j in }\DataTypeTok{range} \NormalTok{(}\DecValTok{0}\NormalTok{,}\DecValTok{10}\NormalTok{):}
                \NormalTok{k += }\DecValTok{1}
                \NormalTok{c = }\DecValTok{10} \NormalTok{+ i}
                \NormalTok{d = c/}\DecValTok{10}\NormalTok{+j/}\DecValTok{100}
                \NormalTok{e = (}\DataTypeTok{int}\NormalTok{(math.log(d,}\DecValTok{10}\NormalTok{)*}\DecValTok{10000}\NormalTok{))}
                \KeywordTok{if} \NormalTok{(k%}\DecValTok{10} \NormalTok{== }\DecValTok{0} \NormalTok{and k<=}\DecValTok{100}\NormalTok{):}
                    \NormalTok{out += }\StringTok{"<td>"}\NormalTok{+}\DataTypeTok{str}\NormalTok{(e)+}\StringTok{"</td></tr><td>"}\NormalTok{+}\DataTypeTok{str}\NormalTok{(c}\DecValTok{+1}\NormalTok{)+}\StringTok{"</td><br />"}
                \KeywordTok{elif} \NormalTok{k%}\DecValTok{100} \NormalTok{== }\DecValTok{0}\NormalTok{:}
                    \NormalTok{out += }\StringTok{"<td>"}\NormalTok{+}\DataTypeTok{str}\NormalTok{(e)+}\StringTok{"</td><br />"}
                \KeywordTok{else}\NormalTok{:}
                    \NormalTok{out +=}\StringTok{"<td>"}\NormalTok{+}\DataTypeTok{str}\NormalTok{(e) + }\StringTok{"</td>"}
                \CommentTok{#print("log=",int(math.log(d,10)*10000))}
            \CommentTok{#print("\textbackslash{}n")}
        \KeywordTok{return} \StringTok{"<table border=1><tr><td>x</td><td>0</td><td>1</td><td>2</td><td>3</td><td>4</td><td>5</td><td>6</td><td>7</td><td>8</td><td>9</td></tr><td>"}\NormalTok{+}\StringTok{"10"}\NormalTok{+}\StringTok{"</td>"}\NormalTok{+out}
 
\CommentTok{# 4. 安排啟動設定}
\CommentTok{# 配合程式檔案所在目錄設定靜態目錄或靜態檔案}
\NormalTok{application_conf = \{}\StringTok{'/static'}\NormalTok{:\{}
        \StringTok{'tools.staticdir.on'}\NormalTok{: }\OtherTok{True}\NormalTok{,}
        \CommentTok{'tools.staticdir.dir'}\NormalTok{: _curdir+}\StringTok{"/static"}\NormalTok{\},}
        \CommentTok{'/downloads'}\NormalTok{:\{}
        \StringTok{'tools.staticdir.on'}\NormalTok{: }\OtherTok{True}\NormalTok{,}
        \CommentTok{'tools.staticdir.dir'}\NormalTok{: data_dir+}\StringTok{"/downloads"}\NormalTok{\}}
    \NormalTok{\}}
 
\CommentTok{# 5. 在近端或遠端啟動程式}
\CommentTok{# 利用 HelloWorld() class 產生案例物件}
\NormalTok{root = HelloWorld()}
\CommentTok{# 假如在 os 環境變數中存在 'OPENSHIFT_REPO_DIR', 表示程式在 OpenShift 環境中執行}
\KeywordTok{if} \StringTok{'OPENSHIFT_REPO_DIR'} \NormalTok{in os.environ.keys():}
    \CommentTok{# 雲端執行啟動}
    \NormalTok{application = cherrypy.Application(root, config = application_conf)}
\KeywordTok{else}\NormalTok{:}
    \CommentTok{# 近端執行啟動}
    \CommentTok{'''}
\CommentTok{    cherrypy.server.socket_port = 8083}
\CommentTok{    cherrypy.server.socket_host = '127.0.0.1'}
\CommentTok{    '''}
    \NormalTok{cherrypy.quickstart(root, config = application_conf)}
\end{Highlighting}
\end{Shaded}

------------我是分隔線------------

第二題:

9×9乘法表,並且有兩input可以改變9×9乘法表的範圍。例:input:9 and 20
則為9×20乘法表。

openshift: http://9x9-cadp13ag8.rhcloud.com/ (備註:i為前值,j為後值)

bitbucket: https://bitbucket.org/40123153/input9x9

\begin{Shaded}
\begin{Highlighting}[]
\CharTok{import} \NormalTok{cherrypy}
\KeywordTok{class} \NormalTok{HelloWorld(}\DataTypeTok{object}\NormalTok{):}
    \OtherTok{@cherrypy.expose}
    \KeywordTok{def} \NormalTok{index(}\OtherTok{self}\NormalTok{, var1=}\DecValTok{9}\NormalTok{, var2=}\DecValTok{9}\NormalTok{):}
        \CommentTok{# initialize outstring}
        \NormalTok{outstring = }\StringTok{""}
        \CommentTok{# initialize count}
        \NormalTok{count = }\DecValTok{0}
        \NormalTok{d = }\DataTypeTok{int}\NormalTok{(var1)+}\DecValTok{1}
        \NormalTok{e = }\DataTypeTok{int}\NormalTok{(var2)+}\DecValTok{1}
        \KeywordTok{for} \NormalTok{i in }\DataTypeTok{range}\NormalTok{(}\DecValTok{1}\NormalTok{, d):}
            \KeywordTok{for} \NormalTok{j in }\DataTypeTok{range}\NormalTok{(}\DecValTok{1}\NormalTok{, e):}
                \NormalTok{count += }\DecValTok{1}
                \KeywordTok{if} \NormalTok{count%}\DataTypeTok{int}\NormalTok{(var2) == }\DecValTok{0}\NormalTok{:}
                    \NormalTok{outstring += }\StringTok{"<td>"}\NormalTok{+}\DataTypeTok{str}\NormalTok{(i) + }\StringTok{"*"} \NormalTok{+ }\DataTypeTok{str}\NormalTok{(j) + }\StringTok{"="} \NormalTok{+ }\DataTypeTok{str}\NormalTok{(i*j) +}\StringTok{"</td>"}\NormalTok{+}\StringTok{"</tr>"} \NormalTok{+ }\StringTok{"<br />"}
                \KeywordTok{else}\NormalTok{:}
                    \NormalTok{outstring += }\StringTok{"<td>"}\NormalTok{+}\DataTypeTok{str}\NormalTok{(i) + }\StringTok{"x"} \NormalTok{+ }\DataTypeTok{str}\NormalTok{(j) + }\StringTok{"="} \NormalTok{+ }\DataTypeTok{str}\NormalTok{(i*j) + }\StringTok{"</td>"}
        \KeywordTok{return} \StringTok{"<table border=3><tr><td>99乘法表</td></tr><tr>"}\NormalTok{+outstring+}\StringTok{'''<br/><form method="POST" action="index">}
\StringTok{                i 我是前值:<input type="text" name="var1"><br />}
\StringTok{                j 我是後值:<input type="text" name="var2"><br />}
\StringTok{                <input type="submit" value="send">}
\StringTok{                </form>}
\StringTok{                '''}
\CommentTok{#http://127.0.0.1:8080/index?var1=10&var2=20}
 
 
\CommentTok{#cherrypy.quickstart(HelloWorld())}
\NormalTok{application=cherrypy.Application(HelloWorld())}
\end{Highlighting}
\end{Shaded}

------------我是分隔線------------

第三題:

新增帳號與密碼都是由 abc001 \textasciitilde{} abc399 字串所組成的 399。

影片解說: https://vimeo.com/92574166

首先,我們必須要先有abc001\textasciitilde{}abc399的帳號密碼值,故先用程式迴圈跑出。
(因為懶得寫輸出檔,或者是不熟悉所以不冒險去寫。)

將值複製至文件內將空白修掉,利用EXCEL把密碼在複製至B欄位。存成CSV檔

同樣利用文件將逗點修掉(如果沒記錯空白處為一個TAB,為了安全還是複製之前的。)

再利用程式,修改讀取的文件,將密碼處改為數列\href{數列第一資料行位為{[}0{]},第二資料故為{[}1{]}}{1}

跑出user之後利用FZ將檔案覆蓋,則可以在雲端上登錄。(為了確認完全傳入,將abc001改為管理權限,確認帳號輸入)

dokuwiki\_openshift: http://dokuwiki-cadp13ag8.rhcloud.com/doku.php

bitbucket: https://bitbucket.org/40123153/dokuwiki\_abc

第三題的程式碼實在不知道要打什麼\ldots{}.

\section{第十二週報告(2ag10)}\label{ux7b2cux5341ux4e8cux9031ux5831ux544a2ag10}

小組網站:http://goo.gl/28PW52

第十二週任務

請各組將第八週與第九週考試的摘要報告放入 Github 協同專案中的分組報告區,
並將內容放入各組控管的同步 OpenShift 網站中. (佔期末成績 5分)

github小組倉儲:https://github.com/2014cdag10

github大分組倉儲:https://github.com/coursemdetw/2014cda

請各組設法利用 CherryPy 與 Pro/Web.Link 技術, 在 Github
協同專案中建立一個能夠透過連結或表單控制 Cube 零件, a, b, 或 c
零件尺寸的網際協同程式, 讓使用者可以藉以利用近端的 Creo 嵌入式瀏覽器控制
Cube 的尺寸後列出該零件的體積大小. (佔期末成績 5分)

\begin{enumerate}
\def\labelenumi{\arabic{enumi}.}
\itemsep1pt\parskip0pt\parsep0pt
\item
  Creo 必須使用 64 位元版本 (配合 Windows 7 操作系統)
\item
  Creo web\_enable\_javascript 設為 on, regen\_failure\_handling 設為
  resolve\_mode
\item
  IE→工具→網際網路選項→安全性→信任的網站→自訂等級 允許信任網站執行
  ActiveX
\item
  cda 專案必須將http://cdag10-40123153.rhcloud.com/設為信任網站
\item
  開啟 Creo 2.0, 建立一個 cube, relations: d0 = a, d1 = b, d2 = c (a, b,
  c 為 local parameters)
\item
  然後在嵌入式 IE, 連接到
  http://cdag10-40123153.rhcloud.com/cdag10/cube10
  (\#在parameter中將函式名稱改為cube10)
\item
  執行 http://cdag10-40123153.rhcloud.com/cdag10/fourbar10
  (\#在parameter中將函式名稱改為fourbar10)
  之前則需要先下載四連桿零件放在 V:\home\fourbar 目錄中,
  並開啟一個空的組立檔案, 執行時 Pro/Web.Link 程式會自動進行連桿組立.
\item
  Pro/Web.Link 應用:
  http://www.ptc.com/company/news/inprint/taiwan/proe5.htm
\item
  參考資料: http://inversionconsulting.blogspot.tw/search/label/WebLink
\end{enumerate}

首先必須將creo 2.0 的
config.pro檔做些設定,必須要將regen\_failure\_handing設為resolv\_mode,使他能重繪。

再來必須將web\_enable\_javascript設為on,這樣才能讓javascript可以動作。

#很重要!!這將會成為成功與失敗的原因\ldots{}.

並且將IE關於ActiveX都通通啟用,將http://cdag10-40123153.rhcloud.com/設為例外網頁。

再來就是繪圖,將零件的三個尺寸設關係式,d0=a , d1=b , d2=c。

所有前置動作都完成了,再來就是利用creo裡面內建的IE跑囉!

將http://cdag10-40123153.rhcloud.com/cdag10/cube10貼至IE的url中,就會開始跑囉!

(原始尺寸為150x150x150的方塊)

第一次:3375000

第二次:4096000

第三次:4913000

第四次:5832000

第五次:6859000

大致上就這樣吧\ldots{}不過 c變數
忘記動到\ldots{}不過只是照a與b同樣的型式\ldots{}將a或b複製下來改成c就可以三軸同動了。

\section{第十二週評分}\label{ux7b2cux5341ux4e8cux9031ux8a55ux5206}

40123116 : 80分

40123118 : 80分

40123153 : 80分

\section{網際鼓式煞車設計(2ag11)}\label{ux7db2ux969bux9f13ux5f0fux715eux8ecaux8a2dux8a082ag11}

有關鼓式煞車

\section{程式設計架構}\label{ux7a0bux5f0fux8a2dux8a08ux67b6ux69cb-6}

鼓式煞車

\section{結果與討論}\label{ux7d50ux679cux8207ux8a0eux8ad6-6}

這裡是結果與討論

\section{網際鼓式煞車設計(2ag12)}\label{ux7db2ux969bux9f13ux5f0fux715eux8ecaux8a2dux8a082ag12}

有關鼓式煞車

\section{程式設計架構}\label{ux7a0bux5f0fux8a2dux8a08ux67b6ux69cb-7}

鼓式煞車

\section{結果與討論}\label{ux7d50ux679cux8207ux8a0eux8ad6-7}

這裡是結果與討論

\section{協同產品設計實習專案(2ag13)}\label{ux5354ux540cux7522ux54c1ux8a2dux8a08ux5be6ux7fd2ux5c08ux68482ag13}

\subsection{組員:}\label{ux7d44ux54e1-1}

40123155

40123124

40123138

OpenShift 網站:
https://python-40123124.rhcloud.com/get\_page?heading=\%3Cspan\%3E\%E6\%9C\%9F\%E4\%B8\%AD\%E8\%80\%83\%3C/span\%3E

請各組將第八週與第九週考試的摘要報告放入 Github 協同專案中的分組報告區,
並將內容放入各組控管的同步 OpenShift 網站中. (佔期末成績 5分)

考試摘要 http://dokuwiki-40123155.rhcloud.com/doku.php

\begin{itemize}
\item
  bitbucket網站: https://40123155@bitbucket.org/40123155/---.git
\item
  dokuwiki網站: http://dokuwiki-40123155.rhcloud.com
\item
  vimeo網站: https://vimeo.com/user26959367
\item
  CMSimply 網站: https://python-40123124.rhcloud.com
\item
  下載網站: https://copy.com/SjKQ8LHZpVFq
\item
  HTML網站:
  http://dokuwiki-40123155.rhcloud.com/lib/exe/fetch.php?media=\%E6\%9C\%9F\%E4\%B8\%AD\%E5\%A0\%B1\%E5\%91\%8A.pdf
\end{itemize}

http://dokuwiki-40123155.rhcloud.com/lib/exe/fetch.php?media=\%E8\%AA\%B2\%E7\%A8\%8B\%E6\%95\%99\%E6\%9D\%90\%E4\%B8\%80\%E7\%BF\%BB\%E8\%AD\%AF\_.pdf

http://dokuwiki-40123155.rhcloud.com/lib/exe/fetch.php?media=\%E8\%AA\%B2\%E7\%A8\%8B\%E6\%95\%99\%E6\%9D\%90\%E4\%BA\%8C\%E7\%BF\%BB\%E8\%AD\%AF.pdf

組員自評:

\begin{verbatim}
40123155:80分

40123124:70分

40123138:75分
\end{verbatim}

\textless{}\textless{}\textless{}\textless{}\textless{}\textless{}\textless{}
HEAD 網際鼓式煞車設計(2ag14) ======= 網際鼓式煞車設計(2ag10)
\textgreater{}\textgreater{}\textgreater{}\textgreater{}\textgreater{}\textgreater{}\textgreater{}
26a5b9324111e15820203a5b96463a4196def80e ===

有關鼓式煞車

\section{程式設計架構}\label{ux7a0bux5f0fux8a2dux8a08ux67b6ux69cb-8}

鼓式煞車

\section{結果與討論}\label{ux7d50ux679cux8207ux8a0eux8ad6-8}

這裡是結果與討論

\section{網際鼓式煞車設計(2ag15)}\label{ux7db2ux969bux9f13ux5f0fux715eux8ecaux8a2dux8a082ag15}

有關鼓式煞車

\section{程式設計架構}\label{ux7a0bux5f0fux8a2dux8a08ux67b6ux69cb-9}

鼓式煞車

\section{結果與討論}\label{ux7d50ux679cux8207ux8a0eux8ad6-9}

這裡是結果與討論

\section{網際鼓式煞車設計(2ag16)}\label{ux7db2ux969bux9f13ux5f0fux715eux8ecaux8a2dux8a082ag16}

有關鼓式煞車

\section{程式設計架構}\label{ux7a0bux5f0fux8a2dux8a08ux67b6ux69cb-10}

鼓式煞車

\section{結果與討論}\label{ux7d50ux679cux8207ux8a0eux8ad6-10}

這裡是結果與討論

\section{第八週摘要報告}\label{ux7b2cux516bux9031ux6458ux8981ux5831ux544a}

以協同方式寫Wiki

本文目的在於讓學生利用開發的wiki與應用快速原型的協作式寫作的開發方法。而本文也由定性數據收集和分析方法的來評價。
最後,對於協作方式來寫作的方式,其影響極其討論協議到組別與軟件開發方面的考慮和教學相關的要求問題
研究方法 wiki的協作式寫作,基於設計的研究包括以下四個步驟: 1.
研究的現狀與認識目的並審查與wiki的協作式寫作相關的問題。 2.
wiki將用於促進以協同方式寫作來設計,參與和群體互動。 3.
使用多種方法收集其經驗數據。 4.
通過系統的評價分析,並通過各種方法收集資料。分析,設計,實施和評價是相互共存的。而缺點提出於每個週期提出,重新設計,重新實現,並重新評估。
維基應用域教育在協同方式寫作方法而成的Wiki是適用於多種情況,期望得到的wiki應用在學習投資效益與協同方式來寫作。它十分靈活的,足以適應的專業條件。更具體地,它的應用領域包括不同級別的更高教育,從研究生到社會教育。除了協同寫作與學科相關,維基可以為一些應用程序的開發系統來使用,如產生教材,網路上互相評論,並收集數據於一種項目。

結論和未來研究方向 1.
這項結果不能限制於狹小的研究的範圍,即使調查結果反映了那些報導中的一致性研究文獻。
2.
用來判斷以不同方式來呈現協同,尤其是同組評議,這擁有極高的教育價值與分析性思維,並參與交流。可以促進團體間的合作和時間,wiki打開了協同寫作上新的視野,而小組互動絕不是容易的事情,解決了技術,教學和文化各種問題。
未來的工作將更顯的協同的重要性,wiki以此設計為基礎下更加精進。重要的是,它更進一步影響高等教育對於wiki的使用。最後,它也進一步實踐於系統評論的基礎

\section{第九週摘要報告}\label{ux7b2cux4e5dux9031ux6458ux8981ux5831ux544a}

題目一 :請寫一個執行時可以列出以十為底對數表的網際 Python 程式, 然後
Push 到個人 bitbucket 空間, 而且同步指到 OpenShift 個人帳號上執行.

資料 : https://bitbucket.org/40123158/test

題目三 : 請在個人的 OpenShift
平台上建立一個能夠列印出與九九乘法表結果完全相同的網際程式,
接著在乘法表上端加上兩個輸入表單, 讓使用者輸入兩個整數, 按下送出鍵後,
程式會列出以此兩個整數為基底的乘法表, 例如: 若兩個欄位都輸入: 9,
則列出九九乘法表, 若輸入 9, 20, 則列出 9×20 的乘法表.

資料 : 第17組dokuwiki , 新增 abc001 \textasciitilde{} abc399 用戶資料
http://wikig17-weis.rhcloud.com/doku.php?id=start

上傳影片- 個人Vimeo https://vimeo.com/92577964

Bitbucket 連結 https://bitbucket.org/40123137/week9 Bitbucket 連結
https://bitbucket.org/40123137/week9

\section{網際鼓式煞車設計(2ag18)}\label{ux7db2ux969bux9f13ux5f0fux715eux8ecaux8a2dux8a082ag18}

有關鼓式煞車g18

\section{程式設計架構}\label{ux7a0bux5f0fux8a2dux8a08ux67b6ux69cb-11}

鼓式煞車

\section{結果與討論}\label{ux7d50ux679cux8207ux8a0eux8ad6-11}

這裡是結果與討論

\section{網際鼓式煞車設計(2ag21)}\label{ux7db2ux969bux9f13ux5f0fux715eux8ecaux8a2dux8a082ag21}

有關鼓式煞車

\section{程式設計架構}\label{ux7a0bux5f0fux8a2dux8a08ux67b6ux69cb-12}

鼓式煞車

\section{結果與討論}\label{ux7d50ux679cux8207ux8a0eux8ad6-12}

這裡是結果與討論00123

\section{網際 OpenJSCAD
程式設計(coursemdetw)}\label{ux7db2ux969b-openjscad-ux7a0bux5f0fux8a2dux8a08coursemdetw}

將 Spur 改為凸輪零件成型

\section{設計程式架構}\label{ux8a2dux8a08ux7a0bux5f0fux67b6ux69cb}

定義凸輪設計公式

\section{結果與討論}\label{ux7d50ux679cux8207ux8a0eux8ad6-13}

有關凸倫程式設計的結果與討論

\end{document}
